%
% main.tex
%

% notes = hide | show | only
\documentclass[xcolor=dvipsnames,dvip,notes=show,handout,table]{beamer}


% Para crear una versión 'handout' (impresa)
%\documentclass[xcolor=dvipsnames,dvip,handout,notes=show,table]{beamer}

%
% cabeceras.tex
%

%\usepackage[T1]{fontenc}

\definecolor{ZurichBlue}{rgb}{.255,.41,.884}

\beamertemplateshadingbackground{white!10}{white!10}

\usepackage{beamerthemeWarsaw}
\usepackage{longtable}

%\usecolortheme[named=OliveGreen]{structure} 
\setbeamertemplate{items}[ball] 
\setbeamertemplate{blocks}[rounded][shadow=true] 
\setbeamertemplate{footline}[page number]
\addtocounter{framenumber}{-1}
%Handout
%\usepackage{handoutWithNotes}
%\usepackage{tikz,times}
%\pgfpagesuselayout{2 on 1 with notes}[a4paper,border shrink=5mm]

\usepackage{beamerthemeshadow}
 \useoutertheme[hooks]{tree}
 
% \setbeamertemplate{headline}[default] % The default is just an empty headline.
% \setbeamertemplate{headline}[infolines theme]
% \setbeamertemplate{headline}[miniframes theme]
% \setbeamertemplate{headline}[sidebar theme]
% \setbeamertemplate{headline}[smoothtree theme]
% \setbeamertemplate{headline}[smoothbars theme]
% \setbeamertemplate{headline}[tree]
\beamertemplatetransparentcovereddynamic

% spanish
\usepackage[spanish]{babel}
\usepackage[utf8]{inputenc}

% diagramas
%\usepackage{pst-eps,epstopdf}
\usepackage{pst-node}
%\usepackage{pst-all}
\usepackage{pst-blur}
%\usepackage{pst-tree}

% incrustaciones de código fuente
\usepackage{listings}

% matemáticas y símbolos
\usepackage{amsmath}
\usepackage{amssymb}
\usepackage[right]{eurosym}
\usepackage{ulem}

% colores
\usepackage{colortbl}

%\usepackage{algorithm2e}
%\usepackage{algorithm}
%\usepackage{algorithmic}

\lstset{language=[90]Fortran,
  basicstyle=\ttfamily,
  keywordstyle=\color{darkred},
  commentstyle=\color{green},
  frame=trBL,
  stringstyle=\color{violet},
  frameround=tttt,
  backgroundcolor=\color{lightyellow},
  morecomment=[l]{!\ }% Comment only with space after !
}


% 
% \lstset{%
%   language=Fortran,
% 	basicstyle=\footnotesize\sffamily,
% 	keywordstyle=\color{darkred}
%  	stringstyle=\color{violet}
%  	commentstyle=\color{blue}
%  	showspaces=false,
%  	showtabs=false,
%  	showstringspaces=false,
%  	frame=trBL,
%         frameround=tttt,
%         backgroundcolor=\color{lightyellow},
%  	extendedchars=true,
%  	numbers=none,
%         aboveskip=0.5cm,
%         belowskip=0.5cm,
%         xleftmargin=1cm,
%         xrightmargin=1cm,
% 	breaklines=true
% }
\definecolor{darkred}{rgb}{0.5, 0, 0}
\definecolor{violet}{rgb}{1, 0, 1}
\definecolor{lightyellow}{rgb}{1,1,0.8}


\usepackage{latexsym}
\usepackage{amsmath}
\usepackage{amssymb}
\usepackage{amsthm}

\usepackage{xspace}



\hyphenation{real}

\newrgbcolor{ColorEncabezadoTabla}{0.7 0.7 0.9}
\newrgbcolor{ColorFila1}{0.8 0.8 0.7}
\newrgbcolor{ColorFila2}{0.8 0.7 0.8}
\newrgbcolor{ColorTotal}{0.7 0.9 0.7}


% \usepackage{tikz,times}
% \usetikzlibrary{mindmap,backgrounds}



%%%%%%%%%%%%%%%%%%%%%%%%%%%%%%%%%%%%%%%%%%%%%%%%%%%%%%%%%%%%%%%%%%%%%%

\title[Java]{Ejemplos básicos \\ Programación en Java}
\author[Jose María Álvarez Rodríguez]{\textbf{Programación} \\ \vspace{0.3cm} Jose María Álvarez Rodríguez}
\subtitle{}
\institute{Departmento de Informática \\ Universidad Carlos III de Madrid}


\date{Curso 2014/2015}

\begin{document}

\frame{
\titlepage

}
% 
\frame{
\tableofcontents

}
% 
% 
\section{Ejemplos básicos}

\frame{

\frametitle{Metodología}
\begin{exampleblock}{Pasos...}
\begin{enumerate}
 \item Analizar y entender el problema. ¿Existe algún modelo formal?
 \item Diseñar una solución al problema.
 \item Implementar el diseño propuesto.
 \item Test: verificar y validar los resultados. Traza y depuración.
 \item Documentar la solución.
 \item Refinar, "refactor" del código fuente ("mejora").
\end{enumerate}
\end{exampleblock}

}

\frame{

\frametitle{Notas}
\begin{alertblock}{...}
\begin{enumerate}
 \item En todos los ejemplos se utiliza una clase auxiliar "MyConsole" para facilitar la lectura y escritura en la consola.
 \item Todos los ejemplos están disponibles en el paquete ``examples''.
 \item ...las directivas ``package'' e ``import'' han sido borradas en estos ejemplos para facilitar su lectura. 
 \item Se presenta una solución al problema planteado pero no es única.
\end{enumerate}

 
\end{alertblock}

}


\frame{

\frametitle{Ejercicio-1}
\begin{block}{Enunciado...}
 Diseña y codifica un programa que lea dos números enteros y muestre por pantalla su suma.
\end{block}

}


\begin{frame}[fragile]
\frametitle{Ejercicio-1: examples/AddNumbers.java}
\scriptsize
\begin{lstlisting}
public class AddNumbers {
  public static void main(String []args){
    MyConsole mc = new MyConsole();
    int op1, op2, result;
    op1 = mc.readInt();
    op2 = mc.readInt();
    result = op1 + op2;
    mc.println("The result is: "+result);
  }
}
\end{lstlisting}
\end{frame}


\frame{
\frametitle{Ejercicio-2}
\begin{block}{Enunciado...}
  Diseña y codifica un programa para intercambiar el valor de dos variables de tipo entero.
\end{block}

}


\begin{frame}[fragile]
\frametitle{Ejercicio-2: examples/SwapVariables.java}
\scriptsize
\begin{lstlisting}
public class SwapVariables {
  public static void main(String []args){
    MyConsole mc = new MyConsole();
    int a,b;
    int temp;
    a = mc.readInt();
    b = mc.readInt();
    mc.println("The values (before swapping) are: "+a+", "+b);
    temp = a;
    a = b;
    b = temp;
    mc.println("The values (after swapping) are: "+a+", "+b);
  }
}
\end{lstlisting}
\end{frame}



\frame{
\frametitle{Ejercicio-3}
\begin{block}{Enunciado...}
 Diseña y codifica un programa para calcular la media de dos números.
\end{block}

}


\begin{frame}[fragile]
\frametitle{Ejercicio-3: examples/AverageTwoNumbers.java}
\scriptsize
\begin{lstlisting}
public class AverageTwoNumbers {

public static void main(String []args){
    MyConsole mc = new MyConsole();
    float op1, op2;
    float average;
    op1 = mc.readFloat();
    op2 = mc.readFloat();
    average = (op1 + op2) / 2;
    mc.println("The average is: "+average);
  }
}
\end{lstlisting}
\end{frame}


\frame{
\frametitle{Ejercicio-4}
\begin{block}{Enunciado...}
  Diseña y codifica un programa que muestre por pantalla el valor más alto de dos números reales.
\end{block}

}


\begin{frame}[fragile]
\frametitle{Ejercicio-4: examples/HighestValue.java}
\scriptsize
\begin{lstlisting}
public class HighestValue {

public static void main(String []args){
  MyConsole mc = new MyConsole();
  float op1, op2;
  op1 = mc.readFloat();
  op2 = mc.readFloat();
  //This is not a secure method to compare floats
  //To compare only int...
  if (op1 > op2){
    mc.println("The highest value is: "+op1);	
  }else{
    mc.println("The highest value is: "+op2);
  }
  }
}
\end{lstlisting}
\end{frame}

\begin{frame}[fragile]
\frametitle{Ejercicio-4 (v2 utilizando el API de Java): examples/HighestValueAPI.java}
\scriptsize
\begin{lstlisting}
public class HighestValueAPI {
  public static void main(String []args){
    MyConsole mc = new MyConsole();
    float op1, op2;
    op1 = mc.readFloat();
    op2 = mc.readFloat();
    int compare = Float.compare(op1, op2); 
    if (compare == 0){
      mc.println("Values are equals.");	
    }else if(compare < 0){
      mc.println("The highest value is: "+op2);
    }else {
      mc.println("The highest value is: "+op1);
    }
  }
}
\end{lstlisting}
\end{frame}


\frame{
\frametitle{Ejercicio-5}
\begin{block}{Enunciado...}
Diseña y codifica un programa que muestre por pantalla el valor más alto de tres números enteros.
\end{block}

}


\begin{frame}[fragile]
\frametitle{Ejercicio-5: examples/HighestThreeIntValue.java}
\scriptsize
\begin{lstlisting}
public class HighestThreeIntValue {
  public static void main(String []args){
    MyConsole mc = new MyConsole();
    int op1, op2, op3;
    op1 = mc.readInt();
    op2 = mc.readInt();
    op3 = mc.readInt();
    if (op1 > op2 && op1 > op3){
      mc.println("The highest value is: "+op1);
     }else if (op2 > op1 && op2 > op3){
      mc.println("The highest value is: "+op2);
     }else{
      mc.println("The highest value is: "+op3);
      }
  }
}
\end{lstlisting}
\end{frame}


\begin{frame}[fragile]
\frametitle{Ejercicio-5 (v2 utilizando el API de Java): examples/HighestThreeValueAPI.java}
\scriptsize
\begin{lstlisting}
public class HighestThreeValueAPI {
  public static void main(String []args){
    MyConsole mc = new MyConsole();
    float op1, op2, op3;
    op1 = mc.readFloat();
    op2 = mc.readFloat();
    op3 = mc.readFloat();
    if (Float.compare(op1, op2)>0 && 
        Float.compare(op1, op3)>0){
      mc.println("The highest value is: "+op1);
     }else if (Float.compare(op2, op3)>0 && 
               Float.compare(op2, op3)>0){
      mc.println("The highest value is: "+op2);
     }else{
      mc.println("The highest value is: "+op3);
      }
  }
}
\end{lstlisting}
\end{frame}


\frame{
\frametitle{Ejercicio-6}
\begin{block}{Enunciado...}
Diseña y codifica un programa que muestre por pantalla los primeros 20 números naturales (ascendente) utilizando diferentes tipos de bucles (\texttt{do-while}, \texttt{while} y \texttt{for}).
\end{block}

}


\begin{frame}[fragile]
\frametitle{Ejercicio-6: examples/ShowFirst20Numbers.java}
\scriptsize
\begin{lstlisting}
public class ShowFirst20Numbers {
  public static void main(String []args){
    int top = 20;
    int i = 0;
    MyConsole mc = new MyConsole();
    do{
      mc.println(i);
      i = i + 1; //i++
    }while(i<=top);

    for(int j = 0; j<=top; j++){
      mc.println(j);
     }
    
    int k = 0;
    while (k<=top){
      mc.println(k);
      k = k+1; //k++
    }
  }
}
\end{lstlisting}
\end{frame}


\frame{
\frametitle{Ejercicio-6}
\begin{block}{Enunciado...}
Diseña y codifica un programa que muestre por pantalla los primeros 20 números naturales (descendente) utilizando diferentes tipos de bucles (\texttt{do-while}, \texttt{while} y \texttt{for}).
\end{block}

}


\begin{frame}[fragile]
\frametitle{Ejercicio-6: examples/ShowFirst20NumbersDesc.java}
\scriptsize
\begin{lstlisting}
public class ShowFirst20NumbersDesc {
  public static void main(String []args){		
    MyConsole mc = new MyConsole();
    int i = 20;
    do{
      mc.println(i);
      i = i-1 ; //i--
    }while(i>=0);

    for(int j = 20; j>=0; j--){
      mc.println(j);
    }
    int k = 20;
    while (k>=0){
      mc.println(k);
      k = k-1; //k--
    }
  }
}
\end{lstlisting}
\end{frame}


\frame{
\frametitle{Ejercicio-7}
\begin{block}{Enunciado...}
Diseña y codifica un programa que muestre por pantalla el resultado de la suma de los primeros 20 números naturales utilizando diferentes tipos de bucles (\texttt{do-while}, \texttt{while} y \texttt{for}).
\end{block}

}


\begin{frame}[fragile]
\frametitle{Ejercicio-7: examples/SumFirst20Numbers.java}
\tiny
\begin{lstlisting}
public class SumFirst20Numbers {
  public static void main(String []args){
    int top = 20;
    int i = 0;
    int sum = 0;
    MyConsole mc = new MyConsole();
    sum = 0;
    do{
      sum = sum + i;
      i = i + 1; //i++
    }while(i<=top);		
    mc.println("After the first loop..."+sum);
    sum=0;
    for(int j = 0; j<=top; j++){
      sum = sum + j;
    }
    mc.println("After the second loop..."+sum);
    int k = 0;
    sum = 0;
    while (k<=top){
      sum = sum + k;
      k = k+1; //k++
    }
  mc.println("After the third loop..."+sum);
 }
}
\end{lstlisting}
\end{frame}


\frame{
\frametitle{Ejercicio-8}
\begin{block}{Enunciado...}
	Diseña y codifica un programa que muestre por pantalla los números pares presentes en los primeros 20 números naturales utilizando diferentes tipos de bucles (\texttt{do-while}, \texttt{while} y \texttt{for}).
\end{block}

}


\begin{frame}[fragile]
\frametitle{Ejercicio-8: examples/EvenIn20Numbers.java}
\tiny
\begin{lstlisting}
public class EvenIn20Numbers {
  public static void main(String []args){
  int top = 20;
  int i = 0;
  MyConsole mc = new MyConsole();
  do{
    if (i%2 == 0){
      mc.println(i);
    }
  i = i + 1; //i++
  }while(i<=top);		
  for(int j = 0; j<=top; j++){
    if (j%2 == 0){
      mc.println(j);
    }
   }
  int k = 0;
  while (k<=top){
    if (k%2 == 0){
      mc.println(k);
    }
  k = k+1; //k++
  }
  while (k++<=top){
  if (k%2 == 0){
    mc.println(k);
    }
  }}}
\end{lstlisting}
\end{frame}



\frame{
\frametitle{Ejercicio-9}
\begin{block}{Enunciado...}
	Diseña y codifica un programa que cuente los números impares presentes en los primeros 20 números naturales utilizando diferentes tipos de bucles (\texttt{do-while}, \texttt{while} y \texttt{for}).
\end{block}

}


\begin{frame}[fragile]
\frametitle{Ejercicio-9: examples/CountOddIn20Numbers.java}
\tiny
\begin{lstlisting}
public class CountOddIn20Numbers {
  public static void main(String []args){
    int top = 20; int i = 0; int counter = 0;
    MyConsole mc = new MyConsole();
    do{
      if (i%2 != 0){
       counter++;
      }
      i = i + 1; //i++
     }while(i<=top);				
    mc.println("Found "+counter+" odd numbers.");
    counter = 0;
    for(int j = 0; j<=top; j++){
      if (j%2 != 0){
      counter++;
      }
    }
   mc.println("Found "+counter+" odd numbers.");
   int k = 0;
   counter = 0;
   while (k<=top){
     if (k%2 != 0){
       counter++;
     }
     k = k+1; //k++
    }
   mc.println("Found "+counter+" odd numbers.");

 }}
\end{lstlisting}
\end{frame}




\frame{
\frametitle{Ejercicio-10}
\begin{block}{Enunciado...}
	Diseña y codifica un programa que muestre el cuadrado de los primeros 20 números naturales utilizando diferentes tipos de bucles (\texttt{do-while}, \texttt{while} y \texttt{for}).
\end{block}

}


\begin{frame}[fragile]
\frametitle{Ejercicio-10: examples/SquareFirst20Numbers.java}
\tiny
\begin{lstlisting}
public class SquareFirst20Numbers {
  public static void main(String []args){
    int top = 20;
    int i = 0;
    MyConsole mc = new MyConsole();
    do{
      mc.println(i*i);
      i = i + 1; //i++
    }while(i<=top);
  
  for(int j = 0; j<=top; j++){
    mc.println(j*j);
    }

  int k = 0;
  while (k<=top){
    mc.println(k*k);
    k = k+1; //k++
  }
 }
}
\end{lstlisting}
\end{frame}


\frame{
\frametitle{Ejercicio-11}
\begin{block}{Enunciado...}
	Diseña y codifica un programa que muestre el cubo de los primeros 20 números naturales utilizando diferentes tipos de bucles (\texttt{do-while}, \texttt{while} y \texttt{for}).
\end{block}

}


\begin{frame}[fragile]
\frametitle{Ejercicio-11: examples/CubeFirst20Numbers.java}
\tiny
\begin{lstlisting}
public class CubeFirst20Numbers {
  public static void main(String []args){
    int top = 20;
    int i = 0;
    MyConsole mc = new MyConsole();
    do{
      mc.println(i*i*i);
      i = i + 1; //i++
    }while(i<=top);

    for(int j = 0; j<=top; j++){
      mc.println(j*j*j);
    }
    int k = 0;
    while (k<=top){
      mc.println(k*k*k);
      k = k+1; //k++
    }
  }
}
\end{lstlisting}
\end{frame}


\frame{
\frametitle{Ejercicio-12}
\begin{block}{Enunciado...}
 ...ahora utilizando el API de Java Math.
\end{block}

}


\begin{frame}[fragile]
\frametitle{Ejercicio-12: examples/CubeFirst20NumbersMathAPI.java}
\tiny
\begin{lstlisting}
public class CubeFirst20NumbersMathAPI {
  public static void main(String []args){
    int top = 20;
    int i = 0;
    MyConsole mc = new MyConsole();
    do{
      mc.println(Math.pow(i, 3));
      i = i + 1; //i++
     }while(i<=top);
    for(int j = 0; j<=top; j++){
      mc.println(Math.pow(j, 3));
    }
    int k = 0;
    while (k<=top){
      mc.println(Math.pow(k, 3));
      k = k+1; //k++
    }
  }
}
\end{lstlisting}
\end{frame}



\frame{
\frametitle{Ejercicio-13}
\begin{block}{Enunciado...}
 Diseña y codifica un programa para calcular la potencia de cualquier número en la forma $a^b$.
\end{block}

}


\begin{frame}[fragile]
\frametitle{Ejercicio-13: examples/CubeFirst20NumbersMathAPI.java}
\tiny
\begin{lstlisting}
public class MyOwnPow {
  public static void main(String []args){
    MyConsole mc = new MyConsole();
    double base = 0;
    int exponent = 0;
    double value = 1;
    base = mc.readDouble();
    exponent = mc.readInt();
    for (int i = 0; i<exponent;i++){
      value = value  * base; 
    }
  mc.println("The pow of "+base+" raised to "+exponent+" is equals to "+value);
  mc.println("Using the Java API: "+Math.pow(base, exponent));
  }
}
\end{lstlisting}
\end{frame}



\begin{frame}[fragile]
\frametitle{Ejercicio-14}
\begin{block}{Enunciado...}
Diseña y codifica un programa para saber si un año es bisiesto ("leap") o no.
\end{block}
\begin{lstlisting}
if year is divisible by 400 then
   is_leap_year
else if year is divisible by 100 then
   not_leap_year
else if year is divisible by 4 then
   is_leap_year
else
   not_leap_year
\end{lstlisting}

Source: \url{http://en.wikipedia.org/wiki/Leap\_year}
\end{frame}


\begin{frame}[fragile]
\frametitle{Ejercicio-14: examples/LeapYear.java}
\scriptsize
\begin{lstlisting}
public class LeapYear {
  public static void main(String []args){
    MyConsole mc = new MyConsole();
    int year = 0;
    year = mc.readInt();
    boolean isLeapYear = (year % 4 == 0);
    // divisible by 4 and not 100
    isLeapYear = isLeapYear && (year % 100 != 0);
    // divisible by 4 and not 100 unless divisible by 400
    isLeapYear = isLeapYear || (year % 400 == 0);
    mc.println(isLeapYear);	
  }
}
\end{lstlisting}
\end{frame}



\begin{frame}[fragile]
\frametitle{Ejercicio-15}
\begin{block}{Enunciado...}
Diseña y codifica un programa para saber si un número es palíndromo.
\end{block}

\begin{exampleblock}{Palíndromo}
  Un palíndromo es una palabra, número o frase que se lee igual hacia adelante que hacia atrás. Si se trata de un número, se llama capicúa.
  \begin{itemize}
   \item 121 es palíndromo.
   \item 123 es palíndromo.
  \end{itemize}

\end{exampleblock}


\end{frame}


\begin{frame}[fragile]
\frametitle{Ejercicio-15: examples/PalindromeNumber.java}
\scriptsize
\begin{lstlisting}
public class PalindromeNumber {
  public static void main(String []args){
    MyConsole mc = new MyConsole();
    int number;
    int palindrome, reverse;
    number = mc.readInt();
    palindrome = number;
    reverse = 0;
    while (palindrome != 0) {
      int remainder = palindrome % 10;
      reverse = reverse * 10 + remainder;
      palindrome = palindrome / 10;
    }
    mc.println("The number "+number+"is palindrome "+ 
      (number==reverse));
  }
}
\end{lstlisting}
\end{frame}


\begin{frame}[fragile]
\frametitle{Ejercicio-16}
\begin{block}{Enunciado...}
Diseña y codifica un programa que muestre por pantalla si un número es un número Armstrong.
\end{block}

\begin{exampleblock}{Armstrong number}
\textit{ An Armstrong number of three digit is a number whose sum of cubes of its digit is equal to its number. }
  \begin{itemize}
   \item  153 = $1^3+5^3+3^3$ = $1+125+27$=153 \textit{is an Armstrong Number.}
   \item  Ver \url{http://en.wikipedia.org/wiki/Narcissistic\_number}
  \end{itemize}

\end{exampleblock}


\end{frame}


\begin{frame}[fragile]
\frametitle{Ejercicio-16: examples/ArmstrongNumber.java}
\scriptsize
\begin{lstlisting}
public class ArmstrongNumber {
  public static void main(String []args){
    MyConsole mc = new MyConsole();
    int number = 0;
    int result, source;
    number = mc.readInt();
    source = number;
    result = 0;
    while(number != 0){
      int remainder = number%10;
      result = result + remainder*remainder*remainder;
      number = number/10;
     }
    mc.println("The number "+number+" is Armstrong number "+
      (source==result));        
  }
}
\end{lstlisting}
\end{frame}


\begin{frame}[fragile]
\frametitle{Ejercicio-17}
\begin{block}{Enunciado...}
 Diseña y codifica un programa que muestre por pantalla la siguiente figura: "Wedge of Stars" (dado un número de estrellas determinado).
\end{block}

\begin{lstlisting}
 *******
 ******
 *****
 ****
 ***
 **
 *
\end{lstlisting}



\end{frame}


\begin{frame}[fragile]
\frametitle{Ejercicio-17: examples/WedgeOfStars.java}
\scriptsize
\begin{lstlisting}
public class WedgeOfStars {
  public static void main(String args[]){
    MyConsole mc = new MyConsole();
    int numberOfStars = 0;
    numberOfStars = mc.readInt();
    for (int i = numberOfStars; i>0; i--){
      for(int j = 0; j<i; j++){
	mc.print("*");
      }
      mc.println("");
      }
  }
}
\end{lstlisting}
\end{frame}




\begin{frame}[fragile]
\frametitle{Ejercicio-18}
\begin{block}{Enunciado...}
 Diseña y codifica un programa que muestre por pantalla un árbol de navidad.
\end{block}

\begin{lstlisting}
       *
      ***
     *****
    *******
   *********
  ***********
 *************
***************
      ***
      ***
      ***
\end{lstlisting}



\end{frame}


\begin{frame}[fragile]
\frametitle{Ejercicio-18: examples/HolidayTree.java}
\tiny
\begin{lstlisting}
public class HolidayTree {
  public static void main(String []args) throws IOException{
    MyConsole mc = new MyConsole();
    int baseStars = 15;
    int halfBlankSpaces = 0;
    for (int i = 1; i<baseStars; i=i+2){
      halfBlankSpaces = (baseStars-i)/2;
      for(int j = 0; j<halfBlankSpaces;j++) {mc.print(" ");}
      for(int k = 0; k<i;k++) {mc.print("*");}
      for(int j = 0; j<halfBlankSpaces;j++) { mc.print(" ");}
      mc.println("");
    }
    halfBlankSpaces = (baseStars/2)-1;
    for(int i = 0; i<3;i++){
      for(int j = 0; j<halfBlankSpaces;j++){mc.print(" ");}
      for(int k = 0; k<3;k++){mc.print("*");}
      for(int j = 0; j<halfBlankSpaces;j++){mc.print(" ");}
    mc.println(" ");
    }
}
}
\end{lstlisting}
\end{frame}


\begin{frame}[fragile]
\frametitle{Ejercicio-19}
\begin{block}{Enunciado...}
  Otros ejemplos utilizando el API de Java Math.
\end{block}

\end{frame}


\begin{frame}[fragile]
\frametitle{Ejercicio-19: examples/MathAPIExample.java}
\tiny
\begin{lstlisting}
public class MathAPIExample {
  public static void main(String []args){
    MyConsole mc = new MyConsole();
    //Max
    mc.println(Math.max(2, 3));
    //Min
    mc.println(Math.min(2, 3));
    //Abs		
    mc.println(Math.abs(-2));
    //Pow
    mc.println(Math.pow(2, 3));
    //Sqrt
    mc.println(Math.sqrt(144));
  }
}
\end{lstlisting}
\end{frame}



\begin{frame}[fragile]
\frametitle{Ejercicio-20}
\begin{block}{Enunciado...}
Diseña y codifica un programa que muestre por pantalla si un número es primo.
\end{block}

\begin{exampleblock}{Definition from Wikipedia}
Un número primo es un número natural mayor que 1 que tiene únicamente dos divisores distintos: él mismo y el 1.
\end{exampleblock}

\end{frame}


\begin{frame}[fragile]
\frametitle{Ejercicio-20: examples/PrimeNumber.java}
\scriptsize
\begin{lstlisting}
public class PrimeNumber {
  public static void main (String []args){
    MyConsole mc = new MyConsole();
    int number = 0;
    boolean isPrime = true;
    number = mc.readInt();
    int i = 2;
    while(isPrime && i!=number){
      if(number%i ==  0){
	isPrime = false;
      }
      i++;
    }
    mc.println("The number "+number+" is prime "+isPrime);
  }
}
\end{lstlisting}
\end{frame}




\begin{frame}[fragile]
\frametitle{Ejercicio-21}
\begin{block}{Enunciado...}
Diseña y codifica un programa que muestre por pantalla los primeros 100 números primos.
\end{block}

\end{frame}


\begin{frame}[fragile]
\frametitle{Ejercicio-21: examples/First100PrimeNumbers.java}
\tiny
\begin{lstlisting}
public class First100PrimeNumbers {
  public static void main (String []args){
    MyConsole mc = new MyConsole();
    int nprimes = 100;
    int matches = 1; //1 is prime Print 1 and 2
    int number = 2; //...we start in 2...
    boolean isPrime = true;			
    int i = 2;
    while (matches<nprimes){
      isPrime = true;
      i = 2;
      while(isPrime && i!=number){
	if(number%i ==  0){
	  isPrime = false;
	}
      i++;
     }
    if (isPrime){
      mc.println(number);
      matches++;
    }
    number++;
   }
 }
}
\end{lstlisting}
\end{frame}




\begin{frame}[fragile]
\frametitle{Ejercicio-22}
\begin{block}{Enunciado...}
Diseña y codifica un programa que sume los primeros $n$ números ($n$ se lee por pantalla).
\end{block}

\end{frame}


\begin{frame}[fragile]
\frametitle{Ejercicio-22: examples/AddingUpIntegers.java}
\scriptsize
\begin{lstlisting}
public class AddingUpIntegers {
  public static void main(String []args){
    MyConsole mc = new MyConsole();
    int numberToRead = 0;
    numberToRead = mc.readInt();
    if(numberToRead>0){
      int value = 0;
      int sum = 0;
      for(int i = 0; i<numberToRead;i++){
	mc.println("Enter a number...");
	value = mc.readInt();
	sum = sum + value;
      }
      mc.println("The sum is "+sum);
      }
  }
}
\end{lstlisting}
\end{frame}




\begin{frame}[fragile]
\frametitle{Ejercicio-23}
\begin{block}{Enunciado...}
Diseña y codifica un programa que calcule el factorial de un número.
\end{block}


\begin{equation}
n!=\prod_{k=1}^n k \!  
\end{equation}


\begin{equation}
n! = \begin{cases}
1 & \text{if } n = 0, \\
(n-1)!\times n & \text{if } n > 0.
\end{cases}
\end{equation}


\end{frame}


\begin{frame}[fragile]
\frametitle{Ejercicio-23: examples/Factorial.java}
\scriptsize
\begin{lstlisting}
public class Factorial {
  public static void main(String []args){
    MyConsole mc = new MyConsole();
    int op1;
    double fact;
    op1 = mc.readInt();
    //fact(n) = n * fact(n-1);
    if (op1 == 0 || op1 == 1){
      fact = 1;
     }else{
     fact = 1;
     for (int i = op1; i>1; i--){
      fact = fact * i; 
      }
    }
    mc.println("The factorial of: "+op1+" is "+fact);
  }
}
\end{lstlisting}
\end{frame}



\begin{frame}[fragile]
\frametitle{Ejercicio-24}
\begin{block}{Enunciado...}
Diseña y codifica un programa para generar la sucesión de Fibonacci para $n$ números.
\end{block}
% 
\begin{exampleblock}{Definición en la Wikipedia}
The Fibonacci sequence is the numbers in the following integer sequence:
0,1,1,2,3,5,8,13,21...
\end{exampleblock}


\begin{equation}
F_n = \begin{cases}
0 & \text{if } n = 0, \\
1 & \text{if } n = 1, \\
F_{n-1} + F_{n-2} & n > 1.\\
\end{cases}
\end{equation}



\end{frame}


\begin{frame}[fragile]
\frametitle{Ejercicio-24: examples/Fibonacci.java}
\scriptsize
\begin{lstlisting}
public class Fibonacci {
  public static void main(String []args){
    MyConsole mc = new MyConsole();
    int numbersToGenerate;
    int fn1, fn2, fcurrent, temp;
    numbersToGenerate = mc.readInt();
    fn2 = 0;
    fn1 = 1;
    mc.println(fn2);
    mc.println(fn1);
    for (int i = 2; i < numbersToGenerate; i++){
      fcurrent = fn1 + fn2;
      temp = fn1;
      fn1 = fcurrent;
      fn2 = temp;
      mc.println(fcurrent);
    }
  }
}
\end{lstlisting}
\end{frame}

\begin{frame}[fragile]
\frametitle{Ejercicio-25}
\begin{block}{Enunciado...}
Diseña y codifica un programa que muestra la tabla de multiplicar (de 1 a 10).
\end{block}
% 
\end{frame}

\begin{frame}[fragile]
\frametitle{Ejercicio-25: examples/MultiplicationTable.java}
\scriptsize
\begin{lstlisting}
public class MultiplicationTable {
  public static void main(String[] args) {
    MyConsole mc = new MyConsole();
    for (int i = 1; i <= 10; i++) {
      for (int j = 1; j <= 10; j++) {
	  mc.print( i * j );
	  mc.print("\t");
	}
      mc.println("");
      }
   }
}
\end{lstlisting}
\end{frame}



\begin{frame}[fragile]
\frametitle{Ejercicio-26}
\begin{block}{Enunciado...}
Diseña y codifica un programa para calcular el número combinatorio.
\end{block}
% 

\begin{equation}
C_{k} = \binom{n}{k} = \frac{n!}{k!(n-k)!} 
\end{equation}

\end{frame}



\begin{frame}[fragile]
\frametitle{Siguiente...}
\begin{exampleblock}{Comentarios...}
\begin{itemize}
 \item Intentar diseñar y codificar los algoritmos por uno mismo.
 \item Si se encuentra algún error, por favor notificadlo.
 \item Un mismo problema puede tener varias soluciones.
 \item Es conveniente seguir la metodología propuesta (Análisis, Diseño, Implementación y Prueba)
 \item ...si hay alguna duda...por favor: ¡PREGUNTEN!
\end{itemize}

\end{exampleblock}



\end{frame}

% 
\frame{
\titlepage

}


% %%%%%%%%%%%%%%%%%%%%%%%%%%%%%%%%%%%%%%%%%%%%%%%%%%%%%%%%%%%%%%%%%%%%%%

\end{document}
