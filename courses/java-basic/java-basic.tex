%
% main.tex
%

% notes = hide | show | only
\documentclass[xcolor=dvipsnames,dvip,notes=show,table]{beamer}


% Para crear una versión 'handout' (impresa)
%\documentclass[xcolor=pst,dvips,handout,notes=show]{beamer}

%
% cabeceras.tex
%

%\usepackage[T1]{fontenc}

\definecolor{ZurichBlue}{rgb}{.255,.41,.884}

\beamertemplateshadingbackground{white!10}{white!10}

\usepackage{beamerthemeWarsaw}
\usepackage{longtable}

%\usecolortheme[named=OliveGreen]{structure} 
\setbeamertemplate{items}[ball] 
\setbeamertemplate{blocks}[rounded][shadow=true] 
\setbeamertemplate{footline}[page number]
\addtocounter{framenumber}{-1}
%Handout
%\usepackage{handoutWithNotes}
%\usepackage{tikz,times}
%\pgfpagesuselayout{2 on 1 with notes}[a4paper,border shrink=5mm]

\usepackage{beamerthemeshadow}
 \useoutertheme[hooks]{tree}
 
% \setbeamertemplate{headline}[default] % The default is just an empty headline.
% \setbeamertemplate{headline}[infolines theme]
% \setbeamertemplate{headline}[miniframes theme]
% \setbeamertemplate{headline}[sidebar theme]
% \setbeamertemplate{headline}[smoothtree theme]
% \setbeamertemplate{headline}[smoothbars theme]
% \setbeamertemplate{headline}[tree]
\beamertemplatetransparentcovereddynamic

% spanish
\usepackage[spanish]{babel}
\usepackage[utf8]{inputenc}

% diagramas
%\usepackage{pst-eps,epstopdf}
\usepackage{pst-node}
%\usepackage{pst-all}
\usepackage{pst-blur}
%\usepackage{pst-tree}

% incrustaciones de código fuente
\usepackage{listings}

% matemáticas y símbolos
\usepackage{amsmath}
\usepackage{amssymb}
\usepackage[right]{eurosym}
\usepackage{ulem}

% colores
\usepackage{colortbl}

%\usepackage{algorithm2e}
%\usepackage{algorithm}
%\usepackage{algorithmic}

\lstset{language=[90]Fortran,
  basicstyle=\ttfamily,
  keywordstyle=\color{darkred},
  commentstyle=\color{green},
  frame=trBL,
  stringstyle=\color{violet},
  frameround=tttt,
  backgroundcolor=\color{lightyellow},
  morecomment=[l]{!\ }% Comment only with space after !
}


% 
% \lstset{%
%   language=Fortran,
% 	basicstyle=\footnotesize\sffamily,
% 	keywordstyle=\color{darkred}
%  	stringstyle=\color{violet}
%  	commentstyle=\color{blue}
%  	showspaces=false,
%  	showtabs=false,
%  	showstringspaces=false,
%  	frame=trBL,
%         frameround=tttt,
%         backgroundcolor=\color{lightyellow},
%  	extendedchars=true,
%  	numbers=none,
%         aboveskip=0.5cm,
%         belowskip=0.5cm,
%         xleftmargin=1cm,
%         xrightmargin=1cm,
% 	breaklines=true
% }
\definecolor{darkred}{rgb}{0.5, 0, 0}
\definecolor{violet}{rgb}{1, 0, 1}
\definecolor{lightyellow}{rgb}{1,1,0.8}


\usepackage{latexsym}
\usepackage{amsmath}
\usepackage{amssymb}
\usepackage{amsthm}

\usepackage{xspace}



\hyphenation{real}

\newrgbcolor{ColorEncabezadoTabla}{0.7 0.7 0.9}
\newrgbcolor{ColorFila1}{0.8 0.8 0.7}
\newrgbcolor{ColorFila2}{0.8 0.7 0.8}
\newrgbcolor{ColorTotal}{0.7 0.9 0.7}


% \usepackage{tikz,times}
% \usetikzlibrary{mindmap,backgrounds}



%%%%%%%%%%%%%%%%%%%%%%%%%%%%%%%%%%%%%%%%%%%%%%%%%%%%%%%%%%%%%%%%%%%%%%

\title[Java]{Basic Examples of \\ the Java Programming Language}
\author[Jose María Álvarez Rodríguez]{\textbf{Programming Course} \\ \vspace{0.3cm} Jose María Álvarez Rodríguez}
\subtitle{}
\institute{Department of Computer Science \\ Carlos III University of Madrid}


\date{Course 2013/2014}

\begin{document}

\frame{
\titlepage

}
% 
\frame{
\tableofcontents

}
% 
% 
\section{Basic Examples}

\frame{

\frametitle{Methodology}
\begin{exampleblock}{Steps...}
\begin{enumerate}
 \item Analyze and understand the problem. Is there any formal model?
 \item Design a solution.
 \item Implement the solution.
 \item Test: verify the results. Trace and Debug the program.
 \item Document the solution.
 \item Refine, refactor the existing code (Improvement).
\end{enumerate}
\end{exampleblock}

}

\frame{

\frametitle{Remarks}
\begin{alertblock}{...}
\begin{enumerate}
 \item In all examples a class called MyConsole is used to read and write from/to the Java console. It is a 
 helper class to deal with simple datatypes.
 \item All examples are available in the package ``examples''.
 \item ...the ``package'' and ``import'' directives have been removed to ease the reading of the following examples
 \item We offer a potential solution more approaches could be applied to solve the same problem.
\end{enumerate}

 
\end{alertblock}

}


\frame{

\frametitle{Exercise-1}
\begin{block}{Statement...}
 Write a program to read and add two numbers.
\end{block}

}


\begin{frame}[fragile]
\frametitle{Exercise-1: examples/AddNumbers.java}
\scriptsize
\begin{lstlisting}
public class AddNumbers {
  public static void main(String []args){
    MyConsole mc = new MyConsole();
    int op1, op2, result;
    op1 = mc.readInt();
    op2 = mc.readInt();
    result = op1 + op2;
    mc.println("The result is: "+result);
  }
}
\end{lstlisting}
\end{frame}


\frame{
\frametitle{Exercise-2}
\begin{block}{Statement...}
 Write a program to swap the values of two integer variables.
\end{block}

}


\begin{frame}[fragile]
\frametitle{Exercise-2: examples/SwapVariables.java}
\scriptsize
\begin{lstlisting}
public class SwapVariables {
  public static void main(String []args){
    MyConsole mc = new MyConsole();
    int a,b;
    int temp;
    a = mc.readInt();
    b = mc.readInt();
    mc.println("The values (before swapping) are: "+a+", "+b);
    temp = a;
    a = b;
    b = temp;
    mc.println("The values (after swapping) are: "+a+", "+b);
  }
}
\end{lstlisting}
\end{frame}



\frame{
\frametitle{Exercise-3}
\begin{block}{Statement...}
 Write a program to calculate the average of two numbers.
\end{block}

}


\begin{frame}[fragile]
\frametitle{Exercise-3: examples/AverageTwoNumbers.java}
\scriptsize
\begin{lstlisting}
public class AverageTwoNumbers {

public static void main(String []args){
    MyConsole mc = new MyConsole();
    float op1, op2;
    float average;
    op1 = mc.readFloat();
    op2 = mc.readFloat();
    average = (op1 + op2) / 2;
    mc.println("The average is: "+average);
  }
}
\end{lstlisting}
\end{frame}


\frame{
\frametitle{Exercise-4}
\begin{block}{Statement...}
 Write a program to calculate the highest value of two float numbers.
\end{block}

}


\begin{frame}[fragile]
\frametitle{Exercise-4: examples/HighestValue.java}
\scriptsize
\begin{lstlisting}
public class HighestValue {

public static void main(String []args){
  MyConsole mc = new MyConsole();
  float op1, op2;
  op1 = mc.readFloat();
  op2 = mc.readFloat();
  //This is not a secure method to compare floats
  //To compare only int...
  if (op1 > op2){
    mc.println("The highest value is: "+op1);	
  }else{
    mc.println("The highest value is: "+op2);
  }
  }
}
\end{lstlisting}
\end{frame}

\begin{frame}[fragile]
\frametitle{Exercise-4 (v2 using the Java API): examples/HighestValueAPI.java}
\scriptsize
\begin{lstlisting}
public class HighestValueAPI {
  public static void main(String []args){
    MyConsole mc = new MyConsole();
    float op1, op2;
    op1 = mc.readFloat();
    op2 = mc.readFloat();
    int compare = Float.compare(op1, op2); 
    if (compare == 0){
      mc.println("Values are equals.");	
    }else if(compare < 0){
      mc.println("The highest value is: "+op2);
    }else {
      mc.println("The highest value is: "+op1);
    }
  }
}
\end{lstlisting}
\end{frame}


\frame{
\frametitle{Exercise-5}
\begin{block}{Statement...}
 Write a program to calculate the highest value of three integer numbers.
\end{block}

}


\begin{frame}[fragile]
\frametitle{Exercise-5: examples/HighestThreeIntValue.java}
\scriptsize
\begin{lstlisting}
public class HighestThreeIntValue {
  public static void main(String []args){
    MyConsole mc = new MyConsole();
    int op1, op2, op3;
    op1 = mc.readInt();
    op2 = mc.readInt();
    op3 = mc.readInt();
    if (op1 > op2 && op1 > op3){
      mc.println("The highest value is: "+op1);
     }else if (op2 > op1 && op2 > op3){
      mc.println("The highest value is: "+op2);
     }else{
      mc.println("The highest value is: "+op3);
      }
  }
}
\end{lstlisting}
\end{frame}


\begin{frame}[fragile]
\frametitle{Exercise-5 (v2 using the Java API): examples/HighestThreeValueAPI.java}
\scriptsize
\begin{lstlisting}
public class HighestThreeValueAPI {
  public static void main(String []args){
    MyConsole mc = new MyConsole();
    float op1, op2, op3;
    op1 = mc.readFloat();
    op2 = mc.readFloat();
    op3 = mc.readFloat();
    if (Float.compare(op1, op2)>0 && 
        Float.compare(op1, op3)>0){
      mc.println("The highest value is: "+op1);
     }else if (Float.compare(op2, op3)>0 && 
               Float.compare(op2, op3)>0){
      mc.println("The highest value is: "+op2);
     }else{
      mc.println("The highest value is: "+op3);
      }
  }
}
\end{lstlisting}
\end{frame}


\frame{
\frametitle{Exercise-6}
\begin{block}{Statement...}
 Write a program to show the first 20 natural numbers (using 3 different loops).
\end{block}

}


\begin{frame}[fragile]
\frametitle{Exercise-6: examples/ShowFirst20Numbers.java}
\scriptsize
\begin{lstlisting}
public class ShowFirst20Numbers {
  public static void main(String []args){
    int top = 20;
    int i = 0;
    MyConsole mc = new MyConsole();
    do{
      mc.println(i);
      i = i + 1; //i++
    }while(i<=top);

    for(int j = 0; j<=top; j++){
      mc.println(j);
     }
    
    int k = 0;
    while (k<=top){
      mc.println(k);
      k = k+1; //k++
    }
  }
}
\end{lstlisting}
\end{frame}


\frame{
\frametitle{Exercise-6}
\begin{block}{Statement...}
 Write a program to show the first 20 natural numbers (descending and using 3 different loops).
\end{block}

}


\begin{frame}[fragile]
\frametitle{Exercise-6: examples/ShowFirst20NumbersDesc.java}
\scriptsize
\begin{lstlisting}
public class ShowFirst20NumbersDesc {
  public static void main(String []args){		
    MyConsole mc = new MyConsole();
    int i = 20;
    do{
      mc.println(i);
      i = i-1 ; //i--
    }while(i>=0);

    for(int j = 20; j>=0; j--){
      mc.println(j);
    }
    int k = 20;
    while (k>=0){
      mc.println(k);
      k = k-1; //k--
    }
  }
}
\end{lstlisting}
\end{frame}


\frame{
\frametitle{Exercise-7}
\begin{block}{Statement...}
 Write a program to add up the first 20 natural numbers (using 3 different loops).
\end{block}

}


\begin{frame}[fragile]
\frametitle{Exercise-7: examples/SumFirst20Numbers.java}
\tiny
\begin{lstlisting}
public class SumFirst20Numbers {
  public static void main(String []args){
    int top = 20;
    int i = 0;
    int sum = 0;
    MyConsole mc = new MyConsole();
    sum = 0;
    do{
      sum = sum + i;
      i = i + 1; //i++
    }while(i<=top);		
    mc.println("After the first loop..."+sum);
    sum=0;
    for(int j = 0; j<=top; j++){
      sum = sum + j;
    }
    mc.println("After the second loop..."+sum);
    int k = 0;
    sum = 0;
    while (k<=top){
      sum = sum + k;
      k = k+1; //k++
    }
  mc.println("After the third loop..."+sum);
 }
}
\end{lstlisting}
\end{frame}


\frame{
\frametitle{Exercise-8}
\begin{block}{Statement...}
 Write a program to print the even numbers between 0 and 20 (using 3 different loops).
\end{block}

}


\begin{frame}[fragile]
\frametitle{Exercise-8: examples/EvenIn20Numbers.java}
\tiny
\begin{lstlisting}
public class EvenIn20Numbers {
  public static void main(String []args){
  int top = 20;
  int i = 0;
  MyConsole mc = new MyConsole();
  do{
    if (i%2 == 0){
      mc.println(i);
    }
  i = i + 1; //i++
  }while(i<=top);		
  for(int j = 0; j<=top; j++){
    if (j%2 == 0){
      mc.println(j);
    }
   }
  int k = 0;
  while (k<=top){
    if (k%2 == 0){
      mc.println(k);
    }
  k = k+1; //k++
  }
  while (k++<=top){
  if (k%2 == 0){
    mc.println(k);
    }
  }}}
\end{lstlisting}
\end{frame}



\frame{
\frametitle{Exercise-9}
\begin{block}{Statement...}
 Write a program to count the odd numbers between 0 and 20 (using different loops).
\end{block}

}


\begin{frame}[fragile]
\frametitle{Exercise-9: examples/CountOddIn20Numbers.java}
\tiny
\begin{lstlisting}
public class CountOddIn20Numbers {
  public static void main(String []args){
    int top = 20; int i = 0; int counter = 0;
    MyConsole mc = new MyConsole();
    do{
      if (i%2 != 0){
       counter++;
      }
      i = i + 1; //i++
     }while(i<=top);				
    mc.println("Found "+counter+" odd numbers.");
    counter = 0;
    for(int j = 0; j<=top; j++){
      if (j%2 != 0){
      counter++;
      }
    }
   mc.println("Found "+counter+" odd numbers.");
   int k = 0;
   counter = 0;
   while (k<=top){
     if (k%2 == 0){
       counter++;
     }
     k = k+1; //k++
    }
   mc.println("Found "+counter+" odd numbers.");

 }}
\end{lstlisting}
\end{frame}




\frame{
\frametitle{Exercise-10}
\begin{block}{Statement...}
 Write a program to calculate the square of numbers between 0 and 20 (using different loops).
\end{block}

}


\begin{frame}[fragile]
\frametitle{Exercise-10: examples/SquareFirst20Numbers.java}
\tiny
\begin{lstlisting}
public class SquareFirst20Numbers {
  public static void main(String []args){
    int top = 20;
    int i = 0;
    MyConsole mc = new MyConsole();
    do{
      mc.println(i*i);
      i = i + 1; //i++
    }while(i<=top);
  
  for(int j = 0; j<=top; j++){
    mc.println(j*j);
    }

  int k = 0;
  while (k<=top){
    mc.println(k*k);
    k = k+1; //k++
  }
 }
}
\end{lstlisting}
\end{frame}


\frame{
\frametitle{Exercise-11}
\begin{block}{Statement...}
 Write a program to calculate the cube of numbers between 0 and 20 (using different loops).
\end{block}

}


\begin{frame}[fragile]
\frametitle{Exercise-11: examples/CubeFirst20Numbers.java}
\tiny
\begin{lstlisting}
public class CubeFirst20Numbers {
  public static void main(String []args){
    int top = 20;
    int i = 0;
    MyConsole mc = new MyConsole();
    do{
      mc.println(i*i*i);
      i = i + 1; //i++
    }while(i<=top);

    for(int j = 0; j<=top; j++){
      mc.println(j*j*j);
    }
    int k = 0;
    while (k<=top){
      mc.println(k*k*k);
      k = k+1; //k++
    }
  }
}
\end{lstlisting}
\end{frame}


\frame{
\frametitle{Exercise-12}
\begin{block}{Statement...}
 ...the previous example but using the Java Math API.
\end{block}

}


\begin{frame}[fragile]
\frametitle{Exercise-12: examples/CubeFirst20NumbersMathAPI.java}
\tiny
\begin{lstlisting}
public class CubeFirst20NumbersMathAPI {
  public static void main(String []args){
    int top = 20;
    int i = 0;
    MyConsole mc = new MyConsole();
    do{
      mc.println(Math.pow(i, 3));
      i = i + 1; //i++
     }while(i<=top);
    for(int j = 0; j<=top; j++){
      mc.println(Math.pow(j, 3));
    }
    int k = 0;
    while (k<=top){
      mc.println(Math.pow(k, 3));
      k = k+1; //k++
    }
  }
}
\end{lstlisting}
\end{frame}



\frame{
\frametitle{Exercise-13}
\begin{block}{Statement...}
 Write a program to calculate the pow of any number in the form $a^b$.
\end{block}

}


\begin{frame}[fragile]
\frametitle{Exercise-13: examples/CubeFirst20NumbersMathAPI.java}
\tiny
\begin{lstlisting}
public class MyOwnPow {
  public static void main(String []args){
    MyConsole mc = new MyConsole();
    double base = 0;
    int exponent = 0;
    double value = 1;
    base = mc.readDouble();
    exponent = mc.readInt();
    for (int i = 0; i<exponent;i++){
      value = value  * base; 
    }
  mc.println("The pow of "+base+" raised to "+exponent+" is equals to "+value);
  mc.println("Using the Java API: "+Math.pow(base, exponent));
  }
}
\end{lstlisting}
\end{frame}



\begin{frame}[fragile]
\frametitle{Exercise-14}
\begin{block}{Statement...}
 Write a program to calculate if a year is a leap year.
\end{block}
\begin{lstlisting}
if year is divisible by 400 then
   is_leap_year
else if year is divisible by 100 then
   not_leap_year
else if year is divisible by 4 then
   is_leap_year
else
   not_leap_year
\end{lstlisting}

Source: \url{http://en.wikipedia.org/wiki/Leap\_year}
\end{frame}


\begin{frame}[fragile]
\frametitle{Exercise-14: examples/LeapYear.java}
\scriptsize
\begin{lstlisting}
public class LeapYear {
  public static void main(String []args){
    MyConsole mc = new MyConsole();
    int year = 0;
    year = mc.readInt();
    boolean isLeapYear = (year % 4 == 0);
    // divisible by 4 and not 100
    isLeapYear = isLeapYear && (year % 100 != 0);
    // divisible by 4 and not 100 unless divisible by 400
    isLeapYear = isLeapYear || (year % 400 == 0);
    mc.println(isLeapYear);	
  }
}
\end{lstlisting}
\end{frame}



\begin{frame}[fragile]
\frametitle{Exercise-15}
\begin{block}{Statement...}
 Write a program to know if a number is a palindrome.
\end{block}

\begin{exampleblock}{Palindrome}
  A number/string is called palindrome if number/string and its reverse is equal.
  \begin{itemize}
   \item 121 is a palindrome.
   \item 123 is not a palindrome.
  \end{itemize}

\end{exampleblock}


\end{frame}


\begin{frame}[fragile]
\frametitle{Exercise-15: examples/PalindromeNumber.java}
\scriptsize
\begin{lstlisting}
public class PalindromeNumber {
  public static void main(String []args){
    MyConsole mc = new MyConsole();
    int number;
    int palindrome, reverse;
    number = mc.readInt();
    palindrome = number;
    reverse = 0;
    while (palindrome != 0) {
      int remainder = palindrome % 10;
      reverse = reverse * 10 + remainder;
      palindrome = palindrome / 10;
    }
    mc.println("The number "+number+"is palindrome "+ 
      (number==reverse));
  }
}
\end{lstlisting}
\end{frame}


\begin{frame}[fragile]
\frametitle{Exercise-16}
\begin{block}{Statement...}
 Write a program to know if a number is a palindrome.
\end{block}

\begin{exampleblock}{Armstrong number}
 An Armstrong number of three digit is a number whose sum of cubes of its digit is equal to its number. 
  \begin{itemize}
   \item  153 = $1^3+5^3+3^3$ = $1+125+27$=153 is an Armstrong Number.
   \item  See \url{http://en.wikipedia.org/wiki/Narcissistic\_number}
  \end{itemize}

\end{exampleblock}


\end{frame}


\begin{frame}[fragile]
\frametitle{Exercise-16: examples/PalindromeNumber.java}
\scriptsize
\begin{lstlisting}
public class ArmstrongNumber {
  public static void main(String []args){
    MyConsole mc = new MyConsole();
    int number = 0;
    int result, source;
    number = mc.readInt();
    source = number;
    result = 0;
    while(number != 0){
      int remainder = number%10;
      result = result + remainder*remainder*remainder;
      number = number/10;
     }
    mc.println("The number "+number+" is Armstrong number "+
      (source==result));        
  }
}
\end{lstlisting}
\end{frame}


\begin{frame}[fragile]
\frametitle{Exercise-17}
\begin{block}{Statement...}
 Write a program to show a Wedge of Stars (given the number of stars).
\end{block}

\begin{lstlisting}
 *******
 ******
 *****
 ****
 ***
 **
 *
\end{lstlisting}



\end{frame}


\begin{frame}[fragile]
\frametitle{Exercise-17: examples/WedgeOfStars.java}
\scriptsize
\begin{lstlisting}
public class WedgeOfStars {
  public static void main(String args[]){
    MyConsole mc = new MyConsole();
    int numberOfStars = 0;
    numberOfStars = mc.readInt();
    for (int i = numberOfStars; i>0; i--){
      for(int j = 0; j<i; j++){
	mc.print("*");
      }
      mc.println("");
      }
  }
}
\end{lstlisting}
\end{frame}




\begin{frame}[fragile]
\frametitle{Exercise-18}
\begin{block}{Statement...}
 Write a program to show a Holiday Tree.
\end{block}

\begin{lstlisting}
       *
      ***
     *****
    *******
   *********
  ***********
 *************
***************
      ***
      ***
      ***
\end{lstlisting}



\end{frame}


\begin{frame}[fragile]
\frametitle{Exercise-18: examples/HolidayTree.java}
\tiny
\begin{lstlisting}
public class HolidayTree {
  public static void main(String []args) throws IOException{
    MyConsole mc = new MyConsole();
    int baseStars = 15;
    int halfBlankSpaces = 0;
    for (int i = 1; i<baseStars; i=i+2){
      halfBlankSpaces = (baseStars-i)/2;
      for(int j = 0; j<halfBlankSpaces;j++) {mc.print(" ");}
      for(int k = 0; k<i;k++) {mc.print("*");}
      for(int j = 0; j<halfBlankSpaces;j++) { mc.print(" ");}
      mc.println("");
    }
    halfBlankSpaces = (baseStars/2)-1;
    for(int i = 0; i<3;i++){
      for(int j = 0; j<halfBlankSpaces;j++){mc.print(" ");}
      for(int k = 0; k<3;k++){mc.print("*");}
      for(int j = 0; j<halfBlankSpaces;j++){mc.print(" ");}
    mc.println(" ");
    }
}
}
\end{lstlisting}
\end{frame}


\begin{frame}[fragile]
\frametitle{Exercise-19}
\begin{block}{Statement...}
  See some examples using the Java Math API
\end{block}

\end{frame}


\begin{frame}[fragile]
\frametitle{Exercise-19: examples/MathAPIExample.java}
\tiny
\begin{lstlisting}
public class MathAPIExample {
  public static void main(String []args){
    MyConsole mc = new MyConsole();
    //Max
    mc.println(Math.max(2, 3));
    //Min
    mc.println(Math.min(2, 3));
    //Abs		
    mc.println(Math.abs(-2));
    //Pow
    mc.println(Math.pow(2, 3));
    //Sqrt
    mc.println(Math.sqrt(144));
  }
}
\end{lstlisting}
\end{frame}



\begin{frame}[fragile]
\frametitle{Exercise-20}
\begin{block}{Statement...}
Write a program to know if a number is a prime number.
\end{block}

\begin{exampleblock}{Definition from Wikipedia}
 A prime number (or a prime) is a natural number greater than 1 that has no positive divisors other than 1 and itself. 
\end{exampleblock}

\end{frame}


\begin{frame}[fragile]
\frametitle{Exercise-20: examples/PrimeNumber.java}
\scriptsize
\begin{lstlisting}
public class PrimeNumber {
  public static void main (String []args){
    MyConsole mc = new MyConsole();
    int number = 0;
    boolean isPrime = true;
    number = mc.readInt();
    int i = 2;
    while(isPrime && i!=number){
      if(number%i ==  0){
      isPrime = false;
     }
    i++;
    }
    mc.println("The number "+number+" is prime "+isPrime);
  }
}
\end{lstlisting}
\end{frame}




\begin{frame}[fragile]
\frametitle{Exercise-21}
\begin{block}{Statement...}
Write a program to show the first 100 primer numbers.
\end{block}

\end{frame}


\begin{frame}[fragile]
\frametitle{Exercise-21: examples/First100PrimeNumbers.java}
\scriptsize
\begin{lstlisting}
public class First100PrimeNumbers {
  public static void main (String []args){
    MyConsole mc = new MyConsole();
    int nprimes = 100;
    int matches = 1; //1 is prime
Print 1 and 2
    int number = 2; //...we start in 2...
    boolean isPrime = true;			
    int i = 2;
    while (matches<nprimes){
      isPrime = true;
      i = 2;
      while(isPrime && i!=number){
	if(number%i ==  0){
	  isPrime = false;
	}
      i++;
     }
    if (isPrime){
      mc.println(number);
      matches++;
    }
    number++;
   }
 }
}
\end{lstlisting}
\end{frame}




\begin{frame}[fragile]
\frametitle{Exercise-22}
\begin{block}{Statement...}
Write a program to add up $n$ numbers (requesting the parameters to the user).
\end{block}

\end{frame}


\begin{frame}[fragile]
\frametitle{Exercise-22: examples/AddingUpIntegers.java}
\scriptsize
\begin{lstlisting}
public class AddingUpIntegers {
  public static void main(String []args){
    MyConsole mc = new MyConsole();
    int numberToRead = 0;
    numberToRead = mc.readInt();
    if(numberToRead>0){
      int value = 0;
      int sum = 0;
      for(int i = 0; i<numberToRead;i++){
	mc.println("Enter a number...");
	value = mc.readInt();
	sum = sum + value;
      }
      mc.println("The sum is "+sum);
      }
  }
}
\end{lstlisting}
\end{frame}




\begin{frame}[fragile]
\frametitle{Exercise-23}
\begin{block}{Statement...}
Write a program to calculate the factorial of a number.
\end{block}


\begin{equation}
n!=\prod_{k=1}^n k \!  
\end{equation}


\begin{equation}
n! = \begin{cases}
1 & \text{if } n = 0, \\
(n-1)!\times n & \text{if } n > 0.
\end{cases}
\end{equation}


\end{frame}


\begin{frame}[fragile]
\frametitle{Exercise-23: examples/Factorial.java}
\scriptsize
\begin{lstlisting}
public class Factorial {
  public static void main(String []args){
    MyConsole mc = new MyConsole();
    int op1;
    double fact;
    op1 = mc.readInt();
    //fact(n) = n * fact(n-1);
    if (op1 == 0 || op1 == 1){
      fact = 1;
     }else{
     fact = 1;
     for (int i = op1; i>1; i--){
      fact = fact * i; 
      }
    }
    mc.println("The factorial of: "+op1+" is "+fact);
  }
}
\end{lstlisting}
\end{frame}



\begin{frame}[fragile]
\frametitle{Exercise-24}
\begin{block}{Statement...}
Write a program to generate a Fibonacci sequence of $n$ numbers.
\end{block}
% 
\begin{exampleblock}{Definition from Wikipedia}
The Fibonacci sequence is the numbers in the following integer sequence:
0,1,1,2,3,5,8,13,21...
\end{exampleblock}


\begin{equation}
F_n = \begin{cases}
0 & \text{if } n = 0, \\
1 & \text{if } n = 1, \\
F_{n-1} + F_{n-2} & n > 1.\\
\end{cases}
\end{equation}



\end{frame}


\begin{frame}[fragile]
\frametitle{Exercise-24: examples/Fibonacci.java}
\scriptsize
\begin{lstlisting}
public class Fibonacci {
  public static void main(String []args){
    MyConsole mc = new MyConsole();
    int numbersToGenerate;
    int fn1, fn2, fcurrent, temp;
    numbersToGenerate = mc.readInt();
    fn2 = 0;
    fn1 = 1;
    mc.println(fn2);
    mc.println(fn1);
    for (int i = 2; i < numbersToGenerate; i++){
      fcurrent = fn1 + fn2;
      temp = fn1;
      fn1 = fcurrent;
      fn2 = temp;
      mc.println(fcurrent);
    }
  }
}
\end{lstlisting}
\end{frame}


\begin{frame}[fragile]
\frametitle{Exercise-25}
\begin{block}{Statement...}
Write a program to implement a binomial coefficient.
\end{block}
% 

\begin{equation}
C_{k} = \binom{n}{k} = \frac{n!}{k!(n-k)!} 
\end{equation}

\end{frame}



\begin{frame}[fragile]
\frametitle{What's next?}
\begin{exampleblock}{Remarks...}
\begin{itemize}
 \item Try to think and implement the examples by yourself.
 \item If you find some typo or error please report me.
 \item The same problem can have several solutions.
 \item Follow the methodology to address the problem.
 \item ...If you are hesitating...Please ASK!
\end{itemize}

\end{exampleblock}



\end{frame}

% 
\frame{
\titlepage

}


% %%%%%%%%%%%%%%%%%%%%%%%%%%%%%%%%%%%%%%%%%%%%%%%%%%%%%%%%%%%%%%%%%%%%%%

\end{document}
