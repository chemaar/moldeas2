%
% main.tex
%

% notes = hide | show | only
\documentclass[xcolor=dvipsnames,dvip,notes=show,table]{beamer}

% Para crear una versión 'handout' (impresa)
%\documentclass[xcolor=pst,dvips,handout,notes=show]{beamer}

\input{headers}



%%%%%%%%%%%%%%%%%%%%%%%%%%%%%%%%%%%%%%%%%%%%%%%%%%%%%%%%%%%%%%%%%%%%%%

\title[FORTRAN]{Basic Examples of FORTRAN Programming}
\author[Jose María Álvarez Rodríguez]{\textbf{Programming Course} \\ \vspace{0.3cm} Jose María Álvarez Rodríguez}
\subtitle{}
\institute{Department of Computer Science \\ Carlos III University of Madrid}


\date{Course 2013/2014}

\begin{document}

\frame{
\titlepage

}
% 
% \frame{
% \tableofcontents
% 
% }
% 
% 
% %%%%%%%%%%%%%%%%%%%%%%%%%%%%%%%%%%%%%%%%%%%%%%%%%%%%%%%%%%%%%%%%%%%%%%
\section{Basic Examples}
% 
\frame{


}
\begin{frame}[fragile]
\frametitle{Hello World!}

\begin{lstlisting}
program HelloWorld
  write (*,*) 'Hello, world!' 
end
\end{lstlisting}
\end{frame}


\begin{frame}[fragile]
\frametitle{Hello World!}

\begin{lstlisting}
program HelloWorld
  write (*,*) 'Hello, world!' 
end
\end{lstlisting}
\end{frame}

%http://www.cs.mtu.edu/~shene/COURSES/cs201/NOTES/chap01/struct.html
%http://nf.nci.org.au/training/FortranBasic/

\begin{frame}[fragile]
\frametitle{Average}

\scriptsize
\begin{lstlisting}
program average
real x, y, z
print *, "What are the two numbers you want to average?"
read *, x, y
call avg(x,y,z)
print *, "The average is", z
end
 
subroutine avg(a,b,c)
real a, b, c
c = (a + b)/2.
end
\end{lstlisting}
\end{frame}



\begin{frame}[fragile]
\frametitle{Factorial}

\scriptsize
\begin{lstlisting}
PROGRAM  FACT 
 INTEGER N 
 REAL FACT, FACTORIAL 
 READ (*,*) N 
 FACT = FACT_FUNCT (N) 
 PRINT *, FACT 
END

FUNCTION FACT_FUNCT (N) 
  INTEGER I, N 
  REAL FACTORIAL,  PROD 
  PROD = 1.0 
  DO I = 2, N 
     PROD = PROD * I 
  END DO 
END
\end{lstlisting}
\end{frame}



\begin{frame}[fragile]
\frametitle{Factorial}

\scriptsize
\begin{lstlisting}
PROGRAM Fibonacci
IMPLICIT NONE
INTEGER:: F1 = 1, F2 = 0, i = 0, n = 0
WRITE(*,*) "Enter how many numbers of the Fibonacci sequence you want to see."
READ(*,*) n
DO ! Calculate and write numbers in sequence
    IF (i >= n) EXIT
    F1 = F2 + F1
    WRITE(*,*) F1
    i = i + 1
    IF (i >= n) EXIT
    F2 = F2 + F1
    WRITE(*,*) F2
    i = i + 1
END DO
END PROGRAM Fibonacci
\end{lstlisting}
\end{frame}





% %%%%%%%%%%%%%%%%%%%%%%%%%%%%%%%%%%%%%%%%%%%%%%%%%%%%%%%%%%%%%%%%%%%%%%

\end{document}
