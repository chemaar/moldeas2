%
% main.tex
%

% notes = hide | show | only
\documentclass[xcolor=dvipsnames,dvip,notes=show,table]{beamer}

% Para crear una versión 'handout' (impresa)
%\documentclass[xcolor=pst,dvips,handout,notes=show]{beamer}

%
% cabeceras.tex
%

%\usepackage[T1]{fontenc}

\definecolor{ZurichBlue}{rgb}{.255,.41,.884}

\beamertemplateshadingbackground{white!10}{white!10}

\usepackage{beamerthemeWarsaw}
\usepackage{longtable}

%\usecolortheme[named=OliveGreen]{structure} 
\setbeamertemplate{items}[ball] 
\setbeamertemplate{blocks}[rounded][shadow=true] 
\setbeamertemplate{footline}[page number]
\addtocounter{framenumber}{-1}
%Handout
%\usepackage{handoutWithNotes}
%\usepackage{tikz,times}
%\pgfpagesuselayout{2 on 1 with notes}[a4paper,border shrink=5mm]

\usepackage{beamerthemeshadow}
 \useoutertheme[hooks]{tree}
 
% \setbeamertemplate{headline}[default] % The default is just an empty headline.
% \setbeamertemplate{headline}[infolines theme]
% \setbeamertemplate{headline}[miniframes theme]
% \setbeamertemplate{headline}[sidebar theme]
% \setbeamertemplate{headline}[smoothtree theme]
% \setbeamertemplate{headline}[smoothbars theme]
% \setbeamertemplate{headline}[tree]
\beamertemplatetransparentcovereddynamic

% spanish
\usepackage[spanish]{babel}
\usepackage[utf8]{inputenc}

% diagramas
%\usepackage{pst-eps,epstopdf}
\usepackage{pst-node}
%\usepackage{pst-all}
\usepackage{pst-blur}
%\usepackage{pst-tree}

% incrustaciones de código fuente
\usepackage{listings}

% matemáticas y símbolos
\usepackage{amsmath}
\usepackage{amssymb}
\usepackage[right]{eurosym}
\usepackage{ulem}

% colores
\usepackage{colortbl}

%\usepackage{algorithm2e}
%\usepackage{algorithm}
%\usepackage{algorithmic}

\lstset{language=[90]Fortran,
  basicstyle=\ttfamily,
  keywordstyle=\color{darkred},
  commentstyle=\color{green},
  frame=trBL,
  stringstyle=\color{violet},
  frameround=tttt,
  backgroundcolor=\color{lightyellow},
  morecomment=[l]{!\ }% Comment only with space after !
}


% 
% \lstset{%
%   language=Fortran,
% 	basicstyle=\footnotesize\sffamily,
% 	keywordstyle=\color{darkred}
%  	stringstyle=\color{violet}
%  	commentstyle=\color{blue}
%  	showspaces=false,
%  	showtabs=false,
%  	showstringspaces=false,
%  	frame=trBL,
%         frameround=tttt,
%         backgroundcolor=\color{lightyellow},
%  	extendedchars=true,
%  	numbers=none,
%         aboveskip=0.5cm,
%         belowskip=0.5cm,
%         xleftmargin=1cm,
%         xrightmargin=1cm,
% 	breaklines=true
% }
\definecolor{darkred}{rgb}{0.5, 0, 0}
\definecolor{violet}{rgb}{1, 0, 1}
\definecolor{lightyellow}{rgb}{1,1,0.8}


\usepackage{latexsym}
\usepackage{amsmath}
\usepackage{amssymb}
\usepackage{amsthm}

\usepackage{xspace}



\hyphenation{real}

\newrgbcolor{ColorEncabezadoTabla}{0.7 0.7 0.9}
\newrgbcolor{ColorFila1}{0.8 0.8 0.7}
\newrgbcolor{ColorFila2}{0.8 0.7 0.8}
\newrgbcolor{ColorTotal}{0.7 0.9 0.7}


% \usepackage{tikz,times}
% \usetikzlibrary{mindmap,backgrounds}



%%%%%%%%%%%%%%%%%%%%%%%%%%%%%%%%%%%%%%%%%%%%%%%%%%%%%%%%%%%%%%%%%%%%%%

\title[FORTRAN]{Basic Examples of FORTRAN Programming}
\author[Jose María Álvarez Rodríguez]{\textbf{Programming Course} \\ \vspace{0.3cm} Jose María Álvarez Rodríguez}
\subtitle{}
\institute{Department of Computer Science \\ Carlos III University of Madrid}


\date{Course 2013/2014}

\begin{document}

\frame{
\titlepage

}
% 
% \frame{
% \tableofcontents
% 
% }
% 
% 
% %%%%%%%%%%%%%%%%%%%%%%%%%%%%%%%%%%%%%%%%%%%%%%%%%%%%%%%%%%%%%%%%%%%%%%
\section{Basic Examples}
% 
\frame{


}
\begin{frame}[fragile]
\frametitle{Hello World!}

\begin{lstlisting}
program HelloWorld
  write (*,*) 'Hello, world!' 
end
\end{lstlisting}
\end{frame}


\begin{frame}[fragile]
\frametitle{Hello World!}

\begin{lstlisting}
program HelloWorld
  write (*,*) 'Hello, world!' 
end
\end{lstlisting}
\end{frame}

%http://www.cs.mtu.edu/~shene/COURSES/cs201/NOTES/chap01/struct.html
%http://nf.nci.org.au/training/FortranBasic/

\begin{frame}[fragile]
\frametitle{Average}

\scriptsize
\begin{lstlisting}
program average
real x, y, z
print *, "What are the two numbers you want to average?"
read *, x, y
call avg(x,y,z)
print *, "The average is", z
end
 
subroutine avg(a,b,c)
real a, b, c
c = (a + b)/2.
end
\end{lstlisting}
\end{frame}



\begin{frame}[fragile]
\frametitle{Factorial}

\scriptsize
\begin{lstlisting}
PROGRAM  FACT 
 INTEGER N 
 REAL FACT, FACTORIAL 
 READ (*,*) N 
 FACT = FACT_FUNCT (N) 
 PRINT *, FACT 
END

FUNCTION FACT_FUNCT (N) 
  INTEGER I, N 
  REAL FACTORIAL,  PROD 
  PROD = 1.0 
  DO I = 2, N 
     PROD = PROD * I 
  END DO 
END
\end{lstlisting}
\end{frame}



\begin{frame}[fragile]
\frametitle{Factorial}

\scriptsize
\begin{lstlisting}
PROGRAM Fibonacci
IMPLICIT NONE
INTEGER:: F1 = 1, F2 = 0, i = 0, n = 0
WRITE(*,*) "Enter how many numbers of the Fibonacci sequence you want to see."
READ(*,*) n
DO ! Calculate and write numbers in sequence
    IF (i >= n) EXIT
    F1 = F2 + F1
    WRITE(*,*) F1
    i = i + 1
    IF (i >= n) EXIT
    F2 = F2 + F1
    WRITE(*,*) F2
    i = i + 1
END DO
END PROGRAM Fibonacci
\end{lstlisting}
\end{frame}





% %%%%%%%%%%%%%%%%%%%%%%%%%%%%%%%%%%%%%%%%%%%%%%%%%%%%%%%%%%%%%%%%%%%%%%

\end{document}
