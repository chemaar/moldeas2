%
% main.tex
%

% notes = hide | show | only
\documentclass[xcolor=dvipsnames,dvip,notes=show,table]{beamer}

% Para crear una versión 'handout' (impresa)
%\documentclass[xcolor=pst,dvips,handout,notes=show]{beamer}

%
% cabeceras.tex
%

%\usepackage[T1]{fontenc}

\definecolor{ZurichBlue}{rgb}{.255,.41,.884}

\beamertemplateshadingbackground{white!10}{white!10}

\usepackage{beamerthemeWarsaw}
\usepackage{longtable}

%\usecolortheme[named=OliveGreen]{structure} 
\setbeamertemplate{items}[ball] 
\setbeamertemplate{blocks}[rounded][shadow=true] 
\setbeamertemplate{footline}[page number]
\addtocounter{framenumber}{-1}
%Handout
%\usepackage{handoutWithNotes}
%\usepackage{tikz,times}
%\pgfpagesuselayout{2 on 1 with notes}[a4paper,border shrink=5mm]

\usepackage{beamerthemeshadow}
 \useoutertheme[hooks]{tree}
 
% \setbeamertemplate{headline}[default] % The default is just an empty headline.
% \setbeamertemplate{headline}[infolines theme]
% \setbeamertemplate{headline}[miniframes theme]
% \setbeamertemplate{headline}[sidebar theme]
% \setbeamertemplate{headline}[smoothtree theme]
% \setbeamertemplate{headline}[smoothbars theme]
% \setbeamertemplate{headline}[tree]
\beamertemplatetransparentcovereddynamic

% spanish
\usepackage[spanish]{babel}
\usepackage[utf8]{inputenc}

% diagramas
%\usepackage{pst-eps,epstopdf}
\usepackage{pst-node}
%\usepackage{pst-all}
\usepackage{pst-blur}
%\usepackage{pst-tree}

% incrustaciones de código fuente
\usepackage{listings}

% matemáticas y símbolos
\usepackage{amsmath}
\usepackage{amssymb}
\usepackage[right]{eurosym}
\usepackage{ulem}

% colores
\usepackage{colortbl}

%\usepackage{algorithm2e}
%\usepackage{algorithm}
%\usepackage{algorithmic}

\lstset{language=[90]Fortran,
  basicstyle=\ttfamily,
  keywordstyle=\color{darkred},
  commentstyle=\color{green},
  frame=trBL,
  stringstyle=\color{violet},
  frameround=tttt,
  backgroundcolor=\color{lightyellow},
  morecomment=[l]{!\ }% Comment only with space after !
}


% 
% \lstset{%
%   language=Fortran,
% 	basicstyle=\footnotesize\sffamily,
% 	keywordstyle=\color{darkred}
%  	stringstyle=\color{violet}
%  	commentstyle=\color{blue}
%  	showspaces=false,
%  	showtabs=false,
%  	showstringspaces=false,
%  	frame=trBL,
%         frameround=tttt,
%         backgroundcolor=\color{lightyellow},
%  	extendedchars=true,
%  	numbers=none,
%         aboveskip=0.5cm,
%         belowskip=0.5cm,
%         xleftmargin=1cm,
%         xrightmargin=1cm,
% 	breaklines=true
% }
\definecolor{darkred}{rgb}{0.5, 0, 0}
\definecolor{violet}{rgb}{1, 0, 1}
\definecolor{lightyellow}{rgb}{1,1,0.8}


\usepackage{latexsym}
\usepackage{amsmath}
\usepackage{amssymb}
\usepackage{amsthm}

\usepackage{xspace}



\hyphenation{real}

\newrgbcolor{ColorEncabezadoTabla}{0.7 0.7 0.9}
\newrgbcolor{ColorFila1}{0.8 0.8 0.7}
\newrgbcolor{ColorFila2}{0.8 0.7 0.8}
\newrgbcolor{ColorTotal}{0.7 0.9 0.7}


% \usepackage{tikz,times}
% \usetikzlibrary{mindmap,backgrounds}



%%%%%%%%%%%%%%%%%%%%%%%%%%%%%%%%%%%%%%%%%%%%%%%%%%%%%%%%%%%%%%%%%%%%%%

\title[FORTRAN]{Basic Examples of \\ the FORTRAN Programming Language}
\author[Jose María Álvarez Rodríguez]{\textbf{Programming Course} \\ \vspace{0.3cm} Jose María Álvarez Rodríguez}
\subtitle{}
\institute{Department of Computer Science \\ Carlos III University of Madrid}


\date{Course 2013/2014}

\begin{document}

\frame{
\titlepage

}
% 
\frame{
\tableofcontents

}
% 
% 
% %%%%%%%%%%%%%%%%%%%%%%%%%%%%%%%%%%%%%%%%%%%%%%%%%%%%%%%%%%%%%%%%%%%%%%
\section{Syntax}
% % 
\begin{frame}[fragile]
\frametitle{Hello World!}

\begin{lstlisting}
program HelloWorld
  write (*,*) 'Hello, FORTRAN world!' 
end program HelloWorld
\end{lstlisting}
\end{frame}

\frame{

\frametitle{Methodology}
\begin{exampleblock}{Steps...}
\begin{enumerate}
 \item Analyze and understand the problem. Is there any formal model?
 \item Design a solution.
 \item Implement the solution.
 \item Test: verify the results. Trace and Debug the program.
 \item Document the solution.
 \item Refine, refactor the existing code (Improvement).
\end{enumerate}
\end{exampleblock}

}

% 
\begin{frame}[fragile]
\frametitle{Program Structure}

\begin{lstlisting}
PROGRAM    program-name
   IMPLICIT   NONE 
   [specification part]
   [execution part]
   [subprogram part]
END PROGRAM program-name
\end{lstlisting}

\begin{alertblock}{IMPLICIT NONE}
\begin{itemize}
 \item In FORTRAN 77, this declaration enables the possibility of using variables that have not been previously declared.
 \item  It is a kind of dynamic declaration, variable on-the-fly.
 \item It is a source of problems. In FORTRAN 90, it is mandatory to declare all variables.
\end{itemize}
\end{alertblock}

\end{frame}

% 
\frame{
  \frametitle{Some Notes...} 
  
\begin{block}{Structure...}
\begin{itemize}
 \item Contents in [ ] are optional.
 \item Keyword IMPLICIT NONE must present.
\item A program starts with the keyword PROGRAM,
\begin{itemize}
 \item ...a program name,
 \item ...the IMPLICIT NONE statement,
 \item ...(my own) some specification statements,
 \item ...the execution part,
 \item ...a set of internal subprograms,
 \item ...keywords END PROGRAM and the program name.
\end{itemize}

\item Comments can be added for improving readability.
\end{itemize}
\end{block}
}

\frame{

\frametitle{Remarks}
\begin{alertblock}{...}
\begin{enumerate}
 \item All examples are available in the folder ``examples''.
 \item We offer a potential solution more approaches could be applied to solve the same problem.
 \item All examples have been tested with the compiler ``gfortran'' (see version below).
\end{enumerate} 
\end{alertblock}

\begin{block}{Fortran Version}
 \begin{itemize}
  \item OS: Linux 3.8.0-31-generic \#46-Ubuntu SMP Tue Sep 10 20:03:44 UTC 2013 x86\_64 x86\_64 x86\_64 GNU/Linux
  \item Install command: sudo apt-get install gfortran
  \item gcc 4.7.3 (Ubuntu/Linaro 4.7.3-1ubuntu1) 
 \end{itemize}
\end{block}


}


% % %http://www.cs.mtu.edu/~shene/COURSES/cs201/NOTES/chap01/struct.html
% % %http://nf.nci.org.au/training/FortranBasic/
% % 
% % 
% \begin{frame}[fragile]
% \frametitle{Comments}
% \scriptsize
% \begin{block}{Types...}
% \begin{enumerate}
%  \item Characters that follow an exclamation mark, !, except in a character string, are comments (ignored by the compiler).
% 
%  \begin{lstlisting}
% salary  = salary + 10   ! add 10 to the salary
%  \end{lstlisting}
% 
%  \item A line comment...
%   \begin{lstlisting}
% ! A comment in the middle of the program
%  \end{lstlisting}
%  
%  \item A blank line is also interpreted as a comment line...
%    \begin{lstlisting}
% 
% ! The above blank line is a comment line
%  \end{lstlisting}
%  
%  
% \end{enumerate}
% 
% \end{block}
% 
% 
% \end{frame}
% % 
% % 
% % %% 
% 
% \begin{frame}[fragile]
% \frametitle{Fortran Continuation Lines}
% 
% \begin{exampleblock}{Statements...}
% Each statement in a new line but long statements can be splitted into several lines...
% \end{exampleblock}
% 
% \tiny
% \begin{block}{Types...}
% \begin{enumerate}
%  \item A line can be ended with an \& (not part of the statement)...
%  \item ...the continuation is usually with the char in the next non-comment line
% 
%  \begin{lstlisting}
% salary = 100.5 * 12   &
%     + 21 / 100
%  \end{lstlisting}
% 
%  \item If the first non-blank character of the continuation line is \&, continuation is to the first character after the \&:
%   \begin{lstlisting}
% salary = 100.5 * 12 + averylooooong&
%      &variablename * 10
%      
% salary = 100.5 * 12 + averylooooongvariablename * 10
% \end{lstlisting}
%   
% \end{enumerate}
% 
% \end{block}
% 
% 
% \end{frame}
% % 
% 
% \begin{frame}[fragile]
%  \frametitle{Alphabet}
%  \begin{block}{Letters}
%   a-zA-Z
%  \end{block}
%  
%  \begin{exampleblock}{Numbers}
%  0-9
%  \end{exampleblock}
% % 
% % 
%  \begin{alertblock}{Special Characters}
%   %space ' " ( ) * + - / : = \_ \! \& \$ ; < > \% ? , .
%  space ' " ( ) * + - / : = \_ \! \& \$ ; < > \% ? , .
%  \end{alertblock}
%  
% \end{frame}
% 
% 
% \begin{frame}[fragile]
% \frametitle{Constants...}
% \scriptsize
%  \begin{exampleblock}{Definition}
% ...are the tokens used to denote the value of a particular type.
% \end{exampleblock}
% 
%  \begin{block}{Types}
%  \begin{enumerate}
%   \item Integer, set of digits with an optional sign. 
%   \item Real: decimal and exponential representation. 
%   \item Complex: not covered in this course. 
%   \item Logical, set of digits with an optional sign. 
%   \item Character String, set of chars enclosed between double quotes or apostrophes (single quotes).  
%  \end{enumerate}
% 
% \end{block} 
%  
%  
% \end{frame}
% 
% 
% \begin{frame}[fragile]
% \frametitle{Identifiers...}
% \scriptsize
%   \begin{exampleblock}{Definition}
%  \end{exampleblock}
%  
% \end{frame}
% 
% 
% 
% \begin{frame}[fragile]
% \frametitle{Variables...}
% \scriptsize
%   \begin{exampleblock}{Definition}
%  \end{exampleblock}
%  
% \end{frame}
% 
% 
% \begin{frame}[fragile]
% \frametitle{Declarations...}
% \scriptsize
%   \begin{exampleblock}{Definition}
%  \end{exampleblock}
%  
% \end{frame}


\section{Basic Examples}

% \begin{frame}[fragile]
% \frametitle{Hello World!}
% 
% \begin{lstlisting}
% program HelloWorld
%   write (*,*) 'Hello, world!' 
% end
% \end{lstlisting}
% \end{frame}
% % 

%1

\frame{

\frametitle{Exercise-1}
\begin{block}{Statement...}
 Write a program to read and add two numbers.
\end{block}

}



\begin{frame}[fragile]
\frametitle{Exercise-1: examples/exercise-1.f90}
\scriptsize
\begin{lstlisting}
program adding2numbers
  real x, y, z
  print *, "What are the two numbers you want to add?"
  read *, x, y  
  z = x + y
  print *, "The result is ", z
end program adding2numbers
\end{lstlisting}
\end{frame}

\frame{
\frametitle{Exercise-2}
\begin{block}{Statement...}
 Write a program to swap the values of two real variables.
\end{block}

}



\begin{frame}[fragile]
\frametitle{Exercise-2: examples/exercise-2.f90}
\scriptsize
\begin{lstlisting}
program swap2variables
 real x, y, z
 print *, "What are the two numbers you want to swap?"
 read *, x, y
 print *, "The values (before swapping) are ", x, y
 z = y
 y = x
 x = z
 print *, "The values (after swapping) are ", x, y
end program swap2variables
\end{lstlisting}
\end{frame}



\frame{
\frametitle{Exercise-3}
\begin{block}{Statement...}
 Write a program to calculate the average of two numbers.
\end{block}

}




\begin{frame}[fragile]
\frametitle{Exercise-3: examples/AverageTwoNumbers.java}
\scriptsize
\begin{lstlisting}
program average
  real x, y, z
  print *, "What are the two numbers you want to average?"
  read *, x, y
  z = (x + y)/2
  print *, "The average is ", z
end program average
\end{lstlisting}
\end{frame}


\frame{
\frametitle{Exercise-4}
\begin{block}{Statement...}
 Write a program to calculate the highest value of two float numbers.
\end{block}

}


\begin{frame}[fragile]
\frametitle{Exercise-4: examples/HighestValue.java}
\scriptsize
\begin{lstlisting}
program highestvalue
  real x, y
  print *, "What are the two numbers you want to compare?"
  read *, x, y
  if (x >y ) then 
    print *, "The highest value is ", x
  else 
    print *, "The highest value is ", y
  end if
end program highestvalue
\end{lstlisting}
\end{frame}

\frame{
\frametitle{Exercise-5}
\begin{block}{Statement...}
 Write a program to calculate the highest value of three integer numbers.
\end{block}

}



\begin{frame}[fragile]
\frametitle{Exercise-5: examples/exercise-5.f90}
\scriptsize
\begin{lstlisting}
program highestvalue3
  real x, y, z
  print *, "What are the three numbers you want to compare?"
  read *, x, y, z
  if (x>=y .AND. x>=c ) then 
    print *, "The highest value is ", x
  else if (y>=x .AND. y>=z) then
    print *, "The highest value is ", y
  else 
    print *, "The highest value is ", z
  end if
end program highestvalue3
\end{lstlisting}
\end{frame}


\frame{
\frametitle{Exercise-6}
\begin{block}{Statement...}
 Write a program to show the first 20 natural numbers (use different loops).
\end{block}

}



\begin{frame}[fragile]
\frametitle{Exercise-6: examples/exercise-6.java}
\scriptsize
\begin{lstlisting}
program asc20
  integer :: a
  
 do a = 1, 20
   print*,a
  end do

 a = 1 
 do while (a <= 20) 
   print*,a
   a = a+1
 end do

end program asc20
\end{lstlisting}
\end{frame}

\frame{
\frametitle{Exercise-7}
\begin{block}{Statement...}
 Write a program to show the first 20 natural numbers (descending and using different loops).
\end{block}

}


\begin{frame}[fragile]
\frametitle{Exercise-7: examples/exercise-7.f90}
\scriptsize
\begin{lstlisting}
program desc20
  integer :: a

  do a = 20, 0, -1
   print*,a
  end do

  a = 20
  do while (a >= 0) 
   print*,a
   a = a - 1
  end do
end program desc20
\end{lstlisting}
\end{frame}


\frame{
\frametitle{Exercise-8}
\begin{block}{Statement...}
 Write a program to add up the first 20 natural numbers (using different loops).
\end{block}

}


\begin{frame}[fragile]
\frametitle{Exercise-8: examples/exercise-8.f90}
\scriptsize
\begin{lstlisting}
program sum20
  integer :: sum,a
  sum = 0
  do a = 1, 20
   sum = sum + a    
  end do
  print*,sum

 a = 1 
 sum = 0
 do while (a <= 20) 
   sum = sum + a    
   a = a+1
 end do

 print*,sum

end program sum20
\end{lstlisting}
\end{frame}


\frame{
\frametitle{Exercise-9}
\begin{block}{Statement...}
 Write a program to print the even numbers between 0 and 20 (using different loops).
\end{block}

}



\begin{frame}[fragile]
\frametitle{Exercise-9: examples/exercise-9.f90}
\scriptsize
\begin{lstlisting}
program showevennumbers
  integer :: a
  do a = 0, 20
    if ( mod(a,2)  == 0.0 ) then 
	 print *, a 
    end if	
  end do

 a = 0
 sum = 0
 do while (a <= 20) 
   if ( mod(a,2)  == 0.0 ) then 
     print *, a 
   end if	
   a = a+1
 end do
end program showevennumbers
\end{lstlisting}
\end{frame}



% 
% %2
% \begin{frame}[fragile]
% \frametitle{Swap variables...}
% 
% \scriptsize
% \begin{lstlisting}
% program swap
%   real x, y, z
%   print *, "What are the two numbers you want to swap?"
%   read *, x, y
%   print *, "The values (before swapping) are ", x, y
%   z = y
%   y = x
%   x = z
%   print *, "The values (after swapping) are ", x, y
% end
% \end{lstlisting}
% \end{frame}
% 
% %3
% \begin{frame}[fragile]
% \frametitle{Average (I)...}
% 
% \scriptsize
% \begin{lstlisting}
% program average
%   real x, y, z
%   print *, "What are the two numbers you want to average?"
%   read *, x, y
%   z = (x + y)/2
%   print *, "The average is ", z
% end
% \end{lstlisting}
% \end{frame}
% 
% 
% %4
% \begin{frame}[fragile]
% \frametitle{Average (II)...}
% 
% \scriptsize
% \begin{lstlisting}
% program average
%   real x, y, z
%   print *, "What are the two numbers you want to average?"
%   read *, x, y
%   call avg(x,y,z)
%   print *, "The average is", z
% end
%  
% subroutine avg(a,b,c)
%   real a, b, c
%   c = (a + b)/2.
% end
% \end{lstlisting}
% \end{frame}
% 
% 
% %5
% \begin{frame}[fragile]
% \frametitle{Highest number...}
% 
% \scriptsize
% \begin{lstlisting}
% program highest
%   real x, y, z
%   print *, "What are the two numbers you want to compare?"
%   read *, x, y
%   if (x >y ) then 
%     print *, "The highest value is ", x
%   else 
%     print *, "The highest value is ", y
%   end if
% end
% \end{lstlisting}
% \end{frame}
% 
% 
% %6
% \begin{frame}[fragile]
% \frametitle{Highest number (among 3)...}
% 
% \scriptsize
% \begin{lstlisting}
% program highest3
%   real x, y, z
%   print *, "What are the three numbers you want to compare?"
%   read *, x, y, z
%   if (x>=y .AND. x>=z ) then 
%     print *, "The highest value is ", x
%   else if (y>=x .AND. y>=z) then
%     print *, "The highest value is ", y
%   else 
%     print *, "The highest value is ", z
%   end if
% end
% \end{lstlisting}
% \end{frame}
% 
% %7
% \begin{frame}[fragile]
% \frametitle{Show the first 20 numbers (ascending)...}
% 
% \scriptsize
% \begin{lstlisting}
% program asc20
%   integer :: a
%   do a = 1, 20
%    print*,a
%   end do
% end
% \end{lstlisting}
% \end{frame}
% 
% 
% %8
% \begin{frame}[fragile]
% \frametitle{Show the first 20 numbers (descending)...}
% 
% \scriptsize
% \begin{lstlisting}
% program desc20
%   integer :: a
%   a = 20
%   do while (a > 0) 
%    print*,a
%    a = a - 1
%   end do
% end
% \end{lstlisting}
% \end{frame}
% 
% 
% %9
% \begin{frame}[fragile]
% \frametitle{Sum of the first 20 numbers...}
% 
% \scriptsize
% \begin{lstlisting}
% program sum20
%   integer :: sum,a
%   sum = 0
%   do a = 1, 20
%    sum = sum + a    
%   end do
%   print*,sum
% end
% \end{lstlisting}
% \end{frame}
% 
% 
% %10
% \begin{frame}[fragile]
% \frametitle{Show even numbers between 1 and 20...}
% 
% \scriptsize
% \begin{lstlisting}
% program even20
%   integer :: a
%   do a = 1, 20
%     if ( mod(a,2) == 0.0 ) then 
%     	print *, a
%     end if	
%   end do
% end
% \end{lstlisting}
% \end{frame}
% 
% %11
% \begin{frame}[fragile]
% \frametitle{Count odd numbers between 1 and 20...}
% 
% \scriptsize
% \begin{lstlisting}
% program countodd
%   integer :: a, count
%   count = 0
%   do a = 1, 20
%     if ( mod(a,2) /= 0.0 ) then 
%     	count = count + 1 
%     end if	
%   end do
%  print *, count 
% end
% \end{lstlisting}
% \end{frame}
% 
% 
% %12
% \begin{frame}[fragile]
% \frametitle{Square of the first 20 numbers...}
% 
% \scriptsize
% \begin{lstlisting}
% program square
%   integer :: a, b
%   do a = 1, 20
%     b = a**2
%     print*,a,"to the power of two = ",b
%   end do
% end
% \end{lstlisting}
% \end{frame}
% 
% %13
% \begin{frame}[fragile]
% \frametitle{Cube of the first 8 numbers...}
% 
% \scriptsize
% \begin{lstlisting}
% program cube
%   integer :: a, b
%   do a = 1, 8
%     b = a*a*a
%     print*,a,"to the power of three = ",b
%   end do
% end
% \end{lstlisting}
% \end{frame}
% 
% %14
% \begin{frame}[fragile]
% \frametitle{Power of a number $a^b$...}
% 
% \scriptsize
% \begin{lstlisting}
% program aexponentb
%   integer :: a, b
%   real :: value
%   read *, a,b
%   value = 1
%   do i = 1, b
%     value = value * a
%   end do
%   print*,a,"to the power of",b,"= ",value
% end
% \end{lstlisting}
% \end{frame}
% 
% %15
% \begin{frame}[fragile]
% \frametitle{Year}
% \scriptsize
% \begin{lstlisting}
% program bisiesto
%   
% end
% \end{lstlisting}
% \end{frame}
% 
% 
% %16
% \begin{frame}[fragile]
% \frametitle{Palindrome Numbers}
% \scriptsize
% \begin{block}{What is a palindrome?}
%   A number/string is called palindrome if number/string and its reverse is equal.
% \end{block}
% 
% \begin{lstlisting}
% program isPalindrome
%   integer:: palindrome, num, reverse, remainder
%   read *, num
%   palindrome = num
%   reverse = 0
%   do while (palindrome /= 0)
%     remainder = mod(palindrome,10)
%     reverse = reverse * 10 + remainder
%     palindrome = palindrome / 10	
%   end do
%   if (num == reverse ) then 
%       print *, "The number ", num, " is a palindrome."
%   else 
%      print *, "The number ", num, " is not a palindrome."
%   end if
% end 
% \end{lstlisting}
% \end{frame}
% 
% 
% %17
% \begin{frame}[fragile]
% \frametitle{Holiday Tree}
% \scriptsize
% \begin{lstlisting}
% program holidaytree
%   
% end
% \end{lstlisting}
% \end{frame}
% 
% 
% %18
% \begin{frame}[fragile]
% \frametitle{Stars}
% \scriptsize
% \begin{lstlisting}
% program stars
%   
% end
% \end{lstlisting}
% \end{frame}
% 
% %19
% \begin{frame}[fragile]
% \frametitle{Factorial}
% \scriptsize
% \begin{lstlisting}
% program fact
%   real fvalue
%   integer n
%   print *, "Enter a number..."
%   read *, n
%   ! fact(n) = n * fact (n-1)
%   if (n < 0) then 
%     print *, "The number is negative..."
%   else if (n>=0 .AND. n<=1) then
%     fvalue = 1
%   else
%     fvalue = 1
%     do i = 2, n
%       fvalue = i * fvalue
%     end do
%   end if
%   print *, "The result is ", fvalue
% end
% \end{lstlisting}
% \end{frame}
% 
% 
% %20
% \begin{frame}[fragile]
% \frametitle{Factorial (I)}
% 
% \scriptsize
% \begin{lstlisting}
% program fact1
%  real, dimension(20) :: fact
%  integer :: n, nmax = 20
%  fact(1) = 1.0
%  do n = 2, nmax
%   fact(n) = fact(n-1) * n
%  end do
%  print*,fact
% end
% \end{lstlisting}
% \end{frame}
% 
% 
% %21
% 
% \begin{frame}[fragile]
% \frametitle{Factorial (III)}
% 
% \scriptsize
% \begin{lstlisting}
% program  Fact 
%  integer N 
%  real fact
%  read (*,*) n 
%  fact = ffact (N) 
%  print *, FACT 
% end
% 
% function FFACT (N) 
%   integer I, N 
%   real factorial,  prod
%   prod = 1.0 
%   do i = 2, N 
%      prod = prod * i 
%   end do 
% end
% \end{lstlisting}
% \end{frame}
% 
% 
% %22
% \begin{frame}[fragile]
% \frametitle{Print the Fibonacci sequence of a number... $fib_n=fib_{n-1}+fib_{n-2}$ // $fib_0 = 0$ and $fib_1 = 1$}
% 
% \scriptsize
% \begin{lstlisting}
% program Fibonacci
%   IMPLICIT NONE
%   integer:: F1 = 1, F2 = 0, i = 0, n = 0
%   print *, "Enter a number..."
%   read (*,*) n
%   do 
%     if (i >= n) exit
%     F1 = F2 + F1
%     print *, F1
%     i = i + 1
%     if (i >= n) exit
%       F2 = F2 + F1
%       print *, F2
%       i = i + 1
%   end do
% end
% \end{lstlisting}
% \end{frame}
% 

\frame{
\titlepage

}


% %%%%%%%%%%%%%%%%%%%%%%%%%%%%%%%%%%%%%%%%%%%%%%%%%%%%%%%%%%%%%%%%%%%%%%

\end{document}
