%
% main.tex
%

% notes = hide | show | only
\documentclass[xcolor=dvipsnames,dvip,notes=show,table]{beamer}

% Para crear una versión 'handout' (impresa)
%\documentclass[xcolor=pst,dvips,handout,notes=show]{beamer}

\input{headers}



%%%%%%%%%%%%%%%%%%%%%%%%%%%%%%%%%%%%%%%%%%%%%%%%%%%%%%%%%%%%%%%%%%%%%%

\title[FORTRAN]{Syntax and Basic Examples of FORTRAN Programming Language}
\author[Jose María Álvarez Rodríguez]{\textbf{Programming Course} \\ \vspace{0.3cm} Jose María Álvarez Rodríguez}
\subtitle{}
\institute{Department of Computer Science \\ Carlos III University of Madrid}


\date{Course 2013/2014}

\begin{document}

\frame{
\titlepage

}
% 
\frame{
\tableofcontents

}
% 
% 
% %%%%%%%%%%%%%%%%%%%%%%%%%%%%%%%%%%%%%%%%%%%%%%%%%%%%%%%%%%%%%%%%%%%%%%
\section{Syntax}
% % 
\begin{frame}[fragile]
\frametitle{Hello World!}

\begin{lstlisting}
program HelloWorld
  write (*,*) 'Hello, FORTRAN world!' 
end
\end{lstlisting}
\end{frame}

% 
\begin{frame}[fragile]
\frametitle{Program Structure}

\begin{lstlisting}
PROGRAM    program-name
   IMPLICIT   NONE
   [specification part]
   [execution part]
   [subprogram part]
END PROGRAM program-name
\end{lstlisting}
\end{frame}

% 
% \frame{
%   \frametitle{Comments} 
%   
% \begin{alertblock}{Structure...}
% \begin{itemize}
%  \item Contents in [ ] are optional.
%  \item Keyword IMPLICIT NONE must present.
% \item A program starts with the keyword PROGRAM,
% \begin{itemize}
%  \item ...a program name,
%  \item ...the IMPLICIT NONE statement,
%  \item ...(my own) some specification statements,
%  \item ...the execution part,
%  \item ...a set of internal subprograms,
%  \item ...keywords END PROGRAM and the program name.
% \end{itemize}
% 
% \item Comments can be added for improving readability.
% \end{itemize}
% \end{alertblock}
% }
% 
% %http://www.cs.mtu.edu/~shene/COURSES/cs201/NOTES/chap01/struct.html
% %http://nf.nci.org.au/training/FortranBasic/
% 
% 
% \begin{frame}[fragile]
% \frametitle{Comments}
% \scriptsize
% \begin{block}{Types...}
% \begin{enumerate}
%  \item All characters following an exclamation mark, !, except in a character string, are commentary, and are ignored by the compiler.
% 
%  \begin{lstlisting}
% Year = Year + 1   ! add 1 to Year
%  \end{lstlisting}
% 
%  \item An entire line may be a comment
%   \begin{lstlisting}
% ! This is a comment line in the middle of a program
%  \end{lstlisting}
%  
%  \item A blank line is also interpreted as a comment line.
%    \begin{lstlisting}
% 
% ! The above blank line is a comment line
%  \end{lstlisting}
%  
%  
% \end{enumerate}
% 
% \end{block}
% 
% 
% \end{frame}
% 
% 
% %% 
% 
% \begin{frame}[fragile]
% \frametitle{Fortran Continuation Lines}
% 
% \begin{exampleblock}{Statements...}
%  In Fortran, a statement must start on a new line. If a statement is too long to fit on a line, it can be continued with the following methods:
% \end{exampleblock}
% 
% \tiny
% \begin{block}{Types...}
% \begin{enumerate}
%  \item If a line is ended with an \& (it is not part of the statement), it will be continued on the next line.
%  \item Continuation is normally to the first character of the next non-comment line.
% 
%  \begin{lstlisting}
% A = 174.5 * Year   &
%     + Count / 100
%  \end{lstlisting}
% 
%  \item If the first non-blank character of the continuation line is \&, continuation is to the first character after the \&:
%   \begin{lstlisting}
% A = 174.5 + ThisIsALong&
%      &VariableName * 123.45
%      
% A = 174.5 + ThisIsALongVariableName * 123.45
%  \end{lstlisting}
%   
% \end{enumerate}
% 
% \end{block}
% 
% 
% \end{frame}
% 
% 
% \begin{frame}[fragile]
% \frametitle{Alphabet}
% \begin{block}{Letters}
%  a-zA-Z
% \end{block}
% 
% \begin{exampleblock}{Numbers}
% 0-9
% \end{exampleblock}
% 
% 
% \begin{alertblock}{Special Characters}
% space ' " ( ) * + - / : = \_ ! \& \$ ; < > \% ? , .
% %space \' \" ( ) * + - 
% \end{alertblock}
% 
% 
% \end{frame}


\section{Basic Examples}

\begin{frame}[fragile]
\frametitle{Hello World!}

\begin{lstlisting}
program HelloWorld
  write (*,*) 'Hello, world!' 
end
\end{lstlisting}
\end{frame}
% 


\begin{frame}[fragile]
\frametitle{Adding two numbers...}
\scriptsize
\begin{lstlisting}
program adding
  real x, y, z
  print *, "What are the two numbers you want to add?"
  read *, x, y  
  z = x + y
  print *, "The result is ", z
end
\end{lstlisting}
\end{frame}



\begin{frame}[fragile]
\frametitle{Swap variables...}

\scriptsize
\begin{lstlisting}
program swap
  real x, y, z
  print *, "What are the two numbers you want to swap?"
  read *, x, y
  print *, "The values (before swapping) are ", x, y
  z = y
  y = x
  x = z
  print *, "The values (after swapping) are ", x, y
end
\end{lstlisting}
\end{frame}


\begin{frame}[fragile]
\frametitle{Average (I)...}

\scriptsize
\begin{lstlisting}
program average
  real x, y, z
  print *, "What are the two numbers you want to average?"
  read *, x, y
  z = (x + y)/2
  print *, "The average is ", z
end
\end{lstlisting}
\end{frame}


\begin{frame}[fragile]
\frametitle{Average (II)...}

\scriptsize
\begin{lstlisting}
program average
  real x, y, z
  print *, "What are the two numbers you want to average?"
  read *, x, y
  call avg(x,y,z)
  print *, "The average is", z
end
 
subroutine avg(a,b,c)
  real a, b, c
  c = (a + b)/2.
end
\end{lstlisting}
\end{frame}


\begin{frame}[fragile]
\frametitle{Highest number...}

\scriptsize
\begin{lstlisting}
program highest
  real x, y, z
  print *, "What are the two numbers you want to compare?"
  read *, x, y
  if (x >y ) then 
    print *, "The highest value is ", x
  else 
    print *, "The highest value is ", y
  end if
end
\end{lstlisting}
\end{frame}


\begin{frame}[fragile]
\frametitle{Highest number (among 3)...}

\scriptsize
\begin{lstlisting}
program highest3
  real x, y, z
  print *, "What are the three numbers you want to compare?"
  read *, x, y, z
  if (x>=y .AND. x>=z ) then 
    print *, "The highest value is ", x
  else if (y>=x .AND. y>=z) then
    print *, "The highest value is ", y
  else 
    print *, "The highest value is ", z
  end if
end
\end{lstlisting}
\end{frame}

\begin{frame}[fragile]
\frametitle{Show the first 20 numbers (ascending)...}

\scriptsize
\begin{lstlisting}
program asc20
  integer :: a
  do a = 1, 20
   print*,a
  end do
end
\end{lstlisting}
\end{frame}


\begin{frame}[fragile]
\frametitle{Show the first 20 numbers (descending)...}

\scriptsize
\begin{lstlisting}
program desc20
  integer :: a
  a = 20
  do while (a > 0) 
   print*,a
   a = a - 1
  end do
end
\end{lstlisting}
\end{frame}


\begin{frame}[fragile]
\frametitle{Sum of the first 20 numbers...}

\scriptsize
\begin{lstlisting}
program sum20
  integer :: sum,a
  sum = 0
  do a = 1, 20
   sum = sum + a    
  end do
  print*,sum
end
\end{lstlisting}
\end{frame}


\begin{frame}[fragile]
\frametitle{Show even numbers between 1 and 20...}

\scriptsize
\begin{lstlisting}
program even20
  integer :: a
  do a = 1, 20
    if ( mod(a,2) == 0.0 ) then 
    	print *, a
    end if	
  end do
end
\end{lstlisting}
\end{frame}

\begin{frame}[fragile]
\frametitle{Count odd numbers between 1 and 20...}

\scriptsize
\begin{lstlisting}
program countodd
  integer :: a, count
  count = 0
  do a = 1, 20
    if ( mod(a,2) /= 0.0 ) then 
    	count = count + 1 
    end if	
  end do
 print *, count 
end
\end{lstlisting}
\end{frame}


\begin{frame}[fragile]
\frametitle{Square of the first 20 numbers...}

\scriptsize
\begin{lstlisting}
program square
  integer :: a, b
  do a = 1, 20
    b = a**2
    print*,a,"to the power of two = ",b
  end do
end
\end{lstlisting}
\end{frame}

\begin{frame}[fragile]
\frametitle{Cube of the first 8 numbers...}

\scriptsize
\begin{lstlisting}
program cube
  integer :: a, b
  do a = 1, 8
    b = a*a*a
    print*,a,"to the power of three = ",b
  end do
end
\end{lstlisting}
\end{frame}

\begin{frame}[fragile]
\frametitle{Power of a number $a^b$...}

\scriptsize
\begin{lstlisting}
program aexponentb
  integer :: a, b
  real :: value
  read *, a,b
  value = 1
  do i = 1, b
    value = value * a
  end do
  print*,a,"to the power of",b,"= ",value
end
\end{lstlisting}
\end{frame}



\begin{frame}[fragile]
\frametitle{Factorial}
\scriptsize
\begin{lstlisting}
program fact
  real fvalue
  integer n
  print *, "Enter a number..."
  read *, n
  ! fact(n) = n * fact (n-1)
  if (n < 0) then 
    print *, "The number is negative..."
  else if (n>=0 .AND. n<=1) then
    fvalue = 1
  else
    fvalue = 1
    do i = 2, n
      fvalue = i * fvalue
    end do
  end if
  print *, "The result is ", fvalue
end
\end{lstlisting}
\end{frame}



\begin{frame}[fragile]
\frametitle{Factorial (I)}

\scriptsize
\begin{lstlisting}
program fact1
 real, dimension(20) :: fact
 integer :: n, nmax = 20
 fact(1) = 1.0
 do n = 2, nmax
  fact(n) = fact(n-1) * n
 end do
 print*,fact
end
\end{lstlisting}
\end{frame}




\begin{frame}[fragile]
\frametitle{Factorial (III)}

\scriptsize
\begin{lstlisting}
program  Fact 
 integer N 
 real fact
 read (*,*) n 
 fact = ffact (N) 
 print *, FACT 
end

function FFACT (N) 
  integer I, N 
  real factorial,  prod
  prod = 1.0 
  do i = 2, N 
     prod = prod * i 
  end do 
end
\end{lstlisting}
\end{frame}



\begin{frame}[fragile]
\frametitle{Print the Fibonacci sequence of a number... $fib_n=fib_{n-1}+fib_{n-2}$ // $fib_0 = 0$ and $fib_1 = 1$}

\scriptsize
\begin{lstlisting}
program Fibonacci
  IMPLICIT NONE
  integer:: F1 = 1, F2 = 0, i = 0, n = 0
  print *, "Enter a number..."
  read (*,*) n
  do 
    if (i >= n) exit
    F1 = F2 + F1
    print *, F1
    i = i + 1
    if (i >= n) exit
      F2 = F2 + F1
      print *, F2
      i = i + 1
  end do
end
\end{lstlisting}
\end{frame}


\frame{
\titlepage

}


% %%%%%%%%%%%%%%%%%%%%%%%%%%%%%%%%%%%%%%%%%%%%%%%%%%%%%%%%%%%%%%%%%%%%%%

\end{document}
