% %
% main.tex
%

% notes = hide | show | only
\documentclass[xcolor=dvipsnames,dvip,notes=show,table]{beamer}

% Para crear una versión 'handout' (impresa)
%\documentclass[xcolor=pst,dvips,handout,notes=show]{beamer}

\input{headers}



%%%%%%%%%%%%%%%%%%%%%%%%%%%%%%%%%%%%%%%%%%%%%%%%%%%%%%%%%%%%%%%%%%%%%%

\title[FORTRAN]{Basic Examples of \\ the FORTRAN Programming Language}
\author[Jose María Álvarez Rodríguez]{\textbf{Programming Course} \\ \vspace{0.3cm} Jose María Álvarez Rodríguez}
\subtitle{}
\institute{Department of Computer Science \\ Carlos III University of Madrid}


\date{Course 2013/2014}

\begin{document}

\frame{
\titlepage

}
% 
\frame{
\tableofcontents

}
% 
% 
% %%%%%%%%%%%%%%%%%%%%%%%%%%%%%%%%%%%%%%%%%%%%%%%%%%%%%%%%%%%%%%%%%%%%%%
\section{Syntax}
% % 
% \begin{frame}[fragile]
% \frametitle{Hello World!}
% 
% \begin{lstlisting}
% program HelloWorld
%   write (*,*) 'Hello, FORTRAN world!' 
% end program HelloWorld
% \end{lstlisting}
% \end{frame}

\frame{

\frametitle{Methodology}
\begin{exampleblock}{Steps...}
\begin{enumerate}
 \item Analyze and understand the problem. Is there any formal model?
 \item Design a solution.
 \item Implement the solution.
 \item Test: verify the results. Trace and Debug the program.
 \item Document the solution.
 \item Refine, refactor the existing code (Improvement).
\end{enumerate}
\end{exampleblock}

}

% 
\begin{frame}[fragile]
\frametitle{Program Structure}

\begin{lstlisting}
PROGRAM    program-name
   IMPLICIT   NONE 
   [specification part]
   [execution part]
   [subprogram part]
END PROGRAM program-name
\end{lstlisting}

\begin{alertblock}{IMPLICIT NONE}
\begin{itemize}
 \item In FORTRAN 77, this declaration enables the possibility of using variables that have not been previously declared.
 \item  It is a kind of dynamic declaration, variable on-the-fly.
 \item It is a source of problems. In FORTRAN 90, it is mandatory to declare all variables.
\end{itemize}
\end{alertblock}

\end{frame}

% 
\frame{
  \frametitle{Some Notes...} 
  
\begin{block}{Structure...}
\begin{itemize}
 \item Contents in [ ] are optional.
 \item Keyword IMPLICIT NONE must present.
\item A program starts with the keyword PROGRAM,
\begin{itemize}
 \item ...a program name,
 \item ...the IMPLICIT NONE statement,
 \item ...(my own) some specification statements,
 \item ...the execution part,
 \item ...a set of internal subprograms,
 \item ...keywords END PROGRAM and the program name.
\end{itemize}

\item Comments can be added for improving readability.
\end{itemize}
\end{block}
}

\frame{

\frametitle{Remarks}
\begin{alertblock}{...}
\begin{enumerate}
 \item All examples are available in the folder ``examples''.
 \item We offer a potential solution more approaches could be applied to solve the same problem.
 \item Integrity checkings such as range and domain of the variables are not included (Robustness improvement).
 \item All examples have been tested with the compiler ``gfortran'' (see version below).
\end{enumerate} 
\end{alertblock}

\begin{block}{Fortran Version}
 \begin{itemize}
  \item OS: Linux 3.8.0-31-generic \#46-Ubuntu SMP Tue Sep 10 20:03:44 UTC 2013 x86\_64 x86\_64 x86\_64 GNU/Linux
  \item Install command: sudo apt-get install gfortran
  \item gcc 4.7.3 (Ubuntu/Linaro 4.7.3-1ubuntu1) 
 \end{itemize}
\end{block}


}


% % %http://www.cs.mtu.edu/~shene/COURSES/cs201/NOTES/chap01/struct.html
% % %http://nf.nci.org.au/training/FortranBasic/
% % 
% % 
% \begin{frame}[fragile]
% \frametitle{Comments}
% \scriptsize
% \begin{block}{Types...}
% \begin{enumerate}
%  \item Characters that follow an exclamation mark, !, except in a character string, are comments (ignored by the compiler).
% 
%  \begin{lstlisting}
% salary  = salary + 10   ! add 10 to the salary
%  \end{lstlisting}
% 
%  \item A line comment...
%   \begin{lstlisting}
% ! A comment in the middle of the program
%  \end{lstlisting}
%  
%  \item A blank line is also interpreted as a comment line...
%    \begin{lstlisting}
% 
% ! The above blank line is a comment line
%  \end{lstlisting}
%  
%  
% \end{enumerate}
% 
% \end{block}
% 
% 
% \end{frame}
% % 
% % 
% % %% 
% 
% \begin{frame}[fragile]
% \frametitle{Fortran Continuation Lines}
% 
% \begin{exampleblock}{Statements...}
% Each statement in a new line but long statements can be splitted into several lines...
% \end{exampleblock}
% 
% \tiny
% \begin{block}{Types...}
% \begin{enumerate}
%  \item A line can be ended with an \& (not part of the statement)...
%  \item ...the continuation is usually with the char in the next non-comment line
% 
%  \begin{lstlisting}
% salary = 100.5 * 12   &
%     + 21 / 100
%  \end{lstlisting}
% 
%  \item If the first non-blank character of the continuation line is \&, continuation is to the first character after the \&:
%   \begin{lstlisting}
% salary = 100.5 * 12 + averylooooong&
%      &variablename * 10
%      
% salary = 100.5 * 12 + averylooooongvariablename * 10
% \end{lstlisting}
%   
% \end{enumerate}
% 
% \end{block}
% 
% 
% \end{frame}
% % 
% 
% \begin{frame}[fragile]
%  \frametitle{Alphabet}
%  \begin{block}{Letters}
%   a-zA-Z
%  \end{block}
%  
%  \begin{exampleblock}{Numbers}
%  0-9
%  \end{exampleblock}
% % 
% % 
%  \begin{alertblock}{Special Characters}
%   %space ' " ( ) * + - / : = \_ \! \& \$ ; < > \% ? , .
%  space ' " ( ) * + - / : = \_ \! \& \$ ; < > \% ? , .
%  \end{alertblock}
%  
% \end{frame}
% 
% 
% \begin{frame}[fragile]
% \frametitle{Constants...}
% \scriptsize
%  \begin{exampleblock}{Definition}
% ...are the tokens used to denote the value of a particular type.
% \end{exampleblock}
% 
%  \begin{block}{Types}
%  \begin{enumerate}
%   \item Integer, set of digits with an optional sign. 
%   \item Real: decimal and exponential representation. 
%   \item Complex: not covered in this course. 
%   \item Logical, set of digits with an optional sign. 
%   \item Character String, set of chars enclosed between double quotes or apostrophes (single quotes).  
%  \end{enumerate}
% 
% \end{block} 
%  
%  
% \end{frame}
% 
% 
% \begin{frame}[fragile]
% \frametitle{Identifiers...}
% \scriptsize
%   \begin{exampleblock}{Definition}
%  \end{exampleblock}
%  
% \end{frame}
% 
% 
% 
% \begin{frame}[fragile]
% \frametitle{Variables...}
% \scriptsize
%   \begin{exampleblock}{Definition}
%  \end{exampleblock}
%  
% \end{frame}
% 
% 
% \begin{frame}[fragile]
% \frametitle{Declarations...}
% \scriptsize
%   \begin{exampleblock}{Definition}
%  \end{exampleblock}
%  
% \end{frame}


\section{Basic Examples}

% \begin{frame}[fragile]
% \frametitle{Hello World!}
% 
% \begin{lstlisting}
% program HelloWorld
%   write (*,*) 'Hello, world!' 
% end
% \end{lstlisting}
% \end{frame}
% % 

%1

\frame{

\frametitle{Exercise-1}
\begin{block}{Statement...}
 Write a program to read and add two numbers.
\end{block}

}



\begin{frame}[fragile]
\frametitle{Exercise-1: examples/exercise-1.f90}
\scriptsize
\begin{lstlisting}
program adding2numbers
  real x, y, z
  print *, "What are the two numbers you want to add?"
  read *, x, y  
  z = x + y
  print *, "The result is ", z
end program adding2numbers
\end{lstlisting}
\end{frame}

\frame{
\frametitle{Exercise-2}
\begin{block}{Statement...}
 Write a program to swap the values of two real variables.
\end{block}

}



\begin{frame}[fragile]
\frametitle{Exercise-2: examples/exercise-2.f90}
\scriptsize
\begin{lstlisting}
program swap2variables
 real x, y, z
 print *, "What are the two numbers you want to swap?"
 read *, x, y
 print *, "The values (before swapping) are ", x, y
 z = y
 y = x
 x = z
 print *, "The values (after swapping) are ", x, y
end program swap2variables
\end{lstlisting}
\end{frame}



\frame{
\frametitle{Exercise-3}
\begin{block}{Statement...}
 Write a program to calculate the average of two numbers.
\end{block}

}




\begin{frame}[fragile]
\frametitle{Exercise-3: examples/exercise-3.f90}
\scriptsize
\begin{lstlisting}
program average
  real x, y, z
  print *, "What are the two numbers you want to average?"
  read *, x, y
  z = (x + y)/2
  print *, "The average is ", z
end program average
\end{lstlisting}
\end{frame}


\frame{
\frametitle{Exercise-4}
\begin{block}{Statement...}
 Write a program to calculate the highest value of two float numbers.
\end{block}

}


\begin{frame}[fragile]
\frametitle{Exercise-4: examples/exercise-4.f90}
\scriptsize
\begin{lstlisting}
program highestvalue
  real x, y
  print *, "What are the two numbers you want to compare?"
  read *, x, y
  if (x >y ) then 
    print *, "The highest value is ", x
  else 
    print *, "The highest value is ", y
  end if
end program highestvalue
\end{lstlisting}
\end{frame}

\frame{
\frametitle{Exercise-5}
\begin{block}{Statement...}
 Write a program to calculate the highest value of three integer numbers.
\end{block}

}



\begin{frame}[fragile]
\frametitle{Exercise-5: examples/exercise-5.f90}
\scriptsize
\begin{lstlisting}
program highestvalue3
  real x, y, z
  print *, "What are the three numbers you want to compare?"
  read *, x, y, z
  if (x>=y .AND. x>=c ) then 
    print *, "The highest value is ", x
  else if (y>=x .AND. y>=z) then
    print *, "The highest value is ", y
  else 
    print *, "The highest value is ", z
  end if
end program highestvalue3
\end{lstlisting}
\end{frame}


\frame{
\frametitle{Exercise-6}
\begin{block}{Statement...}
 Write a program to show the first 20 natural numbers (use different loops).
\end{block}

}



\begin{frame}[fragile]
\frametitle{Exercise-6: examples/exercise-6.f90}
\scriptsize
\begin{lstlisting}
program asc20
  integer :: a
  
 do a = 1, 20
   print*,a
  end do

 a = 1 
 do while (a <= 20) 
   print*,a
   a = a+1
 end do

end program asc20
\end{lstlisting}
\end{frame}

\frame{
\frametitle{Exercise-7}
\begin{block}{Statement...}
 Write a program to show the first 20 natural numbers (descending and using different loops).
\end{block}

}


\begin{frame}[fragile]
\frametitle{Exercise-7: examples/exercise-7.f90}
\scriptsize
\begin{lstlisting}
program desc20
  integer :: a

  do a = 20, 0, -1
   print*,a
  end do

  a = 20
  do while (a >= 0) 
   print*,a
   a = a - 1
  end do
end program desc20
\end{lstlisting}
\end{frame}


\frame{
\frametitle{Exercise-8}
\begin{block}{Statement...}
 Write a program to add up the first 20 natural numbers (using different loops).
\end{block}

}


\begin{frame}[fragile]
\frametitle{Exercise-8: examples/exercise-8.f90}
\scriptsize
\begin{lstlisting}
program sum20
  integer :: sum,a
  sum = 0
  do a = 1, 20
   sum = sum + a    
  end do
  print*,sum

 a = 1 
 sum = 0
 do while (a <= 20) 
   sum = sum + a    
   a = a+1
 end do

 print*,sum

end program sum20
\end{lstlisting}
\end{frame}


\frame{
\frametitle{Exercise-9}
\begin{block}{Statement...}
 Write a program to print the even numbers between 0 and 20 (using different loops).
\end{block}

}



\begin{frame}[fragile]
\frametitle{Exercise-9: examples/exercise-9.f90}
\scriptsize
\begin{lstlisting}
program showevennumbers
  integer :: a
  do a = 0, 20
    if ( mod(a,2)  == 0.0 ) then 
	 print *, a 
    end if	
  end do

 a = 0
 sum = 0
 do while (a <= 20) 
   if ( mod(a,2)  == 0.0 ) then 
     print *, a 
   end if	
   a = a+1
 end do
end program showevennumbers
\end{lstlisting}
\end{frame}



\frame{
\frametitle{Exercise-10}
\begin{block}{Statement...}
 Write a program to calculate the square of numbers between 1 and 20 (using different loops).
\end{block}

}


\begin{frame}[fragile]
\frametitle{Exercise-10: examples/exercise-10.f90}
\tiny
\begin{lstlisting}
program squares
  integer :: a, b
  do a = 1, 20
    b = a**2
    print*,a,"to the power of two = ",b
  end do
  a = 1
  b = 0
  do while (a <= 20) 
   b = a**2
   print*,a,"to the power of two = ",b
   a = a+1
 end do
end program squares
\end{lstlisting}
\end{frame}

\frame{
\frametitle{Exercise-11}
\begin{block}{Statement...}
 Write a program to calculate the cube of numbers between 1 and 20 (using different loops).
\end{block}

}


\begin{frame}[fragile]
\frametitle{Exercise-11: examples/exercise-11.f90}
\tiny
\begin{lstlisting}
program cubes
  integer :: a, b
  integer :: exponent
  exponent = 3
  do a = 1, 20
    b = a**exponent
    print*,a,"to the power of two = ",b
  end do
  a = 1
  b = 0
  do while (a <= 20) 
   b = a**exponent
   print*,a,"to the power of ", exponent," = ",b
   a = a+1
 end do
end program cubes
\end{lstlisting}
\end{frame}


\frame{
\frametitle{Exercise-12}
\begin{block}{Statement...}
 Write a program to calculate the pow of any number in the form $a^b$.
\end{block}

\begin{exampleblock}{Notes...}
 Implement your own loop instead of using the Fortran operator $pow=base**exponent$.
\end{exampleblock}


}


\begin{frame}[fragile]
\frametitle{Exercise-12: examples/exercise-12.f90}
\tiny
\begin{lstlisting}
program mypow
  integer :: base, exponent
  real :: value
  integer :: i
  
  print*,"Intro the base..."
  read *, base
  print*,"Intro the exponent..."
  read *, exponent
 
  i = 0
  value = 1
  do while (i < exponent) 
    value = value * base
    i = i+1
  end do 

  print*,base,"to the power of ", exponent," = ",value 	
 
end program mypow
\end{lstlisting}
\end{frame}



\begin{frame}[fragile]
\frametitle{Exercise-13}
\begin{block}{Statement...}
 Write a program to calculate if a year is a leap year.
\end{block}
\begin{lstlisting}
if year is divisible by 400 then
   is_leap_year
else if year is divisible by 100 then
   not_leap_year
else if year is divisible by 4 then
   is_leap_year
else
   not_leap_year
\end{lstlisting}

Source: \url{http://en.wikipedia.org/wiki/Leap\_year}
\end{frame}


\begin{frame}[fragile]
\frametitle{Exercise-13: examples/exercise-13.f90}
\scriptsize
\begin{lstlisting}
program leapyear
 integer :: year
 logical :: isLeapYear

 print*,"Intro the year..."
 read *, year

 isLeapYear = mod (year, 4) == 0
 ! divisible by 4 and not 100
 isLeapYear = isLeapYear .AND. .NOT.(mod(year, 100) == 0.0)
 ! divisible by 4 and not 100 unless divisible by 400
 isLeapYear = isLeapYear .OR. mod(year, 400) == 0

 print*,"The year ",year," is leap year ",isLeapYear
 
end program leapyear
\end{lstlisting}
\end{frame}




\begin{frame}[fragile]
\frametitle{Exercise-14}
\begin{block}{Statement...}
 Write a program to know if a number is a palindrome.
\end{block}

\begin{exampleblock}{Palindrome}
  A number/string is called palindrome if number/string and its reverse is equal.
  \begin{itemize}
   \item 121 is a palindrome.
   \item 123 is not a palindrome.
  \end{itemize}

\end{exampleblock}


\end{frame}


\begin{frame}[fragile]
\frametitle{Exercise-14: examples/exercise-14.f90}
\scriptsize
\begin{lstlisting}
program isPalindrome
 integer :: number
 integer :: palindrome, reverse, remainder

 print*,"Intro a number..."
 read *, number

 palindrome = number
 reverse = 0
 do while (palindrome .NE. 0)
  remainder = mod(palindrome,10)
  reverse = reverse * 10 + remainder
  palindrome = palindrome / 10
 end do

 print*,"The number ",number," is palindrome ",  (number .EQ. reverse)
end program isPalindrome
\end{lstlisting}
\end{frame}



\begin{frame}[fragile]
\frametitle{Exercise-15}
\begin{block}{Statement...}
 Write a program to know if a number is an Armstrong Number.
\end{block}

\begin{exampleblock}{Armstrong number}
 An Armstrong number of three digit is a number whose sum of cubes of its digit is equal to its number. 
  \begin{itemize}
   \item  153 = $1^3+5^3+3^3$ = $1+125+27$=153 is an Armstrong Number.
   \item  See \url{http://en.wikipedia.org/wiki/Narcissistic\_number}
  \end{itemize}

\end{exampleblock}


\end{frame}


\begin{frame}[fragile]
\frametitle{Exercise-15: examples/exercise-15.f90}
\scriptsize
\begin{lstlisting}
program isArmstrong
 integer :: number
 integer :: result, source, remainder

 print*,"Intro a number..."
 read *, number

 source = number
 result = 0
 do while (source .NE. 0)
  remainder = mod(source,10)
  result = result + remainder**3
  source = source / 10
 end do

 print*,"The number ",number," is an Armstrong number ",  
  (number .EQ. result)
end program isArmstrong
\end{lstlisting}
\end{frame}


\begin{frame}[fragile]
\frametitle{Exercise-16}
\begin{block}{Statement...}
 Write a program to show a Wedge of Stars (given the number of stars).
\end{block}


\begin{lstlisting}
 *******
 ******
 *****
 ****
 ***
 **
 *
\end{lstlisting}

\begin{block}{Hint...}
 To write something without a new line use this sentence: write(*,``(A)'',advance=``no'') ``*''
\end{block}


\end{frame}


\begin{frame}[fragile]
\frametitle{Exercise-16: examples/exercise-16.f90}
\scriptsize
\begin{lstlisting}
program wedgeOfStars
 integer :: numberofStars
 integer :: i,j

 print*,"Intro a number of stars..."
 read *, number

 i = number
 do while (i>0)
	j = 0
	do while (j<i)
	  write(*,"(A)",advance="no") "*"
	 j = j + 1
	end do
   print*,""
   i = i - 1
 end do
end program wedgeOfStars
\end{lstlisting}
\end{frame}



\begin{frame}[fragile]
\frametitle{Exercise-17}
\begin{block}{Statement...}
 Write a program to show a Holiday Tree.
\end{block}

\begin{lstlisting}
       *
      ***
     *****
    *******
   *********
  ***********
 *************
***************
      ***
      ***
      ***
\end{lstlisting}



\end{frame}


\begin{frame}[fragile]
\frametitle{Exercise-17: examples/exercise-17.f90}
\tiny

\begin{columns}[c] % the "c" option specifies center vertical alignment
\column{.5\textwidth} % column designated by a command

\begin{lstlisting}
program holidayTree
 integer :: baseStars = 15
 integer :: halfBlankSpaces = 0
 integer :: i, j, k

 i = 1
 do while (i<baseStars)
        halfBlankSpaces = (baseStars-i)/2
	j = 0  
	do while (j<halfBlankSpaces)
	  write(*,"(A)",advance="no") " "
	  j = j + 1
	end do

	k = 0  
	do while (k<i)
	  write(*,"(A)",advance="no") "*"
	  k = k + 1
	end do

	j = 0  
	do while (j<halfBlankSpaces)
	  write(*,"(A)",advance="no") " "
	  j = j + 1
	end do
   print*,""
   i = i + 2
 end do
 ...
\end{lstlisting}

\column{.5\textwidth}



\begin{lstlisting}
...
halfBlankSpaces = (baseStars/2)-1
i = 0
do while (i<3)
	j = 0  
	do while (j<halfBlankSpaces)
	  write(*,"(A)",advance="no") " "
	  j = j + 1
	end do

	k = 0  
	do while (k<3)
	  write(*,"(A)",advance="no") "*"
	  k = k + 1
	end do

	j = 0  
	do while (j<halfBlankSpaces)
	  write(*,"(A)",advance="no") " "
	  j = j + 1
	end do
   print*,""
   i = i + 1
 end do

end program holidayTree
\end{lstlisting}


\end{columns}

\end{frame}


\begin{frame}[fragile]
\frametitle{Exercise-18}
\begin{block}{Statement...}
Write a program to know if a number is a prime number.
\end{block}

\begin{exampleblock}{Definition from Wikipedia}
 A prime number (or a prime) is a natural number greater than 1 that has no positive divisors other than 1 and itself. 
\end{exampleblock}

\end{frame}


\begin{frame}[fragile]
\frametitle{Exercise-18: examples/exercise-18.f90}
\scriptsize
\begin{lstlisting}
program isPrimeNumber
  integer :: number
  integer :: i
  logical :: isPrime = .TRUE.
  
  print*,"Intro a number..."
  read *, number
  
  if (number .NE. 1) then 
	  i = 2
	  do while ( (isPrime) .AND. (i .NE. number)) 
		if (mod(number,i) == 0.0) then
			isPrime = .FALSE.
		endif
	    i = i+1
	  end do 
  endif
  print*,"The number ", number," is prime ",isPrime
 
end program isPrimeNumber
\end{lstlisting}
\end{frame}


\begin{frame}[fragile]
\frametitle{Exercise-19}
\begin{block}{Statement...}
Write a program to show the first 100 primer numbers.
\end{block}

\end{frame}


\begin{frame}[fragile]
\frametitle{Exercise-19: examples/exercise-19.f90}
\tiny
\begin{lstlisting}
program oneHundredPrimeNumbers
  integer :: number
  integer :: i
  integer :: matches
  logical :: isPrime = .TRUE.
  
  matches = 0
  number = 1  
  do while (matches < 100)
     isPrime = .TRUE.
     if (number .NE. 1) then 
	  i = 2	 
	  do while ( (isPrime) .AND. (i .NE. number)) 
		if (mod(number,i) == 0.0) then
			isPrime = .FALSE.
		endif
	    i = i+1
	  end do 
      endif
     if (isPrime) then
	print *,number
        matches = matches + 1
     end if
     number = number + 1
 end do
 
end program oneHundredPrimeNumbers
\end{lstlisting}
\end{frame}






\begin{frame}[fragile]
\frametitle{Exercise-20}
\begin{block}{Statement...}
Write a program to add up $n$ numbers (requesting the parameters to the user).
\end{block}

\end{frame}


\begin{frame}[fragile]
\frametitle{Exercise-20: examples/exercise-20.f90}
\scriptsize
\begin{lstlisting}
program addUp
 integer :: toRead = 0
 integer :: i = 0
 integer :: value, sum
 print *, "Enter the number of values to read..."
 read *, toRead 
 if (toRead > 0) then
   value = 0
   sum = 0
   i = 0
   do while (i<toRead)
	 print *, "Enter a value..."
	 read *, value 
 	 sum = sum + value
         i = i + 1
   end do
 end if
 print*, "The sum is ",sum
end program addUp
\end{lstlisting}
\end{frame}



\begin{frame}[fragile]
\frametitle{Exercise-21}
\begin{block}{Statement...}
Write a program to calculate the factorial of a number.
\end{block}


\begin{equation}
n!=\prod_{k=1}^n k \!  
\end{equation}


\begin{equation}
n! = \begin{cases}
1 & \text{if } n = 0, \\
(n-1)!\times n & \text{if } n > 0.
\end{cases}
\end{equation}


\end{frame}


\begin{frame}[fragile]
\frametitle{Exercise-21: examples/exercise-21.f90}
\scriptsize
\begin{lstlisting}
program factorial
  real fvalue
  integer n
  print *, "Which is the number you want to calculate 
    the factorial?"
  read *, n
  ! fact(n) = n * fact (n-1)
  if (n < 0) then 
    print *, "You cannot calculate the factorial of 
      negative numbers"
  else if (n>=0 .AND. n<=1) then
    fvalue = 1
  else
    fvalue = 1
    do i = 2, n
      fvalue = i * fvalue
    end do
  end if
  print *, "The result is ", fvalue
end program factorial
\end{lstlisting}
\end{frame}





\begin{frame}[fragile]
\frametitle{Exercise-22}
\begin{block}{Statement...}
Write a program to generate a Fibonacci sequence of $n$ numbers.
\end{block}
% 
\begin{exampleblock}{Definition from Wikipedia}
The Fibonacci sequence is the numbers in the following integer sequence:
0,1,1,2,3,5,8,13,21...
\end{exampleblock}


\begin{equation}
F_n = \begin{cases}
0 & \text{if } n = 0, \\
1 & \text{if } n = 1, \\
F_{n-1} + F_{n-2} & n > 1.\\
\end{cases}
\end{equation}



\end{frame}


\begin{frame}[fragile]
\frametitle{Exercise-22: examples/exercise-22.f90}
\scriptsize
\begin{lstlisting}
program Fibonacci
  IMPLICIT NONE
  integer:: F1 = 1, F2 = 0, i = 0, n = 0
  print *, "Enter a number..."
  read (*,*) n
  do 
      if (i >= n) exit
	F1 = F2 + F1
	print *, F1
	i = i + 1
      if (i >= n) exit
	F2 = F2 + F1
	print *, F2
	i = i + 1
    end do
end program Fibonacci
\end{lstlisting}
\end{frame}

\begin{frame}[fragile]
\frametitle{Exercise-23}
\begin{block}{Statement...}
Write a program to generate the multiplication table (from 1 to 10).
\end{block}
% 
\end{frame}

\begin{frame}[fragile]
\frametitle{Exercise-23: examples/exercise-23.f90}
\scriptsize
\begin{lstlisting}
program multiplicationTable
 integer :: i, j
 i = 1
 do while (i <= 10)
  j = 1
  do while (j<=10)
    write(*,"(I4)",advance="no") i*j
    j = j + 1
  end do
  print*,""
  i = i + 1
 end do
end
\end{lstlisting}
\end{frame}



\begin{frame}[fragile]
\frametitle{Exercise-24}
\begin{block}{Statement...}
Write a program to implement a binomial coefficient.
\end{block}
% 

\begin{equation}
C_{k} = \binom{n}{k} = \frac{n!}{k!(n-k)!} 
\end{equation}

\end{frame}


\begin{frame}[fragile]
\frametitle{What's next?}
\begin{exampleblock}{Remarks...}
\begin{itemize}
 \item Try to think and implement the examples by yourself.
 \item If you find some typo or error please report me.
 \item The same problem can have several solutions.
 \item Follow the methodology to address the problem.
 \item ...If you are hesitating in some of the examples...Please ASK!
\end{itemize}

\end{exampleblock}



\end{frame}


\frame{
\titlepage

}


% %%%%%%%%%%%%%%%%%%%%%%%%%%%%%%%%%%%%%%%%%%%%%%%%%%%%%%%%%%%%%%%%%%%%%%

\end{document}
