\documentclass[conference]{IEEEtran}
\usepackage{cite}
\ifCLASSINFOpdf
  % \usepackage[pdftex]{graphicx}
  % declare the path(s) where your graphic files are
  % \graphicspath{{../pdf/}{../jpeg/}}
  % and their extensions so you won't have to specify these with
  % every instance of \includegraphics
  % \DeclareGraphicsExtensions{.pdf,.jpeg,.png}
\else
  % or other class option (dvipsone, dvipdf, if not using dvips). graphicx
  % will default to the driver specified in the system graphics.cfg if no
  % driver is specified.
  % \usepackage[dvips]{graphicx}
  % declare the path(s) where your graphic files are
  % \graphicspath{{../eps/}}
  % and their extensions so you won't have to specify these with
  % every instance of \includegraphics
  % \DeclareGraphicsExtensions{.eps}
\fi

%
\usepackage[cmex10]{amsmath}

\usepackage{algorithmic}
\usepackage{array}
\usepackage{eqparbox}
% *** SUBFIGURE PACKAGES ***
%\usepackage[tight,footnotesize]{subfigure}
% subfigure.sty was written by Steven Douglas Cochran. This package makes it
% easy to put subfigures in your figures. e.g., "Figure 1a and 1b". For IEEE
% work, it is a good idea to load it with the tight package option to reduce
% the amount of white space around the subfigures. subfigure.sty is already
% installed on most LaTeX systems. The latest version and documentation can
% be obtained at:
% http://www.ctan.org/tex-archive/obsolete/macros/latex/contrib/subfigure/
% subfigure.sty has been superceeded by subfig.sty.



\usepackage{graphicx}
\usepackage[T1]{fontenc}
\usepackage[english]{babel}
\usepackage[utf8]{inputenc}

%\usepackage{natbib}
\usepackage{url}
\usepackage{rotating}

\usepackage{latexsym}
 \usepackage{amsmath}
% \usepackage{amssymb}
% \usepackage{amsthm}
%\usepackage{eurosans}

\usepackage{eurosym}

\usepackage{longtable}

\usepackage{listings}

\usepackage{color}
\usepackage{textcomp}


\definecolor{gray}{gray}{0.5}
\definecolor{green}{rgb}{0,0.5,0}

\usepackage{rotating}
\usepackage{xspace}

\newcommand{\si}{$\oplus$\xspace}
\newcommand{\no}{$\ominus$\xspace}
\newcommand{\na}{$\odot$\xspace}
\newcommand{\linkeddata}{\textit{Linked Data}\xspace}
\newcommand{\opendata}{\textit{Open Data}\xspace}
\newcommand{\lod}{\textit{Linking Open Data}\xspace}
\newcommand{\ogd}{\textit{Open Government Data}\xspace}
\newcommand{\datasets}{\textit{datasets}\xspace}
\newcommand{\dataset}{\textit{dataset}\xspace}
% correct bad hyphenation here
\hyphenation{op-tical net-works semi-conduc-tor}


\begin{document}
%
% paper title
% can use linebreaks \\ within to get better formatting as desired
\title{A multidimensional criteria-based technique for quality assessment of Linked Open Data. \\ A case study in the European e-Procurement context.}


% author names and affiliations
% use a multiple column layout for up to three different
% affiliations

\author{\IEEEauthorblockN{Jose María Alvarez-Rodríguez\IEEEauthorrefmark{1}, Juan Miguel Gómez-Berbís\IEEEauthorrefmark{1}, \\Juan Llorens-Morillo\IEEEauthorrefmark{1}}
\IEEEauthorblockA{\IEEEauthorrefmark{1}Knowledge Reuse Group\\Department of Computer Science\\
Carlos III University of Madrid,\\
Leganés, Madrid, Spain, 28911\\
Email: \{josemaria.alvarez,juanmiguel.gomez,juan.llorens\}@uc3m.es}
\and
\IEEEauthorblockN{José Emilio Labra-Gayo}
\IEEEauthorblockA{WESO Research Group\\Department of Computer Science\\
University of Oviedo\\
Oviedo, Asturias, Spain, 33007\\
Email: labra@uniovi.es}

}

\maketitle


\begin{abstract}
The present paper introduces a technique based on checking a set of 196
predefined criteria to assess linked data quality according to different
dimensions. The growing interest in promoting raw data to the Linked Data
initiative has implied the definition of life cycles that enable the right
management of all processes involved in the deployment of a data-based
infrastructure. Although the most relevant steps have been covered and defined
by these methodologies or best practices the real support to validate the
promoted data from structural, access or data integrity points of view is far
from being fully developed. That is why in this work authors compile a set of
criteria coming from existing specifications, best practices, methodologies and
tools to establish a multidimensional set of validation criteria that can be
checked through  a semi-automatic approach. Moreover the proposed technique is
applied to evaluate the quality of public procurement data coming from different
and relevant data sources in the European Union with the aim of demonstrating
both: 1) the adequacy of this technique to assess quality in the new of web of
data and 2) the status of data quality in the European e-Procurement sector.
Finally the learned lessons and restrictions of the presented method are
discussed  to conclude with some remarks and future directions.


\end{abstract}
% IEEEtran.cls defaults to using nonbold math in the Abstract.
% This preserves the distinction between vectors and scalars. However,
% if the conference you are submitting to favors bold math in the abstract,
% then you can use LaTeX's standard command \boldmath at the very start
% of the abstract to achieve this. Many IEEE journals/conferences frown on
% math in the abstract anyway.

% no keywords




% For peer review papers, you can put extra information on the cover
% page as needed:
% \ifCLASSOPTIONpeerreview
% \begin{center} \bfseries EDICS Category: 3-BBND \end{center}
% \fi
%
% For peerreview papers, this IEEEtran command inserts a page break and
% creates the second title. It will be ignored for other modes.
\IEEEpeerreviewmaketitle


\section{Introduction}
Government bodies and public institutions as a whole are the most important buyers in the European Union (EU), since public procurement spending represents around 19\% of 
EU Gross Domestic Product (GDP)~\cite{d2010}. Given this situation there is a growing interest and commitment~\cite{d2010a} to ensure that these funds are 
well managed and most of inefficiencies are eliminated. Electronic public procurement or e-Procurement~\cite{Podlogar2007} emerges as an alternative to link and 
integrate inter-organizational business processes and systems with the automation of the requisitioning, the approval purchase order 
management, accounting processes among others through the Internet-based protocol. However there is much more at stake than the mere changeover 
from paper based systems to ones using electronic communications. It should have the potential to yield important improvements in the access to information and data such as the efficiency of individual purchases, 
the overall administration of public procurement or the functioning of the markets for government contracts. 

In this context the European Commission (EC) is trying to unlock this potential, the 2004 Public Procurement Directives 2004/17/EC and 2004/18/EC 
introduced several provisions and projects~\cite{peppol,e-certis} aimed at enabling e-Procurement uptake in all Member States. In this light of 
modernizing the European public procurement sector to support growth and employment the EC also identified~\cite{siemensEval} 
both regulatory and natural barriers in the access to public procurement in the EU context, especially for SMEs. 
According to this evaluation, a real European Single Market use~\cite{d2011} has not yet been achieved causing losses, 
being cost-inefficient, missing opportunities for society and leading to a situation where more than 27 national markets co-exist instead of an EU-wide public market~\cite{monti2010}. 
In this sense, one relevant action to ease the interconnectivity and interoperability in this landscape was the creation of the Tenders Electronic Daily~\cite{eNotices,formsTed} (TED) 
by the EC. It is the on-line version of the ``Supplement to the Official Journal of the European Union'', dedicated to European public procurement notices 
($1500$ new announcements every day) but a unified pan-European information system dealing with: dispersion of the information, 
duplication of the same notice in more than one source, different publishing formats, problems with regards to a multilingual environment and 
aggregation of low-value procurement opportunities, is still missing. As a consequence some of the potential advantages 
outlined by the EC~\cite{d2010} such as increased accessibility and transparency, benefits for individual procedures, 
benefits in terms of more efficient procurement administration and potential for integration of EU procurement markets cannot be reached in a 
short-term.

Furthermore other EU actions in the e-Government context have been focused on the development of conceptual/terminological maps available in the Eurostat's metadata server (RAMON): 
in the Health field, the ``European Schedule of Occupational Diseases'' or  the ``International Classification of Diseases''; in the Education field,  
thesauri as the ``European Education Thesaurus''; in the Employment field, the ``International Standard Classification of Occupations''; 
in the European Parliament activities the ``Eurovoc Thesaurus'' and in the e-Procurement field the ``Common Procurement Vocabulary'' (CPV). 
The structure and features of these systems are very heterogeneous, although some common aspects can be found in all of them: hierarchical relationships between terms or concepts and multilingual character of the information. 
These knowledge organization systems (KOS) enable users to annotate information objects providing an agile mechanism for performing 
tasks such as exploration, searching, automatic classification or reasoning. Nevertheless depending on the country and the scope 
distinct classifications are commonly used. In the specific case of e-Procurement, the United Nations Standard Products and Services Code (UNSPSC) in Australia, 
the North American Industry Classification System (NAICS) in United States or the CPV and TARIC (Integrated Tariff of the European Communities) in the European Union 
are examples of similar efforts to unify and model procurement-related data. As a consequence a real, standardized and integrated environment 
for e-Procurement (and business) data to encourage the creation of knowledge-based services cannot be easily deployed. In this sense the 
``European Code of Best Practices Facilitating Access by SMEs to Public Procurement Contracts''~\cite{d2008} pointed out 
two groups of difficulties regarding the barriers faced by SMEs in accessing relevant information about public procurement opportunities. 
The document stated that ``the (big) number of such web portals being used by the government and by regional and local authorities makes it difficult 
for tenderers to maintain an overview''. By aggregating data on public procurement in all Member States 
and at all administrative levels in a standardized way and reusing existing efforts of modeling 
and classifying domain knowledge, some open and critical issues can be tackled to improve a situation which affects more 
than \euro $20$ million non-financial companies established in the EU.


On the other hand and following the principles of the Open Data initiative, the vice-president Neelie Kroes is leading the Open Data Strategy for Europe. 
She also outlined in December 2011 several types of actions that will help to unleash potential of data held by governments in Europe. The strategy is 
strongly focused on the commercial value of the re-use of Public Sector Information (PSI), by which the Commission expects to deliver a \euro 40 billion boost to the EU's economy each year. 
A new version of the 2003 Public Sector Information Directive~\cite{d2003} is also expected and the EC is strongly determined to act as a global leader along with United States, 
Canada or Australia. This open data is supposed to enable greater transparency; delivers more efficient public services; and encourages greater 
public and commercial use. More specifically in the context of e-Procurement the access to public contracts announcements is the first natural 
step to encourage SME participation and create a real public market of procurement opportunities.


Moreover, the emerging Web of Data and the sheer mass of information now available make it possible the deployment of new 
added-value services and applications based on the reuse of existing vocabularies and datasets. 
The popular diagram of the Linked Data Cloud~\cite{linked-data-cloud}, generated from metadata extracted from the 
Comprehensive Knowledge Archive Network (CKAN) out, contains $336$ datasets, with more than 25 billion RDF triples and 395 million links. 
With regards to e-Procurement, a new group of 130 CKAN datasets have been released from the ``OpenSpending.org'' site. In this realm, 
data coming from different sources and domains have been promoted following the principles of the 
Linked Data initiative~\cite{Berners-Lee-2006} to improve the access to large documental databases, 
e.g. e-Government resources, scientific publications or e-Health records. In the case of KOS, such as thesauri, taxonomies or classification systems 
are developed by specific communities and institutions in order to organize huge collections of information objects. 
These vocabularies allow users to annotate~\cite{Leukel-standard,Leukel-automating,Leukel-comparative} the information objects and easily retrieve them, 
promoting lightweight reasoning in the Semantic Web. Topic or subject indexing is an easy way to introduce machine-readable metadata for a resource's content 
description. Indeed, Product Scheme Classifications (also known as PSCs), such as the CPV, are a kind of KOS that have been built to solve specific problems 
of interoperability and communication between e-Commerce agents~\cite{FenselOmel2001,Leukel-findings}, in the particular case of PSCs they are considered to be a key-enabler of 
the next European e-Procurement domain and other supply chain processes~\cite{DBLP:journals/tcci/Alor-HernandezAJPRMBG10}.

Obviously the public information published by governmental contracting authorities, more specifically PSCs, are a suitable candidate to apply the Linked (Open) Data 
(LOD) principles and semantic web technologies providing a new environment in which the conjunction of these initiatives can provide the building blocks for an 
innovative unified pan-European information system that encourages standardization of key processes and systems and gives economic operators the tools to overcome 
technical interoperability easing and enhancing the access and reuse of public information. Therefore main contributions of this paper lay in:
\begin{itemize}
\item Modeling, transforming and interlinking the structure and data of PSCs developed by public bodies following the LOD principles.
\item Publishing all information via an SPARQL endpoint providing a public procurement Linked Data node and, more specifically, a new PSCs catalogue as Linked Data.
\item Exploiting the aforementioned catalogue via a contract descriptor recommendation service.
\item Demonstrating the gain (in terms of number of descriptors to retrieve public procurement notices) and the semantic exploitation 
capabilities of the new PSCs catalogue.
\end{itemize}

The remainder of this paper is structured as follows. Section~\ref{sect:related-work} reviews the relevant literature. Next thesauri 
conversion methods are outlined. Section~\ref{sect:method} presents the application of a method for promoting raw data to the LOD initiative 
and its application to the PSCs in the European e-Procurement sector. Section~\ref{sect:evaluation} describes the evaluation of applying the LOD principles to 
these product schemes through two studies that include a description of the sample, the method, results and discussion. Finally, 
the paper ends with a discussion of research findings, limitations and some concluding remarks.


\section{Related Work}
According to the topics of the paper the relevant works can be divided into the next points:

\begin{itemize}
 \item  The Semantic Web area, coined by Tim Berners-Lee in 2001, has experienced during last years a growing commitment from both academia and industrial areas 
 with the objective of elevating the meaning of web information resources through a common and shared data model (graphs) and 
 an underlying semantics based on different logic formalisms (ontologies). The Resource Description Framework (RDF), based on a graph model, 
 and the Web Ontology Language (OWL), designed to formalize and model domain knowledge, are the two main \textit{ingredients} to reuse information and data 
 in a knowledge-based realm. Thus data, information and knowledge can be easily shared, exchanged and linked~\cite{Maali_Cyganiak_2011} 
 to other knowledge and databases through the use URIs, more specifically HTTP-URIs. Therefore the broad objective of this effort can be summarized 
 as a new environment of added-value services that can encourage and improve B2B (Business to Business), B2C (Business to Client) or 
 A2A (Administration to Administration) relationships by means of the implementation of new contex-awareness expert systems to tackle existing 
 cross-domain problems~\cite{DBLP:journals/cbm/GonzalezA13,DBLP:journals/eswa/Casado-LumbrerasGAP12} such as medical reasoning,
 analysis of social media, etc. in which data heterogeneities, lack of standard knowledge representation 
 and interoperability problems are common factors. As a practical view of the Semantic Web, 
 the Linked Data~\cite{Berners-Lee-2006,Heath_Bizer_2011} emerges to create a large and distributed database on the Web. 
 In order to reach this major objective the publication of information and data under a common data model (RDF) 
 and formats with a specific formal query language (SPARQL~\cite{Sparql11}) provide the required building blocks to turn the Web of documents 
 into a real database or ``Web of Data''. Research works are focused in two main areas: 1) production/publishing~\cite{bizer07how} and 2) consumption of 
 Linked Data. In the first case data quality~\cite{bizer2007,wiqa,ld-quality,DBLP:journals/ws/BizerC09,lodq,link-qa}, conformance~\cite{DBLP:journals/ws/HoganUHCPD12}, 
 provenance~\cite{w3c-prov,DBLP:conf/ipaw/HartigZ10}, trust~\cite{Carroll05namedgraphs}, description of datasets~\cite{void,Cyganiak08semanticsitemaps,ckanValidator} and 
 entity reconciliation~\cite{Serimi,Maali_Cyganiak_2011} are becoming major objectives since a mass of amount data is already available through 
 RDF repositories and SPARQL endpoints. 
 
 On the other hand, consumption of Linked Data is being addressed to provide new ways of data visualization~\cite{DBLP:journals/semweb/DadzieR11,hoga-etal-2011-swse-JWS}, 
 faceted browsing~\cite{Pietriga06fresnel, citeulike:8529753,Sparallax} and searching~\cite{hoga-etal-2011-swse-JWS}, processing~\cite{Harth:2011:SIP:1963192.1963318} and exploitation of data applying 
 different approaches such as sensors~\cite{Jeung:2010:EMM:1850003.1850235,ontology-search} and techniques such as distributed queries\cite{Hartig09executingsparql,Ankolekar07thetwo,sparqlOpt}, 
 scalable reasoning process~\cite{Urbani2010WebPIE,HoganHarthPolleres2009,DBLP:conf/semweb/HoganPPD10}, 
 annnotation of web pages~\cite{rdfa-primer} or information retrieval~\cite{Pound} to name a few.
  
 \item In the field of e-Procurement there are projects trying to exploit the 
 information of public procurement notices like the ``Linked Open Tenders Electronic Daily'' project~\cite{loted} 
 where they use the RSS feeds of TED.  In the European project LOD2~\cite{lod2-project}, there is a specific workpackage, 
 WP9a ``LOD2 for A Distributed Marketplace for Public Sector Contracts'', to explore and demonstrate the 
 application of linked data principles for procuring contracts in the public sector and, 
 the Media Lab research group at the Technical University of Athens has recently published the 
 ``PublicSpending.net''~\cite{publicspending} portal to visualize and manage statistics about public spending around the world. 
 Finally the ``OpenSpending.org''~\cite{open-spending} portal also presents some specifications to model public procurement data and 
 visualize where the money goes. In general, the ``LOD Around-the-Clock''~\cite{latc-project} (LATC) and PlanetData~\cite{planet-data-project} 
 projects are also increasing the awareness of LOD across Europe delivering specific research and dissemination activities such as the 
 ``European Data Forum''. Furthermore legal aspects of public sector information are being reviewed in the 
 LAPSI project~\cite{lapsi-project}. In the case of vocabularies and datasets, GoodRelations and ProductOntology are two o
 f the most prominent approaches for tagging products and services using semantic web technologies, 
 for instance Renault UK has GoodRelations in RDFa in their UK merchandise store.

\end{itemize}

 Since an overview of semantic technologies and Linked Data and their application to e-Procurement has been presented 
 some remarks and discussion about related works are outlined below.
 
 \begin{enumerate}
  \item  On the one hand, entity reconciliation is becoming a major challenge in the Linked Data community due to its relevance 
 to enrich data with existing datasets and to perform some kind of reasoning process. Existing techniques are based 
 on natural language processing (NLP) algorithms that perform some kind of string comparison (labels~\cite{Serimi} or URIs~\cite{Maali_Cyganiak_2011}) 
 to establish a similarity value between two RDF resources under a threshold of confidence. Tools such as the Silk~\cite{DBLP:conf/semweb/JentzschIB10} provides a 
 a tool for discovering relationships between data items within different Linked Data sources or the DBPedia Spotlight~\cite{DBLP:conf/i-semantics/MendesJGB11}, a 
 ``tool for automatically annotating mentions of DBpedia resources in text, providing a solution for linking unstructured information sources to the Linked Open Data 
 cloud through DBpedia'' are relevant tools for linking existing RDF resources. Other approaches coming from the ontology mapping and alignment areas try to create 
 links according to the structure (relationships) and naming convention of the RDF resources. Finally other approaches based on learning algorithms such as 
 genetic programming~\cite{DBLP:conf/semweb/IseleB11} are emerging to learn linkage rules from existing datasets. As conclusion interlinking RDF resources is 
 consider to be a key-enabler for a better data consumption. Neverthless the main drawback of these approaches lies in the necessity of human validation 
 to ensure the validaty and quality of the link. Furthermore these tools are based on two main assumptions: 1) resources are already available in RDF and 
 2) parameters such as stopwords cannot be easily configured. That is why it is necessary to provide a custom PSC reconciliation service that takes into account 
 the specific characteristics of PSCs descriptors after the promotion to RDF.
 
 \item On the other hand there is an increasing interest in the creation of methodologies, best practices/recipes~\cite{best-gld,linked-data-cookbook} and lifecycles~\cite{gld-lifecycle,lod2-stack}. In this sense, some 
 Linked Data design considerations can be found in~\cite{bizer07how} covering from the design or URIs~\cite{Sauermann+2007a,bernerslee1998uri,uris-uk}, design patterns~\cite{linked-data-patterns}, 
 publication of RDF datasets and vocabularies~\cite{Berr08}, etc. to the establishment of Linked Data profiles~\cite{basic-profile-w3c}. Neverthless all these guidelines present 
 a tangled environment of aspects with different levels of abstraction that prevent a clear application to a specific problem such as the promotion of PSCs to the Linked Data initiative.
 
%  \item Finally information retrieval and recommending processes have been widely studied to FIXME. In the case of Spreading Activation~\cite{Collins_Loftus_1975} (SA) it has been applied to the resolution
%   of problems trying to simulate the behavior of the brain using a connectionist method to provide an ``intelligent'' way to retrieve information and data. The use of SA was motivated 
%   due to the research on graph exploration~\cite{Scott1981,AndersonTheory}. Nevertheless the success of this technique is specially relevant to the fields of Document~\cite{turtle91inference} 
%   and Information Retrieval~\cite{Cohen1987}. It has been also demonstrated its application to extract correlations between query terms and documents analyzing user 
%   logs~\cite{Cui03} and to retrieve resources amongst multiple systems~\cite{Schumacher+2008search} 
%   in which ontologies are used to link and annotate resources. In recent years and regarding the emerging use of ontologies in the Semantic Web area new applications of SA have 
%   appeared to explore concepts~\cite{Qiu93,Chen95} addressing the two important issues: 1) the selection and 2) the weighting of 
%   additional search terms and to measure conceptual similarity~\cite{gouws-vanrooyen-engelbrecht:2010:CCSR}. 
%   On the other hand, there are works~\cite{DBLP:journals/cogsr/KatiforiVD10} exploring the application of the SA on ontologies in order to create context inference models. The 
%   semi-automatically extension and refinement of ontologies~\cite{liu_et_al_2005} is other trending topic to apply SA in combination with other techniques based on natural language processing. 
%   Data mining, more specifically mining socio-semantic networks\cite{paper:troussov:2008}, and applications to collaborative filtering (community detection based on tag recommendations, expertise location, etc.) 
%   are other potential scenarios to apply the SA theory due to the high performance and high scalability of the technique. In particular, 
%   annotation and tagging~\cite{labraTagging2007} services to gather meta-data~\cite{GelgiVD05} from the Web or to predict social annotation~\cite{Chen:2007:PSA:1780653.1780702} and recommending 
%   systems based on the combination of ontologies and SA~\cite{citeulike:3779904} are taken advantage of using SA technique. Besides 
%   semantic search~\cite{conf-sofsem-Suchal08} is a highlight area to apply SA following hybrid approaches~\cite{bopaEstonia,RochaSA04} or user query expansion~\cite{767402} combining metadata 
%   and user information.

 \end{enumerate}
  


\section{A multidimensional criteria-based technique for quality assessment of Linked Open Data}
% \input{sections/computex}
% 
\section{Use Case: The European e-Procurement context}
\subsection{The MOLDEAS project}
%\input{sections/use-case-webindex}

\section{Research Study}
\subsection{Design}
\subsection{Sample}
\subsection{Results and Discussion}

% \begin{table*}[t]
%   \centering
%   \begin{tabular}{*{15}{c}}
%     \hline
%     One & Two & Three & Four & Five & Six & Seven & Eight & Nine & Ten & Eleven & Twelve & Thirteen & Fourteen & Fifteen \\
%     \hline
%     Fifteen & Fourteen & Thirteen & Twelve & Eleven & Ten & Nine & Eight & Seven & Six & Five & Four & Three & Two & One \\
%     \hline
%   \end{tabular}
%   \caption{Here is a caption.}
% \end{table*}


%\begin{sidewaystable}[ht!]
\begin{table*}[t]
\scriptsize
\renewcommand{\arraystretch}{1.3}
\begin{center}
\begin{tabular}{|p{2cm}||c|c|c||c|c|c||c|c|c||c|c|c||c|c|c||c|c|c||c|c|c||c|c|c|}
\hline
\textbf{Table}&\multicolumn{3}{|c||}{$T^{1}$} & \multicolumn{3}{|c||}{$T^{2}$}& \multicolumn{3}{|c||}{$T^{3}$} & \multicolumn{3}{|c||}{$T^{3}_1$} & \multicolumn{3}{|c||}{$T^{4}$} & \multicolumn{3}{|c||}{$T^{4}_1$} & \multicolumn{3}{|c||}{$T^{5}$} & \multicolumn{3}{|c|}{$T^{6}$} \\ \hline
 &\si&\no&\na&	\si&\no&\na&	\si&\no&\na&	\si&\no&\na&	\si&\no&\na&	\si&\no&\na&	\si&\no&\na&	\si&\no&\na \\ \hline
 \textbf{Reference} &\textbf{60}&\textbf{0}&\textbf{9}&	\textbf{44}&\textbf{0}&\textbf{0}&	\textbf{4}&\textbf{0}&\textbf{0}&	\textbf{5}&\textbf{0}&\textbf{0}&	\textbf{8}&\textbf{0}&\textbf{0}&	\textbf{14}&\textbf{0}&\textbf{0}&	\textbf{5}&\textbf{0}&\textbf{0}&	\textbf{33}&\textbf{0}&\textbf{14}\\ \hline \hline
  \multicolumn{25}{|c|}{\textbf{Public Procurement Notices}} \\ \hline
 \textbf{TED}	     			 & 13 & 7 & 49 	& 0 & 0 & 44  	& 1 & 0 & 3  & 1 & 0 & 4  & 6 & 2 & 0  & 11 & 3 & 0  	& 0 & 0 & 5  & 0 & 0 & 47 \\ \hline
 \textbf{CONTRATACION}& 15 & 5 & 49 	& 0 & 0 & 44  	& 1 & 0 & 3  & 1 & 0 & 4  & 7 & 1 & 0  & 11 & 3 & 0  	& 0 & 0 & 5  & 0 & 0 & 47 \\ \hline 
 \textbf{BOE}	     			& 10 & 9 & 50 	& 0 & 0 & 44  	& 1 & 0 & 3  & 1 & 0 & 4  & 8 & 0 & 0  & 10 & 3 & 1  	& 0 & 0 & 5  & 0 & 0 & 47 \\ \hline 
 \textbf{Third-parties services}     	& 12 & 8 & 49 	& 0 & 0 & 44  	& 1 & 0 & 3  & 1 & 0 & 4  & 5 & 3 & 0  & 6 & 3 & 5  	& 0 & 0 & 5  & 0 & 0 & 47 \\ \hline 
 \textbf{LOTED}	     		& 35 & 24 & 10 	& 23 & 8 & 13  	& 4 & 0 & 0  & 5 & 0 & 0  & 8 & 0 & 0  & 12 & 2 & 0  	& 5 & 0 & 0  & 0 & 0 & 47 \\ \hline 
 \textbf{MOLDEAS}	     	& 54 & 7 & 8  	& 31 & 3 & 10 	& 4 & 0 & 0  & 5 & 0 & 0  & 8 & 0 & 0  & 14 & 0 & 0 	& 5 & 0 & 0  & 0 & 0 & 47 \\ \hline 
 \multicolumn{25}{|c|}{\textbf{Catalogue of Product Scheme Classifications}} \\ \hline
 \textbf{CSV/MSExcel} 		& 9 & 4 & 56 & 0 & 0 & 44 & 0 & 0 & 4 & 2 & 0 & 3 & 8 & 0 & 0 & 6 & 8 & 0 & 0 & 0 & 5 & 0 & 0 & 47 \\ \hline 
 \textbf{Servicios on-line} 	& 11 & 11 & 47 & 0 & 0 & 44 & 0 & 0 & 4 & 0 & 0 & 5 & 5 & 2 & 1 & 5 & 8 & 1 & 0 & 0 & 5 & 0 & 0 & 47 \\ \hline 
 \textbf{MOLDEAS}		& 54 & 7 & 8 & 31 & 3 & 10 & 4 & 0 & 0 & 5 & 0 & 0 & 8 & 0 & 0 & 14 & 0 & 0 & 5 & 0 & 0 & 32 & 0 & 15 \\ \hline 
 \multicolumn{25}{|c|}{\textbf{Organizations}} \\ \hline
 \textbf{TED} 			&  3 & 3 & 63 & 0 & 0 & 44 & 0 & 0 & 4 & 1 & 0 & 4 & 6 & 2 & 0 & 10 & 4 & 0 & 0 & 0 & 5 & 0 & 0 & 47 \\ \hline 
 \textbf{CONTRATACION} &    13 & 4 & 52 & 0 & 0 & 44 & 1 & 0 & 3 & 1 & 4 & 0 & 8 & 0 & 0 & 12 & 2 & 0 & 0 & 0 & 5 & 0 & 0 & 47 \\ \hline 
 \textbf{BORME}			&    1 & 0 & 68 & 0 & 0 & 44 & 0 & 0 & 4 & 1 & 0 & 4 & 8 & 0 & 0 & 13 & 1 & 0 & 0 & 0 & 5 & 0 & 0 & 47 \\ \hline 
 \textbf{Third-parties services} 	&  12 & 8 & 49 & 0 & 0 & 44 & 1 & 0 & 3 & 1 & 0 & 4 & 1 & 7 & 0 & 5 & 5 & 4 & 0 & 0 & 5 & 0 & 0 & 47 \\ \hline 
 \textbf{Commercial Databases} &     1 & 0 & 68 & 0 & 0 & 44 & 0 & 0 & 4 & 0 & 0 & 5 & 1 & 5 & 2 & 10 & 4 & 0 & 0 & 0 & 5 & 0 & 0 & 47 \\ \hline 
 \textbf{OpenCorporates} 	&    35 & 22 & 12 & 21 & 10 & 13 & 4 & 0 & 0 & 4 & 0 & 1 & 8 & 0 & 0 & 13 & 1 & 0 & 0 & 0 & 5 & 0 & 0 & 47 \\ \hline 
 \textbf{MOLDEAS} 		&    54 & 7 & 8 & 44 & 0 & 0 & 4 & 0 & 0 & 5 & 0 & 0 & 8 & 0 & 0 & 14 & 0 & 0 & 5 & 0 & 0 & 0 & 0 & 47 \\ \hline 
\hline
  \end{tabular}
  \caption{Aggregated validation table including parcial evaluation.}
  \label{tabla:agregado-full}
  \end{center}
%\end{sidewaystable}
\end{table*} 




\section{Conclusions and Future Work}
\input{sections/conclusions}

\section*{Acknowledgment}
The research leading to these results has received funding from the European Union’s Seventh Framework Programme (FP7-PEOPLE-2010-ITN) 
under grant agreement n° 264840 and developed in the context of the workpackage 4 and more specifically under the project 
``Quality Management in Service-based Systems and Cloud Applications''.

\nocite{*}
\bibliographystyle{IEEEtran}
% argument is your BibTeX string definitions and bibliography database(s)
\bibliography{references}

\clearpage
\appendix[Validation Tables]

\begin{table}[t]
\scriptsize
\renewcommand{\arraystretch}{1.3}
\begin{center}
\begin{tabular}[c]{|l|p{5cm}|c|} 
\hline 
  \textbf{ID} & \textbf{Question} &  \textbf{Expected value} \\\hline  
  1& \multicolumn{2}{c|}{\textbf{\textit{Design of an URI scheme}}} \\ \hline
  1.1&  Are the resources accessible via their URIs? \textit{Minting HTTP URIs}? & \si \\ \hline 
  1.2&  Is the namespace under our control? & \si \\ \hline
  1.3&  Is the HTTP scheme used? &\si  \\ \hline
  1.4&  Can the URIs consider as \textit{Cool URIs}? &\si  \\ \hline
  1.5&  Are the URIs of type \textit{hash URIs}? & \na  \\ \hline
  1.6&  Are the URIs of type \textit{slash URIs}? & \si  \\ \hline
  1.7&  Are the URIs of type \textit{param URIs}? & \na  \\ \hline
  1.8&  Is not any implementation detail in the URI?& \si  \\ \hline
  1.9&  Are primary-keys used to identify resources (ID URIs)?  & \si  \\ \hline
  1.10& Are the URIs of type  \textit{Meaningful URIs}? & \na  \\ \hline
  1.11& Has an URI base been defined for all resources? & \si  \\ \hline
  1.12& Has an URI been defined for the data model?  & \si  \\ \hline
  1.13& Has a complete URI scheme been defined for RDF \dataset RDF? & \si  \\ \hline
  1.14& Has a complete URI scheme been defined for the data model? & \si  \\ \hline
  1.15& Has any metadata been included in the URIs? & \si  \\ \hline
 2&\multicolumn{2}{c|}{\textbf{\textit{RDF resources description}}}\\ \hline
  2.1& Have vocabularies such as SKOS, RDFS or OWL been used to create the domain model? & \si  \\ \hline
  2.2& Have the selected vocabularies been selected according to their adequacy and up-to-date?& \si  \\ \hline
  2.3& Have annotations in RDFS or SKOS been included?& \si  \\ \hline
  2.4& Have the new classes and properties been aligned to existing ones?& \si \\ \hline
  2.5& Have some existing classes and properties been re-used?& \si  \\ \hline
  2.6& Have new classes and properties been defined?&\si  \\ \hline
  2.7& Have the RDF resources been enriched with existing and widely-accepted RDF datasets?& \si  \\ \hline
  2.8& Have the RDF resources been locally described when they are linked to existing RDF resources in external datasets?& \no  \\ \hline
  2.9& Is there any extra-metainformation in all RDF resources?& \si  \\ \hline
  2.10& Is it possible to browse all RDF resources?& \si  \\ \hline
  2.11& Is there any useful information in the URI of RDF resources?& \si  \\ \hline
  2.12& Are the URIs accessible via Internet protocols? & \si  \\ \hline  
 3&\multicolumn{2}{c|}{\textbf{\textit{RDF \dataset description}}}\\ \hline
  3.1& Has an URI been defined for all the RDF datasets? & \si  \\ \hline
  3.2& Have the voID vocabulary or similar been used to describe the RDF dataset? & \si  \\ \hline
  3.3& Is there any Semantic Sitemap? & \na  \\ \hline
  3.4& Is there any provenance information? & \si  \\ \hline
  3.5& Is there any information about the license, copyright, etc.?&  \si  \\ \hline
  3.6& Is there any information about the authors? & \si  \\ \hline
  3.7& Is there any example of the RDF resources? & \si  \\ \hline
  3.8& Have annotations in RDFS or SKOS been included? &  \si  \\ \hline
  3.9& Have multilingual annotations in RDFS or SKOS been included ?& \si  \\ \hline
 \hline
  \end{tabular}
\caption{$T^{1}$-List of Linked Data features (I).}\label{table:validation-t11}  
  \end{center}
\end{table} 



\begin{table}[t]
\scriptsize
\renewcommand{\arraystretch}{1.3}
\begin{center}
\begin{tabular}[c]{|l|p{5cm}|c|} 
\hline 
  \textbf{ID} & \textbf{Question} &  \textbf{Expected value} \\\hline  
 4&\multicolumn{2}{c|}{\textbf{\textit{Others}}}\\ \hline
  4.1& Has any tool/automatic method been used to generate the RDF?& \si  \\ \hline
  4.2& Does the dataset contain consolidated RDF?& \no  \\ \hline
  4.3& Is the process of entity reconciliation performed automatically?& \no  \\ \hline
  4.4& Is the source dataset dynamic?& \na  \\ \hline
  4.5& Is the source dataset static? & \si  \\ \hline
  4.6& Is the target dataset size in order of million of triples?& \si  \\ \hline
  4.7& Is the target dataset size in order of billion of triples?& \na  \\ \hline
5&\multicolumn{2}{c|}{\textbf{\textit{Publishing \linkeddata}}}\\ \hline
  5.1&  Is there any RDF data dump publicly available? & \no  \\ \hline 
  5.2&  Has any automatic method been used for publishing RDF? & \si  \\ \hline
  5.3&  Is there any formal query language to access data? & \si  \\ \hline
  5.4&  Is any SPARQL endpoint publicly available? & \si  \\ \hline
  5.5&  Is there any content negotiation method to access RDF resources? & \si  \\ \hline
  5.6&  Is it possible to create links between the RDF resources and other documents such as HTML? & \si  \\ \hline    
  5.7&  Is there any Linked Data Frontend available to browse the RDF repository? & \si  \\ \hline  
  5.8&  Is there any metadescription of the RDF dataset? & \si  \\ \hline
  5.9&  Has the RDF dataset (data, namespaces, etc.) been published through services such as CKAN, Prefix.cc, etc.? & \si  \\ \hline
  5.10&  Is there any API or web service to access data? & \si  \\ \hline
  5.11&  Is there any restriction to query data? & \no  \\ \hline
  5.12&  Is there any privacy mechanism available? & \no  \\ \hline
  5.13&  Is there any information about the size of the dataset? & \si  \\ \hline  
  5.14&  Are the RDF datasets (or resources) published as a static file? & \na  \\ \hline
  5.15&  Are the RDF datasets (or resources) generated on-the-fly from a database?& \na  \\ \hline
  5.16&  Are the RDF datasets (or resources) published via a RDF repository? & \si  \\ \hline    
  5.17&  Are the RDF datasets (or resources) generated on-the-fly from a customized application? & \si  \\ \hline
  5.18&  Do the RDF datasets (or resources) contain temporal or evolution information?& \no  \\ \hline     
  5.19&  Does the RDF dataset contain examples for debugging and consumption? & \si  \\ \hline        
  5.20&  Does the RDF dataset contain alias to specific dirs or names? & \si  \\ \hline
  5.21&  In case of an error, is there any default RDF resource? &  \no  \\ \hline      
  5.22&  ¿Se utilizan protocolos estándar? & \si  \\ \hline    
  5.23&  ¿Se provee algún mecanismo de realimentación? & \no  \\ \hline    
  5.24&  ¿Se provee documentación sobre los datos publicados? & \si  \\ \hline
  5.25&  ¿Se proveen estadísticas de los datos publicados? & \si  \\ \hline    
  5.26&  ¿Se utilizan mecanismos de sellado en el tiempo o similares? & \si  \\ \hline            
 \hline
  \end{tabular}
\caption{$T^{1}$-List of Linked Data features (II).}\label{table:validation-t12}  
  \end{center}
\end{table} 



\begin{table}[t]
\scriptsize
\renewcommand{\arraystretch}{1.3}
\begin{center}
\begin{tabular}[c]{|l|p{5cm}|c|} 
\hline 
   1& \multicolumn{2}{c|}{\textbf{\textit{Identifier Patterns}}}\\ \hline
  1.1 &  \textit{Hierarchical URIs} &\si \\ \hline
  1.2 &  \textit{Literal Keys} &\si \\ \hline
  1.3 &  \textit{Natural Keys} &\si \\ \hline
  1.4 &  \textit{Patterned URIs} &\si \\ \hline
  1.5 &  \textit{Proxy URIs} &\na \\ \hline
  1.6 &  \textit{Shared Keys} &\na \\ \hline
  1.7 &  \textit{URL Slug} &\na \\ \hline    
    2& \multicolumn{2}{c|}{\textbf{\textit{Modelling Patterns}}}\\ \hline
  2.1 &  \textit{Custom Datatype} &\si \\ \hline    
  2.2 &  \textit{Index Resources} &\na \\ \hline    
  2.3 &  \textit{Label Everything} &\si \\ \hline     
  2.4 &  \textit{Link Not Label} &\si \\ \hline    
  2.5 &  \textit{Multi-Lingual Literal} &\si \\ \hline    
  2.6 &  \textit{N-Ary Relation} &\na \\ \hline    
  2.7 &  \textit{Ordered List} &\na \\ \hline     
  2.8 &  \textit{Ordering Relation} &\na \\ \hline     
  2.9 &  \textit{Preferred Label} &\si \\ \hline    
  2.10 &  \textit{Qualified Relation} &\si \\ \hline   
  2.11 &  \textit{Reified Statement} &\na \\ \hline    
  2.12 &  \textit{Topic Relation} &\si \\ \hline      
  2.13 &  \textit{Typed Literal} &\si \\ \hline        
    3& \multicolumn{2}{c|}{\textbf{\textit{Publishing Patterns}}}\\ \hline
  3.1 &  \textit{Annotation} &\si \\ \hline    
  3.2 &  \textit{Autodiscovery} &\si \\ \hline    
  3.3 &  \textit{Document Type} &\si \\ \hline     
  3.4 &  \textit{Edit Trail} &\no \\ \hline    
  3.5 &  \textit{Embedded Metadata} &\si \\ \hline    
  3.6 &  \textit{Equivalence Links} &\si \\ \hline    
  3.7 &  \textit{Link Base} &\si \\ \hline     
  3.8 &  \textit{Materialize Inferences} &\na \\ \hline     
  3.9 &  \textit{Named Graphs} &\si \\ \hline    
  3.10 &  \textit{Primary Topic Autodiscovery} &\si \\ \hline    
  3.11 &  \textit{Progressive Enrichment} &\si \\ \hline      
  3.12 &  \textit{See Also} &\si \\ \hline    
        4& \multicolumn{2}{c|}{\textbf{\textit{Application Patterns}}}\\ \hline
  4.1 &  \textit{Assertion Query} &\si \\ \hline    
  4.2 &  \textit{Blackboard} &\si \\ \hline    
  4.3 &  \textit{Bounded Description} &\si \\ \hline     
  4.4 &  \textit{Composite Descriptions} &\si \\ \hline    
  4.5 &  \textit{Follow Your Nose}&\no \\ \hline    
  4.6 &  \textit{Missing Isn't Broken} &\no \\ \hline    
  4.7 &  \textit{Parallel Loading} &\no \\ \hline     
  4.8 &  \textit{Parallel Retrieval} &\no \\ \hline
  4.9 &  \textit{Resource Caching} &\no \\ \hline    
  4.10 &  \textit{Schema Annotation} &\si \\ \hline    
  4.11 &  \textit{Smushing} &\si \\ \hline
  4.12 &  \textit{Transformation Query} &\no \\ \hline        
\hline 

   \end{tabular}
  \caption{$T^{2}$-List of \textit{Linked Data Patterns}.}\label{table:validation-t2}   
  \end{center}
\end{table} 



\begin{table}[t]
\scriptsize
\renewcommand{\arraystretch}{1.3}
\begin{center}
\begin{tabular}[c]{|l|p{5cm}|c|} 
\hline
  \textbf{ID} & \textbf{Pregunta} &  \textbf{Cumplimiento}  \\\hline
   1.1&\textit{Use URIs as names for things} & \si  \\ \hline
   1.2&\textit{When someone looks up a URI, provide useful information, using the standards (RDF*, SPARQL)} & \si \\ \hline  
   1.3&\textit{Include links to other URIs} & \si \\ \hline    
   1.4&\textit{Use HTTP URIs} & \si \\ \hline    
   \hline
   \end{tabular}
  \caption{$T^{3}$-List of Linked Data Principles.}
  \label{table:validation-t3}
  \end{center}
\end{table} 

\begin{table}[t]
\scriptsize
\renewcommand{\arraystretch}{1.3}
\begin{center}
\begin{tabular}[c]{|l|p{5cm}|c|} 
\hline
  \textbf{ID} & \textbf{Pregunta} &  \textbf{Cumplimiento}  \\\hline
    1.1&$\star$	& \si \\ \hline 
    1.2&$\star \star$	 & \si \\ \hline 
    1.3&$\star \star \star$	& \si  \\ \hline 
    1.4&$\star \star \star \star$ & \si \\ \hline 
   1.5&$\star \star \star \star \star$ & \si \\ \hline 
   \end{tabular}
   \caption{$T^{3}_1$-List of features of the 5 $\star$ Model.}
   \label{table:validation-t31}
  \end{center}
\end{table} 


\begin{table}[t]
\scriptsize
\renewcommand{\arraystretch}{1.3}
\begin{center}
\begin{tabular}[c]{|l|p{5cm}|c|} 
\hline
  \textbf{ID} & \textbf{Pregunta} &  \textbf{Cumplimiento}  \\\hline
  \multicolumn{3}{|c|}{\textbf{8 Principios}}  \\ \hline
   1.1& \textit{Complete} & \si  \\ \hline
   1.2&\textit{Primary} & \si  \\ \hline  
   1.3&\textit{Timely} & \si  \\ \hline  
   1.4&\textit{Accessible} & \si  \\ \hline  
   1.5&\textit{Machine processable} & \si  \\ \hline  
   1.6&\textit{Non-Discriminatory} & \si  \\ \hline  
   1.7&\textit{Non-Proprietary} &\si  \\ \hline
   1.8&\textit{License-free} & \si  \\ \hline                                                               
  \hline
  \end{tabular}
 \caption{$T^{4}$-List of Open Data Principles.}
  \label{table:validation-t4}
  \end{center}
\end{table} 



\begin{table}[t]
\scriptsize
\renewcommand{\arraystretch}{1.3}
\begin{center}
\begin{tabular}[c]{|l|p{5cm}|c|} 
\hline
  \textbf{ID} & \textbf{Pregunta} &  \textbf{Cumplimiento}  \\\hline
   \multicolumn{3}{|c|}{\textbf{Producción}}  \\ \hline
   1.1& ¿Se ha definido una misión y estrategia para la apertura de los datos? & \no  \\ \hline
   1.2& ¿Los datos proceden de una fuente segura? & \no  \\ \hline
   1.3& ¿Se puede conocer la procedencia de los datos? & \si  \\ \hline    
   1.4& ¿Existe algún mecanismo para asegurar la calidad de los datos? & \si  \\ \hline  
  \multicolumn{3}{|c|}{\textbf{Ventajas}}  \\ \hline
   2.1& ¿Facilitan los datos la inclusión? & \si  \\ \hline
   2.2& ¿Mejoran la transparencia? & \si  \\ \hline    
   2.3& ¿Existe alguna responsabilidad sobre los datos? & \no  \\ \hline
  \multicolumn{3}{|c|}{\textbf{Beneficios}}  \\ \hline
   3.1& ¿Pueden las aplicaciones servirse de estos datos para generar servicios, reutilización? & \si  \\ \hline
   3.2& ¿Se pueden generar múltiples vistas de los datos? & \si  \\ \hline
   3.3& ¿Se pueden integrar con otras fuentes de datos? & \si  \\ \hline     
   \multicolumn{2}{|c|}{\textbf{Consumo}}  \\ \hline
   4.1& Uso de anotaciones & \si  \\ \hline        
   4.2& ¿Se provee un API o servicio web de consumo? & \si  \\ \hline
   4.3& ¿Se provee algún mecanismo de sindicación para obtener los datos?& \no  \\ \hline
   4.4& ¿Existe algún modelo formal o especificación de los datos publicados? & \si  \\ \hline                                                             
  \hline
  \end{tabular}
  \caption{$T^{4}_1$-List of Open Data features.}
  \label{table:validation-t41}
  \end{center}
\end{table} 




\begin{table}[t]
\scriptsize
\renewcommand{\arraystretch}{1.3}
\begin{center}
\begin{tabular}[c]{|l|p{5cm}|c|} 
\hline
  \textbf{ID} & \textbf{Pregunta} &  \textbf{Cumplimiento}  \\\hline
   1& ¿Son los recursos RDF accesibles mediante HTTP o HTTPS? & \si  \\ \hline
   2& ¿Se provee negociación de contenido? & \si  \\ \hline
   3& ¿El \dataset contiene más de 1000 tripletas? & \si  \\ \hline
   4& ¿Se provee, al menos, 50 enlaces a \datasets ya disponibles en el diagrama? & \si \\ \hline
   5& ¿Se provee acceso al \dataset completo? & \si  \\ \hline
  \hline
  \end{tabular}
  \caption{$T^{5}$-List of features to join \textit{The Linking Open Data Cloud}.}
  \label{table:validation-t5}
  \end{center}
\end{table} 




\begin{table}[t]
\scriptsize
\renewcommand{\arraystretch}{1.3}
\begin{center}
\begin{tabular}[c]{|l|p{5cm}|c|} 
  \textbf{ID} & \textbf{Pregunta} &  \textbf{Cumplimiento}  \\\hline
  1&\multicolumn{2}{|c|}{\textbf{Standard CKAN fields}}  \\ \hline
  1.1&  \textit{Name} & \si  \\ \hline
  1.2 &  \textit{Title} & \si  \\ \hline
  1.3 &  \textit{URL} & \si  \\ \hline
  \multicolumn{3}{|c|}{\textbf{Enhanced CKAN fields}}  \\ \hline
   1.4&  \textit{Version} & \si  \\ \hline
   1.5&  \textit{Notes} & \no  \\ \hline
   1.6&  \textit{Author} & \si  \\ \hline
   1.7&  \textit{Author email} &\si  \\ \hline
   1.8&  \textit{License} &\si  \\ \hline
   \multicolumn{3}{|c|}{\textbf{Custom CKAN fields}}  \\ \hline
   1.9&  \textit{shortname} & \si  \\ \hline
   1.10&  \textit{license\_link} & \si  \\ \hline
   1.11&  \textit{sparql\_graph\_name} & \no  \\ \hline
   1.12&  \textit{namespace} & \si  \\ \hline
   1.13&  \textit{triples} & \si  \\ \hline
   1.14&  \textit{links:xxx} &\si  \\ \hline
  2&\multicolumn{2}{|c|}{\textbf{CKAN tags}}  \\ \hline
  2.1&  \textit{<topic>} &\si \\ \hline
  \multicolumn{3}{|c|}{\textbf{Metainformation CKAN tags}}  \\ \hline  
  2.2&\textit{format-<prefix>}&\si \\ \hline
  2.3&\textit{no-proprietary-vocab}&\na \\ \hline
  2.4 &\textit{deref-vocab}&\si \\ \hline
  2.5&\textit{no-deref-vocab}&\na \\ \hline
  2.6&\textit{vocab-mappings}&\si \\ \hline
  2.7&\textit{no-vocab-mappings}&\na \\ \hline
  2.8&\textit{provenance-metadata}&\si \\ \hline
  2.9&\textit{no-provenance-metadata}&\na \\ \hline
  2.10&\textit{license-metadata}&\si \\ \hline
  2.11&\textit{no-license-metadata}&\na \\ \hline	
  2.12&\textit{published-by-producer}&\na \\ \hline
  2.13&\textit{published-by-third-party}&\si \\ \hline		
  2.14&\textit{limited-sparql-endpoint}&\no \\ \hline
  2.15&\textit{lodcloud.nolinks}&\no \\ \hline		
  2.16&\textit{lodcloud.unconnected} &\no \\ \hline
  2.17&\textit{lodcloud.needsinfo}&\no \\ \hline				
  2.18&\textit{lodcloud.needsfixing}&\no \\ \hline				
  3& \multicolumn{2}{|c|}{\textbf{CKAN resource links}}  \\ \hline
  3.1&  \textit{Download Page} &\si \\ \hline
  3.2&  \textit{meta/sitemap} &\na \\ \hline
  3.3&  \textit{api/sparql} &\si \\ \hline
  3.4&  \textit{meta/void} &\si \\ \hline
  3.5&  \textit{application/rdf+xml} &\si \\ \hline
  3.6&  \textit{text/turtle} &\na \\ \hline
  3.7&  \textit{application/x-ntriples} &\na \\ \hline
  3.8&  \textit{application/x-nquads} &\na \\ \hline
  3.9&  \textit{meta/rdf-schema} &\si \\ \hline
  3.10&  \textit{example/rdf+xml} &\si \\ \hline
  3.11& \textit{example/turtle} &\na \\ \hline
  3.12&  \textit{example/ntriples }&\na \\ \hline
  3.13&  \textit{example/rdfa} &\na \\ \hline
  3.14&  \textit{mapping/\{format\}} &\no \\ \hline
    \hline
  \end{tabular}
  \caption{$T^{6}$-List of features to register a \dataset in a CKAN repository.}
  \label{table:validation-t6}
  \end{center}
\end{table} 








\end{document}


