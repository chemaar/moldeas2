Government bodies and public institutions as a whole are the most important buyers in the European Union (EU), since public procurement spending represents around 19\% of 
EU Gross Domestic Product (GDP)~\cite{d2010}. Given this situation there is a growing interest and commitment~\cite{d2010a}. to ensure that these funds are 
well managed and most of inefficiencies are eliminated. Electronic public procurement or e-Procurement emerges as an alternative to link and 
integrate inter-organizational business processes and systems with the automation of the requisitioning, the approval purchase order 
management, accounting processes among others through the Internet-based protocol~\cite{Podlogar2007}.  However, there is much more at stake than the mere changeover 
from paper based systems to ones using electronic communications. It should have the potential to yield important improvements in the access to information and d
ata such as the efficiency of individual purchases, the overall administration of public procurement or the functioning of the markets for government contracts. 
The technology in this area may make it possible to automate the processes involved in the public procurement sector. Furthermore, features for supplier 
management and complex auctions should be included in by means of applying new technologies and methods to accomplish the requirements of this new realm of electronic businesses.


The European Commission is trying to unlock this potential, the 2004 Public Procurement Directives 2004/17/EC and 2004/18/EC 
introduced several provisions aimed at enabling e-Procurement uptake in all Member States. In this light of 
modernizing the European public procurement to support growth and employment the EC also identified~\cite{siemensEval} both regulatory and natural barriers in the access to 
public procurement in the EU context, especially for SMEs. According to this evaluation, a real European Single Market use~\cite{d2011} 
has not yet been achieved causing losses, being cost-inefficient, missing opportunities for society and leading to a 
situation where more than 27 national markets co-exist instead of an EU-wide public market~\cite{monti2010}.

As a consequence the EC outlined the following advantages in the wider use of e-Procurement~\cite{d2010}: increased accessibility and transparency, benefits for individual procedures, 
benefits in terms of more efficient procurement administration and potential for integration of EU procurement markets. 
However several interlinked challenges to achieve a successful transition to e-Procurement are missing: overcoming inertia 
and fears on the side of contracting authorities and suppliers, lack of standards in e-Procurement processes, no means 
to facilitate mutual recognition of national electronic solutions, onerous technical requirements, 
particularly for bidder authentication and managing multi-speed transition to e-Procurement. 


In this sense, one relevant action to ease the interconnectivity and interoperability in this landscape was the creation of the Tenders Electronic Daily (TED) 
by the EC. It is the on-line version of the ``Supplement to the Official Journal of the European Union'', dedicated to European public procurement 
($1500$ new public procurement notices every day) but a unified pan-European information system dealing with: dispersion of the information,
duplication of the same notice in more than one source, different publishing formats, problems with regards to a multilingual environment and 
aggregation of low-value procurement opportunities, is still missing. 


Other EU actions in the e-Government context were focused on the development of conceptual/terminological maps available in RAMON, the Eurostat's metadata server: 
in the Health field, the ``European Schedule of Occupational Diseases'' or ``International Classification of Diseases''; in the Education field,  
thesauri as ``European Education Thesaurus''; European Glossary on Education; in the Employment field, 
the ``International Standard Classification of Occupations'';  in the European Parliament activities the ``Eurovoc Thesaurus'' and in the 
e-Procurement field the ``Common Procurement Vocabulary'' (CPV). The structure and features of these systems are very heterogeneous, 
although some common aspects can be found in all of them: hierarchical relationships between terms or concepts and multilingual character of the information. 
These knowledge organization systems (KOS) enable users to annotate information objects providing an agile mechanism for performing 
tasks such as exploration, searching, automatic classification or reasoning. These actions constitute a part of the strategy that EC 
is carrying out to encourage the creation of a knowledge-based and standardized environment to boost and improve access to 
public information and more specifically to public procurement opportunities. Nevertheless, a question about further steps is not yet answered: ``\textit{What further steps might be 
taken to improve the access of all interested parties, particularly SMEs, to e-Procurement systems?}''

On the other hand and following the principles of the Open Data initiative, the vice-president Neelie Kroes is leading the Open Data Strategy for Europe. 
She also outlined in December 2011 several types of measures that will help to unleash potential of data held by governments in Europe. The strategy is 
strongly focused on the commercial value of the re-use of public sector information, by which the Commission expects to deliver a \euro 40 billion boost to the EU's economy each year. 
A new version of the 2003 Public Sector Information (PSI) Directive is also expected and the EC is strongly determined to act as a global leader along with United States, 
Canada or Australia. This open data is supposed to enable greater transparency; delivers more efficient public services; and encourages greater public and commercial use.


Moreover, the emerging Web of Data and the sheer mass of information now available make it possible the deployment of new 
added-value services and applications based on the reuse of existing vocabularies and datasets. 
The popular diagram of the Linked Data Cloud~\cite{linked-data-cloud}, generated from metadata extracted from the 
Comprehensive Knowledge Archive Network (CKAN) out, contains 337 datasets, with more than 25 billion RDF triples and 395 million links. 
With regards to e-Procurement, a new group of 130 CKAN datasets have been released from OpenSpending.org site. In this realm, 
data coming from different sources and domains have been promoted following the principles of the 
Linked Data initiative~\cite{Berners-Lee-2006} to improve the access (in terms of expressivity) to large documental databases, 
e.g. e-Government resources, scientific publications or e-Health records. In the case of KOS, such as thesauri, taxonomies or classification systems 
are developed by specific communities and institutions in order to organize huge collections of information objects. 
These vocabularies allow users to annotate~\cite{Leukel-standard,Leukel-automating,Leukel-comparative} the information objects and easily retrieve them, 
promoting lightweight reasoning in the Semantic Web. Topic or subject indexing is an easy way to introduce machine-readable metadata for a resource's content 
description. Indeed, Product Scheme Classifications (also known as PSCs), such as the CPV, are a kind of KOS that have been built to solve specific problems 
of interoperability and communication between e-Commerce agents~\cite{FenselOmel2001,Leukel-findings}, in the particular case of they are considered to be a key-enabler of 
the next European e-Procurement domain and other supply chain processes~\cite{DBLP:journals/tcci/Alor-HernandezAJPRMBG10}. 

Obviously the public information published by governmental contracting authorities, more specifically PSCs, are a suitable candidate to apply the Linked (Open) Data 
(LOD) principles and semantic web technologies providing a new environment in which the conjunction of these initiatives can provide the building blocks for an 
innovative unified pan-European information system that encourages standardization of key processes and systems and gives economic operators the tools to overcome 
technical interoperability easing and enhancing the access and the reuse of public information.

Finally the ``European Code of Best Practices Facilitating Access by SMEs to Public Procurement Contracts''~\cite{d2008} pointed out 
two groups of difficulties regarding the barriers faced by SMEs in accessing relevant information about public procurement opportunities. 
The document stated that ``the (big) number of such web portals being used by the government and by regional and local authorities makes it difficult 
for tenderers to maintain an overview''. By aggregating data on public procurement in all Member States 
and at all administrative levels in a standardized way and reusing existing efforts of modelling 
and classifying domain knowledge, some open and critical issues can be tackled to improve a situation which affects more 
than \euro $20$ million non-financial companies established in the EU.


The remainder of this paper is structured as follows. Section~\ref{sect:related-work} reviews the relevant literature. Next thesauri 
conversion methods are outlined. Section~\ref{sect:method} presents a method for promoting raw data to the LOD initiative 
and its application to the PSCs in the European e-Procurement sector. Section~\ref{sect:evaluation} describes the evaluation of applying the LOD principles to 
these product schemes through two studies that include a description of the sample, the method, results and discussion. Finally, 
the paper ends with a discussion of research findings, limitations and concluding remarks.
