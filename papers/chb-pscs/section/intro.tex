Government bodies and public institutions as a whole are the most important buyers in the European Union (EU), since public procurement spending represents around 19\% of 
EU Gross Domestic Product (GDP)~\cite{d2010}. Given this situation there is a growing interest and commitment~\cite{d2010a} to ensure that these funds are 
well managed and most of inefficiencies are eliminated. Electronic public procurement or e-Procurement~\cite{Podlogar2007} emerges as an alternative to link and 
integrate inter-organizational business processes and systems with the automation of the requisitioning, the approval purchase order 
management, accounting processes among others through the Internet-based protocol. 

In this context the European Commission (EC) is trying to unlock this potential, the 2004 Public Procurement Directives 2004/17/EC and 2004/18/EC 
introduced several provisions and projects~\cite{peppol,e-certis} aimed at enabling e-Procurement uptake in all Member States. In this light of 
modernizing the European public procurement sector to support growth and employment the EC also identified~\cite{siemensEval} 
both regulatory and natural barriers in the access to public procurement in the EU context, especially for SMEs~\cite{d2008}. 
According to this evaluation, a real European Single Market~\cite{d2011} has not yet been achieved causing losses, 
being cost-inefficient, missing opportunities for society and leading to a situation where more than 27 national markets 
co-exist instead of an EU-wide public market~\cite{monti2010}. In this sense, one relevant action to ease the interconnectivity and interoperability in this landscape was the creation of the Tenders Electronic Daily~\cite{eNotices,formsTed} (TED) 
by the EC. It is the on-line version of the ``Supplement to the Official Journal of the European Union'', dedicated to European public procurement notices 
($1500$ new announcements every day). Taking into account that the type of contract and the geographical information are two of the main variables 
that serve to be aware of new business opportunities, the EU has established the use of the ``Common Procurement Vocabulary'' (hereafter CVP refers to CPV 2008) 
and the ``Nomenclature of territorial units for statistics'' (NUTS) as mandatory to annotate public procurement notices. Nevertheless depending on the country and the 
scope of use distinct classifications have been also created. In the specific case of e-Procurement, the ``United Nations Standard Products and Services Code'' (UNSPSC) in Australia, the ``North American Industry Classification System'' (NAICS) in United States or 
the ``Integrated Tariff of the European Communities'' (TARIC) in the EU are examples of similar efforts to unify 
and model procurement-related data. As a consequence a real, standardized and integrated environment for e-Procurement (and business) data 
to encourage the creation of knowledge-based services cannot be easily deployed. That is why some of the potential advantages 
outlined by the EC~\cite{d2010} such as increased accessibility and transparency, benefits for individual procedures, 
benefits in terms of more efficient procurement administration and potential for integration of EU procurement markets and systems cannot be reached in a 
short-term. 

Additionally public bodies are also defining knowledge organization systems (KOS), such as the classifications in the Eurostat's metadata server (RAMON), 
to enable users to annotate information objects providing an agile mechanism for performing tasks such as exploration, searching, 
automatic classification or reasoning. The structure and features of these systems are usually very heterogeneous, 
although some common aspects can be found in all of them: hierarchical relationships between terms or concepts and multilingual character of the information. 
However the lack of mappings among them makes almost impossible a proper reuse of these systems out of their scope. As motivating and 
explanatory examples of the necessity of linking knowledge organization systems in the e-Procurement context, 
four common situations are presented below:

\begin{itemize}
 \item An American SME that usually manages the NAICS vocabulary for being aware of public procurement 
opportunities in United States is looking for new business opportunities in Europe. Although they perfectly 
know that the CPV is used in European public procurement notices they are not able to retrieve 
any announcement due to the fact there is not mapping between the CPV and NAICS classification systems.
\item A Spanish family-owned company is also looking for new business opportunities in Bulgaria. They have been informed 
that the national government is requiring providers for low-value procurement opportunities that match their services but they 
are finding three main issues: 1) Bulgarian civil servants annotate these procurement opportunities with different codes as Spanish ones do; 2) 
these notices are not published in TED (because of their value) and 3) the language is a simple barrier they cannot overcome.
\item A department of statistics in some international organism, such as the WorldBank, is gathering data and information 
from different public institutions and scopes to compare some economical indicators. Due to the fact that data and information are managed 
using distinct knowledge organization systems, a simple task of integrating this information to create a report is becoming a major challenge.
\item A company has implemented a commercial alert service that enables the possibility of getting latest public 
procurement opportunities according to a customer profile. Although business users perfectly know how contracts are described using 
some classification scheme, sometimes they are not able to translate a client query to contract descriptors. If a 
service could suggest these descriptors exploiting some kind of knowledge-base, business users decisions would be supported by a tool, 
the alert service would be more accurate and customer satisfaction would be increased.
\end{itemize}

On the other hand and following the principles of the Open Data initiative, the vice-president Neelie Kroes is 
leading the Open Data Strategy~\cite{d2003} for Europe. The strategy is strongly focused on the commercial 
value of the reuse of Public Sector Information (PSI). In this case e-Procurement data is considered to be 
a key-enabler of the new data-based economy enabling a greater transparency; delivers more efficient public services; and encourages 
greater public and commercial use. Therefore an enriched version of public procurement data to ease the access 
to business opportunities can be the first action to encourage SME participation and create a real and competitive 
public market of procurement opportunities. 


Moreover, the emerging Web of Data and the sheer mass of information now available make it possible the deployment of new 
added-value services and applications based on the reuse of existing vocabularies and datasets. 
The popular diagram of the Linked Data Cloud~\cite{linked-data-cloud}, generated from metadata extracted from the 
Comprehensive Knowledge Archive Network (CKAN) out, contains $336$ datasets, with more than 25 billion RDF triples and 395 million links. 
With regards to e-Procurement, a new group of 130 CKAN datasets have been released from the ``OpenSpending.org'' site. In this realm, 
data coming from different sources and domains have been promoted following the principles of the 
Linked Data initiative~\cite{Berners-Lee-2006} to improve the access to large documental databases, 
e.g. e-Government resources, scientific publications or e-Health records. In the case of KOS, such as thesauri, taxonomies or classification systems 
are developed by specific communities and institutions in order to organize huge collections of information objects. 
These vocabularies allow users to annotate~\cite{Leukel-standard,Leukel-automating,Leukel-comparative} the information objects and easily retrieve them, 
promoting lightweight reasoning in the Semantic Web. Topic or subject indexing is an easy way to introduce machine-readable metadata for a resource's content 
description. Indeed, Product Scheme Classifications (also known as PSCs), such as the CPV, are a kind of KOS that have been built to solve specific problems 
of interoperability and communication between e-Commerce agents~\cite{FenselOmel2001,Leukel-findings}. In fact PSCs are considered to be a key-enabler of 
the next European e-Procurement domain and other supply chain processes~\cite{DBLP:journals/tcci/Alor-HernandezAJPRMBG10}.

Obviously the public information published by governmental contracting authorities, more specifically PSCs, are a suitable candidate to apply the Linked (Open) Data 
(LOD) principles and semantic web technologies providing a new data environment. The conjunction of these initiatives provides the adequate building blocks for an 
innovative and unified pan-European information system that encourages standardization of key processes and systems. In fact economic operators can 
take advantage of this approach to overcome some technical interoperability issues and make a better reuse of public sector information. According to the 
aforementioned points main contributions of this paper can be summarized as follows:
\begin{itemize}
\item Modeling, promotion and inter-linkage of PSCs, structure and data, following the LOD principles.
\item Publication of all information via a SPARQL endpoint providing a public procurement Linked Data node and, more specifically, a new PSCs catalogue as Linked Data.
\item Exploitation of the PSCS catalogue and an existing dataset of public procurement notices via a contract descriptor recommendation service.
\item Demonstration of the gain (in terms of number of descriptors to retrieve public procurement notices) and the semantic exploitation 
capabilities of the new PSCs catalogue.
\end{itemize}

The remainder of this paper is structured as follows. Section~\ref{sect:related-work} reviews the relevant literature. 
Section~\ref{sect:method} briefly presents the MOLDEAS lifecycle for promoting raw data to the LOD initiative. 
Afterwards next section reviews existing methods to code PSCs as RDF/OWL to finally apply the MOLDEAS lifecycle. 
Next section describes the evaluation of applying the LOD principles to the product 
schemes through two studies that include a description of the sample, the method, results and discussion. Finally, 
the paper ends with a discussion of research findings, limitations and some concluding remarks.
