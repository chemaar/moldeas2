According to the previous section, some relevant works can be found and grouped by the topics covered in this paper:

\begin{itemize}
 \item  The Semantic Web area, coined by Tim Berners-Lee in 2001, has experienced during last years a growing 
 commitment from both academia and industrial areas  with the objective of elevating the meaning of web 
 information resources through a common and shared data model (graphs) and an underlying semantics based 
 on different logic formalisms (ontologies). The Resource Description Framework (RDF), based on a graph model, and the Web Ontology Language (OWL), designed to formalize and model domain knowledge, are the two main \textit{ingredients} to reuse information and data 
 in a knowledge-based realm. Thus data, information and knowledge can be easily shared, exchanged and linked~\cite{Maali_Cyganiak_2011} 
 to other knowledge and databases through the use URIs, more specifically HTTP-URIs. Therefore the broad objective of this effort can be summarized 
 as a new environment of added-value information and services that can encourage and improve B2B (Business to Business), B2C (Business to Client) or 
 A2A (Administration to Administration) relationships. The implementation of new context-awareness expert systems to tackle existing 
 cross-domain problems such as medical reasoning, analysis of social media, etc. in which data heterogeneities, 
 lack of standard knowledge representation and interoperability problems are common factors that the use of semantics can improve. As a practical view of the Semantic Web, 
 the Linked Data~\cite{Berners-Lee-2006,Heath_Bizer_2011} initiative emerges to create a large and distributed database on the Web. 
 In order to reach this major objective the publication of information and data under a common data model (RDF) 
 with a specific formal query language (SPARQL~\cite{Sparql11}) provides the required building blocks to turn the Web of documents 
 into a real database or ``Web of Data''. Research works are focused in two main areas: 1) production/publishing~\cite{bizer07how} and 2) consumption of 
 Linked Data. In the first case data quality~\cite{bizer2007,Bizer2009QA,lodq,link-qa}, conformance~\cite{DBLP:journals/ws/HoganUHCPD12}, 
 provenance~\cite{w3c-prov,DBLP:conf/ipaw/HartigZ10}, trust~\cite{Carroll05namedgraphs}, description of 
 datasets~\cite{void,Cyganiak08semanticsitemaps,ckanValidator} and entity reconciliation~\cite{Serimi,Maali_Cyganiak_2011} issues are 
 becoming major objectives since a mass of amount data is already available through RDF repositories and SPARQL endpoints. 
 
 On the other hand, consumption of Linked Data is being addressed to provide new ways of data 
 visualization~\cite{DBLP:journals/semweb/DadzieR11,hoga-etal-2011-swse-JWS}, faceted browsing~\cite{Pietriga06fresnel,citeulike:8529753}, 
 searching~\cite{hoga-etal-2011-swse-JWS} and data exploitation~\cite{Harth:2011:SIP:1963192.1963318}. Some approaches 
 based on sensors~\cite{Jeung:2010:EMM:1850003.1850235,ontology-search}, distributed queries\cite{Hartig09executingsparql,Ankolekar07thetwo,sparqlOpt}, 
 scalable reasoning processes~\cite{DBLP:journals/ws/UrbaniKMHB12,DBLP:journals/ws/BonattiHPS11}, 
 annotation of web pages~\cite{rdfa-primer} or information retrieval~\cite{Pound} are key-enablers for easing the access 
 to information and data.
  
 \item In the particular case of e-Procurement there are projects trying to exploit the 
 information of public procurement notices like the ``Linked Open Tenders Electronic Daily'' project~\cite{loted} 
 where they use the RSS feeds of TED. In the European project LOD2~\cite{lod2-project}, there is a specific workpackage, 
 WP9a ``LOD2 for A Distributed Marketplace for Public Sector Contracts'', to explore and demonstrate the 
 application of linked data principles for procuring contracts in the public sector and, 
 the Media Lab research group at the Technical University of Athens has recently published the 
 ``PublicSpending.net''~\cite{publicspending} portal to visualize and manage statistics about public spending around the world. 
 Finally the ``OpenSpending.org''~\cite{open-spending} portal also presents some specifications to model public procurement data and 
 visualize where the money goes. In general, the ``LOD Around-the-Clock''~\cite{latc-project} (LATC) and the PlanetData~\cite{planet-data-project} 
 projects are also increasing the awareness of LOD across Europe delivering specific research and dissemination activities such as the 
 ``European Data Forum''. Furthermore legal aspects of public sector information are being reviewed in the 
 LAPSI project~\cite{lapsi-project}. In the case of vocabularies and datasets, GoodRelations and ProductOntology are two of 
 the most prominent approaches for tagging products and services using semantic web technologies, 
 for instance Renault UK uses GoodRelations in RDFa in their UK merchandise store. Although these previous efforts to apply Semantic Web 
 principles to procurement data, a system to link existing PSCs to enable a better consumption of 
 public procurement notices and interoperability among e-Procurement systems is still missing 

\end{itemize}

Since an overview of semantic technologies and Linked Data and their application to e-Procurement has been presented 
some remarks and discussion about related works are outlined below.
 
 \begin{enumerate}
  \item  On the one hand, entity reconciliation is becoming a major challenge in the Linked Data community due to its relevance 
 to enrich data with existing datasets. Existing techniques are mainly based on natural language processing (NLP) algorithms 
 that perform some kind of string comparison (labels~\cite{Serimi} or URIs~\cite{Maali_Cyganiak_2011}) to establish a similarity 
 value between two RDF resources under a particular threshold of confidence. The Silk framework~\cite{DBLP:conf/semweb/JentzschIB10} 
 provides an API for discovering relationships between data items within different Linked Data sources. In this sense 
 the DBPedia Spotlight~\cite{DBLP:conf/i-semantics/MendesJGB11} gives also a ``tool for automatically annotating mentions of DBpedia resources in text,
 a solution for linking unstructured information sources to the Linked Open Data cloud through DBpedia''. Other approaches 
 coming from the ontology mapping and alignment areas try to create links according to the structure (relationships) and naming convention 
 of the RDF resources. Finally some machine learning techniques such as genetic programming~\cite{DBLP:conf/semweb/IseleB11} are emerging 
 to learn linkage rules from existing datasets. As conclusion, the inter-linkage of RDF resources is consider 
 to be a key-enabler for an enriched data consumption. Nevertheless the main drawback of these approaches lies in the necessity of human validation 
 to ensure the quality of the link. Furthermore these tools have been designed with a general purpose (parameters such as stopwords cannot 
 be easily configured) and based on the assumption that resources are already available in RDF. That is why it is necessary 
 to provide a custom PSC reconciliation service that takes into account the specific characteristics of PSCs descriptors.
 
 \item On the other hand there is an increasing interest in the creation of methodologies, 
 best practices/recipes~\cite{best-gld,linked-data-cookbook} and lifecycles~\cite{gld-lifecycle,lod2-stack} in the Linked Data community. 
 In this sense, some Linked Data design considerations can be found in~\cite{bizer07how} covering from the design or URIs~\cite{Sauermann+2007a,bernerslee1998uri,uris-uk}, design patterns~\cite{linked-data-patterns}, 
 publication of RDF datasets and vocabularies~\cite{Berr08}, etc. to the establishment of Linked Data profiles~\cite{basic-profile-w3c}. Nevertheless all these guidelines present 
 a tangled environment of aspects with different levels of abstraction that prevent a clear application to 
 a specific problem such as the promotion of PSCs to the Linked Data initiative.
 
 \item Finally information retrieval and recommending techniques have been widely studied to tackle existing problems of searching 
 and filtering information according to an user profile in different domains~\cite{freews}. In the particular 
 case of Spreading Activation~\cite{Collins_Loftus_1975,Scott1981} (SA) it has been demonstrated its application to 
 extract correlations between query terms and documents analyzing user logs~\cite{Cui:2003:QEM:1435677.858986} and to retrieve resources amongst multiple 
 systems~\cite{Schumacher:2008:CFD:1789394.1789447} in which ontologies are used to link and annotate resources. In conjunction with ontologies some works 
 have studied this technique to explore concepts addressing two important issues: 1) the selection and
 2) the ranking of  additional search terms and to measure conceptual similarity~\cite{gouws-vanrooyen-engelbrecht:2010:CCSR}. 
 On the other hand, data mining, more specifically mining socio-semantic networks\cite{Troussov08miningsocio-semantic}, and applications to collaborative filtering 
 (community detection based on tag recommendations, expertise location, etc.) are other potential scenarios to apply the SA theory due to 
 its high performance and scalability. In particular, annotation and tagging~\cite{LabraGayo:2010:WAS:1668126.1668147,DBLP:journals/ijksr/RodriguezGP12} services 
 to gather metadata~\cite{GelgiVD05} from the Web or to predict social annotation~\cite{Chen:2007:PSA:1780653.1780702} and recommending
 systems based on the combination of ontologies and SA~\cite{Gao:2008:SAR:1441425.1441845} are taken advantage of using this technique. Besides 
 semantic search is a relevant area to apply SA following hybrid approaches~\cite{DBLP:conf/jckbse/BerruetaGP06,Rocha:2004:HAS:988672.988723} or user/concept
 query expansion~\cite{Nie:2003:QEQ:767396.767402} combining metadata and user information.

 \end{enumerate}
  
