According to the topics of the paper the relevant works can be divided into the next points:

\begin{itemize}
 \item  The emerging Semantic Web area, coined by Tim Berners-Lee et al.~\cite{Berners-Lee2001} in 2001, 
 has experienced during last years a growing commitment from both academia and industrial areas 
 with the objective of elevating the meaning of web information resources through a common and shared data model (graphs) and 
 an underlying semantics based on different logic formalisms (ontologies) such as Description Logics. 
 In the same way the effort carried out by international standardization organisms such as the W3C has produced and is currently delivering 
 new specifications, vocabularies and models to boost the use of semantic technologies in a broad scope 
 of domains such as e-Government, e-Health, Biomedicine, Education, Bibliography or Geograph to name a few, 
 with the aim of solving existing problems of integration and interoperability among applications and create 
 a proper knowledge environment under Web-based protocolos. The Resource Description Framework (RDF), based on a graph model, 
 and the Web Ontology Language (OWL), designed to formalize and model domain knowledge, 
 are the two main \textit{ingredients} to reuse information and data in a knowledge-based realm. Thus data, information 
 and knowledge can be easily shared, exchanged and linked~\cite{Maali_Cyganiak_2011} to other knowledge and databases 
 through the use URIs, more specifically HTTP-URIs. Therefore the broad objective of this effort can be summarized 
 as a new environment of added-value services that can encourage and improve B2B (Business to Business), B2C (Business to Client) or 
 A2A (Administration to Administration) relationships by means of the implementation of new contex-awareness expert systems to tackle existing 
 cross-domain problems such as medical reasoning~\cite{DBLP:journals/cbm/GonzalezA13,DBLP:journals/eswa/Casado-LumbrerasGAP12}, 
 analysis of social media, etc. in which data heterogeneities, lack of standard knowledge representation 
 and interoperability problems are common factors. 
 
 \item The Linked Data~\cite{Berners-Lee-2006,Heath_Bizer_2011} initiative. This initiative seeks for the application of semantic web concepts to 
 create a large and distributed database on the Web. In order to reach this major objective the publication 
 of information and data under a common data model (RDF) and formats with a specific formal 
 query language (SPARQL~\cite{Sparql11}) provide the required building blocks to turn the Web of documents 
 into a real database of data. Research works are focused in two main areas: 1) production/publishing~\cite{bizer07how} and 2) consumption of 
 Linked Data. In the first case data quality~\cite{bizer2007,wiqa,ld-quality,DBLP:journals/ws/BizerC09,lodq,link-qa}, conformance~\cite{DBLP:journals/ws/HoganUHCPD12}, 
 provenance~\cite{w3c-prov,DBLP:conf/ipaw/HartigZ10}, trust~\cite{Carroll05namedgraphs}, description of datasets~\cite{void,Cyganiak08semanticsitemaps,ckanValidator} and 
 entity reconciliation~\cite{Serimi,Maali_Cyganiak_2011} are becoming major objectives since a mass of amount data is already available~\cite{Triplify} 
 through SPARQL endpoints deployed on the top of RDF repositories such as OpenLink Virtuoso or OWLim. Furthermore and with regards to the promotion of 
 raw data to RDF, there is an increasing interest in the creation of methodologies, best practices/recipes~\cite{best-gld,linked-data-cookbook} and lifecycles~\cite{gld-lifecycle,lod2-stack}. In this sense, some 
 Linked Data design considerations can be found in~\cite{bizer07how} covering from the design or URIs~\cite{Sauermann+2007a,bernerslee1998uri,uris-uk}, design patterns~\cite{linked-data-patterns}, 
 publication of RDF datasets and vocabularies~\cite{Berr08}, etc. to the establishment of Linked Data profiles~\cite{basic-profile-w3c}.
 
 On the other hand, consumption of Linked Data is being addressed to provide new ways of data visualization~\cite{DBLP:journals/semweb/DadzieR11,hoga-etal-2011-swse-JWS}, 
 faceted browsing~\cite{Pietriga06fresnel, citeulike:8529753,Sparallax} and searching~\cite{hoga-etal-2011-swse-JWS}, processing~\cite{Harth:2011:SIP:1963192.1963318} and exploitation of data applying 
 different approaches such as sensors~\cite{Jeung:2010:EMM:1850003.1850235,ontology-search} and techniques such as distributed queries\cite{Hartig09executingsparql,Ankolekar07thetwo,sparqlOpt}, 
 scalable reasoning process~\cite{Urbani2010WebPIE,HoganHarthPolleres2009,DBLP:conf/semweb/HoganPPD10}, 
 annnotation of web pages~\cite{rdfa-primer} or information retrieval~\cite{Pound} to name a few.
  
  FIXME: linking+lifecycles+retrieval
  
 \item In the field of Open Government Data (OGD) there are projects trying to exploit the 
 information of public procurement notices like the ``Linked Open Tenders Electronic Daily'' project~\cite{loted} 
 where they use the RSS feeds of TED.  In the European project LOD2~\cite{lod2-project}, there is a specific workpackage, 
 WP9a ``LOD2 for A Distributed Marketplace for Public Sector Contracts'', to explore and demonstrate the 
 application of linked data principles for procuring contracts in the public sector and, 
 the Media Lab research group at the Technical University of Athens has recently published the 
 ``PublicSpending.net''~\cite{publicspending} portal to visualize and manage statistics about public spending around the world. Finally the 
 ``OpenSpending.org''~\cite{open-spending} portal also presents some specifications to model public procurement data and 
 visualize where the money goes. The ``LOD Around-the-Clock''~\cite{latc-project} (LATC) and PlanetData~\cite{planet-data-project} projects are also increasing the awareness of 
 LOD across Europe delivering specific research and dissemination activities such as the 
 ``European Data Forum''. Furthermore legal aspects of public sector information are being reviewed in the 
 LAPSI project~\cite{lapsi-project}. In the case of vocabularies and datasets, GoodRelations 
 and ProductOntology are two of the most prominent approaches for tagging products and services using semantic web technologies, 
 for instance Renault UK has GoodRelations in RDFa in their UK merchandise store.

\end{itemize}


