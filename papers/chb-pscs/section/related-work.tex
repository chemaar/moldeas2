As part of the LOD effort other similar initiatives have also been deployed such as the Linked Geo Data effort, 
the GeoNames service, the Linked Open Drug Data initiative or the OpenLink Data Spaces network among others. 
More specifically in the field of Open Government Data (OGD) there are projects trying to exploit the 
information of public procurement notices like the ``Linked Open Tenders Electronic Daily'' project~\cite{loted}
where they use the RSS feeds of TED.  In the European project LOD2~\cite{lod2-project}, there is a specific workpackage, 
WP9a ``LOD2 for A Distributed Marketplace for Public Sector Contracts'', to explore and demonstrate the 
application of linked data principles for procuring contracts in the public sector and, 
the Media Lab research group at the Technical University of Athens has recently published the 
``PublicSpending.net''~\cite{publicspending} portal to visualize and manage statistics about public spending around the world. Finally the 
``OpenSpending.org''~\cite{open-spending} portal also presents some specifications to model public procurement data and 
visualize where the money goes. The ``LOD Around-the-Clock''~\cite{latc-project} (LATC) and PlanetData~\cite{planet-data-project} projects are also increasing the awareness of 
LOD across Europe delivering specific research and dissemination activities such as the 
``European Data Forum''. Furthermore legal aspects of public sector information are being reviewed in the 
LAPSI project~\cite{lapsi-project}. In the case of vocabularies and datasets, GoodRelations 
and ProductOntology are two of the most prominent approaches for tagging products and services using semantic web technologies, 
for instance Renault UK has GoodRelations in RDFa in their UK merchandise store.

FIXME: from COMIND survey