In order to evaluate if the approach to generate Linked Data from PSCs can improve the access to public procurement data 
exploiting the advantages of this initiative, an evaluation of the number of links and its application to implement a decision 
support system has been carried out. The first studio check if we can access more public procurement notices when the 
links between a PSC and the CPV are created. The aim of this study is to compare the initial input vocabulary (containing the CPV descriptors) and 
the final input vocabulary (containing the descriptors of the PSCs linked to the CPV). If we extend the input vocabulary we can access to more 
public procurement notices because we have a greater number of descriptors to build SPARQL queries going from any PSC to the CPV. 
The second study tries to exploit the semantic relationships between the CPV concepts to build a recommender of CPV codes for business users. 
In order to make a real comparison we have used $11$ user queries and the translation into CPV codes made by the company Euroalert.net 
(the real user query and the descriptions of the CPV codes are skipped in order to keep the privacy of this company and their clients).
\subsection{Study 1}
\subsection{Research Design}
The purpose of this study is to compare the size of the initial input vocabulary for retrieving public procurement notices 
with regards to the final input vocabulary when links between a PSC and the CPV are created. The exploitation of these links can be used 
to build new SPARQL queries that can take advantage of Linked Data principles to provide a greater input vocabulary. Let's assume the following points about the 
public procurement notices and the PSCs used in this study:
\begin{itemize}
 \item All public procurement notices are annotated by, at least, one CPV element.
 \item Each PSC is a controlled vocabulary comprised of a finite set of elements. The CPV is a controlled vocabulary, $\mathcal{V}$, comprised of $10357$ elements/terms/codes.
 \item The public procurement notices is a dataset, $\mathcal{D}$, comprised of 1M of documents, 
 already available as RDF resources. Each document, $d \in \mathcal{D}$ is tagged, at least, using a CPV code, $v \in \mathcal{V}$. Thus, 
 if a query contains all elements, $v \in \mathcal{V}$, the entire dataset of notices will be retrieved.
\end{itemize}
Once these assumptions are defined, the gain of linking a new finite PSC to CPV can be calculated as follows:
\begin{itemize}
 \item The source controlled vocabulary, $\mathcal{V}_{psc}$, is comprised of \#$\mathcal{V}_{psc}$ elements.
 \item The linking of terms between a source vocabulary, $\mathcal{V}_{psc}$, and  a target vocabulary, $\mathcal{V}$ can be carried out in the next ways:
 \begin{enumerate}
  \item $1-1$ link, there are elements $v^k_{psc} \in \mathcal{V}_{psc}$ that are linked to only one element $v \in \mathcal{V}$.
  \item $1-n$ link, there are elements $v^k_{psc} \in \mathcal{V}_{psc}$ that are linked to some elements $v \in \mathcal{V}$ generating $K$ links.  
  \item The result of the previous operation generates links or pairs in the form $p_k=(v^k_{psc}, v^k)$ building a set of pairs $\mathcal{P}=\{p_1,p_2,...,p_k,p_n\}$. Taking into account this situation, 
  the initial vocabulary $V$ is increased in all elements $v^k_{psc} \in \mathcal{V}_{psc}$ that have a link to an element $v \in \mathcal{V}$. The new 
  derivate vocabulary $V'_{psc}$ is a controlled vocabulary comprising all elements in $\mathcal{V}$ and all $v^k_{psc}: v^k_{psc} \in p_i$.
 \end{enumerate}
 
 \item The percentage of gain in terms of expressivity (number of elements to be used in queries) is related to the number of elements that enables go from $\mathcal{V}_{psc}$ to $\mathcal{V}$:	
 \begin{equation}\label{percentage}
  \%=100*(\frac{\#\mathcal{V}_{psc} + \#\mathcal{V}}{\#\mathcal{V}}-1)
 \end{equation}
 
  As an example of this approach, two controlled vocabularies and a set of links/pairs are presented to calculate the gain in terms of expressivity:
  \begin{itemize}
  \item Let $\mathcal{V} = \{ 1, 2, 3 \}$  and  $\mathcal{V}_{psc} = \{A, B, C, D, E\}$ the source and target controlled vocabularies.
  \item Let $P = \{ (A,1), (B,2), (C,1) (C,2) \}$ the set of links/pairs between $\mathcal{V}$ and $\mathcal{V}_{psc}$.
  \item Therefore, the derivate controlled vocabulary is $\mathcal{V'}_{psc} = \mathcal{V}\,\cup\,\{A, B, C\}$ and the percentage of gain in terms of expressivity 
  according to the Equation~\ref{percentage} is:

  \begin{equation}
      \% = 100 - \{ \langle (3+3) / 3 \rangle -1 \} = 2-1 = 100 
  \end{equation}
      \item As a consequence the number of final terms to create queries is just two times than the initial set, increasing the expressivity in a $100\%$.
  \end{itemize}

 \item Finally, in this study, there is a specific scenario in which elements $v^k_{psc} \in \mathcal{V}_{psc}$ are not directly mapped to elements $v \in \mathcal{V}$ but 
 assuming that there are relations $r_k$  among elements $v^j_{psc}$  and $v^k_{psc}$ new links could emerge between $v^j_{psc}$ and $v$ through $r_k$. Nevertheless this 
 situation should be avoided in order to prevent an infinite and recursive linking process and keep the advantages of using finite controlled vocabularies, 
 e.g. in the ongoing example, Product Ontology (PO) is used as a bridge classification implying an infinite max gain ($\infty$).
 
\end{itemize}

\subsection{Sample}
The PSCs in Table that have been selected to be promoted to RDF are also used to check if we get any gain when a link between a PSC and the CPV is created. 


\begin{longtable}[c]{|p{2.6cm}|p{1.8cm}|p{1.8cm}|p{1.8cm}|p{1.8cm}|p{1.8cm}|p{1.8cm}|} 
\hline
  $\mathcal{V}_{psc}$ & $\#\mathcal{V}_{psc}$  & $\#\mathcal{V'}_{psc}$ &$\#\mathcal{V'''}_{psc}$ &  $\%$ real &  $\%$ real trough PO  &  $\%$ max   \\\hline
\endhead
CPV 2008 	& $10357$  	& $803311$	& $0$	 	& $0$	 	& $10000$	& $96,55$  \\ \hline
CPV 2003 	& $8323$  	& $462$		& $8312$ 	& $4.46$ 	& $80.25$	& $80.36$  \\ \hline
CN 2012  	& $14552$	& $2390$	& $2390$ 	& $23.07$	& $23.07$	& $140.50$  \\ \hline
CPC 2008 	& $4408$	& $4402$   	& $4403$	& $42.50$	& $42.51$ 	& $42.56$  \\ \hline
CPA 2008 	& $5429$	& $5399$   	& $5410$	& $52.12$	& $52.23$	& $52.41$  \\ \hline
ISIC v4  	& $766$		& $765$   	& $765$ 	& $7.38$ 	& $7.38$	& $7.39$    \\ \hline
NAICS 2007 	& $2328$	& $2300$ 	& $2300$	& $22.20$	& $22.20$	& $22.47$  \\ \hline
NAICS 2012 	& $2212$	& $2186$ 	& $2186$	& $21.10$	& $21.10$	& $21.35$  \\ \hline
SITC v4 	& $4017$	& $3811$   	& $3820$	& $36.79$	& $36.88$	& $38.78$  \\ \hline
\multicolumn{7}{|c|}{\textbf{Total}} \\ \hline
$\star$ & $42035$ 		& $21715$   	& $29586$	& $209.66$ 	& $285.66$	& $405.86$ \\ \hline
\hline
\multicolumn{7}{|c|}{\textbf{Linking CPV 2008 and \textit{ProductOntology}}} \\ \hline
\textit{PO}& $\infty$	& $10000$   	& N/A	& $96.55$	& $96.55$ 	& $\infty$  \\ \hline
\multicolumn{7}{|c|}{\textbf{Total including \textit{ProductOntology}}} \\ \hline
$\star$	 & $\infty$	& $31715$   	& 39586	& $306.21$	& $382.21$	& $\infty$ \\ \hline
\hline
\caption{Statistics of promoting to RDF selected PSCs and linking them to the CPV 2008.}\label{ganancia-terminos}\\    
\end{longtable}

\subsection{Results and Discussion}


Additionally, the lifecycle applied to promote and validate the data generated from public procurements notices is based on 
a checklist~\cite{} created by a compilation of $196$ checkpoints which a value can be ``Applicable and positive (Yes)'', ``Applicable and not found (No)'' or ``Not Applicable (N/A)''; 
from books, W3C specifications, Open Data and Linked Data principles, CKAN validator, LOD Cloud requirements, etc., 
and know-how acquired in previous projects that seeks for ensuring the principles of the LOD initiative, 
easing the reuse, maintenance, governance, coverage and expressivity of data. 
In order to compare the new version of a PSCs catalogue to existing public versions of these classifications, 
we have carried out this survey obtaining the next results, see Figure . Finally, this PSCs catalogue has been released to the LOD community, added to the ``datahub.io'' 
register and joined the LOD Cloud Diagram under the dataset id ``\texttt{pscs-catalogue}''.


