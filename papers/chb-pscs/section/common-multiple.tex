Knowledge organizing systems are used for managing large collections of information objects and efficient information retrieval. 
Existing controlled vocabularies are currently available in several formats: PDF, MSExcel, CSV or XML. However promoting them 
to the Semantic Web is not a merely process of RDF/OWL conversions. Transformations need to fulfill some requirements. Firstly, 
a common RDF/OWL representation is required to ensure: a) the semantic compatibility between different vocabularies,
b) the processing of vocabularies in a standard way and c) the sharing of vocabularies for third-parties adoption. In this sense, 
the W3C vocabulary ``Simple Knowledge Organization System'' (SKOS) has been especially designed to fully support 
both structural and lexical features of knowledge organization systems and must be necessarily used in this context. 
Secondly, although controlled vocabularies have some specific non-shared features, in practice a distinction between them is very hard to draw. 
We have identified a minimum set of common features for them. The data model should be expressive enough to preserve as much as 
possible the original semantics of primary sources for these common features. Thirdly, a generic method is needed to ensure quality 
of data conversions to correct SKOS instances. That is why the MOLDEAS lifecycle has been selected to perform the promotion 
of PSCs to RDF taking into account their aforementioned special features and making special emphasis in the next points:
\begin{enumerate}
 \item \textbf{URI Scheme generation.} Controlled structured vocabularies and conceptual resources are interpreted in SKOS as RDF 
 resources, in particular, instances of \texttt{skos:ConceptScheme} and \texttt{skos:Concept}. Thus they are referenced by 
 means of URIs.  Although namespaces are out of the scope of this analysis, one of the tasks is the generation of 
 the \texttt{rdf:ID}s of \texttt{skos:Concept} and \texttt{skos:ConceptScheme} from the original source's information. 
 Usually controlled vocabularies provide unique identifiers for their terms or concepts as follows:
 \begin{itemize}
  \item Generating new identifiers for the concepts of the vocabulary. This option introduces additional management. 
  A mapping between elements of the original source and identifiers should be maintained for updating purposes.
  \item Using the string of the preferred term. We would like to point out here that multilingual sources introduce a 
  factor of complexity that it is not present in monolingual systems. In multilingual sources, this solution 
  implies selecting a preferred term in a given natural language, thus promoting one language over the others with a 
  possible non-desired political impact. In addition, a control procedure has to be established to ensure URI updating if the source term changes.
  \item Using the identifier code of an element. This solution avoids the problem of selecting one preferred language 
  to encode concept URIs. Moreover, codes are usually strings composed of a serial number 
  (legal URI characters) and it preserves the original semantics of a multilingual resource,
  where these codes identify unique terms or concepts and establish mappings between different languages. 
  This last option has been chosen keeping in mind the desired feature of having ``Cool URIs'' to identify RDF resources.
 \end{itemize}

 \item \textbf{Hierarchy formalization.} From our point of view, one of the common aspects shared by 
 KOS is a hierarchy-based structure, at least by thesauri, taxonomies and by most of 
 classification schemes. Hierarchical relations establish links between conceptual resources, 
 showing that the semantics of a resource is, in some way, more general (``broader'') than other (``narrower''). 
 In SKOS, the properties \texttt{skos:broader} and \texttt{skos:narrower} are only used to assert hierarchical 
 statements between two conceptual resources. By the way, these properties are currently not defined 
 as transitive properties. Nevertheless, third-parties, if they consider valuable, can use an OWL reasoner 
 to infer the transitive closure of the hierarchy by means of using both transitive sub properties 
 \texttt{skos:broaderTransitive} and \texttt{skos:narrowerTransitive}. From a theoretical point of view, the transitive closure of hierarchical relations 
 of KOS is still an open issue. Transitive logical formalizations (e.g. using a Description Logics-based language such as SKOS/OWL) 
 of broader/narrower properties have some risks. Cycles can appear in the hierarchical-based structured 
 of controlled vocabularies. Even though, transitive closure of these properties can be useful for 
 search applications to expand original user-queries with terms hierarchically related. In our conversion-method, 
 we have followed the recommendation of the current SKOS specification.
 
 \item \textbf{Multilingual and lexical features.} Regarding controlled vocabularies, 
 multilinguism is a critical issue, especially in European vocabularies such as the CPV. 
 Thesaurus are accessible in $21$ ($23$) official languages of the European Union 
 (Bulgarian, Spanish, Czech, Danish, German, Estonian, Greek, English, French, Italian, Latvian, Lithuanian, 
 Hungarian, Dutch, Polish, Portuguese, Romanian, Slovak, Slovene, Finnish and Swedish) and others such as Croatian. 
 In SKOS conceptual resources are labeled with any number of lexical strings, in any given natural 
 language identified by the \texttt{xml:lang} attribute, following normative RDF/XML syntax. 
 One of these labels is selected as the \texttt{skos:prefLabel} for any given language, and the 
 others as values of \texttt{skos:altLabel}.
 
\end{enumerate}
