This section specifically discusses existing methods to convert knowledge organizations systems distinguishing between RDF/OWL 
conversions methods for thesauri and product classification systems.

\begin{itemize}
 \item \textbf{Thesauri Conversion Methods}. A thesaurus is a controlled vocabulary, with equivalent terms explicitly 
 identified and with ambiguous words or phrases (e.g. homographs) made unique. This set of terms also may include broader-narrower 
 or other relationships. Usually they are considered to be the most complex of controlled vocabularies. Thesauri as a KOS 
 system can be converted by means of different procedures. On one hand, there are methods~\cite{DBLP:conf/jcdl/Soergel05} that propose specific 
 techniques for thesauri conversions into ontology. However the method does not target a specific output format and it considers the 
 hierarchical structure of thesauri as logical is-a relationships. On the other hand, there are some generic methods 
 for thesauri conversions, as the step-wise method defined in~\cite{DBLP:conf/esws/AssemMMS06}. This method selects a common output data model, the 
 SKOS vocabulary, and is comprised of the following sequenced steps: generation of the RDF encoding, error checking and validation 
 and publishing the RDF triples on the Web. In addition, this method has been refined in~\cite{siedl2008} adding three new sub steps 
 for generating the RDF encoding: analyzing the vocabulary, mapping the vocabulary to SKOS properties and classes and building a conversion program. 
 This initial stepwise method can be considered as a previous effort to the abovementioned linked data lifecycles in which all tasks are already defined and these
 steps are embedded as part of the whole lifecycle.
 
 \item \textbf{Product Scheme Classification Conversion Methods.}  Product Scheme Classifications (PSCs) have been developed to 
 organize the marketplace~\cite{Leukel-automating,Leukel-comparative} in distinct vertical sectors that reflect the 
 activity (or some activities) of economy and commerce. They have been built to solve specific problems of 
 interoperability and communication in e-commerce~\cite{Leukel-findings} providing a structural organization 
 of different kind of products collected together by some economic criteria. The aim of a PSC is to be used 
 as a standard~\cite{Leukel-standard} de facto by different agents for information interchange 
 in marketplaces~\cite{FenselOmel2001,FenselDing2001}. Many approaches for product classification systems adaptation to the Semantic Web, 
 like~\cite{Lonsdale:2010:ROL:1743778.1744005}, present methods with the goal to convert them to domain-ontologies. 
 The tree-based structure between product terms is then interpreted as a logical is-a hierarchy. 
 From our point of view and following the discussion in~\cite{Hepp:2007:POR:1256315.1256337,Hepp:2006:SWS:1128590.1128683}, hierarchical 
 links between the elements of each economic sector have not the semantics of subsumption relationships. The next example 
 taken directly from the CPV 2008 shows how the relationship between the element ``Parts and accessories for bicycles'' (34432000-4) 
 and its direct antecessor, ``Bicycles'' (34430000-0), does not seem an is-a relation. In this case, an ontological 
 property for object composition like \texttt{hasPart} would be much better. Moreover, there are further 
 remarks against the idea of using the PSCs as domain-ontologies. It is difficult to assert 
 that the CPV element, ``Bars, rods, wire and profiles used in construction'' (44330000-2), represents 
 any specific product. Rather it should be regarded as a collection of products. 
 To convert correctly this element into a domain ontology concept, it should be considered 
 as equivalent to the union of several concepts (e.g. $Bar \cup Rod \cup Wire \cup Profile$).
 
 Our approach instead will not consider PCSs as domain ontologies, but a specific kind of knowledge organization system. Any PSC is 
 interpreted as a conceptual scheme comprised of conceptual resources. Thus, hierarchical relationships are not 
 considered to be any more logical \texttt{is-a} relations, but \texttt{broader/narrower} ones. 

\end{itemize}
