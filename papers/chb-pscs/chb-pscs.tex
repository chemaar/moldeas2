\documentclass[preprint,12pt]{elsarticle}


%% The `ecrc' package must be called to make the CRC functionality available
%\usepackage{ecrc}

%% set the volume if you know. Otherwise `00'
%\volume{00}

%% set the starting page if not 1
%\firstpage{1}

%% Give the name of the journal
%\journalname{Expert Systems With Applications}

%% Give the author list to appear in the running head
%% Example \runauth{C.V. Radhakrishnan et al.}
%\runauth{}

%% The choice of journal logo is determined by the \jid and \jnltitlelogo commands.
%% A user-supplied logo with the name <\jid>logo.pdf will be inserted if present.
%% e.g. if \jid{yspmi} the system will look for a file yspmilogo.pdf
%% Otherwise the content of \jnltitlelogo will be set between horizontal lines as a default logo

%% Give the abbreviation of the Journal.  Contact the journal editorial office if in any doubt
%\jid{eswa}

%% Give a short journal name for the dummy logo (if needed)
%\jnltitlelogo{ESWA Logo}

%% Provide the copyright line to appear in the abstract
%% Usage:
%   \CopyrightLine[<text-before-year>]{<year>}{<restt-of-the-copyright-text>}
%   \CopyrightLine[Crown copyright]{2011}{Published by Elsevier Ltd.}
%   \CopyrightLine{2011}{Elsevier Ltd. All rights reserved}
%\CopyrightLine{2013}{Published by Elsevier Ltd.}



%\usepackage{llncsdoc}
\usepackage[figuresright]{rotating}
%\usepackage{makeidx}  % allows for indexgeneration
\usepackage{graphicx}
\usepackage[T1]{fontenc}
\usepackage[english]{babel}
\usepackage[utf8]{inputenc}
\usepackage{multirow}

\usepackage{url}
\usepackage{rotating}

%%%Math
\usepackage{latexsym}
% \usepackage{amsmath}
% \usepackage{amssymb}
% \usepackage{amsthm}
%\usepackage{eurosans}

\usepackage{eurosym}

\usepackage{longtable}

\usepackage{listings}

\usepackage{color}
\usepackage{textcomp}


\definecolor{gray}{gray}{0.5}
\definecolor{green}{rgb}{0,0.5,0}


\begin{document}


\begin{frontmatter}

%% Title, authors and addresses

%% use the tnoteref command within \title for footnotes;
%% use the tnotetext command for the associated footnote;
%% use the fnref command within \author or \address for footnotes;
%% use the fntext command for the associated footnote;
%% use the corref command within \author for corresponding author footnotes;
%% use the cortext command for the associated footnote;
%% use the ead command for the email address,
%% and the form \ead[url] for the home page:
%%
%% \title{Title\tnoteref{label1}}
%% \tnotetext[label1]{}
%% \author{Name\corref{cor1}\fnref{label2}}
%% \ead{email address}
%% \ead[url]{home page}
%% \fntext[label2]{}
%% \cortext[cor1]{}
%% \address{Address\fnref{label3}}
%% \fntext[label3]{}

%\dochead{}
%% Use \dochead if there is an article header, e.g. \dochead{Short communication}
%% \dochead can also be used to include a conference title, if directed by the editors
%% e.g. \dochead{17th International Conference on Dynamical Processes in Excited States of Solids}


\title{Empowering the access to public procurement opportunities by means of linking controlled vocabularies.  A case study of Product Scheme Classifications in the European e-Procurement sector.}


%% use optional labels to link authors explicitly to addresses:
\author[label1]{Jose María Alvarez-Rodríguez\corref{cor1}}
\address[label1]{The South East European Research Center, Thessaloniki, Greece.}
\ead{jmalvarez@seerc.org}
\ead[url]{http://www.seerc.org}

\author[label2]{José Emilio Labra-Gayo}
\address[label2]{WESO Research Group, Department of Computer Science, University of Oviedo, 33007, Oviedo, Spain.}
\ead{labra@uniovi.es}

\author[label3]{Patricia Ordoñez De Pablos}
\address[label3]{WESO Research Group, Department of Business Administration, University of Oviedo, 33007, Oviedo, Spain.}
\ead{patriop@uniovi.es}

\author[label4]{Alejandro Rodríguez-González}
\address[label4]{Bioinformatics at Centre for Plant Biotechnology and Genomics UPM-INIA, Polytechnic University of Madrid, Madrid, Spain.}
\ead{alejandro.rodriguezg@upm.es}



\author{}

\address{}

\begin{abstract}
This article introduces a method to promote existing controlled vocabularies to 
the Linked Data initiative. A common data model and an enclosed conversion 
method for knowledge organization systems based on semantic web technologies and 
vocabularies such as SKOS are presented.  This method is applied to well-known 
taxonomies and controlled vocabularies in the business sector, more specifically 
to Product Scheme Classifications created by governmental institutions such as 
the European Union or the United Nations. Once these product schemes are 
available in a common and shared data model, the needs of the European 
e-Procurement sector are outlined to finally demonstrate how Linked Data can 
address some of the challenges for publishing and retrieving information 
resources. As a consequence, two experiments are also provided in order to 
validate the gain, in terms of expressivity, and the goodness of this emerging 
approach to help expert users to make decisions on the selection of descriptors 
for public procurement notices.
\end{abstract}

\begin{keyword}
%% keywords here, in the form: keyword \sep keyword
e-Procurement \sep product scheme classifications \sep  linked open data \sep  semantic web \sep expert systems
%% PACS codes here, in the form: \PACS code \sep code

%% MSC codes here, in the form: \MSC code \sep code
%% or \MSC[2008] code \sep code (2000 is the default)

\end{keyword}


\end{frontmatter}


\section{Introduction}

Government bodies and public institutions as a whole are the most important buyers in the European Union (EU), since public procurement spending represents around 19\% of 
EU Gross Domestic Product (GDP)~\cite{d2010}. Given this situation there is a growing interest and commitment~\cite{d2010a}. to ensure that these funds are 
well managed and most of inefficiencies are eliminated. Electronic public procurement or e-Procurement emerges as an alternative to link and 
integrate inter-organizational business processes and systems with the automation of the requisitioning, the approval purchase order 
management, accounting processes among others through the Internet-based protocol~\cite{Podlogar2007}.  However, there is much more at stake than the mere changeover 
from paper based systems to ones using electronic communications. It should have the potential to yield important improvements in the access to information and d
ata such as the efficiency of individual purchases, the overall administration of public procurement or the functioning of the markets for government contracts. 
The technology in this area may make it possible to automate the processes involved in the public procurement sector. Furthermore, features for supplier 
management and complex auctions should be included in by means of applying new technologies and methods to accomplish the requirements of this new realm of electronic businesses.


The European Commission is trying to unlock this potential, the 2004 Public Procurement Directives 2004/17/EC and 2004/18/EC 
introduced several provisions aimed at enabling e-Procurement uptake in all Member States. In this light of 
modernizing the European public procurement to support growth and employment the EC also identified~\cite{siemensEval} both regulatory and natural barriers in the access to 
public procurement in the EU context, especially for SMEs. According to this evaluation, a real European Single Market use~\cite{d2011} 
has not yet been achieved causing losses, being cost-inefficient, missing opportunities for society and leading to a 
situation where more than 27 national markets co-exist instead of an EU-wide public market~\cite{monti2010}.





\section{Acknowledgements}
% This work is part of the FP7 Marie Curie Initial Training Network ``RELATE'' (cod. 264840) and developed in the context 
% of the Workpackage 4 and more specifically under the project ``Quality Management in Service-based Systems and Cloud Applications''. It is 
% also funded by the ROCAS project with code TIN2011-27871, a research project partially funded by the Spanish Ministry of Science and Innovation.

\clearpage

\bibliographystyle{plain}
% %\bibliographystyle{unsrt}
% % %\bibliographystyle{acm}
\bibliography{references}
 % \renewcommand{\bibname}{References}


\end{document}
