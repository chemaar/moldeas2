\documentclass[preprint,12pt]{elsarticle}


%% The `ecrc' package must be called to make the CRC functionality available
%\usepackage{ecrc}

%% set the volume if you know. Otherwise `00'
%\volume{00}

%% set the starting page if not 1
%\firstpage{1}

%% Give the name of the journal
%\journalname{Expert Systems With Applications}

%% Give the author list to appear in the running head
%% Example \runauth{C.V. Radhakrishnan et al.}
%\runauth{}

%% The choice of journal logo is determined by the \jid and \jnltitlelogo commands.
%% A user-supplied logo with the name <\jid>logo.pdf will be inserted if present.
%% e.g. if \jid{yspmi} the system will look for a file yspmilogo.pdf
%% Otherwise the content of \jnltitlelogo will be set between horizontal lines as a default logo

%% Give the abbreviation of the Journal.  Contact the journal editorial office if in any doubt
%\jid{eswa}

%% Give a short journal name for the dummy logo (if needed)
%\jnltitlelogo{ESWA Logo}

%% Provide the copyright line to appear in the abstract
%% Usage:
%   \CopyrightLine[<text-before-year>]{<year>}{<restt-of-the-copyright-text>}
%   \CopyrightLine[Crown copyright]{2011}{Published by Elsevier Ltd.}
%   \CopyrightLine{2011}{Elsevier Ltd. All rights reserved}
%\CopyrightLine{2013}{Published by Elsevier Ltd.}



%\usepackage{llncsdoc}
\usepackage[figuresright]{rotating}
%\usepackage{makeidx}  % allows for indexgeneration
\usepackage{graphicx}
\usepackage[T1]{fontenc}
\usepackage[english]{babel}
\usepackage[utf8]{inputenc}
\usepackage{multirow}

\usepackage{url}
\usepackage{rotating}

%%%Math
\usepackage{latexsym}
% \usepackage{amsmath}
% \usepackage{amssymb}
% \usepackage{amsthm}
%\usepackage{eurosans}

\usepackage{eurosym}

\usepackage{longtable}

\usepackage{listings}

\usepackage{color}
\usepackage{textcomp}


\definecolor{gray}{gray}{0.5}
\definecolor{green}{rgb}{0,0.5,0}


\begin{document}


\begin{frontmatter}

%% Title, authors and addresses

%% use the tnoteref command within \title for footnotes;
%% use the tnotetext command for the associated footnote;
%% use the fnref command within \author or \address for footnotes;
%% use the fntext command for the associated footnote;
%% use the corref command within \author for corresponding author footnotes;
%% use the cortext command for the associated footnote;
%% use the ead command for the email address,
%% and the form \ead[url] for the home page:
%%
%% \title{Title\tnoteref{label1}}
%% \tnotetext[label1]{}
%% \author{Name\corref{cor1}\fnref{label2}}
%% \ead{email address}
%% \ead[url]{home page}
%% \fntext[label2]{}
%% \cortext[cor1]{}
%% \address{Address\fnref{label3}}
%% \fntext[label3]{}

%\dochead{}
%% Use \dochead if there is an article header, e.g. \dochead{Short communication}
%% \dochead can also be used to include a conference title, if directed by the editors
%% e.g. \dochead{17th International Conference on Dynamical Processes in Excited States of Solids}


\title{Empowering the access to public procurement opportunities by means of linking controlled vocabularies.  A case study of Product Scheme Classifications in the European e-Procurement sector.}


%% use optional labels to link authors explicitly to addresses:
\author[label1]{Jose María Alvarez-Rodríguez\corref{cor1}}
\address[label1]{The South East European Research Center, Thessaloniki, Greece.}
\ead{jmalvarez@seerc.org}
\ead[url]{http://www.seerc.org}

\author[label2]{José Emilio Labra-Gayo}
\address[label2]{WESO Research Group, Department of Computer Science, University of Oviedo, 33007, Oviedo, Spain.}
\ead{labra@uniovi.es}

\author[label3]{Patricia Ordoñez De Pablos}
\address[label3]{WESO Research Group, Department of Business Administration, University of Oviedo, 33007, Oviedo, Spain.}
\ead{patriop@uniovi.es}

\author[label4]{Alejandro Rodríguez-González}
\address[label4]{Bioinformatics at Centre for Plant Biotechnology and Genomics UPM-INIA, Polytechnic University of Madrid, Madrid, Spain.}
\ead{alejandro.rodriguezg@upm.es}



\author{}

\address{}

\begin{abstract}
This article introduces a method to promote existing controlled vocabularies to 
the Linked Data initiative. A common data model and an enclosed conversion 
method for knowledge organization systems based on semantic web technologies and 
vocabularies such as SKOS are presented.  This method is applied to well-known 
taxonomies and controlled vocabularies in the business sector, more specifically 
to Product Scheme Classifications created by governmental institutions such as 
the European Union or the United Nations. Once these product schemes are 
available in a common and shared data model, the needs of the European 
e-Procurement sector are outlined to finally demonstrate how Linked Data can 
address some of the challenges for publishing and retrieving information 
resources. As a consequence, two experiments are also provided in order to 
validate the gain, in terms of expressivity, and the goodness of this emerging 
approach to help expert users to make decisions on the selection of descriptors 
for public procurement notices.
\end{abstract}

\begin{keyword}
%% keywords here, in the form: keyword \sep keyword
e-Procurement \sep product scheme classifications \sep  linked open data \sep  semantic web \sep expert systems
%% PACS codes here, in the form: \PACS code \sep code

%% MSC codes here, in the form: \MSC code \sep code
%% or \MSC[2008] code \sep code (2000 is the default)

\end{keyword}


\end{frontmatter}


\section{Introduction}
Government bodies and public institutions as a whole are the most important buyers in the European Union (EU), since public procurement spending represents around 19\% of 
EU Gross Domestic Product (GDP)~\cite{d2010}. Given this situation there is a growing interest and commitment~\cite{d2010a} to ensure that these funds are 
well managed and most of inefficiencies are eliminated. Electronic public procurement or e-Procurement~\cite{Podlogar2007} emerges as an alternative to link and 
integrate inter-organizational business processes and systems with the automation of the requisitioning, the approval purchase order 
management, accounting processes among others through the Internet-based protocol. However there is much more at stake than the mere changeover 
from paper based systems to ones using electronic communications. It should have the potential to yield important improvements in the access to information and data such as the efficiency of individual purchases, 
the overall administration of public procurement or the functioning of the markets for government contracts. 

In this context the European Commission (EC) is trying to unlock this potential, the 2004 Public Procurement Directives 2004/17/EC and 2004/18/EC 
introduced several provisions and projects~\cite{peppol,e-certis} aimed at enabling e-Procurement uptake in all Member States. In this light of 
modernizing the European public procurement sector to support growth and employment the EC also identified~\cite{siemensEval} 
both regulatory and natural barriers in the access to public procurement in the EU context, especially for SMEs. 
According to this evaluation, a real European Single Market use~\cite{d2011} has not yet been achieved causing losses, 
being cost-inefficient, missing opportunities for society and leading to a situation where more than 27 national markets co-exist instead of an EU-wide public market~\cite{monti2010}. 
In this sense, one relevant action to ease the interconnectivity and interoperability in this landscape was the creation of the Tenders Electronic Daily~\cite{eNotices,formsTed} (TED) 
by the EC. It is the on-line version of the ``Supplement to the Official Journal of the European Union'', dedicated to European public procurement notices 
($1500$ new announcements every day) but a unified pan-European information system dealing with: dispersion of the information, 
duplication of the same notice in more than one source, different publishing formats, problems with regards to a multilingual environment and 
aggregation of low-value procurement opportunities, is still missing. As a consequence some of the potential advantages 
outlined by the EC~\cite{d2010} such as increased accessibility and transparency, benefits for individual procedures, 
benefits in terms of more efficient procurement administration and potential for integration of EU procurement markets cannot be reached in a 
short-term.

Furthermore other EU actions in the e-Government context have been focused on the development of conceptual/terminological maps available in the Eurostat's metadata server (RAMON): 
in the Health field, the ``European Schedule of Occupational Diseases'' or  the ``International Classification of Diseases''; in the Education field,  
thesauri as the ``European Education Thesaurus''; in the Employment field, the ``International Standard Classification of Occupations''; 
in the European Parliament activities the ``Eurovoc Thesaurus'' and in the e-Procurement field the ``Common Procurement Vocabulary'' (CPV). 
The structure and features of these systems are very heterogeneous, although some common aspects can be found in all of them: hierarchical relationships between terms or concepts and multilingual character of the information. 
These knowledge organization systems (KOS) enable users to annotate information objects providing an agile mechanism for performing 
tasks such as exploration, searching, automatic classification or reasoning. Nevertheless depending on the country and the scope 
distinct classifications are commonly used. In the specific case of e-Procurement, the United Nations Standard Products and Services Code (UNSPSC) in Australia, 
the North American Industry Classification System (NAICS) in United States or the CPV and TARIC (Integrated Tariff of the European Communities) in the European Union 
are examples of similar efforts to unify and model procurement-related data. As a consequence a real, standardized and integrated environment 
for e-Procurement (and business) data to encourage the creation of knowledge-based services cannot be easily deployed. In this sense the 
``European Code of Best Practices Facilitating Access by SMEs to Public Procurement Contracts''~\cite{d2008} pointed out 
two groups of difficulties regarding the barriers faced by SMEs in accessing relevant information about public procurement opportunities. 
The document stated that ``the (big) number of such web portals being used by the government and by regional and local authorities makes it difficult 
for tenderers to maintain an overview''. By aggregating data on public procurement in all Member States 
and at all administrative levels in a standardized way and reusing existing efforts of modeling 
and classifying domain knowledge, some open and critical issues can be tackled to improve a situation which affects more 
than \euro $20$ million non-financial companies established in the EU.


On the other hand and following the principles of the Open Data initiative, the vice-president Neelie Kroes is leading the Open Data Strategy for Europe. 
She also outlined in December 2011 several types of actions that will help to unleash potential of data held by governments in Europe. The strategy is 
strongly focused on the commercial value of the re-use of Public Sector Information (PSI), by which the Commission expects to deliver a \euro 40 billion boost to the EU's economy each year. 
A new version of the 2003 Public Sector Information Directive~\cite{d2003} is also expected and the EC is strongly determined to act as a global leader along with United States, 
Canada or Australia. This open data is supposed to enable greater transparency; delivers more efficient public services; and encourages greater 
public and commercial use. More specifically in the context of e-Procurement the access to public contracts announcements is the first natural 
step to encourage SME participation and create a real public market of procurement opportunities.


Moreover, the emerging Web of Data and the sheer mass of information now available make it possible the deployment of new 
added-value services and applications based on the reuse of existing vocabularies and datasets. 
The popular diagram of the Linked Data Cloud~\cite{linked-data-cloud}, generated from metadata extracted from the 
Comprehensive Knowledge Archive Network (CKAN) out, contains $336$ datasets, with more than 25 billion RDF triples and 395 million links. 
With regards to e-Procurement, a new group of 130 CKAN datasets have been released from the ``OpenSpending.org'' site. In this realm, 
data coming from different sources and domains have been promoted following the principles of the 
Linked Data initiative~\cite{Berners-Lee-2006} to improve the access to large documental databases, 
e.g. e-Government resources, scientific publications or e-Health records. In the case of KOS, such as thesauri, taxonomies or classification systems 
are developed by specific communities and institutions in order to organize huge collections of information objects. 
These vocabularies allow users to annotate~\cite{Leukel-standard,Leukel-automating,Leukel-comparative} the information objects and easily retrieve them, 
promoting lightweight reasoning in the Semantic Web. Topic or subject indexing is an easy way to introduce machine-readable metadata for a resource's content 
description. Indeed, Product Scheme Classifications (also known as PSCs), such as the CPV, are a kind of KOS that have been built to solve specific problems 
of interoperability and communication between e-Commerce agents~\cite{FenselOmel2001,Leukel-findings}, in the particular case of PSCs they are considered to be a key-enabler of 
the next European e-Procurement domain and other supply chain processes~\cite{DBLP:journals/tcci/Alor-HernandezAJPRMBG10}.

Obviously the public information published by governmental contracting authorities, more specifically PSCs, are a suitable candidate to apply the Linked (Open) Data 
(LOD) principles and semantic web technologies providing a new environment in which the conjunction of these initiatives can provide the building blocks for an 
innovative unified pan-European information system that encourages standardization of key processes and systems and gives economic operators the tools to overcome 
technical interoperability easing and enhancing the access and reuse of public information. Therefore main contributions of this paper lay in:
\begin{itemize}
\item Modeling, transforming and interlinking the structure and data of PSCs developed by public bodies following the LOD principles.
\item Publishing all information via an SPARQL endpoint providing a public procurement Linked Data node and, more specifically, a new PSCs catalogue as Linked Data.
\item Exploiting the aforementioned catalogue via a contract descriptor recommendation service.
\item Demonstrating the gain (in terms of number of descriptors to retrieve public procurement notices) and the semantic exploitation 
capabilities of the new PSCs catalogue.
\end{itemize}

The remainder of this paper is structured as follows. Section~\ref{sect:related-work} reviews the relevant literature. Next thesauri 
conversion methods are outlined. Section~\ref{sect:method} presents the application of a method for promoting raw data to the LOD initiative 
and its application to the PSCs in the European e-Procurement sector. Section~\ref{sect:evaluation} describes the evaluation of applying the LOD principles to 
these product schemes through two studies that include a description of the sample, the method, results and discussion. Finally, 
the paper ends with a discussion of research findings, limitations and some concluding remarks.

\section{Literature Review}\label{sect:related-work}
According to the previous section, some relevant works can be found and grouped by the topics covered in this paper:

\begin{itemize}
 \item  The Semantic Web area, coined by Tim Berners-Lee in 2001, has experienced during last years a growing 
 commitment from both academia and industrial areas  with the objective of elevating the meaning of web 
 information resources through a common and shared data model (graphs) and an underlying semantics based 
 on different logic formalisms (ontologies). The Resource Description Framework (RDF), based on a graph model, and the Web Ontology Language (OWL), designed to formalize and model domain knowledge, are the two main \textit{ingredients} to reuse information and data 
 in a knowledge-based realm. Thus data, information and knowledge can be easily shared, exchanged and linked~\cite{Maali_Cyganiak_2011} 
 to other knowledge and databases through the use URIs, more specifically HTTP-URIs. Therefore the broad objective of this effort can be summarized 
 as a new environment of added-value information and services that can encourage and improve B2B (Business to Business), B2C (Business to Client) or 
 A2A (Administration to Administration) relationships by means of the implementation of new contex-awareness expert systems to tackle existing 
 cross-domain problems such as medical reasoning, analysis of social media, etc. in which data heterogeneities, 
 lack of standard knowledge representation and interoperability problems are common factors. As a practical view of the Semantic Web, 
 the Linked Data~\cite{Berners-Lee-2006,Heath_Bizer_2011} initiative emerges to create a large and distributed database on the Web. 
 In order to reach this major objective the publication of information and data under a common data model (RDF) 
 with a specific formal query language (SPARQL~\cite{Sparql11}) provides the required building blocks to turn the Web of documents 
 into a real database or ``Web of Data''. Research works are focused in two main areas: 1) production/publishing~\cite{bizer07how} and 2) consumption of 
 Linked Data. In the first case data quality~\cite{bizer2007,wiqa,ld-quality,DBLP:journals/ws/BizerC09,lodq,link-qa}, conformance~\cite{DBLP:journals/ws/HoganUHCPD12}, 
 provenance~\cite{w3c-prov,DBLP:conf/ipaw/HartigZ10}, trust~\cite{Carroll05namedgraphs}, description of 
 datasets~\cite{void,Cyganiak08semanticsitemaps,ckanValidator} and entity reconciliation~\cite{Serimi,Maali_Cyganiak_2011} are 
 becoming major objectives since a mass of amount data is already available through RDF repositories and SPARQL endpoints. 
 
 On the other hand, consumption of Linked Data is being addressed to provide new ways of data 
 visualization~\cite{DBLP:journals/semweb/DadzieR11,hoga-etal-2011-swse-JWS}, faceted browsing~\cite{Pietriga06fresnel, citeulike:8529753,Sparallax}, 
 searching~\cite{hoga-etal-2011-swse-JWS} and data exploitation~\cite{Harth:2011:SIP:1963192.1963318}. Some approaches 
 based on sensors~\cite{Jeung:2010:EMM:1850003.1850235,ontology-search}, distributed queries\cite{Hartig09executingsparql,Ankolekar07thetwo,sparqlOpt}, 
 scalable reasoning processes~\cite{Urbani2010WebPIE,HoganHarthPolleres2009,DBLP:conf/semweb/HoganPPD10}, 
 annnotation of web pages~\cite{rdfa-primer} or information retrieval~\cite{Pound} are key-enablers for easing the access 
 to information and data.
  
 \item In the particular case of e-Procurement there are projects trying to exploit the 
 information of public procurement notices like the ``Linked Open Tenders Electronic Daily'' project~\cite{loted} 
 where they use the RSS feeds of TED. In the European project LOD2~\cite{lod2-project}, there is a specific workpackage, 
 WP9a ``LOD2 for A Distributed Marketplace for Public Sector Contracts'', to explore and demonstrate the 
 application of linked data principles for procuring contracts in the public sector and, 
 the Media Lab research group at the Technical University of Athens has recently published the 
 ``PublicSpending.net''~\cite{publicspending} portal to visualize and manage statistics about public spending around the world. 
 Finally the ``OpenSpending.org''~\cite{open-spending} portal also presents some specifications to model public procurement data and 
 visualize where the money goes. In general, the ``LOD Around-the-Clock''~\cite{latc-project} (LATC) and PlanetData~\cite{planet-data-project} 
 projects are also increasing the awareness of LOD across Europe delivering specific research and dissemination activities such as the 
 ``European Data Forum''. Furthermore legal aspects of public sector information are being reviewed in the 
 LAPSI project~\cite{lapsi-project}. In the case of vocabularies and datasets, GoodRelations and ProductOntology are two of 
 the most prominent approaches for tagging products and services using semantic web technologies, 
 for instance Renault UK has GoodRelations in RDFa in their UK merchandise store. Although these previous efforts to apply Semantic Web 
 principles to procurement data, a system to link existing PSCs to enable a better consumption of 
 public procurement notices and interoperability among e-Procurement systems is still missing 

\end{itemize}

Since an overview of semantic technologies and Linked Data and their application to e-Procurement has been presented 
some remarks and discussion about related works are outlined below.
 
 \begin{enumerate}
  \item  On the one hand, entity reconciliation is becoming a major challenge in the Linked Data community due to its relevance 
 to enrich data with existing datasets. Existing techniques are mainly based on natural language processing (NLP) algorithms 
 that perform some kind of string comparison (labels~\cite{Serimi} or URIs~\cite{Maali_Cyganiak_2011}) to establish a similarity 
 value between two RDF resources under a particular threshold of confidence. The Silk framework~\cite{DBLP:conf/semweb/JentzschIB10} 
 provides an API for discovering relationships between data items within different Linked Data sources. In this sense 
 the DBPedia Spotlight~\cite{DBLP:conf/i-semantics/MendesJGB11} gives a ``tool for automatically annotating mentions of DBpedia resources in text,
 a solution for linking unstructured information sources to the Linked Open Data cloud through DBpedia''. Other approaches 
 coming from the ontology mapping and alignment areas try to create links according to the structure (relationships) and naming convention 
 of the RDF resources. Finally some machine learning techniques such as genetic programming~\cite{DBLP:conf/semweb/IseleB11} are emerging 
 to learn linkage rules from existing datasets. As conclusion, the interlinkage of RDF resources is consider 
 to be a key-enabler for an enriched data consumption. Neverthless the main drawback of these approaches lies in the necessity of human validation 
 to ensure the validaty and quality of the link. Furthermore these tools have been designed with a general purpose (parameters such as stopwords cannot 
 be easily configured) and based on the assumption that resources are already available in RDF. That is why it is necessary 
 to provide a custom PSC reconciliation service that takes into account the specific characteristics of PSCs descriptors.
 
 \item On the other hand there is an increasing interest in the creation of methodologies, 
 best practices/recipes~\cite{best-gld,linked-data-cookbook} and lifecycles~\cite{gld-lifecycle,lod2-stack}. 
 In this sense, some Linked Data design considerations can be found in~\cite{bizer07how} covering from the design or URIs~\cite{Sauermann+2007a,bernerslee1998uri,uris-uk}, design patterns~\cite{linked-data-patterns}, 
 publication of RDF datasets and vocabularies~\cite{Berr08}, etc. to the establishment of Linked Data profiles~\cite{basic-profile-w3c}. Neverthless all these guidelines present 
 a tangled environment of aspects with different levels of abstraction that prevent a clear application to 
 a specific problem such as the promotion of PSCs to the Linked Data initiative.
 
%  \item Finally information retrieval and recommending processes have been widely studied to FIXME. In the case of Spreading Activation~\cite{Collins_Loftus_1975} (SA) it has been applied to the resolution
%   of problems trying to simulate the behavior of the brain using a connectionist method to provide an ``intelligent'' way to retrieve information and data. The use of SA was motivated 
%   due to the research on graph exploration~\cite{Scott1981,AndersonTheory}. Nevertheless the success of this technique is specially relevant to the fields of Document~\cite{turtle91inference} 
%   and Information Retrieval~\cite{Cohen1987}. It has been also demonstrated its application to extract correlations between query terms and documents analyzing user 
%   logs~\cite{Cui03} and to retrieve resources amongst multiple systems~\cite{Schumacher+2008search} 
%   in which ontologies are used to link and annotate resources. In recent years and regarding the emerging use of ontologies in the Semantic Web area new applications of SA have 
%   appeared to explore concepts~\cite{Qiu93,Chen95} addressing the two important issues: 1) the selection and 2) the weighting of 
%   additional search terms and to measure conceptual similarity~\cite{gouws-vanrooyen-engelbrecht:2010:CCSR}. 
%   On the other hand, there are works~\cite{DBLP:journals/cogsr/KatiforiVD10} exploring the application of the SA on ontologies in order to create context inference models. The 
%   semi-automatically extension and refinement of ontologies~\cite{liu_et_al_2005} is other trending topic to apply SA in combination with other techniques based on natural language processing. 
%   Data mining, more specifically mining socio-semantic networks\cite{paper:troussov:2008}, and applications to collaborative filtering (community detection based on tag recommendations, expertise location, etc.) 
%   are other potential scenarios to apply the SA theory due to the high performance and high scalability of the technique. In particular, 
%   annotation and tagging~\cite{labraTagging2007} services to gather meta-data~\cite{GelgiVD05} from the Web or to predict social annotation~\cite{Chen:2007:PSA:1780653.1780702} and recommending 
%   systems based on the combination of ontologies and SA~\cite{citeulike:3779904} are taken advantage of using SA technique. Besides 
%   semantic search~\cite{conf-sofsem-Suchal08} is a highlight area to apply SA following hybrid approaches~\cite{bopaEstonia,RochaSA04} or user query expansion~\cite{767402} combining metadata 
%   and user information.

 \end{enumerate}
  

\section{Existing approaches for PROMOTING controlled vocabularies to RDF/OWL}\label{sect:thesauri}
This section discusses existing methods to convert knowledge organizations systems distinguishing between RDF/OWL 
conversions methods for thesauri and product classification systems.
\begin{itemize}
 \item \textbf{Thesauri Conversion Methods}. A thesaurus is a controlled vocabulary, with equivalent terms explicitly 
 identified and with ambiguous words or phrases (e.g. homographs) made unique. This set of terms also may include broader-narrower 
 or other relationships. Usually they are considered to be the most complex of controlled vocabularies. Thesauri as a KOS 
 system can be converted by means of different procedures. On one hand, there are methods~\cite{DBLP:conf/jcdl/Soergel05} that propose specific 
 techniques for thesauri conversions into ontology. However the method does not target a specific output format and it considers the 
 hierarchical structure of thesauri as logical is-a relationships. On the other hand, there are some generic methods 
 for thesauri conversions, as the step-wise method defined by Miles et al~\cite{DBLP:conf/esws/AssemMMS06}. This method selects a common output data model, the 
 SKOS vocabulary, and is comprised of the following sequenced steps: generation of the RDF encoding, error checking and validation 
 and publishing the RDF triples on the Web. In addition, this method has been refined (FIXME: Polo-Paredes, Álvarez-Rodríguez \& Rubiera-Azcona, 2008) 
 adding three new sub steps for generating the RDF encoding: analyzing the vocabulary, mapping the vocabulary to 
 SKOS properties and classes and building a conversion program. This initial stepwise method can be considered as 
 a previous effort to the abovementioned linked data lifecycles in which all tasks are already defined and these
 steps are embedded as part of the whole lifecycle.
 
 \item \textbf{Product Scheme Classification Conversion Methods.}

\end{itemize}


\section{A Method to promote controlled vocabularies to the Linked Data initiative}\label{sect:method}
\section{Evaluation}\label{sect:evaluation}
\section{Conclusions and Future work}\label{sect:conclusions}
\section{Acknowledgements}
% This work is part of the FP7 Marie Curie Initial Training Network ``RELATE'' (cod. 264840) and developed in the context 
% of the Workpackage 4 and more specifically under the project ``Quality Management in Service-based Systems and Cloud Applications''. It is 
% also funded by the ROCAS project with code TIN2011-27871, a research project partially funded by the Spanish Ministry of Science and Innovation.

\clearpage

\bibliographystyle{plain}
% %\bibliographystyle{unsrt}
% % %\bibliographystyle{acm}
\bibliography{references}
 % \renewcommand{\bibname}{References}


\end{document}
