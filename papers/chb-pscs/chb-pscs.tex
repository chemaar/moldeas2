\documentclass[preprint,12pt]{elsarticle}


%% The `ecrc' package must be called to make the CRC functionality available
%\usepackage{ecrc}

%% set the volume if you know. Otherwise `00'
%\volume{00}

%% set the starting page if not 1
%\firstpage{1}

%% Give the name of the journal
%\journalname{Expert Systems With Applications}

%% Give the author list to appear in the running head
%% Example \runauth{C.V. Radhakrishnan et al.}
%\runauth{}

%% The choice of journal logo is determined by the \jid and \jnltitlelogo commands.
%% A user-supplied logo with the name <\jid>logo.pdf will be inserted if present.
%% e.g. if \jid{yspmi} the system will look for a file yspmilogo.pdf
%% Otherwise the content of \jnltitlelogo will be set between horizontal lines as a default logo

%% Give the abbreviation of the Journal.  Contact the journal editorial office if in any doubt
%\jid{eswa}

%% Give a short journal name for the dummy logo (if needed)
%\jnltitlelogo{ESWA Logo}

%% Provide the copyright line to appear in the abstract
%% Usage:
%   \CopyrightLine[<text-before-year>]{<year>}{<restt-of-the-copyright-text>}
%   \CopyrightLine[Crown copyright]{2011}{Published by Elsevier Ltd.}
%   \CopyrightLine{2011}{Elsevier Ltd. All rights reserved}
%\CopyrightLine{2013}{Published by Elsevier Ltd.}



%\usepackage{llncsdoc}
\usepackage[figuresright]{rotating}
%\usepackage{makeidx}  % allows for indexgeneration
\usepackage{graphicx}
\usepackage[T1]{fontenc}
\usepackage[english]{babel}
\usepackage[utf8]{inputenc}
\usepackage{multirow}

\usepackage{url}
\usepackage{rotating}

%%%Math
\usepackage{latexsym}
% \usepackage{amsmath}
% \usepackage{amssymb}
% \usepackage{amsthm}
%\usepackage{eurosans}

\usepackage{eurosym}

\usepackage{longtable}

\usepackage{listings}

\usepackage{color}
\usepackage{textcomp}


\definecolor{gray}{gray}{0.5}
\definecolor{green}{rgb}{0,0.5,0}


\begin{document}


\begin{frontmatter}

%% Title, authors and addresses

%% use the tnoteref command within \title for footnotes;
%% use the tnotetext command for the associated footnote;
%% use the fnref command within \author or \address for footnotes;
%% use the fntext command for the associated footnote;
%% use the corref command within \author for corresponding author footnotes;
%% use the cortext command for the associated footnote;
%% use the ead command for the email address,
%% and the form \ead[url] for the home page:
%%
%% \title{Title\tnoteref{label1}}
%% \tnotetext[label1]{}
%% \author{Name\corref{cor1}\fnref{label2}}
%% \ead{email address}
%% \ead[url]{home page}
%% \fntext[label2]{}
%% \cortext[cor1]{}
%% \address{Address\fnref{label3}}
%% \fntext[label3]{}

%\dochead{}
%% Use \dochead if there is an article header, e.g. \dochead{Short communication}
%% \dochead can also be used to include a conference title, if directed by the editors
%% e.g. \dochead{17th International Conference on Dynamical Processes in Excited States of Solids}


\title{Empowering the access to public procurement opportunities by means of linking controlled vocabularies.  A case study of Product Scheme Classifications in the European e-Procurement sector.}


%% use optional labels to link authors explicitly to addresses:
\author[label1]{Jose María Alvarez-Rodríguez\corref{cor1}}
\address[label1]{The South East European Research Center, Thessaloniki, Greece.}
\ead{jmalvarez@seerc.org}
\ead[url]{http://www.seerc.org}

\author[label2]{José Emilio Labra-Gayo}
\address[label2]{WESO Research Group, Department of Computer Science, University of Oviedo, 33007, Oviedo, Spain.}
\ead{labra@uniovi.es}

\author[label3]{Patricia Ordoñez De Pablos}
\address[label3]{WESO Research Group, Department of Business Administration, University of Oviedo, 33007, Oviedo, Spain.}
\ead{patriop@uniovi.es}

\author[label4]{Alejandro Rodríguez-González}
\address[label4]{Bioinformatics at Centre for Plant Biotechnology and Genomics UPM-INIA, Polytechnic University of Madrid, Madrid, Spain.}
\ead{alejandro.rodriguezg@upm.es}



\author{}

\address{}

\begin{abstract}
This article introduces a method to promote existing controlled vocabularies to 
the Linked Data initiative. A common data model and an enclosed conversion 
method for knowledge organization systems based on semantic web technologies and 
vocabularies such as SKOS are presented.  This method is applied to well-known 
taxonomies and controlled vocabularies in the business sector, more specifically 
to Product Scheme Classifications created by governmental institutions such as 
the European Union or the United Nations. Once these product schemes are 
available in a common and shared data model, the needs of the European 
e-Procurement sector are outlined to finally demonstrate how Linked Data can 
address some of the challenges for publishing and retrieving information 
resources. As a consequence, two experiments are also provided in order to 
validate the gain, in terms of expressivity, and the goodness of this emerging 
approach to help expert users to make decisions on the selection of descriptors 
for public procurement notices.
\end{abstract}

\begin{keyword}
%% keywords here, in the form: keyword \sep keyword
e-Procurement \sep product scheme classifications \sep  linked open data \sep  semantic web \sep expert systems
%% PACS codes here, in the form: \PACS code \sep code

%% MSC codes here, in the form: \MSC code \sep code
%% or \MSC[2008] code \sep code (2000 is the default)

\end{keyword}

\end{frontmatter}



% \titlerunning{}
% 
% 
% \author{Jose Mar\'{i}a \'{A}lvarez\inst{1} \and Patricia Ordoñez de Pablos~\inst{2} \and Michail Vafolopoulos\inst{3} \and Jos\'{e} Emilio Labra\inst{4}}
% 
% 
% \authorrunning{Jose Mar\'{i}a Alvarez \and Michail Vafolopoulos \and  Jos\'{e} Emilio Labra}
% 
% 
% \institute{South East European Research Center, \\ 54622, Thessaloniki, Greece\\
%   \email{{jmalvarez@seerc.org}}\\
%    \and  WESO Research Group, Department of Business Administration, University of Oviedo \\ 33007, Oviedo, Spain.\\
%     \email{{patriop@uniovi.es}},\\
%   \and  Multimedia Technology Laboratory, National Technical University of Athens, \\ 15773, Athens, Greece.\\
%     \email{{vafopoulos@medialab.ntua.gr}},
% \and WESO Research Group, Department of Computer Science, University of Oviedo, \\ 33007, Oviedo, Spain.\\
%   \email{{labra@uniovi.es.es}}\\  
% }
% 
% 
% \authorrunning{Jose Mar\'{i}a Alvarez-Rodr\'{i}guez}
% 
% \date{}
% 
% \maketitle
% 
% \renewcommand{\labelitemi}{$\bullet$}



\section{Introduction}


\end{document}
