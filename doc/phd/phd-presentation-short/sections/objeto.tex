% \frame{
%   \frametitle{Objeto de la Investigación}
% 
% \begin{block}{Etapas de la Investigación}
% 
%  \begin{enumerate}
%  \item Objeto de Estudio.
% \begin{itemize}
%  \item Formulación del Problema.
%  \item Justificación del Estudio.
%  \item Objetivo General.
%  \item Objetivos Específicos.
% \end{itemize}
%  \item Marco Teórico y Conceptual.
%  \item Marco Metodológico.
%  \item Resultados y Evaluación.
%  \item Conclusiones.
% \end{enumerate}
% \end{block}
% 
% }

\frame{
\large
\begin{enumerate}
 \item Objeto de la Investigación.
  \begin{itemize}
  \item Formulación del Problema.
  \item Justificación del Estudio.
  \item Objetivo General.
  \item Objetivos Específicos.
  \item Hipótesis.
  \end{itemize}

\end{enumerate}

}

\frame{
  \frametitle{Formulación del Problema}

\begin{exampleblock}{Problema}<1->
¿Por qué aplicar métodos semánticos y los principios 
de \linkeddata y \opendata al dominio de las licitaciones públicas?
\end{exampleblock}

\begin{alertblock}{Contexto}<2->
¿Cuáles son las licitaciones públicas sobre «construcción de puentes y carreteras» publicadas en la región de Bélgica cuyo idioma es holandés durante 2011 y cuyo 
importe está entre 100 y 200K euros?
\end{alertblock}

}


\frame{
  \frametitle{Justificación del Estudio}

\begin{alertblock}{Motivación}
\begin{itemize}
\item Profundización en el área de estudio.
\item Corriente de investigación en continua evolución.
\item Investigación e Innovación.
\item Servicios de valor añadido.
\item Sector estratégico en las Administraciones Públicas (AAPP).
\end{itemize}
\end{alertblock}

}

\frame{
  \frametitle{Justificación del Estudio}

\begin{block}{Motivación Científica}
\begin{itemize}
\item Formalización del conocimiento de un dominio.
\item Aplicación de modelos estándar a un dominio.
\item Integración de fuentes de datos.
\item Mejora y consolidación de las técnicas actuales.
\item Generación de conocimiento, nuevos datos y enfoques.
\item Aportaciones a otros investigadores, comunidad.
\end{itemize}
\end{block}

}


\frame{
  \frametitle{Justificación del Estudio}

\begin{exampleblock}{Motivación Tecnológica}
\begin{itemize}
\item Aplicación de nuevas soluciones técnicas.
\item Gestión avanzada de la información y datos.
\item Mejora de los procesos de acceso a la información.
\item Cobertura a las necesidades de la cadena de valor del proceso administrativo.
\item Impulso de un sector trascendente: económico, social, financiero, etc. 
\item Generación de nuevas oportunidades de negocio. 
\end{itemize}
\end{exampleblock}

}

\frame{
  \frametitle{Objetivo General}

\begin{block}{Semántica + \linkeddata+ \opendata}
 Aplicación de la tecnología y métodos semánticos al dominio de la 
contratación pública electrónica para dar soporte a los principios de las 
iniciativas de \linkeddata y \opendata.
\end{block}

}

\frame{
  \frametitle{Objetivos Específicos}

\begin{enumerate}
\item Estudiar, analizar y valorar:
 \begin{itemize}
  \item Contratación Pública y \eproc.
  \item Web Semántica, \linkeddata y \opendata.
 \end{itemize}
\item Definir métodos basados en semántica para gestionar la información de los anuncios de licitación.
\item Implementar y reutilizar los componentes software.%5
\item Promover el uso de estándares, la reutilización de información y los modelos de conocimiento compartido.%6
\item Aplicar los métodos semánticos definidos al contexto de \eproc.%7
\item Establecer un conjunto de prueba y validación.%8
\item Difundir, formar y transferir la tecnología y conocimiento.%9
\end{enumerate}

}

\frame{
  \frametitle{Hipótesis}
\begin{exampleblock}{Hipótesis de la Investigación}
\textbf{Es posible }\textbf{mejorar el acceso} a la \textbf{información} contenida en los \textbf{anuncios de licitación} de 
las distintas instituciones públicas europeas, tanto en términos \textbf{cuantitativos} como \textbf{cualitativos}, \textbf{mediante} \textbf{métodos semánticos} basados 
en \textbf{aplicar} y \textbf{cumplir} los \textbf{principios} de la iniciativa \textbf{\linkeddata} y de la misma forma \textbf{mantener} y \textbf{favorecer}
los principios de la corriente \textbf{Open Data}.
\end{exampleblock}

}


\frame{
  \frametitle{Hipótesis}
\begin{block}{Términos cualitativos}<1->
¿Se puede mejorar el acceso a la información y datos utilizando semántica?
\end{block}
\begin{exampleblock}{Términos cuantitativos}<2->
¿Se puede acceder a un mayor número de anuncios de licitación utilizando semántica?
\end{exampleblock}
\begin{alertblock}{Aplicar, cumplir, mantener y favorecer}<3->
¿Se puede realizar este enfoque cumpliendo con los principios y corrientes actuales?
\end{alertblock}

}


