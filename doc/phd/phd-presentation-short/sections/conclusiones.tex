\frame{
\large
\begin{enumerate}
\setcounter{enumi}{4}
 \item Conclusiones.
  \begin{itemize}
  \item Grado de Cumplimiento de Objetivos.
  \item Consecución Hipótesis de la Investigación.
  \item Principales Aportaciones.
  \item Visión Científica.  
  \item Trabajo Futuro.
  \item Impacto y Difusión.
 \end{itemize}

\end{enumerate}

}

\frame{
  \frametitle{Grado de Cumplimiento de Objetivos}
\begin{block}{Objetivo General}<1->
 Aplicación de la tecnología y métodos semánticos al dominio de la 
contratación pública electrónica para dar soporte a los principios de las 
iniciativas de \linkeddata y \opendata.
\end{block}
\begin{exampleblock}{Se cumple}<2->
\begin{itemize}
 \item \textbf{Definición} de un \textbf{ciclo de vida} para los datos enlazados abiertos.
 \item \textbf{Aplicación} del ciclo de vida al dominio de \textbf{e-Procurement}, componentes de \textbf{MOLDEAS}.
 \item \textbf{Experimentación} y \textbf{validación} de los principios de \linkeddata y \opendata.
\end{itemize}

\end{exampleblock}

}

\frame{
  \frametitle{Grado de Cumplimiento de Objetivos}
\scriptsize
\begin{table}[!htb]
\renewcommand{\arraystretch}{1.3}
\begin{center}
\begin{tabular}{|p{8cm}|p{2cm}|}
\hline
  \textbf{Objetivo}& \textbf{Capítulo} \\ \hline
  \textbf{Estudiar, analizar y valorar}: Contratación Pública y \eproc & 2\\ \hline
  \ldots Web Semántica, \linkeddata y \opendata &  3\\ \hline
  \textbf{Definir métodos} basados en \textbf{semántica} para producir, publicar, consumir y validar información de los anuncios de licitación. & 4\\ \hline
  \textbf{Definir} los \textbf{algoritmos} y \textbf{procesos} para dar soporte a los métodos anteriores. & 5-6\\ \hline
  \textbf{Implementar} y \textbf{reutilizar} los componentes software. & 6 \\ \hline
  \textbf{Promover}: uso de \textbf{estándares}, \textbf{reutilización de información} y modelos de \textbf{conocimiento} \textbf{compartido}. &  Todos \\ \hline
  \textbf{Aplicar} los \textbf{métodos} semánticos definidos al contexto de \textbf{e-Procurement}. & 5 \\ \hline
  \textbf{Establecer} un conjunto de \textbf{prueba} y \textbf{validación}. &  7\\ \hline
  \textbf{Difundir}, formar y transferir la tecnología y conocimiento. & A-B \\ \hline
  \hline
  \end{tabular}
  \end{center}
\end{table} 

}


\frame{
  \frametitle{Consecución Hipótesis de la Investigación}
\begin{block}{Hipótesis de la Investigación}<1->
Es posible \textbf{mejorar el acceso} a la \textbf{información} contenida en los \textbf{anuncios de licitación} de 
las distintas instituciones públicas europeas, tanto en términos \textbf{cuantitativos} como \textbf{cualitativos}, mediante \textbf{métodos semánticos} basados 
en \textbf{aplicar} y \textbf{cumplir} los \textbf{principios} de la iniciativa \textbf{\linkeddata} y de la misma forma \textbf{mantener} y \textbf{favorecer}
los principios de la corriente \textbf{Open Data}.
\end{block}

\begin{exampleblock}{Se cumple}<2->
\begin{itemize}
 \item Resultados del \textbf{experimento} desde un punto de vista \textbf{cuantitativo}.
 \item Resultados del \textbf{experimento} desde un punto de vista \textbf{cualitativo}.
\end{itemize}

\end{exampleblock}

}




\frame{
  \frametitle{Principales Aportaciones}
\begin{enumerate}
 \item \textbf{Repaso} del estado actual de \textbf{\eproc} y aplicación de \textbf{tecnologías} \textbf{semánticas}.
 \item \textbf{Ciclo de vida} para \linkeddata.
 \item Sistema \textbf{MOLDEAS}.
  \begin{itemize}
  \item \textbf{Modelo} de \textbf{información} y \textbf{datos} en \eproc: anuncios, catálogo de clasificaciones, organizaciones, etc.
  \item \textbf{Reconciliación} de entidades.
  \item \textbf{Implementación} de un conjunto de \textbf{componentes} para el \textbf{consumo} y \textbf{explotación} de datos.
  \item Diseño de \textbf{criterios} de \textbf{validación} de los datos generados.
  \item \textbf{Demostración} de la \textbf{mejora} cuantitativa y cualitativa en el \textbf{acceso} a la información y datos.
  \end{itemize}
 \item Generación de \textit{know-how}.
\end{enumerate}

}

\frame{
  \frametitle{Principales Problemas en \eproc}

\begin{exampleblock}{Puntos Clave}
\begin{itemize}
 \item \textbf{Mitigación} de la \textbf{dispersión} de información.
 \item \textbf{Identificación} \textbf{única} de la información y datos.
 \item \textbf{Estandarización} de los modelos y formatos de representación de la información y datos.
 \item \textbf{Soporte} intrínseco al \textbf{multiling\"{u}ismo/multiculturalidad}.
 \item Enfoque \textbf{no intrusivo e integrador} de soluciones existentes.
\end{itemize}
\end{exampleblock}

}


\frame{
  \frametitle{Visión Científica}

\begin{exampleblock}{Puntos Clave}
\begin{itemize}
 \item \textbf{Aumento} del \textbf{vocabulario de entrada} del CPV 2008 con \linkeddata.
 \item \textbf{Mejora} de la \textbf{expresividad} para la realización de consultas en SPARQL.
 \item \textbf{Incremento} del número de \textbf{anuncios} de licitación a los que se puede \textbf{acceder}.
 \item \textbf{Establecimiento} de una \textbf{fórmula} para el cálculo de la \textbf{ganancia} del enlazado de datos.
 \item \textbf{Mejora} \textbf{cualitativa} la información y datos mediante la aplicación intensiva de estándares.
 \item \textbf{Aumento} de la visión global de los datos, \textbf{expresividad} y \textbf{estructuración}. 
\end{itemize}

\end{exampleblock}


}

\frame{
  \frametitle{Visión Científica}

\begin{exampleblock}{Puntos Clave}
\begin{itemize}
  \item \textbf{Incremento} del \textbf{conocimiento} en el dominio de \eproc.
 \item \textbf{Impulso} de la \textbf{reutilización} de la \textbf{información} y datos, mayor poder de redistribución.
 \item \textbf{Minimización} de \textbf{restricciones} tecnológicas.
 \item \textbf{Minimización} de \textbf{aspectos discriminatorios}.
 \item \textbf{Aumento} de la \textbf{transparencia}, \textbf{inclusión} y \textbf{responsabilidad}.
 \item \textbf{Alineación} con las actuales \textbf{propuestas estratégicas} de futuro.
 \item \textbf{Determinación} de aspectos claves en la creación de consultas en SPARQL.
\end{itemize}

\end{exampleblock}


}


% \frame{
%   \frametitle{Visión Científica}
% 
% \begin{exampleblock}{Puntos Clave}
% \begin{itemize}
%  \item La optimización de consultas en SPARQL es clave para la consulta de grandes \datasets.
%  \item La semántica colabora en la creación de sistemas expertos en un dominio determinado.
% % \item La aplicación práctica de semántica pone de manifiesto nuevas necesidades: SPARQL 1.1, OWL2, etc.
% % \item La información de carácter público facilitar la creación de oportunidades de negocio.
% % \item La contratación pública electrónica es un proceso clave en las AAPP.
% \end{itemize}
% 
% \end{exampleblock}
% 
% 
% }


\subsection*{Trabajo Futuro}


\frame{
  \frametitle{Líneas de Investigación}

\begin{block}{Semántica + \linkeddata+ \opendata}
\begin{itemize}
 \item Catalogación de vocabularios y conjuntos de datos de forma precisa.
 \item Mejora de los algoritmos de reconciliación de entidades.
 \item Establecimiento de métricas de calidad.
 \item Mejora del rendimiento de las consultas. Creación de un \textit{benchmark} con datos reales.
 \item Estudio de la aplicación de \lod a otras etapas de \eproc, otros procesos administrativos, etc.
 \item Mejora del sistema de recuperación de información (operadores de agregación).
 \item \ldots
\end{itemize}

\end{block}
}

\frame{
  \frametitle{Líneas de Investigación}

\begin{block}{Semántica + \linkeddata+ \opendata}
\begin{itemize}
 \item Procesamiento de consultas federadas de forma eficiente.
 \item Búsqueda sobre fuentes de datos heterogéneas.
 \item Descubrimiento automático de \datasets.
 \item Gestión de \datasets dinámicos como los provenientes de sensores, sistemas reactivos, etc.
 \item Calidad de los datos: procedencia, valores, etc.
 \item Usabilidad en la interacción con datos enlazados.
\end{itemize}

\end{block}
}

\frame{
  \frametitle{Líneas de Tecnología}

\begin{exampleblock}{Campos de Mejora}
\begin{itemize}
 \item Mejora del sistema de consumo de datos enlazados de MOLDEAS, personalización y prueba intensiva de los algoritmos disponibles.
 \item Mejora del sistema de visualización y consumo de datos enlazados desde el punto de vista del usuario final.
 \item Continuación del desarrollo del sistema de validación de datos enlazados.
 \item Contribución con nuevas herramientas a la comunidad de \linkeddata.
 \item Capitalización del conocimiento y de la propiedad industrial e intelectual. Nuevas oportunidades de negocio y servicios.
  \item \ldots
\end{itemize}
\end{exampleblock}
}



\subsection*{Impacto y Difusión}
\frame{
  \frametitle{Publicaciones derivadas del estudio}

\begin{exampleblock}{Revistas internacionales con índice de impacto}
\begin{enumerate}
\small
 \item Jose María Alvarez Rodríguez, José Emilio Labra Gayo, Francisco Cifuentes Silva, Giner Alor-Hernández, Cauthemoc Sánchez y Jaime
Alberto Guzman Luna. \textbf{Towards a Pan-European E-Procurement platform to Aggregate, Publish and Search Public Procurement Notices 
powered by Linked Open Data: The MOLDEAS Approach}. International Journal of Software Engineering and Knowledge Engineering (IJSEKE). 
2011. \textbf{IF}: $0.262$.
 \item Jose María Alvarez Rodríguez, José Emilio Labra Gayo y Patricia Ordoñez De Pablos. \textbf{Survey of New Trends on \eproc Applying Semantics}.
International Journal of Computers in Industry Focused Topic Issue on New Trends on \eproc Applying Semantics. 2014. \textbf{IF}: $1.620$.
\end{enumerate}
\end{exampleblock}

}

\frame{
  \frametitle{Publicaciones derivadas del estudio}
\begin{block}{Revistas internacionales}
\scriptsize
\begin{enumerate}
\item Jose María Alvarez Rodríguez, José Emilio Labra Gayo y Patricia Ordoñez De Pablos.\textbf{ An Extensible Framework to Sort Out 
Nodes in Graph-based Structures Powered by the Spreading Activation Technique: The ONTOSPREAD approach}. IJKSR. 2011. 
\item Jorge González Lorenzo, José Emilio Labra Gayo y Jose María Alvarez Rodríguez. \textbf{A MapReduce implementation of the Spreading Activation 
algorithm for processing large knowledge bases based on semantic networks}. IJKSR. 
\item Jose María Alvarez Rodríguez, José Emilio Labra Gayo, Ramón Calmeau, Ángel Marín y Jose Luis Marín. \textbf{Query Expansion Methods and Performance Evaluation for Reusing Linking 
Open Data of the European Public Procurement Notices}. Current Topics in Artificial Intelligence. 14th Conference of 
the Spanish Association for Artificial Intelligence, CAEPIA 2011, La Laguna, Spain, November 8-11, 2011, Selected Papers.
\item Jose María Alvarez Rodríguez, José Emilio Labra Gayo, Ramón Calmeau, Ángel Marín y Jose Luis Marín.\textbf{ Innovative Services to ease the Access to the Public Procurement Notices 
using Linking Open Data and Advanced Methods based on Semantics}. International Journal of Electronic Government. 2012.
\end{enumerate}
\end{block}}



\frame{
  \frametitle{Publicaciones derivadas del estudio}
\begin{exampleblock}{Capítulos de libros}
\begin{enumerate}
\item Jose Luis Marín, Mai Rodríguez, Ramón Calmeau, Ángel Marín, Jose María Alvarez Rodríguez y José Emilio Labra Gayo.
 \textbf{Euroalert.net: aggregating public procurement data to deliver commercial services to SMEs}. 
``E-Procurement Management for Successful Electronic Government System''. IGI Global. 2012.

\item Jose María Alvarez Rodríguez, Luis Polo Paredes, Emilio Rubiera Azcona, José Emilio Labra Gayo y Patricia Ordoñez De Pablos.
 \textbf{Enhancing the Access to Public Procurement Notices by Promoting Product Scheme Classifications to the 
Linked Open Data Initiative}. ``Cases on Open-Linked Data and Semantic Web Applications''. IGI Global. 2012
\end{enumerate}
\end{exampleblock}
}


\frame{
  \frametitle{Publicaciones derivadas del estudio}
\begin{block}{Conferencias Internacionales}
\begin{enumerate}
\scriptsize
\item Jose María Alvarez Rodríguez, José Emilio Labra Gayo y Patricia Ordoñez De Pablos.\textbf{ Enhancing the Access to Large Data Sets by means of Linking Controlled Vocabularies}. WSKS 2012. 2012.
\item Jose María Alvarez Rodríguez, José Emilio Labra Gayo y Patricia Ordoñez De Pablos.\textbf{ An Extensible Framework to Sort Out 
Nodes in Graph-based Structures Powered by the Spreading Activation Technique: The ONTOSPREAD approach}. WSKS 2011. 2011.
\item Jorge González Lorenzo, José Emilio Labra Gayo y Jose María Alvarez Rodríguez. \textbf{A MapReduce implementation of the Spreading Activation 
algorithm for processing large knowledge bases based on semantic networks}. WSKS 2011. 2011.
\item Jose María Alvarez Rodríguez, José Emilio Labra Gayo, Ramón Calmeau, Ángel Marín y Jose Luis Marín.\textbf{ Innovative Services to ease the Access to the Public Procurement Notices 
using Linking Open Data and Advanced Methods based on Semantics}. MeTTeG. 2011.
\end{enumerate} 
\end{block}
}


\frame{
  \frametitle{Publicaciones derivadas del estudio}
\begin{block}{Otros}
\begin{enumerate}
\item 3 artículos en \textbf{Workshops} internacionales.
\item 2 posters en eventos internacionales.
\item 4 artículos en \textbf{Workshops} nacionales.
\end{enumerate} 
\end{block}

}


\frame{
  \frametitle{Otras publicaciones relevantes en el ámbito de la Semántica}
\begin{exampleblock}{Revistas internacionales con índice de impacto}
\begin{enumerate}
\scriptsize
\item Alejandro Rodríguez González, Javier Torres-Niño, Gandhi Hernández-Chan, Enrique Jiménez-Domingo y Jose
María Alvarez Rodríguez. \textbf{Using Agents to Parallelize a Medical Reasoning System Based on Ontologies and Description Logics}. International Journal Expert Systems With Applications. 2012. 
\textbf{IF}: $1.926$. Aceptado.
\item Miguel García Rodríguez, Jose María Alvarez Rodríguez, Diego Berrueta Muñoz, Luis Polo Paredes, José Emilio Labra Gayo y 
Patricia Ordoñez De Pablos.\textbf{ Towards a Practical Solution for Data Grounding in a Semantic Web Services Environment. Journal of Universal Computer Science}. 2012. \textbf{IF}: $0.788$. Aceptado.
\item Cristina Casado Lumbreras, Alejandro Rodríguez González, Jose María Alvarez Rodríguez y Ricardo Colomo Palacios. 
\textbf{PsyDis: towards a diagnosis support system for psychological disorders.}. International Journal Expert Systems With Applications. 2012. 
\textbf{IF}: $1.926$. Aceptado.
\end{enumerate} 
\end{exampleblock}

}




