%
% resumen.tex
%

\chapter*{Resumen}
Las Administraciones Públicas son uno de los mayores compradores de la Unión Europea, ya que
sus adquisiciones en conjunto representan un $17$\% del Producto Interior Bruto. La tendencia busca 
la agilización de este gran mercado de oportunidades para facilitar el acceso a las empresas, especialmente a 
las pequeñas y medianas empresas (PYMEs), posibilitando el acceso a la información presente 
en los contratos públicos y permitiendo así la construcción de un entorno de mercado competitivo. 

Desde otro punto de vista, el creciente uso de Internet durante los últimos años ha puesto de manifiesto un
nuevo entorno de ejecución para las aplicaciones, utilizando como nueva plataforma la Web. Nuevas tecnologías y paradigmas
están emergiendo para dar soporte al desarrollo y despliegue de aplicaciones y servicios, así como para la publicación de datos e información. 
En este sentido, iniciativas como la Web Semántica, en concreto el esfuerzo de \linkeddata dentro de la nueva ``Web de Datos'',
 intentan elevar el significado de los elementos y recursos que están disponibles en la web con el objetivo de mejorar 
la integración e interoperabilidad entre aplicaciones y facilitar el acceso a la información y los datos a 
través de modelos y formatos de datos de conocimiento compartido unificados.

En el caso particular del proceso de contratación pública electrónica surge la necesidad de abordar problemas de 
gran calado como la dispersión de la información, la heterogeneidad de los formatos de los anuncios, 
la diversidad de formatos de explotación o el multiling\"{u}ismo y multiculturalidad. Con la meta de suministrar 
una solución a estas necesidades mediante la aplicación de la iniciativa de Web Semántica y el esfuerzo 
propuesto por \linkeddata, se realiza un estudio de una serie de métodos semánticos, materializados a 
través de un ciclo de vida de datos enlazados de carácter general, y su aplicación 
específica al dominio de la contratación pública electrónica.

De esta forma se mejora el acceso a la información y a los datos de los anuncios de licitación, como clave para 
el incremento de las oportunidades de participación en los procesos de contratación pública, favoreciendo la 
publicidad de los anuncios de licitación e impulsando un entorno de datos abiertos, 
estratégico para las Administraciones Públicas tanto por su carácter económico 
como corporativo debido al movimiento \opendata. 

%%%%%%%%%%%%%%%%%%%%%%%%%%%%%%%%%%%%%%%%%%%%%%%%%%%%%%%%%%%%%%%%%%%%%%
\section*{Palabras clave}

contratación pública electrónica, \textit{e-Procurement}, \textit{e-Government}, \textit{linked data}, \textit{open data}, 
\textit{linking open data}, \textit{linking open government data}, Web de Datos, web semántica

\chapter*{Abstract}
Public administrations are one of the largest buyers of the European Union representing the 
 $17$\% of the total GDP. Current trends try to provide a new agile market of opportunities to 
ease the access to public procurement notices to companies, more specifically to Small 
and Medium Companies (SMEs), making the construction of a competitive pan-European market possible.

From other point of view, the emerging use of Internet in the last years has generate a new realm for the 
execution of applications and the deployment of new services using the Web as platform. New technologies 
and development models are arising to deal with the requirements of this new context in which new added-value 
services, publication of data and information are required. In this sense, initiatives such as Semantic 
Web, in particular through the \linkeddata effort within the new Web of Data, seeks for raising the meaning 
of information resources on the Web with the major objective of improving the integration and interoperability 
among applications and easing the access to existing information and data by means of shared common models and formats.

In the e-Procurement context new relevant issues have to be resolved in order to fulfill the needs 
of existing problems with regards to the information dispersion, diverse publishing and exploitation formats or 
the multilingualism and multiculturalism. Taking into account the main goal of delivering a new solution 
to deal with these requirements applying the principles of the Semantic Web and \linkeddata initiatives, an innovative and 
in-depth study to establish a set of semantic methods as part of a \linkeddata life cycle and 
their application to the public contracts domain is carried out.

Thus the access to existing information and data extracted from the notices out is improved as key-enabler 
to increase the opportunity of joining in cross-border public procurement processes, boosting a new environment 
in which new advertising techniques for the notices based on open data are provided and aligning the outcomes 
of this work with the current corporate strategy of public administrations related to open data in one of 
the main economic domains.

%%%%%%%%%%%%%%%%%%%%%%%%%%%%%%%%%%%%%%%%%%%%%%%%%%%%%%%%%%%%%%%%%%%%%%
\section*{Key-words}

\textit{e-Procurement}, \textit{e-Government}, \textit{linked data}, \textit{open data}, \textit{linking open data}, 
\textit{linking open government data}, \textit{Web of Data}, \textit{semantic web}
