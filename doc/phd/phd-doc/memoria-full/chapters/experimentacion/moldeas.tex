\subsection{Diseño del experimento sobre el Sistema MOLDEAS}
Uno de los puntos clave para el triunfo de la iniciativa de \linkeddata consiste 
en asegurar que las aplicaciones pueden consumir los datos para la construcción de servicios 
de valor añadido aprovechando el modelo semántico subyacente. En este sentido, 
en el caso objeto de estudio de este documento y tras el diseño e implementación 
del sistema \gls{MOLDEAS}, en concreto del componente \texttt{moldeas-api} y teniendo 
en cuenta la ejecución del proyecto ``10ders Information Services'' se consideró 
la necesidad de completar el proceso de consumo mediante una el diseño de 
una serie de métodos que suministren servicios de recuperación de información 
sobre las licitaciones.

Inicialmente, este sistema se encuadra dentro de la tecnología existente 
para la recuperación información, pudiendo así considerarse como un sistema 
experto o bien un buscador híbrido de licitaciones. En ambos casos, la casuística 
y la gran cantidad y variedad de información contemplada en los documentos de los anuncios 
de licitación convierten a un sistema de recuperación de estos documentos en realmente 
un objeto de estudio de gran profundidad. Por ello, en el desarrollo de esta tesis 
se ha optado por la realización de un demostrador público que haciendo uso 
de datos enlazados y de las capacidades de las tecnologías semánticas sea capaz de 
abordar los siguientes puntos:
\begin{itemize}
 \item Consumir los datos enlazados desde un lenguaje de programación.
 \item Crear un sistema de recuperación de información.
\end{itemize}

Bajo estos dos grandes objetivos se ha diseñado una primera versión de un 
prototipo experimental prestando especial interés en su diseño para ser extendido 
en siguientes iteraciones.

\subsubsection{Recuperación de Información}
Este término más conocido como \textit{Information Retrieval}~\cite{Manning:2008:IIR:1394399} es el área 
de conocimiento sobre la tecnología para la adquisición, representación, almacenamiento, 
organización y acceso a recursos de información. El objetivo de la misma 
se basa en ofrecer al usuario una serie de métodos para la obtención 
de documentos de su interés de acuerdo a una consulta. Como ejemplo 
de este tipo de sistemas se encuentran los motores de búsqueda sintácticos 
basados en la búsqueda de cadenas de texto coincidentes y que han demostrado 
en la última década su increíble capacidad para localizar y recuperar 
información en los contenidos presentes en la web. El uso de algoritmos 
como \textit{PageRank}~\cite{Page:1998:PCR}, en el caso de Google, para establecer la relevancia 
de un documento combinado con el refinamiento de las técnicas estadísticas 
de búsqueda de texto han aumentado la potencia y precisión de la búsqueda 
en la Web. Por eso, el uso de semántica como potenciador de los sistemas 
de búsqueda no trata de suplantar la búsqueda tradicional sino ofrecer soluciones 
en dominios verticales en los cuales el uso de estas técnicas puede ser mejorado.

En general, en un dominio vertical el espacio de búsqueda y la terminología 
utilizada por un usuario es más complicada por lo que la formulación de las 
consultas con criterios adecuados se convierte en un problema~\cite{Maslov2009} en el momento 
de obtener resultados satisfactorios.

En el caso que nos ocupa de recuperación de anuncios de licitación, la aplicación 
de un algoritmo como \textit{PageRank} es relativamente no aplicable ya que 
no se dispone de enlaces entra los documentos presentes y las relaciones 
entre los anuncios son conceptuales, clasificadas de acuerdo un tipo de 
contrato u otras variables de información como el emisor, la región, etc. La 
propuesta diseñada se basa en la aplicación de semántica para la recuperación 
de los anuncios de licitación combinando tanto técnicas semánticas como otras 
que exploten las capacidades semánticas y los sistemas de recomendación. 

Un anuncio de licitación se puede ver como un documento en el cual existen dos 
tipos de propiedades:
\begin{description}
 \item [Extensionales.] Estas propiedades son independientes del contenido 
del texto y se refieren propiamente a metadatos del documento como el tipo 
de anuncio, el contratante o la localización.
\item [Intensionales.] El contenido propio de un documento, es decir, el  
discurso, conjunto de oraciones que constituyen una unidad de significado
\end{description}

La formalización y la explotación de las propiedades de los documentos 
difiere según el caso. Para el tratamiento de las propiedades extensionales 
se suelen utilizar herramientas del ámbito de la gestión documental y utilizadas 
ampliamente en biblioteconomía y archivística utilizando vocabularios 
controlados para su gestión. En cambio, el caso de las propiedades 
intensionales cambia radicalmente ya que aunque la teoría del análisis 
del discurso ha alcanzado una madurez importante en los últimos años 
todavía su representación semántica es bastante compleja.

En el caso objeto de estudio de este documento y referente a los anuncios 
de licitación queda de manifiesto que las propiedades extensionales permiten 
actualmente dar soporte a los sistemas de recuperación de información. A través 
del concepto de metadato y el uso de un vocabulario controlado como el \gls{CPV} 
es posible clasificar los anuncios de licitación de forma facetada 
de acuerdo a un sistema homogéneo. De esta manera, se obtienen estructuras 
lógicas basadas en conceptos que permiten la recuperación efectiva 
de información. De acuerdo al modelo realizado en el Capítulo~\ref{capitulo:metodos-separados} se observa 
como los metadatos propios de los anuncios de licitación son modelados permitiendo la realización de 
procesos de clasificación y búsqueda. El uso de este tipo propiedades facilita enormemente 
la recuperación de información para este dominio ya que se el esfuerzo realizado 
por facilitar estos procesos compensa y en cierta manera evita el uso de 
propiedades intensionales que en la mayor parte de los casos inducen 
a recuperación de elementos de información no deseados.

\subsubsection{Tareas, Modelos y Métodos de Recuperación de Información}
En general, existen dos grandes tipos de sistemas para el procesamiento 
de elementos de información: los de recuperación de información 
y los sistemas gestores de bases de datos orientadas al manejo 
de datos estructurados~\cite{Chen:1976:EMU:320434.320440}. La diferencia 
entre ambos enfoques reside en que en el primer de los casos el sistema 
intenta obtener la mejor relación entre documentos relevantes recuperados 
y documentos relevantes, mientras que en el segundo caso existe un álgebra 
definido que devuelve todos los resultados de acuerdo al mismo.

Se pueden establecer una serie de tareas en los sistemas de 
recuperación de información:
\begin{itemize}
 \item Recuperación \textit{ad-hoc}. El paradigma sería un conjunto de
documentos fijo sobre el cual se ejecutan diferentes consultas.
\item Categorización o clasificación de documentos. En este caso, 
se trata de proporcionar propiedades intensionales a los documentos 
para su recuperación y filtrado.
\item \textit{Clustering} de documentos. En el caso de que no se puedan 
agrupar documentos mediante propiedades intensionales o vocabularios 
controlados se utilizan distintas técnicas para agrupar documentos 
con características comunes.
\item Segmentación de documentos. Consiste en la división de un documento 
en unidades de información coherentes para ser tratado con una mejor 
eficacia.
\end{itemize}

Evidentemente las tareas de la recuperación de información 
guían hacia los distintos de modelos de representación interna, entre los cuales 
se puede destacar:
\begin{itemize}
 \item \textit{Bag-of-Terms}. Cada documento se representa a través 
de una serie de términos índices, por ejemplo los documentos de una 
biblioteca. Se basa en el principio de composicionalidad, según la cual 
un documento está formado por únicamente por los términos índices y en 
consecuencia se puede asumir que ese documento trata sobre ese tema. 

Este tipo de representación es eficiente pero carece de semántica entre 
los términos salvo que, como en el caso objeto de estudio, el vocabulario 
controlado utilizado ya contenga una semántica definida, en este 
caso una taxonomía con relaciones de especificidad.

\item Términos con pesos. La semántica asociada a un documento 
con un número determinado de términos no es completa en el sentido 
de que no todos los términos índices utilizados tienen el mismo peso. Por ello, 
la evolución natural del modelo anterior consiste en asignar un peso $w_{ij}$ para 
cada término $i$ en un documento $j$.

Existen dos puntos clave para la asignación de pesos en un documento: su frecuencia 
de aparición y su distribución dentro del conjunto de todos los documentos, lo que 
permite realizar dos suposiciones:
\begin{enumerate}
 \item Los términos que más se repiten deben ser considerados como representativos, 
grado de significatividad. En este caso, esta medida se refiere a la frecuencia.
 \item El número de documentos en los que aparece un determinado término implica una medida 
del grado de discriminación, por lo que aquellos que aparecen en muchos documentos 
deberían de alguna forma ser obtener un peso menor. En este segundo caso la medida 
se refiere a la frecuencia inversa.
\end{enumerate}

Si $N$ es el número de documentos total y $n_i$ el número de documentos en los que aparece 
un término $t_i$, se puede definir la frecuencia de un término $t_i$ en un documento 
$j$ como $tf_{ij}$. La frecuencia inversa de un término se calcula como $idf_{i}=log(N/n_{i})$.

Estas fórmulas deben ser tenidas en cuenta en el momento del cálculo de los pesos y en la mayoría 
de los casos se asume que los términos no están correlados con el objetivo de 
que la asignación de pesos sea independiente.

\end{itemize}

Una vez que se ha proporcionado una visión sintética de los modelos 
de representación cabe especificar los modelos de recuperación más 
habituales:

\begin{itemize}
 \item Modelo Booleano. Se trata del modelo de recuperación más sencillo y está basado 
en teoría de conjuntos y álgebra de Bool. El acceso a la recuperación de la información 
en este modelo se basa en la ejecución de una consulta con distintos términos estableciendo 
una operación booleana entre los mismos para la recuperación de los documentos que cumplan 
dicha expresión. El conjunto total de documentos se divide por tanto en relevantes y 
no relevantes. 

Este método se ha utilizado inicialmente debido a su sencillez tanto desde un punto 
de vista formal como de implementación, se ejecuta con una gran eficiencia. El principal 
problema de la expresión de consultas mediante un conjunto de términos y operadores booleanos 
recae en que usuarios no expertos encuentran dificultades para la creación de consultas 
lo que impide rescatar los documentos realmente relevantes para los usuarios. Otra 
de las desventajas de este modelo reside en que no existe un encaje parcial de la consulta 
respecto a los documentos al tratarse de un modelo binario, por lo tanto, un documento 
o cumple con la expresión de la consulta entera o no.
 
\item Modelo Vectorial. Este modelo surge para abordar los problemas encontrados en el modelo 
anterior para la realización de encajes parciales y la asignación de relevancia de un documento 
a una consulta. Para ello, se utiliza una representación tanto de las consultas como los 
documentos mediante un vector en el cual se indican los términos y pesos. Si tanto 
el vector de la consulta como el del documento están próximos, desde un punto de 
vista geométrico, se puede asumir que el documento es relevante.

La mejora de este modelo reside en que no sólo se basa en la creación de dos conjuntos 
como el modelo booleano sino que se basa en la similitud de los términos utilizados 
en la consulta y la definida en los documentos. Con todo ello, el modelo vectorial 
ha obtenido buenos resultados tanto desde un punto de vista formal como en la práctica.
 
\item Modelo Probabilístico. Una vez repasados dos de los grandes modelos de recuperación 
de información basados en conjuntos y en geometría, cabe citar este modelo basado en términos 
de teoría de probabilidades. La teoría que subyace a este modelo consiste en asignar la probabilidad 
de que un documento sea relevante para una determinada consulta.
\end{itemize}

El repaso de los métodos de representación de información y los modelos de recuperación 
permiten establecer una primera visión de la complejidad de un sistema de recuperación 
de información con el cual caracterizar el entorno para la búsqueda de anuncios 
de licitación públicos.

\subsubsection{Búsqueda y Generación de Consultas}
El proceso de búsqueda de documentos tras la definición del método de representación y el 
modelo de recuperación debe además acometer determinadas tareas para el procesamiento 
del lenguaje de natural. 

La principal tarea consiste en la extracción de términos para el indexado. 
La estrategia habitual para llevar a cabo esta tarea se basa en técnicas de transformación de los términos iniciales 
mediante procesamiento del lenguaje natural~\cite{Wilcock:2009:ILA:1717999} con el 
objetivo de eliminar texto superfluo y normalizar los términos realmente interesantes 
para el indexado. La ejecución básica de esta tarea consiste en la eliminación 
de las denominadas \textit{stop-words}, normalización de mayúsculas y minúsculas, 
aplicación de técnicas de \textit{stemming}, etc. La aplicación de estas 
técnicas puede obtener diferentes términos de indexado conllevando un comportamiento 
diferente en el sistema de recuperación de información.

Una vez que el índice de búsqueda se ha generado conteniendo para cada documento 
un conjunto de términos índices obtenidos tras el análisis léxico de los elementos 
presentes en el documento ya se está en disposición de realizar búsquedas y 
recuperar información sobre este índice. De forma habitual el proceso a seguir consiste 
en la introducción de una consulta que es tratada inicialmente mediante el mismo tipo de análisis 
utilizado para la generación de términos de indexado. Dependiendo del modelo 
de recuperación utilizado se obtendrán resultados o documentos con una determinada 
relevancia para la consulta. Como mejora a este proceso se utilizan técnicas de 
expansión de consultas que de forma automática o semi-automática realizan transformaciones 
en la consulta inicial para obtener una consulta enriquecida con nuevos términos. Estas 
técnicas se basan principalmente en el uso de tesauros para la obtención de relaciones de sinonimia, 
antonimia, hiperonimia e hiponimia, etc. para la construcción de una nueva 
consulta, también se suele utilizar la realimentación del usuario 
y el estudio de los registros de las consultas para optimizar la generación 
de consultas extendidos, en ambos casos tesauros como Wordnet~\cite{Gong:Cheang:U:2006} son relevantes 
y existen enfoques híbridos~\cite{Lisa2010}. Finalmente, otro tipo de sistemas de búsqueda realizan 
un enfoque compuesto combinando técnicas de búsqueda sintáctica añadiendo semántica para la expansión 
de conceptos y generación de consultas avanzadas, este tipo de enfoque es considerado como 
búsqueda semántica previa alineación de la consulta con conceptos de una ontología.

En este sentido y en los últimos años con la aparición de las ontologías como elemento 
vertebrador del conocimiento han aparecido técnicas como \textit{Spreading Activation} (implementadas en la biblioteca ONTOSPREAD) 
para la exploración de conceptos~\cite{Qiu93,Chen95} en redes semánticas, 
con el objetivo de facilitar dos funciones: 1) la selección de conceptos y 2) la ponderación de términos 
adicionales de búsqueda de acuerdo a medidas de similaridad~\cite{gouws-vanrooyen-engelbrecht:2010:CCSR}. El objetivo 
final de aplicación de estos enfoques es la construcción de sistemas de búsqueda 
híbridos~\cite{bopaEstonia,RochaSA04,conf-sofsem-Suchal08,767402} basados en semántica, en realidad se pueden alinear 
con los tradicionales enfoques de expansión de consultas comentados anteriormente.

\subsubsection{Medidas de Evaluación}
El éxito de un sistema de recuperación de información vendrá 
determinado por la evaluación realizar para la comprobación del ajuste 
de los documentos recuperados de acuerdo a las expectativas del usuario. En general, 
se pueden evaluar múltiples aspectos: eficiencia, en cuanto a el coste 
espacio-temporal; efectividad, cantidad de documentos relevantes recuperados; 
esfuerzo, en el momento de la construcción de consultas y usabilidad. 

Entre las medidas más importantes se encuentra la efectividad del sistema ya que 
permite establecer la bondad de la recuperación y comparar distintos modelos 
de recuperación de información. Entre este tipo de métricas se estable la siguiente 
Tabla~\ref{tabla:pr}, característica de los documentos recuperados. Este tipo de medidas 
son ampliamente utilizadas en las clasificaciones binarias en las cuales un individuo 
pertenece o no a un conjunto

\begin{table}[!htb]
\renewcommand{\arraystretch}{1.3}
\begin{center}
\begin{tabular}{|l|c|c|}
\hline
  &\multicolumn{2}{|c|}{\textbf{\textit{Actual class (expected)}}} \\ \hline
   \multirow{2}{*}{\textbf{\textit{Predicted class (observation)}}}&tp (\textit{true positive})&fp (\textit{false positive}) \\ 
  &fn (\textit{false negative})&tn (\textit{true negative}) \\ \hline
  \hline
  \end{tabular}
  \caption{Clasificación de resultados de la Recuperación de Información.}
  \label{tabla:pr}
  \end{center}
\end{table} 


\begin{description}
 \item [Precisión.] Capacidad del sistema para recuperar sólo los documentos relevantes de acuerdo 
a una consulta. La intepretación de esta medida será mejor cuanto más cercana a $1$ se encuentre ya 
que indicará que la proporción de todos los documentos o recursos relevantes extraídos de la consulta 
respecto al número total de documentos extraídos.

\begin{center}
$Precision = \frac{tp}{tp+fp} $

$Precision = \frac{D_{relevantes}\cap D_{recuperados}}{D_{recuperados}} $
\end{center}


 \item [\textit{Recall}.] Capacidad del sistema para todos los documentos relevantes 
de acuerdo a una consulta. Dependiendo del contexto de uso también se denomina \textit{Sensivity}, por ejemplo 
en clasificación binaria. La interpretación de esta medida es evidente ya que un valor cercano a $1$ indica 
que se han devuelto todos los documentos relevantes a la consulta por ello se debe combinar con la precisión.

\begin{center}
$Recall = \frac{tp}{tp+fn} $

$Recall = \frac{D_{relevantes}\cap D_{recuperados}}{D_{relevantes}} $
 
\end{center}

Precisión y \textit{recall} también se pueden combinar para la obtención de la medida $F-measure$, calculada 
como la media armónica. 

 \item [\textit{Accuracy}.] Es la proporción de resultados positivos (tanto positivos como negativos) de la población 
observada. Es un valor de medición positivo.

\begin{center}
$Accuracy = \frac{tp+tn}{tp+fp+fn+tn} $

\end{center}



 \item [\textit{Specifity}.] Es la capacidad del sistema para detectar resultados negativos. Por ejemplo, en un sistema 
de diagnóstico médico un valor alto de esta medida implica una alta probabilidad de presencia de la enfermedad 
diagnosticada.

\begin{center}
$Specifity = \frac{tn}{tn+fp} $
 
\end{center}
 
\end{description}

Estas medidas permiten valorar y comparar la bondad de un sistema diagnóstico, un sistema de recuperación de información, etc., 
en los cuales resulta de vital de importancia la mejora constante para cumplir con las expectativas del usuario.

\subsubsection{Caracterización de la Recuperación de Información en Anuncios de Licitación}
La especificación del experimento sobre la recuperación de información en anuncios de licitación se ha abordado 
con los dos grandes objetivos previamente mencionados de consumir los datos enlazados producidos e implementar 
un demostrador público que permita realizar búsquedas sobre la información disponible. La recuperación 
de documentos de los anuncios de licitación se puede abordar desde diferentes puntos de vista, de acuerdo 
a los modelos y métodos de representación y de recuperación en la sección anterior se establece que 
el modelo de representación es similar a \textit{Bag-of-Terms} ya que el contenido completo de los anuncios 
no está disponible y siempre se disponen de códigos \gls{CPV} para etiquetar los mismos, el tipo de propiedades 
utilizadas en este caso serían intensionales y de acuerdo a los metadatos propios de un anuncio de licitación. En cuanto al método de 
recuperación se encontraría entre un híbrido entre el modelo booleano y el vectorial dependiendo del sistema del tipo 
de consulta que se pretenda proveer el usuario.

El sistema de recuperación de información diseñado se basa en un repositorio \gls{RDF} en el cual se almacenan las descripciones 
RDF de cada uno de los anuncios de licitación suministrando a través de un \textit{endpoint} de \gls{SPARQL} el acceso a los 
mismos. 

El experimento diseñado busca realizar las pruebas sobre el componente \texttt{moldeas-api} teniendo en cuenta 
los métodos desarrollados para la expansión de consultas. Las variables de información presentes en los anuncios 
de licitación y su correlación se pueden explotar de muchas formas distintas. No obstante, para este experimento 
se ha seleccionado el tipo de licitación, es decir, los códigos \gls{CPV} de la misma para comprobar los métodos de 
recuperación de información debido principalmente a las siguientes causas:

\begin{itemize}
 \item Es la variable de información cuya presencia está asegurada en todos los anuncios. Otras como el emisor, fechas, etc., también 
se pueden utilizar para la recuperación de información pero su importancia es ciertamente inferior en cuanto a trascendencia respecto 
al tipo de contrato público.
\item Dentro de la ejecución del proyecto ``10ders Information Services'' se han solicitado consultas de los clientes del 
servicio ``Euroalert.net'' y la información proporcionada se basa en los códigos CPV exclusivamente.
\end{itemize}

Sintetizando el experimento que se realizará evaluará los códigos CPV generados (o recuperados) mediante el sistema MOLDEAS en contraposición 
con los suministrados por ``10ders Information Services'' y utilizando las medidas de evaluación Precisión (P), \textit{Recall} (R), \textit{Accuracy} (A) y 
\textit{Especificidad} (S).

Una vez determinadas los puntos clave a tener en cuenta para la evaluación de la recuperación de información es conveniente la caracterización 
propia del experimento que consta de las siguientes etapas:
\begin{enumerate}
 \item Definición de los objetivos del experimento. Las preguntas a responder por el experimento serán:
\begin{itemize}
 \item ¿Es posible implementar un sistema de recuperación de información utilizando datos enlazados?
 \item ¿Es posible explotar las relaciones semánticas establecidas para mejorar la recuperación de información?
 \item ¿Cuál es el mejor enfoque para la recuperación de información en los anuncios de licitación? 
 \item ¿Cómo afectan los resultados en la implementación actual del sistema \gls{MOLDEAS}?
\end{itemize}
 \item Selección de una regla de asignación de las unidades experimentales a las condiciones de estudio. En este caso, 
la unidad experimental de este estudio será un repositorio \gls{RDF} en el cual se encuentran almacenadas 
un $1M$ de anuncios de licitación y las clasificaciones de productos, especialmente el CPV 2008 con $10357$ términos. 
\begin{itemize}
 \item Cualitativos: tipo de entorno hardware y software, códigos CPV generados.
 \item Cuantitativos: tamaño de la muestra y nº de códigos \gls{CPV} iniciales.
\end{itemize}

En este caso y centrándose en el tamaño de la muestra se han obtenido las consultas presentadas en la Tabla~\ref{table:queries-ir}, de las 
cuales se tiene la siguiente información: consulta proporcionada por un cliente, $Q_{str}$, y códigos CPV seleccionados por los expertos 
de dominio de ``Euroalert.net'', $Q_{cpv}$. En este caso y debido al gran número de códigos CPV tan sólo se muestra la cantidad de los mismos esquivando 
el código y la descripción. Realmente y desde un punto de vista de la recuperación de información las medidas de evaluación dependen 
tan sólo del número de códigos que se hayan conseguido generar adecuadamente no así su descripción.

\begin{longtable}[c]{|l|p{10cm}|p{4cm}|} 
\hline
\multirow{2}{*}\textbf{$Q_{i}$} &  \textbf{Consulta de Usuario-$Q_{str}$} &  \textbf{Nº de Códigos CPV relevantes-$\#Q^{i}_{cpv}$} \\\hline
\endhead
$Q_1$ & ``Comprehensive overview over all environmental technologies renewable energy products'' & $463$ \\ \hline
$Q_2$ & ``Tendering of public works: housing, hospitals, roads, housing developments, station drinking water treatment, reforestation'' & $35$ \\ \hline
$Q_3$ & ``Prefabricated buildings'' & $7$ \\ \hline
$Q_4$ & ``Games for children (parks swings, slides, land of play equipment in the public sphere'' & $26$ \\ \hline
$Q_5$ & ``Vital signs monitor'' &  $277$\\ \hline
$Q_6$ & ``Museum and exhibition and product launch services'' & $1$ \\ \hline
$Q_7$ & ``Voltmeters, instruments measuring electrical quantities, Ammeters, Instruments for checking physical characteristics, hygrometers, thermometers, measuring equipment and control, leak detector, Analyzers, 
Cable Splicing insulated cable joints kits, screwdrivers, hand tools , screwdriver'' & $117$ \\ \hline
$Q_8$ & ``Conservation Maintenance of pavements for roads, airfields, bridges, tunnels'' & $13$ \\ \hline
$Q_9$ & ``Wood poles, Wooden sleepers , Lattice towers'' & $10$ \\ \hline
$Q_{10}$ & ``Architectural, construction, engineering and inspection services'' &  $173$\\ \hline
$Q_{11}$ & ``Medical practice and related services'' &  $13$\\ \hline

\hline
\caption{Consultas suministradas en el proyecto ``10ders Information Services''.}\label{table:queries-ir}\\    
\end{longtable}


Respecto a los métodos empleados para la generación de códigos \gls{CPV}, ver Tabla~\ref{table:metodos-ir}, es necesario destacar que uno de los mismos está basado 
en búsqueda sintáctica, $M^1$, y además sirve como primer filtro para la ejecución de los siguientes métodos ya que 
el primer conjunto de códigos CPV se obtiene de esta forma para ser utilizado como parámetro de entrada en los siguientes 
métodos. En el caso del segundo método se intentan explotar las relaciones de la jerarquía del CPV (\texttt{skos:broader}) para 
obtener un nuevo conjunto de salida. De la misma forma se procede en el método $M^3$ pero en este caso se utiliza la 
técnica de \textit{Spreading Activation} a través de la implementación de ONTOSPREAD para obtener el conjunto de salida 
final. Finalmente, se ha utilizado un motor de recomendación para la obtención de códigos CPV similares a los de entrada 
en el método $M^4$, en este caso se supone que el comportamiento difiere de los anteriores métodos ya que la semántica 
de un sistema de búsqueda o de expansión de consulta difiere del comportamiento de un motor de recomendación, en el primer 
caso se busca recuperar exactamente un recurso mientras que en el segundo caso, dado un recurso se buscan similares. Este enfoque, 
también se ha utilizado para interpretar si el sistema de casamiento de consultas y usuarios se puede asociar a un sistema 
de búsqueda a un motor de recomendación.

\begin{longtable}[c]{|l|p{7cm}|p{5cm}|} 
\hline
\textbf{Método} &  \textbf{Descripción} &  \textbf{Tecnología} \\\hline
\endhead
$M^1$ & Se indexan las descripciones de los códigos CPV realizando un proceso de búsqueda sintáctica de las consultas preparadas para así 
obtener un conjunto de códigos CPV resultado. Similar a la reconciliación de entidades llevada a cabo para las clasificaciones de productos. & Apache Lucene y Solr \\ \hline
$M^2$ & Se extraen una serie de códigos CPV candidatos, atendiendo a las relaciones de la jerarquía se obtiene un nuevo 
conjunto de códigos CPV de salida ponderado. & $M^1$ + ponderación jerarquía \\ \hline
$M^3$ & Se extraen una serie de códigos CPV candidatos, atendiendo a las relaciones de la jerarquía y mediante \textit{Spreading Activation} 
conjunto de códigos CPV de salida ponderado. & $M^1$ + ONTOSPREAD \\ \hline
$M^4$ & Se extraen una serie de códigos CPV candidatos, atendiendo al histórico de las relaciones 
entre códigos del $1$ M de anuncios se obtiene con el motor de recomendación un nuevo 
conjunto de códigos CPV de salida ponderado. & $M^1$ + Apache Mahout \\ \hline
\hline
\caption{Métodos de generación de códigos CPV.}\label{table:metodos-ir}\\    
\end{longtable}

 \item Especificación de las medidas de trabajo en cuanto a la respuesta. Para cada uno de los métodos a evaluar se recogen los códigos 
CPV devueltos con el objetivo de calcular los elementos de la Tabla~\ref{tabla:pr} en contraste con los disponibles en las 
consultas preparadas. De esta forma y atendiendo a las características de la recuperación de anuncios de licitación se 
recuperarán tantas licitaciones como códigos coincidentes existan entre la consulta $Q^{i}_{cpv}$ formada por los códigos 
CPV generados por los expertos del dominio y el conjunto $Q^{M^i}_{cpv}$ generado por los respectivos métodos.

 \item Ejecución de un experimento piloto. Con el objetivo de cubrir todo el proceso transversalmente se ha realizado una primera iteración 
con una sola consulta extrayendo los códigos CPV para los distintos métodos y realizando los cálculos \textit{PRAS} tanto para 
la validación de las fórmulas como la comparación con los resultados esperados.

 \item Especificación de un modelo. En este caso, se podría pensar en la realización de un contraste de hipótesis para establecer 
y asegurar que un método mejora o no la generación de códigos CPV, sin embargo y teniendo en cuenta que la muestra de consultas 
es de tan sólo $11$ no es considerada significativa para el uso de distribuciones estadísticas y el tipo de test a practicar 
para este tipo de experimento correspondería a un Test De Wilcoxon en el caso de que las muestras sean apareadas.
 \item Esquematización de los pasos a seguir. La ejecución del experimento sigue los siguientes pasos:
\begin{enumerate}
 \item Para cada una de las consultas del usuario $Q_{str}$ identificadas a través de $Q_{i}$ se aplica 
un método $M_{i}$ de recuperación de códigos CPV.
\item Cada uno de los métodos $M_{i}$ se configura para que devuelva al menos $\#Q^{i}_{cpv}$ elementos para así poder 
comparar conjuntos de cardinalidades iguales. En algunos casos está situación los métodos utilizados no son capaces 
de generar $\#Q^{i}_{cpv}$ elementos debido a la baja representatividad de los últimos códigos generados y al gran número de 
códigos presentes en $Q^{i}_{cpv}$. Se ha desestimado utilizar la media del número de elementos de cada 
conjunto $Q^{i}_{cpv}$ debido a su excesiva divergencia.
\item Cada conjunto resultado $Q^{M^i}_{cpv}$ se compara con el conjunto de resultado esperado $Q^{i}_{cpv}$ a través de un \textit{script} en 
AWK, ver Figura~\ref{figure:extrac-pras}, para obtener las medidas tp, fp, fn y tn (mediante hoja de cálculo).
\item Finalmente con estos valores se generan los valores \textit{PRAS} para cada uno de los métodos $M_{i}$ y cada 
consulta de entrada $Q_{i}$.

\begin{figure}[!ht] 
\begin{center}
\begin{lstlisting}[language=AWK]
awk 'NR==FNR{a[$0];next} $0 in a' $1 $2 > "$1-$2-tp"
TP=` cat "$1-$2-tp" | wc -l`
FP=`awk 'NR==FNR { a[$0]; next } !($0 in a)' "$1" "$2" | wc -l`
FN=`awk 'NR==FNR { a[$0]; next } $0 in a { delete a[$0]; next } 1; END 
  { for (b in a) print b }' "$1" "$1-$2-tp" | wc -l`
echo "TP, FP, FN"
echo $TP "," $FP "," $FN
rm -f "$1-$2-tp"
\end{lstlisting}
\caption{Extracción de valores tp, fp y fn.}
\label{figure:extrac-pras}
\end{center}
\end{figure}

\end{enumerate}

 \item Determinación del tamaño muestral. Como ya se ha comentado anteriormente, en el apartado de factores cuantitativos, se utilizarán
 los datos generados durante el proceso de promoción de datos a \linkeddata de los anuncios de licitación, de las clasificaciones de productos y 
de las consultas preparadas suministradas por el servicio de ``Euroalert.net''.
 \item Revisión de las decisiones anteriores.

\end{enumerate}

\subsection{Ejecución del experimento sobre el Sistema MOLDEAS}
La ejecución del plan diseñado para el experimento arroja los resultados que se presentan 
a través de la Tabla~\ref{tabla:results-dss}, en la cual para cada una de las consultas 
de usuario $Q_{str}$ identificadas a través de $Q_{i}$, se ha ejecutado la generación 
de códigos \gls{CPV} a través de los distintos métodos, $M^i$, para obtener un conjunto 
de salida de códigos CPV $Q^{M^i}_{cpv}$ permitiendo establecer las métricas 
de evaluación: Precisión, \textit{Recall}, \textit{Accuracy} y Especificidad.

\begin{sidewaystable}[!htb]
\renewcommand{\arraystretch}{1.3}
\begin{center}
\begin{tabular}{|c||c|c|c|c||c|c|c|c||c|c|c|c||c|c|c|c|}
\hline
 \textbf{$Q_i$}&\multicolumn{4}{|c||}{$M^{1}$} & \multicolumn{4}{|c||}{$M^{2}$}& \multicolumn{4}{|c||}{$M^{3}$} & \multicolumn{4}{|c|}{$M^{4}$} \\ \hline
	  &\textbf{P} & \textbf{R} & \textbf{A} & \textbf{S} &		\textbf{P} & \textbf{R} & \textbf{A} & \textbf{S} &			\textbf{P} & \textbf{R} & \textbf{A} & \textbf{S} &		\textbf{P} & \textbf{R} & \textbf{A} & \textbf{S}  \\ \hline \hline
$Q_1$  	  &$0,15$ & $0,08$ & $0,94$ & $0,98$ &		$0,15$ & $0,15$ & $0,92$ & $0,96$ &	$0,12$ & $0,06$ & $0,94$ & $0,98$ &	$0,06$ & $0,06$ & $0,68$ & $0,81$ \\ \hline
$Q_2$  	  &$0,09$ & $0,09$ & $0,99$ & $1,00$ & 		$0,06$ & $0,06$ & $0,99$ & $1,00$ & 	$0,03$ & $0,03$ & $0,99$ & $1,00$ & 	$0,03$ & $0,03$ & $0,99$ & $1,00$ \\ \hline
$Q_3$  	  &$0,14$ & $0,14$ & $1,00$ & $1,00$ & 		$0,14$ & $0,14$ & $1,00$ & $1,00$ & 	$0,14$ & $0,14$ & $1,00$ & $1,00$ & 	$0,00$ & $0,00$ & $1,00$ & $1,00$ \\ \hline
$Q_4$  	  &$0,19$ & $0,19$ & $1,00$ & $1,00$ &		$0,00$ & $0,00$ & $0,99$ & $1,00$ & 	$0,12$ & $0,12$ & $1,00$ & $1,00$ & 	$0,00$ & $0,00$ & $0,99$ & $1,00$ \\ \hline
$Q_5$  	  &$0,12$ & $0,01$ & $0,97$ & $1,00$ & 		$0,01$ & $0,01$ & $0,95$ & $0,97$ & 	$0,08$ & $0,01$ & $0,97$ & $1,00$ & 	$0,03$ & $0,03$ & $0,95$ & $0,97$ \\ \hline
$Q_6$  	  &$1,00$ & $1,00$ & $1,00$ & $1,00$ & 		$0,00$ & $0,00$ & $1,00$ & $1,00$ & 	$1,00$ & $1,00$ & $1,00$ & $1,00$ & 	$0,10$ & $0,67$ & $0,98$ & $0,98$ \\ \hline
$Q_7$  	  &$0,20$ & $0,20$ & $0,98$ & $0,99$ & 		$0,09$ & $0,09$ & $0,98$ & $0,99$ & 	$0,15$ & $0,16$ & $0,98$ & $0,99$ & 	$0,03$ & $0,03$ & $0,98$ & $0,99$ \\ \hline
$Q_8$  	  &$0,08$ & $0,08$ & $1,00$ & $1,00$ & 		$0,08$ & $0,08$ & $1,00$ & $1,00$ & 	$0,08$ & $0,08$ & $1,00$ & $1,00$ & 	$0,00$ & $0,00$ & $1,00$ & $1,00$ \\ \hline
$Q_9$  	  &$0,50$ & $0,50$ & $1,00$ & $1,00$ & 		$0,00$ & $0,00$ & $1,00$ & $1,00$ & 	$0,30$ & $0,38$ & $1,00$ & $1,00$ & 	$0,00$ & $0,00$ & $1,00$ & $1,00$ \\ \hline
$Q_{10}$  &$0,39$ & $0,39$ & $0,98$ & $0,99$ & 		$0,42$ & $0,42$ & $0,98$ & $0,99$ & 	$0,34$ & $0,35$ & $0,98$ & $0,99$ & 	$0,16$ & $0,16$ & $0,97$ & $0,99$ \\ \hline
$Q_{11}$  &$0,23$ & $0,23$ & $1,00$ & $1,00$ & 		$0,23$ & $0,23$ & $1,00$ & $1,00$ & 	$0,15$ & $0,17$ & $1,00$ & $1,00$ & 	$0,00$ & $0,00$ & $1,00$ & $1,00$ \\ \hline
\multicolumn{17}{|c|}{\textbf{Medias Totales de Métricas PRAS}} \\ \hline
\textbf{Total}  &$0,28$ & $0,26$ & $0,99$ & $1,00$ & 	$0,11$ & $0,11$ & $0,98$ & $0,99$ & 	$0,23$ & $0,23$ & $0,99$ & $1,00$ & 	$0,03$ & $0,03$ & $0,96$ & $0,98$ \\ \hline
\hline
 \end{tabular}
\caption{Resultados PRAS de las consultas suministradas en el proyecto ``10ders Information Services''.}\label{table:queries-ir-results}
  \label{tabla:results-dss}
  \end{center}
\end{sidewaystable} 

%FIXME \cleardoublepage
\subsection{Validación del experimento sobre el Sistema MOLDEAS}
La realización del experimento sobre la recuperación de códigos \gls{CPV} y en consecuencia recuperación 
de información de los contratos públicos es motivado por el objetivo de ayudar al experto del dominio 
a decidir los códigos CPV que se ajustan a las consultas en lenguaje natural de los clientes o 
usuarios. La dificultad de selección de códigos reside en varios puntos que la experiencia ha 
dejado patente:
\begin{itemize}
 \item Extensión del posible conjunto de entrada, $10357$ términos del vocabulario controlado CPV 2008. La necesidad de 
facilitar la expresividad del conjunto de términos de entrada viene determinada para el aumento de las posibilidades 
de expresar consultas y enlazar códigos de forma automática, como se ha presentado en la Sección~\ref{sect:validation-tablas}. Ahora 
bien, un conjunto de entrada más amplio, no implica mayor dificultad para seleccionar los códigos CPV, sino que el punto 
clave consiste en establecer a partir de un código CPV sus posibles códigos relacionados, independientemente 
de la extensión del vocabulario de entrada. En este sentido, el experto del dominio debe conocer cómo se etiquetan 
los anuncios de licitación y estimar las intenciones del usuario ya que en la mayoría de los casos no existe relación 
entre la expresión del usuario (intuición sobre cómo se han etiquetado los anuncios de licitación) y el etiquetado 
real de los anuncios. En este contexto, se pone de manifiesto la intervención del experto y las herramientas 
que faciliten la selección de códigos de forma automática, un sistema en el cual existan más relaciones permite 
establecer conjuntos de códigos CPV más completos.

\item Explotación de las relaciones jerárquicas de la taxonomía del CPV 2008. En general, la búsqueda sintáctica 
tradicional se basa en el encaje de cadenas texto preprocesadas mediante distintas técnicas (\textit{stopwords}, 
\textit{stemming}, \textit{n-grams}, etc.) con el objetivo de facilitar el acceso al mayor número de documentos 
o recursos de acuerdo a una consulta de entrada. Tanto el proceso de indexación de recursos como el preprocesamiento 
de la consulta permite que la comparación se realice en las mismas condiciones. No obstante, la explotación de relaciones 
más allá del lenguaje natural se delega en la propia aplicación antes de proveer la verdadera consulta de entrada. El enriquecimiento 
o expansión de la consulta a través de la navegación por relaciones semánticas presentes en taxonomías o vocabularios 
controlados permite obtener un conjunto de términos estrechamente relacionadas y validados ya que han sido consensuados 
por una comunidad. Este enfoque de expansión en el caso del CPV 2008 se considera perfectamente aplicable ya que no 
es delegado en una taxonomía genérica y externa como puede ser Wordnet, sino que se utilizan las relaciones 
entre los distintos códigos, que como se ha visto en la estructura del CPV 2008 no son necesariamente de 
herencia sino de grado de especificidad. De esta forma, la extensión del conjunto de entrada mediante la navegación 
por la jerarquía de la taxonomía se considera válida para obtener un conjunto enriquecido que pueda ser 
trasladado al sistema de búsqueda final, en este caso a una consulta sobre un motor sintáctico 
en el cual se han indexado las descripciones de los códigos CPV 2008.

\item Diferenciación de comportamiento entre búsqueda y recomendación. La distinción 
entre estos dos enfoques tecnológicos para la recuperación de información reside 
en que los sistemas de búsqueda están orientados principalmente a la obtención 
de respuestas concretas para consultas de usuario planteadas como cadenas 
de texto para necesidades específicas. En general este enfoque también es correcto 
para la obtención de recursos relacionados con las consultas del usuario, aunque 
se basen en el procesamiento de lenguaje natural en muchos casos las respuestas 
obtenidas son bien aceptadas por el usuario. En el caso de recomendación 
el comportamiento es significativamente diferente ya que en este caso se buscan 
recursos relacionados no directamente obtenidos por una consulta de usuario sino 
más bien de acuerdo a un comportamiento histórico o la explotación de relaciones. En ambos casos, 
el objetivo es suministrar un mecanismo para el filtrado de la información. No obstante,
en los últimos años pese al gran triunfo de los sistemas de búsqueda, los sistemas 
de recomendación se han convertido en su gran competidor y la tendencia 
dicta que se pueden convertir en su sucesor natural. Sin embargo, la deformación provocada 
por el uso de sistemas de búsqueda provoca que en algunos casos las respuestas proporcionadas 
por un sistema de recomendación no dispongan de la acogida apropiada por los usuarios. En este sentido, 
la recomendación de códigos CPV de acuerdo al histórico presente en los anuncios de licitación 
se plantea estratégica para futuras iteraciones de un sistema de soporte a la decisión 
sobre qué códigos CPV seleccionar para la mejora de la recuperación de información 
proveniente de anuncios de licitación.
\end{itemize}

Las evidencias destacadas de estos puntos conducen la validación del experimento con la 
consiguiente extracción de conclusiones:

\begin{itemize}
 \item Las consultas, $Q_{i}$, preparadas por los expertos de acuerdo a clientes reales dejan patente 
que la búsqueda de códigos CPV se basa en un encaje sintáctico inicial mediante el cual se 
establece un conjunto de códigos potenciales a partir de los cuales se seleccionan todos los 
elementos jerárquicamente inferiores, aumento del grado de especificidad, como nuevos elementos 
a buscar. Esta situación se refleja en la cardinalidad de cada uno de los conjuntos de códigos 
CPV de las consultas $Q_{i}$. De esta forma si uno de los elementos de entrada fuera una división 
se obtienen todos los descendientes, grupos, clases y categorías, aumentando el espectro de códigos 
válidos enormemente. La disposición de un conjunto esperado de resultados tan numeroso implica por una parte, que 
la búsqueda real de anuncios de licitación se puede ver distorsionada y por otra parte, que es más sencillo 
que los métodos de generación de códigos CPV obtengan una precisión mayor. En el punto opuesto se encuentran 
aquellas consultas muy específicas en las cuales tan sólo se espera un código de salida, en este caso, la precisión 
de los métodos se ve menoscabada y en muchos casos en un encaje directo es más que suficiente.

\item En general el método más preciso es $M^1$, correspondiente a un motor de búsqueda 
sintáctico basado en Apache \gls{Lucene} y \gls{Solr} mediante la aplicación de filtros sobre el texto tanto 
para el indexado de los elementos del CPV 2008 como para el texto de la consulta del usuario. De nuevo, 
esta situación evidencia que el punto anterior sobre el grado de cumplimiento de las expectativas del experto 
del dominio sobre la búsqueda y recomendación de códigos CPV se basa en técnicas puramente sintácticas. Evidentemente, 
la afinación de las técnicas de procesamiento de descripciones de códigos de clasificaciones de productos realizadas 
para la reconciliación de entidades y enlazado con tras clasificaciones se ha visto recompensado por su posterior 
reutilización para la búsqueda de códigos en general. No obstante, las aspiraciones del experto de dominio no se ven 
totalmente colmadas y existe un porcentaje importante de mejora para conseguir la máxima precisión con valor $1$. Si bien 
en la mayoría de los sistemas de búsqueda actuales está situación se admite, un estudio posterior debería sondear con los 
expertos del dominio si los códigos sugeridos realmente son falsos positivos o bien son códigos que tendrían sentido 
para la consulta de entrada.

\item El método $M^2$ hace uso de las mismas técnicas que el anterior para obtener un conjunto inicial de conceptos 
similar a $M^1$ para a partir de ese momento obtener la jerarquía completa para cada uno de los códigos repartiendo 
un valor de activación entre los hijos. Por ello, se debería mantener la precisión del método $M^1$ pero en este caso 
se ve decrementada debido principalmente a dos causas: 1) reordenación de la ponderación de los códigos y 2) la selección 
de códigos no sólo está basada en la navegación jerárquica. Esta situación es tremendamente crítica ya que si bien el experto 
tiende a situarse en una jerarquía y a partir de ese nivel seleccionar todos los descendientes, dependiendo del caso y de la 
extensa casuística de los anuncios de licitación, la experiencia conlleva a seleccionar códigos no relacionadas directamente 
en ningún nivel. Por ejemplo, en los anuncios de licitación relativos a equipamiento de un hospital se encuentran códigos 
relacionados con la adquisición de \textit{software} y de colchones cuya regla es difícilmente representable con 
una sola taxonomía. Algún tipo de información extra o el dominio en general podrían ayudar a solventar esta información suministrando
así un sistema experto con una precisión elevada.

\item El método $M^3$ utiliza las técnicas propuestas en $M^1$ para realizar el salto entre las descripciones textuales 
de una consulta de usuario a los códigos CPV de la taxonomía. Este conjunto inicial es utilizado por la técnica de \textit{Spreading 
Activation} para descubrir y enriquecer el conjunto de códigos de salida. No obstante, los resultados no son especialmente satisfactorios 
debido de nuevo a que los resultados esperados en las consultas preparadas parte de una creación basada en el encaje de descripciones 
textuales y conocimiento experto de dominio de difícil formalización de forma genérica en una técnica como \textit{Spreading Activation}. Una 
posible propuesta consiste en establecer un contexto de búsqueda en el cual las técnicas avanzadas sean más ``informadas'' de cuáles 
son los caminos a recorrer para simular el conocimiento experto.

\item Finalmente, el método $M^4$ cuya ejecución es similar a los anteriores pero que difiere en la técnica de extracción 
de códigos CPV tampoco obtiene mejoras respecto al método básico $M^1$. Si bien este método reutiliza la información histórica 
de un millón de anuncios de licitación no consigue aproximarse en cuanto a precisión al conocimiento experto. La causa de esta 
situación se debe precisamente a que el comportamiento esperado de un sistema de búsqueda respecto a uno de recomendación 
difiere diametralmente. En este caso y teniendo en cuenta que las consultas preparadas se basan en encajes textuales, 
la recomendación como técnica de aproximación de códigos CPV se aprecia que no es apropiada. No obstante, las posibilidades 
de variación de parámetros en los algoritmos de recomendación tienen un carácter tan extenso que si podrían acercarse 
en implementaciones posteriores.

\item La implementación de estos métodos para la generación de códigos CPV a partir de descripciones textuales de búsqueda 
de anuncios de licitación pone de manifiesto la posibilidad de consumo de datos enlazados para la construcción de servicios 
de soporte a la decisión para la recuperación de información de anuncios de licitación. Esta primera aproximación constituye 
un demostrador público de consumo de \linkeddata y sienta las bases para el perfeccionamiento de la técnica de \textit{Spreading Activation} 
en este ámbito y de los motores de recomendación como futuro para la recuperación de información. No obstante y de igual 
forma que en otros muchos contextos la búsqueda tradicional basada en técnicas sintácticas sigue manteniéndose como 
principal herramienta para obtener los resultados deseados de un usuario. Si bien esta situación puede deberse 
al comportamiento que nos han acostumbrado los sistemas de búsqueda existentes, la tendencia es obtener búsquedas 
más exactas y no directamente relacionadas con la consulta inicial, intentando discernir las intenciones del usuario 
y no ceñirse a una serie de términos.
\end{itemize}


\subsection{Evaluación del experimento sobre el Sistema MOLDEAS}
La ejecución y validación del experimento mediante la comparación de un conjunto de resultados 
esperado $Q^{i}_{cpv}$ y un conjunto de resultados obtenido, $Q^{M^i}_{cpv}$, a través de distintos métodos 
permite dar respuesta a las preguntas planteadas al inicio del experimento.

\begin{itemize}
 \item ¿Es posible implementar un sistema de recuperación de información utilizando datos enlazados?

Evidentemente, el tipo y formato de una fuente de datos no debe ser impedimento para la construcción de servicios 
en un dominio determinado. En este caso particular, los datos enlazados provenientes de los anuncios de licitación 
pueden ser perfectamente consumidos desde un lenguaje de programación como Java mediante el acceso a un 
\textit{endpoint} de \gls{SPARQL} y realizando las consultas pertinentes para cargar los objetos de negocio con la 
información y datos necesarios. Hasta este punto, el consumo de datos enlazados no representa ninguna mejora 
respecto a otro tipo de almacenamiento y representación, la ventaja estratégica reside en su flexibilidad 
para añadir e integrar nuevos datos y a la posibilidad de explotación de las relaciones establecidas entre los distintos 
recursos. Sin embargo, el esfuerzo para el consumo de datos todavía es elevado ya que es complicado mantener esta 
flexibilidad de representación en su consumo desde un lenguaje de programación. Las APIs y bibliotecas disponibles 
facilitan esta labor pero distan en mucho a las herramientas disponibles para los sistemas de gestión de bases 
de datos relacionales. No obstante, este esfuerzo es recompensado ya que los servicios generados están verdaderamente 
más informados y son capaces de explotar eficientemente la información y los datos para la construcción de servicios 
de valor añadido. Además en el nuevo contexto de la \wode este requisito de consumo será prácticamente indispensable 
para cualquier aplicación. Finalmente, el consumo de datos enlazados presenta algunas cuestiones abiertas como la 
confianza y la verificación de la procedencia de los datos que deben ser abordadas para impulsar este proceso.


 \item ¿Es posible explotar las relaciones semánticas establecidas para mejorar la recuperación de información?

En el demostrador público del sistema \gls{MOLDEAS} se permite la consulta y recuperación de anuncios de licitación de acuerdo 
a un método preestablecido y ciertos parámetros de entrada. La experimentación realizada con los distintos métodos permite 
adecuar este demostrador a los resultados obtenidos pero la explotación de las relaciones semánticas no se considera 
suficientemente relevante, al menos desde el punto de vista de un experto de dominio, por lo que aunque el uso de 
semántica implica una serie de ventajas intrínsecas, su aplicación actual en comparación con las expectativas 
del usuario deben ser mejoradas para conseguir una diferenciación real entre usar o no semántica en las 
aplicaciones según es entendida en las iniciativas de Web Semántica y \linkeddata. No sólo se trata de la posibilidad 
de formular consultas muy complejas navegando por las relaciones, sino más bien de comprender cómo se ha de utilizar la semántica para dar respuesta 
a los problemas del usuario, actualmente el uso de semántica se centra en facilitar la interoperabilidad e integración de aplicaciones que 
en el suministro de servicios de valor añadido para el gran público. La adición de las grandes compañías de software y servicios a 
esta iniciativa provocará un vuelco a esta situación, impulsando y facilitando el uso de técnicas semánticas para el usuario 
final.


 \item ¿Cuál es el mejor enfoque para la recuperación de información en los anuncios de licitación? 

Según los resultados obtenidos y de acuerdo al comportamiento de los expertos del dominio, queda patente que el 
método $M^1$ es el más cercano a los resultados esperados. La explicación, como ya se ha detallado en párrafos anteriores, 
se debe principalmente al grado de implantación que la búsqueda tradicional posee. No obstante, tan sólo se han utilizado 
$11$ consultas preparadas (no se disponía de una muestra mayor) por lo que los resultados bajo estas condiciones y 
sin un contraste de hipótesis estadístico, deben ser tomados como una guía en la cual la valoración con una muestra 
de la población relevante e informativa podrían variar.

 \item ¿Cómo afectan los resultados en la implementación actual del sistema MOLDEAS?

Este experimento sirve como prueba del consumo de datos enlazados en un dominio y de la aplicación de técnicas 
tanto tradicionales como procedentes de otros ámbitos para la recuperación de códigos \gls{CPV}. La principal conclusión 
que se debe extraer, consiste en el planteamiento de que un sistema experto para la recuperación de anuncios de licitación 
públicos tiene un gran calado debido a diversos factores: variables de información múltiples, correlaciones entre las 
mismas, cantidad de datos a procesar, etc., que pueden ser optimizados a través de las pruebas con distintos algoritmos 
y técnicas. Es por ello que las próximas iteraciones de MOLDEAS se centrarán en optimizar estas técnicas en el campo 
de las licitaciones para que puedan ser promocionadas a otros dominios de carácter general. Con ello, se consigue 
realizar una demostración pública y abrir una nueva e interesante línea de investigación en el campo de los sistemas 
expertos para la recomendación de anuncios de licitación.

\end{itemize}

\cleardoublepage