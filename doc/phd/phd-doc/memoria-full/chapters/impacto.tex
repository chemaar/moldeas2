Las actividades de difusión y de explotación forman parte significativa
de las actividades transversales que hay que realizar en la promoción
de un trabajo de investigación y se configuran como fundamentales
para la colaboración con otras personas o entidades desde un punto
de vista científico como para iniciar el proceso de transferencia
tecnológica y su posible explotación en el mercado.

El principal objetivo de este tipo de actividades es crear
concienciación de los resultados conseguidos durante
el estudio y ejecución de los trabajos de investigación
tanto a nivel de calidad científica como de su 
posible explotación industrial. Para llevar a cabo este objetivo
se deben seleccionar los canales apropiados de distribución así 
como las entidades que deben hacer eco de los avances realizados. Para conseguir
el impacto deseado estas actividades de difusión deben ejecutarse en paralelo
con la investigación para demostrar los avances y el carácter práctico
atrayendo tanto las comunidades científicas como industriales.

En general las actividades de difusión y demostración se pueden plantear
en dos planos diferentes:

\begin{description}
 \item [Interna.] Permitir que la comunicación entre los miembros participantes
en los trabajos de investigación puedan colaborar de forma ágil. Para ello 
se han utilizado: listas de correo, un wiki y herramientas de desarrollo
basadas en el control de versiones. De esta forma, se involucra y se obtiene
realimentación tanto del Director de la tesis, crucial en este caso, como 
de los participantes en el marco del proyecto \gls{10ders}, ver Sección~\ref{10ders}.
\item [Externa.] Difundir el trabajo realizado para la creación de concienciación
pública. Habitualmente la participación en eventos como conferencias, \textit{workshops}, etc.
o el envío de artículos a revistas son los medios más utilizados para este tipo de difusión.
Hoy en día y con la eclosión de las redes sociales, el uso de sitios como Twitter, Facebook o
Linkedin pueden aportar un gran valor a este tipo de difusión consiguiendo un gran impacto
real. Para ello se han realizado actividades de distinto tipo: a) Formacionales:
celebración de cursos relacionados con la tecnología utilizada, b)
Creación de un demostrador público. c) Creación de un sitio web y presencia en
las principales redes sociales como canales de distribución de la información y
sitios de referencia para consultas. d) Realización de publicaciones científicas
e industriales (aplicación) en conferencias y revistas tanto nacionales como
internacionales. e) Informe de resultados y conocimiento a través de las
distintas plataformas tecnológicas a las que pertenece el grupo \gls{WESO}, en proceso.
\end{description}

Finalmente, la difusión de este trabajo busca integrarse en la comunidad científica situándose como referencia en
la problemática tratada, \lod en el campo de la contratación pública, así como en el tejido empresarial proporcionando un nuevo
enfoque para la gestión de información y conocimiento de la información de carácter público. En las siguientes secciones detallaremos
de forma concisa algunas de las actividades realizadas.

\subsection{Contacto a través de Correo Electrónico y Página Web}
Con el objetivo de difundir los resultados parciales que se iban consiguiendo en el desarrollo
de los trabajos y estudios realizados durante la elaboración de tesis se realizaron
algunos artículos (en formato \textit{blog}) para su posterior publicación
a través de las redes sociales. Para ello se conformo la siguiente estructura:
\begin{itemize}
 \item Página web con un demostrador público del trabajo realizado.
 \item Uso de la cuenta de Twitter del grupo \gls{WESO}.
 \item Uso de la página web personal del autor (con correo electrónico) y del grupo WESO para publicitar los avances.
\end{itemize}

En este sentido y tras la publicación del clasificaciones de productos como \linkeddata hemos obtenido la siguiente
realimentación:
\begin{itemize}

 \item Richard Cyganiak Linked \textit{data technologist at the LiDRC}, DERI Galway, Ireland.

\begin{Frame} 
\begin{itshape}
Hi Jose,

This looks great! I think it's more appropriate to add the “umbrella” dataset at
pscs-catalogue to the LOD Cloud, given that all the individual classifications
are published in the same SPARQL endpoint and look more like parts of one large
dataset to me. (This is a judgment call, so please let us know if you think that
any particular sub-datasets should be highlighted as individual datasets – we
would consider it.)

So for now I've added pscs-catalogue to the lodcloud group and it will be in the
next update of the LOD Cloud diagram.

The nomenclator-asturias-2010 dataset fulfils the technical criteria, so I'm
adding it as well. Good job! ;-)

I've done a change to both of the CKAN records: The links:xxx extra field is
supposed to use the name of the respective CKAN record, so it should be
links:dbpedia to indicate the number of DBpedia links. I've fixed this for
pscs-catalogue and nomenclator-asturias-2010, but you may want to still fix it
for the individual sub-datasets.

I've also updated the description of pscs-catalogue to list the individual
datasets. Feel free to change this further as you see fit.

Finally, I see that the blog post about the nomenclator dataset lists a number
of further links that are not present in the CKAN entry. If there's no
particluar reason for omitting them, then I encourage you to add those links as
well to the CKAN record.

All the best, and keep up the good work,
Richard
\end{itshape}
\end{Frame}

\item Jindřich Mynarz, University of Economics Prague, miembro del consorcio LOD2 Project~\cite{lod2-project}.

\begin{Frame}
\begin{itshape}
Hi Jose,

thanks for our Sunday Twitter discussion. It's always a pleasure to
find someone who works on a closely related topic as you do!
...

Of the classifications you have mentioned, CPV and NACE are of most interest to
me. We have started using you CPV codes to describe the public contracts we
scraped from the Czech central public contracts repository. Your effort with
NACE interests me because I'm working on a conversion of NACE rev. 2 to RDF. We
have an agreement with the Czech Statistical Office to provide us with some
of their classifications that we can convert to RDF and expose as linked data.
One of these classification is NACE rev. 2, to which Czech Statistical Office
has added one level of the most specific concepts (i.e., the activities in NACE
have more narrower concepts as leaf nodes). It would make sense to me just to
release this Czech extension of NACE and link to either to your converted NACE
or to EUROSTAT's NACE in RDF on their metadata server. Unfortunately, there
are some issues with the data dump I've got from the Czech Statistical Office
that block progress on this work.

..
\end{itshape}
\end{Frame}

\item Andreas Radinger, Professur f\"{u}r Allgemeine BWL, insbesondere E-Business e-business and web science research group. Universit\"{a}t der Bundeswehr M\"{u}nchen.

\begin{Frame}
\begin{itshape}
Hi Jose,

congratulations for the very nice dataset at \url{http://www.josemalvarez.es/web/2011/11/05/cpv/}.
One suggestion:
The lemma of the URIs of the ProductOntology are case-sensitive.
...
and perhaps you can also fix the correct spelling in some cases?
...
Best regards,
Andreas
\end{itshape}
\end{Frame}


\item Dominique Guardiola, empresa QuiNode en Francia.

\begin{Frame}
\begin{itshape}
\textit{Thanks for this great work, this is just what I was looking for
Just to mention that stores are complaining about the rdf: namespace not being defined in the 2008 dump.}
\end{itshape}
\end{Frame}


\end{itemize}

Por otra parte y a través de los mecanismos que provee Twitter, ya existen trabajos~\cite{tweet-impact} para evaluar
el impacto de las publicaciones utilizando este herramienta, como los \textit{retweets} o las menciones de otros
usuarios podemos destacar los siguientes mensajes:

\begin{itemize}

\item Stefano Bertolo. Project Officer at European Commission DG INFSO/E2.

\begin{Frame}
\begin{itshape}
Common Procurement Vocabulary released as \#linkeddata using \#goodrelations and \#productontology via @wesoviedo @mfhepp
\end{itshape}
\end{Frame}

\item Jindřich Mynarz.

\begin{Frame}
\begin{itshape}
Improving performance of SPARQL for querying EU public procurement notices. \url{http://slidesha.re/tIIUmw} by @wesoviedo
\end{itshape}
\end{Frame}


\begin{Frame}
\begin{itshape}
@wesoviedo @chema\_ar Great! It seems superior to our attempt at publishing CPV as \#linkeddata: \url{http://bit.ly/mTMmry}
\end{itshape}
\end{Frame}


\begin{Frame}
\begin{itshape}
Methods On Linked Data for E-procurement Applying Semantics - demo \url{http://purl.org/weso/moldeas/} by @wesoviedo. \#linkeddata
\end{itshape}
\end{Frame}


\item Nelson Piedra. Computer Sciences School, Director at UTPL - Universidad Técnica Particular de Loja. Ecuador.

\begin{Frame}
\begin{itshape}
Product Scheme Clas.: CPV, CN, CPA, etc. (beta release) as \#LinkedData \#goodrelations \&... \url{http://shar.es/o30mV} @Euroalert (via @wesoviedo)
\end{itshape}
\end{Frame}


\end{itemize}

En esta línea del uso de Twitter también hemos figurado dos veces en las ``historias del día'' que recoge Ralph Hodgson, Co-founder/CTO de TopQuadrant:
\begin{itemize}
 \item 5 de noviembre de 2011 (\url{http://paper.li/ralphtq/1316980404/2011/11/05})
 \item 16 de noviembre de 2011 (\url{http://paper.li/ralphtq/1316980404/2011/11/16})
\end{itemize}

Finalmente, el desarrollo del trabajo de investigación de esta tesis ha permitido establecer colaboración a través
de distintos contactos:
\begin{itemize}
 \item Jindřich Mynarz. Colaboración en la elaboración de una ontología para formalizar la información
contenida en los anuncios de licitación.
\item Giner Alor-Hernández (Instituto Tecnológico de Orizaba, México). Elaboración de artículos de científicos para el 
caso de \eproc pero más centrado en las cadenas suministros industriales.
\item Otras colaboraciones con entidades a nivel regional y nacional, por ejemplo reaprovechando el trabajo de la Universidad de Zaragoza 
en el campo de la georreferenciación para mejorar al acceso contextualizado a las licitaciones.
\end{itemize}

Todos estos resultados nos animan a seguir trabajando en esta línea de actuación en el campo de los
anuncios de licitación y a promocionar la investigación y resultados realizados.


\subsection{Contribución a las iniciativas \linkeddata y a \lod}
El trabajo realizado también se ha plasmado en la contribución de los datos generados a las iniciativas 
de \linkeddata y \lod. En concreto, el catálogo de clasificaciones de productos se ha difundido 
a través del registro \gls{CKAN} provisto por ``TheDataHub.org'' y solicitado su incorporación 
a la nube de datos enlazados de forma exitosa. De esta manera, en la próxima 
actualización de este diagrama, ampliamente usado y extendido en muchas de las presentaciones 
relacionadas con esta iniciativa, estará presente el catálogo de clasificaciones de productos.

Para consultar esta información se puede acceder a la siguiente \gls{URL}: \url{http://thedatahub.org/dataset/pscs-catalogue}, 
en la cual el catálogo está descrito y definido convenientemente. Igualmente, cada una de las clasificaciones de productos 
están disponibles individualmente. 

Se ha seleccionado este catálogo para su difusión debido a que los datos son abiertos y están disponibles 
en las fuentes oficiales, respecto a la información y datos concernientes a los anuncios de licitación y organizaciones 
se ha pospuesto su adición ya que han sido procesados, generados y cedidos de forma conveniente por la empresa 
Gateway S.C.S. dentro de la ejecución del proyecto ``\gls{10ders} Information Services''.

No obstante es necesario marcar siempre como objetivo la difusión de los datos promocionados a la iniciativa 
de \linkeddata para facilitar su posterior reutilización, colaborando en el despegue de este nuevo entorno 
que constituye la \wode.













