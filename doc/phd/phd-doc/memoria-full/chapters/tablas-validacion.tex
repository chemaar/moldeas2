\section{Consideraciones sobre las Tablas de Validación}
El objetivo de esta sección es presentar las tablas de validación que se utilizan para realizar 
la evaluación del grado de cumplimiento de criterios relativos a las iniciativas de 
\linkeddata y \opendata para verificar que verdaderamente los datos generados se pueden enclavar 
bajo estos enfoques asegurando por tanto los beneficios y ventajas propugnados por los mismos. La elaboración 
y diseño de estas tablas ha considerado el trabajo relacionado presentado en la Sección~\ref{metodologias} así 
como la experiencia personal en el desarrollo de distintas actividades relacionadas con estas iniciativas. 

Los criterios presentados en estas tablas pueden adquirir tres valores:
\begin{itemize}
 \item Valor positivo, \si, si es un criterio que debe tener y se cumple.
 \item Valor negativo, \no, si es un criterio que debe tener y no se cumple.
 \item Valor no aplicable, \na, si es un criterio que se desconoce, se solapa con otro o bien 
o no está asociado al enfoque evaluado.
\end{itemize}

Para ejemplificar una validación en las tablas siguientes se realiza la valoración de un conjunto 
de datos de referencia en general, y \datasets RDF en particular. El grado de cumplimiento de criterios 
ideal se fija en un total de $173$ criterios positivos (\si) y $23$ no aplicables (\na) realizando 
una encuesta total de $196$ criterios de valoración. De esta forma, es posible delimitar si una enfoque 
pertenece a la iniciativa de \opendata, \linkeddata, a la nube de datos enlazados abiertos o 
a un registro \gls{CKAN}, adicionalmente se valora el grado en el que los procesos de producción, publicación, 
consumo, validación y realimentación de los conjuntos de datos se facilitan. 

\subsection{Tabla de Validación $T^{1}$}
En la Tabla~\ref{table:validation-t1} se recogen criterios de validación relativos a los 
procesos de producción, publicación y consumo de datos enlazados, realizando un especial hincapié 
en el diseño de \gls{URI}s, modelado de información y datos utilizando tecnologías semánticas así como factores relativos a la 
descripción de la metainformación de los recursos. Se ha realizado la valoración ideal 
de un \dataset \gls{RDF} de referencia que contaría con $60$ criterios positivos (\si) y $9$ no aplicables 
(\na).

\begin{longtable}[c]{|l|p{7cm}|c|} 
\hline 
  \textbf{ID} & \textbf{Pregunta} &  \textbf{Cumplimiento} \\\hline
\endhead  
  1& \multicolumn{2}{|c|}{\textbf{\textit{Uso de URIs}}} \\ \hline
  1.1&  ¿Las URIs utilizadas permiten acceder a los recursos, \textit{Minting HTTP URIs}? & \si \\ \hline 
  1.2&  ¿El \textit{namespace} utilizado en las URIs está bajo nuestro control? & \si \\ \hline
  1.3&  ¿Se utiliza el esquema  HTTP? &\si  \\ \hline
  1.4&  ¿Las URIs siguen las directrices de \textit{Cool URIs}? &\si  \\ \hline
  1.5&  ¿Se utilizan \textit{hash URIs}? & \na  \\ \hline
  1.6&  ¿Se utilizan \textit{slash URIs}? & \si  \\ \hline
  1.7&  ¿Se utilizan \textit{param URIs}? & \na  \\ \hline
  1.8&  ¿No se incluyen detalles de implementación en la URI? & \si  \\ \hline
  1.9&  ¿Se utilizan claves primarias para identificar los recursos, ID URIs? & \si  \\ \hline
  1.10&  ¿Se utilizan nombres para identificar a los recursos, \textit{Meaningful URIs}? & \na  \\ \hline
  1.11&  ¿Se ha definido una URI base para los recursos? & \si  \\ \hline
  1.12&  ¿Se ha definido una URI base para el modelo formal? & \si  \\ \hline
  1.13&  ¿Se ha definido un esquema de URIs para el \dataset RDF? & \si  \\ \hline
  1.14&  ¿Se ha definido un esquema de URIs para el modelo formal? & \si  \\ \hline
  1.15&  ¿Se incluye metainformación en las URIs? & \si  \\ \hline
 2&\multicolumn{2}{|c|}{\textbf{\textit{Descripción de recursos en \gls{RDF}}}}\\ \hline
  2.1& ¿Se utilizan vocabularios como \gls{SKOS}, RDFS, OWL para modelar el dominio?& \si  \\ \hline
  2.2& ¿Se reutilizan vocabularios de acuerdo a su uso y actualización?& \si  \\ \hline
  2.3& ¿Se utilizan anotaciones de RDFS o SKOS?& \si  \\ \hline
  2.4& ¿Se relacionan las clases y propiedades con otras ya existentes?& \si \\ \hline
  2.5& ¿Se reutilizan las clases y propiedades ya existentes?& \si  \\ \hline
  2.6& ¿Se definen nuevas clases y propiedades?&\si  \\ \hline
  2.7& ¿Se enriquecen las descripciones RDF con otros \datasets RDF?& \si  \\ \hline
  2.8& ¿Se describe parcialmente los recursos de otros \datasets RDF a los que se enlaza desde el actual?& \no  \\ \hline
  2.9& ¿Se añade metainformación a cada recurso RDF individualmente?& \si  \\ \hline
  2.10& ¿Son las descripciones de los recursos RDF navegables?& \si  \\ \hline
  2.11& ¿Se provee información útil del recurso RDF a través de la URI?& \si  \\ \hline
  2.12& ¿Son las URIs reutilizadas referenciables?& \si  \\ \hline  
 3&\multicolumn{2}{|c|}{\textbf{\textit{Descripción del \dataset RDF}}}\\ \hline
  3.1& ¿Se ha definido una \gls{URI} para el \dataset RDF? & \si  \\ \hline
  3.2& ¿Se utiliza el vocabulario \gls{voID} para describir el \dataset RDF? & \si  \\ \hline
  3.3& ¿Se utiliza el \textit{Semantic Sitemaps} para describir el \dataset RDF? & \na  \\ \hline
  3.4& ¿Se provee información de \textit{provenance}? & \si  \\ \hline
  3.5& ¿Se provee licencia de uso, información sobre derechos de copia, etc.?&  \si  \\ \hline
  3.6& ¿Se provee información sobre la autoría? & \si  \\ \hline
  3.7& ¿Se proveen ejemplos de uso del \dataset RDF? & \si  \\ \hline
  3.8& ¿Se utilizan anotaciones de RDFS o SKOS?&  \si  \\ \hline
  3.9& ¿Se utilizan anotaciones multilingües RDFS o SKOS?& \si  \\ \hline
 4&\multicolumn{2}{|c|}{\textbf{\textit{Otros}}}\\ \hline
  4.1& ¿Se utilizan herramientas automáticas para la producción de RDF?& \si  \\ \hline
  4.2& ¿Se consolidan parte de los datos en RDF?& \no  \\ \hline
  4.3& ¿Se realiza la reconciliación de entidades de forma automática?& \no  \\ \hline
  4.4& ¿Los datos son dinámicos?& \na  \\ \hline
  4.5& ¿Los datos son estáticos?& \si  \\ \hline
  4.6& ¿El tamaño del \dataset es del orden de millones de tripletas?& \si  \\ \hline
  4.7& ¿El tamaño del \dataset es del orden de billones de tripletas?& \na  \\ \hline
5&\multicolumn{2}{|c|}{\textbf{\textit{Publicación de \linkeddata}}}\\ \hline
  5.1&  ¿Se publica un volcado de los datos en RDF? & \no  \\ \hline 
  5.2&  ¿Se utiliza algún método de publicación de datos enlazados? & \si  \\ \hline
  5.3&  ¿Se provee algún lenguaje de consulta formal? & \si  \\ \hline
  5.4&  ¿Se provee un \textit{endpoint de \gls{SPARQL}}? & \si  \\ \hline
  5.5&  ¿Se provee negociación de contenido? & \si  \\ \hline
  5.6&  ¿Se pueden referenciar los recursos desde otros documentos tipo \gls{HTML}? & \si  \\ \hline    
  5.7&  ¿Se provee un \linkeddata \textit{Frontend}? & \si  \\ \hline  
  5.8&  ¿Se provee metainformación del \dataset RDF? & \si  \\ \hline
  5.9&  ¿Se difunde el \dataset RDF en distintos medios tipo CKAN, Prefix.cc, etc.? & \si  \\ \hline
  5.10&  ¿Se publica algún API o servicio web para la consulta de los datos? & \si  \\ \hline
  5.11&  ¿Existe alguna restricción en la consulta de los datos? & \no  \\ \hline
  5.12&  ¿Se establece algún mecanismo de privacidad? & \no  \\ \hline
  5.13&  ¿Se informa del tamaño del \dataset? & \si  \\ \hline  
  5.14&  ¿Se publican los datos RDF como un fichero estático? & \na  \\ \hline
  5.15&  ¿Se publican los datos RDF bajo demanda desde base de datos?& \na  \\ \hline
  5.16&  ¿Se publican los datos RDF en un repositorio? & \si  \\ \hline    
  5.17&  ¿Se publican los datos RDF bajo demanda desde una aplicación? & \si  \\ \hline
  5.18&  ¿Se publican los datos con información temporal y evolución en el tiempo?& \no  \\ \hline     
  5.19&  ¿Se proveen ejemplos para depuración y consumo? & \si  \\ \hline        
  5.20&  ¿Se proveen alias a ciertos directorios o nombres? & \si  \\ \hline
  5.21&  En caso de error, ¿se devuelve algún recurso por omisión? &  \no  \\ \hline      
  5.22&  ¿Se utilizan protocolos estándar? & \si  \\ \hline    
  5.23&  ¿Se provee algún mecanismo de realimentación? & \no  \\ \hline    
  5.24&  ¿Se provee documentación sobre los datos publicados? & \si  \\ \hline
  5.25&  ¿Se proveen estadísticas de los datos publicados? & \si  \\ \hline    
  5.26&  ¿Se utilizan mecanismos de sellado en el tiempo o similares? & \si  \\ \hline            
 \hline
\caption{$T^{1}$-Tabla de Validación de Características \linkeddata.}\label{table:validation-t1}\\    
\end{longtable}



\subsection{Tabla de Validación $T^{2}$}

En la Tabla~\ref{table:validation-t2} se recogen criterios de validación relativos a los 
a los patrones de diseño utilizados para la generación de datos enlazados. Si bien su uso no es obligatorio, 
la idea subyacente coincide con los patrones de diseño en ingeniería del software en los cuales 
se ofrecen soluciones para casuísticas habituales con el objetivo de mejorar tanto los procesos de 
producción, publicación, consumo, validación y realimentación. Se ha realizado la valoración ideal 
de un \dataset \gls{RDF} de referencia que contaría con $44$ criterios positivos (\si), incluyendo la aplicación 
de todos los patrones, se ha tomado esta decisión para obtener una medida del grado de aplicación de los 
mismos ya que en muchos casos la información generada en un \dataset RDF puede utilizar unos u otros no es 
conveniente plantear su uso como disyuntivo.


\begin{longtable}[c]{|l|p{7cm}|c|} 
\hline 
  \textbf{ID} & \textbf{Pregunta} &  \textbf{Cumplimiento}  \\\hline
\endhead 
   1& \multicolumn{2}{|c|}{\textbf{\textit{Identifier Patterns}}}\\ \hline
  1.1 &  \textit{Hierarchical \gls{URI}s} &\si \\ \hline
  1.2 &  \textit{Literal Keys} &\si \\ \hline
  1.3 &  \textit{Natural Keys} &\si \\ \hline
  1.4 &  \textit{Patterned URIs} &\si \\ \hline
  1.5 &  \textit{Proxy URIs} &\na \\ \hline
  1.6 &  \textit{Shared Keys} &\na \\ \hline
  1.7 &  \textit{\gls{URL} Slug} &\na \\ \hline    
    2& \multicolumn{2}{|c|}{\textbf{\textit{Modelling Patterns}}}\\ \hline
  2.1 &  \textit{Custom Datatype} &\si \\ \hline    
  2.2 &  \textit{Index Resources} &\na \\ \hline    
  2.3 &  \textit{Label Everything} &\si \\ \hline     
  2.4 &  \textit{Link Not Label} &\si \\ \hline    
  2.5 &  \textit{Multi-Lingual Literal} &\si \\ \hline    
  2.6 &  \textit{N-Ary Relation} &\na \\ \hline    
  2.7 &  \textit{Ordered List} &\na \\ \hline     
  2.8 &  \textit{Ordering Relation} &\na \\ \hline     
  2.9 &  \textit{Preferred Label} &\si \\ \hline    
  2.10 &  \textit{Qualified Relation} &\si \\ \hline   
  2.11 &  \textit{Reified Statement} &\na \\ \hline    
  2.12 &  \textit{Topic Relation} &\si \\ \hline      
  2.13 &  \textit{Typed Literal} &\si \\ \hline        
    3& \multicolumn{2}{|c|}{\textbf{\textit{Publishing Patterns}}}\\ \hline
  3.1 &  \textit{Annotation} &\si \\ \hline    
  3.2 &  \textit{Autodiscovery} &\si \\ \hline    
  3.3 &  \textit{Document Type} &\si \\ \hline     
  3.4 &  \textit{Edit Trail} &\no \\ \hline    
  3.5 &  \textit{Embedded Metadata} &\si \\ \hline    
  3.6 &  \textit{Equivalence Links} &\si \\ \hline    
  3.7 &  \textit{Link Base} &\si \\ \hline     
  3.8 &  \textit{Materialize Inferences} &\na \\ \hline     
  3.9 &  \textit{Named Graphs} &\si \\ \hline    
  3.10 &  \textit{Primary Topic Autodiscovery} &\si \\ \hline    
  3.11 &  \textit{Progressive Enrichment} &\si \\ \hline      
  3.12 &  \textit{See Also} &\si \\ \hline    
        4& \multicolumn{2}{|c|}{\textbf{\textit{Application Patterns}}}\\ \hline
  4.1 &  \textit{Assertion Query} &\si \\ \hline    
  4.2 &  \textit{Blackboard} &\si \\ \hline    
  4.3 &  \textit{Bounded Description} &\si \\ \hline     
  4.4 &  \textit{Composite Descriptions} &\si \\ \hline    
  4.5 &  \textit{Follow Your Nose}&\no \\ \hline    
  4.6 &  \textit{Missing Isn't Broken} &\no \\ \hline    
  4.7 &  \textit{Parallel Loading} &\no \\ \hline     
  4.8 &  \textit{Parallel Retrieval} &\no \\ \hline     
  4.9 &  \textit{Resource Caching} &\no \\ \hline    
  4.10 &  \textit{Schema Annotation} &\si \\ \hline    
  4.11 &  \textit{Smushing} &\si \\ \hline
  4.12 &  \textit{Transformation Query} &\no \\ \hline        
 
\hline
\caption{$T^{2}$-Tabla de Validación de \textit{Linked Data Patterns}.}\label{table:validation-t2}\\    
\end{longtable}



\subsection{Tabla de Validación $T^{3}$}
En la Tabla~\ref{table:validation-t3} se recogen los principios de \linkeddata que sirven 
para valorar si un enfoque se encuentra dentro de esta iniciativa. Se ha realizado la valoración ideal 
de un \dataset \gls{RDF} de referencia que contaría con $4$ criterios positivos (\si).

\begin{longtable}[c]{|l|p{7cm}|c|} 
\hline 
  \textbf{ID} & \textbf{Pregunta} &  \textbf{Cumplimiento}  \\\hline
\endhead
   1.1&\textit{Use \gls{URI}s as names for things} & \si  \\ \hline
   1.2&\textit{When someone looks up a URI, provide useful information, using the standards (RDF*, \gls{SPARQL})} & \si \\ \hline  
   1.3&\textit{Include links to other URIs} & \si \\ \hline    
   1.4&\textit{Use \gls{HTTP URI}s} & \si \\ \hline    
   \hline
  \caption{$T^{3}$-Tabla de Validación de Principios de \linkeddata.}
  \label{table:validation-t3}
\end{longtable}

\subsection{Tabla de Validación $T^{3}_1$}
En la Tabla~\ref{table:validation-t31} y en combinación con la anterior, se recogen los criterios 
relacionados con el modelo de $\star$ propuesto por Tim Berners-Lee y de esta forma se puede 
establecer el nivel de datos enlazados de un determinado \dataset \gls{RDF}, el valor de referencia 
máximo contaría con $5$ criterios positivos (\si), uno por cada $\star$.

\begin{longtable}[c]{|l|p{7cm}|c|} 
\hline 
  \textbf{ID} & \textbf{Pregunta} &  \textbf{Cumplimiento}  \\\hline
\endhead
 
   1.1&$\star$	& \si \\ \hline 

1.2&$\star \star$	 & \si \\ \hline 

1.3&$\star \star \star$	& \si  \\ \hline 

1.4&$\star \star \star \star$ & \si \\ \hline 
 
1.5&$\star \star \star \star \star$ & \si \\ \hline 
  \hline
  \caption{$T^{3}_1$-Tabla de Validación del Modelo $\star$.}
  \label{table:validation-t31}
\end{longtable}



\subsection{Tabla de Validación $T^{4}$}
En la Tabla~\ref{table:validation-t4} se recogen los principios de \opendata que sirven 
para valorar si un enfoque se encuentra dentro de esta iniciativa. Se ha realizado la valoración ideal 
de un \dataset \gls{RDF} de referencia, o en general un conjunto de datos públicos, contaría con $8$ criterios positivos (\si).

\begin{longtable}[c]{|l|p{7cm}|c|} 
\hline 
  \textbf{ID} & \textbf{Pregunta} &  \textbf{Cumplimiento}  \\\hline
\endhead
  \multicolumn{2}{|c|}{\textbf{8 Principios}}  \\ \hline
   1.1& \textit{Complete} & \si  \\ \hline
   1.2&\textit{Primary} & \si  \\ \hline  
   1.3&\textit{Timely} & \si  \\ \hline  
   1.4&\textit{Accessible} & \si  \\ \hline  
   1.5&\textit{Machine processable} & \si  \\ \hline  
   1.6&\textit{Non-Discriminatory} & \si  \\ \hline  
   1.7&\textit{Non-Proprietary} &\si  \\ \hline
   1.8&\textit{License-free} & \si  \\ \hline                                                               
  \hline
  \caption{$T^{4}$-Tabla de Validación de Principios de \opendata.}
  \label{table:validation-t4}
\end{longtable}

\subsection{Tabla de Validación $T^{4}_1$}
En la Tabla~\ref{table:validation-t41} se recogen beneficios y ventajas de la aplicación de los principios de \opendata. 
Se ha realizado la valoración ideal de un \dataset \gls{RDF}, o en general de un conjunto de datos públicos, 
de referencia contaría con $14$ criterios positivos (\si).

\begin{longtable}[c]{|l|p{7cm}|c|} 
\hline 
  \textbf{ID} & \textbf{Pregunta} &  \textbf{Cumplimiento}  \\\hline
\endhead
   \multicolumn{2}{|c|}{\textbf{Producción}}  \\ \hline
   1.1& ¿Se ha definido una misión y estrategia para la apertura de los datos? & \no  \\ \hline
   1.2& ¿Los datos proceden de una fuente segura? & \no  \\ \hline
   1.3& ¿Se puede conocer la procedencia de los datos? & \si  \\ \hline    
   1.4& ¿Existe algún mecanismo para asegurar la calidad de los datos? & \si  \\ \hline  
  \multicolumn{2}{|c|}{\textbf{Ventajas}}  \\ \hline
   2.1& ¿Facilitan los datos la inclusión? & \si  \\ \hline
   2.2& ¿Mejoran la transparencia? & \si  \\ \hline    
   2.3& ¿Existe alguna responsabilidad sobre los datos? & \no  \\ \hline
  \multicolumn{2}{|c|}{\textbf{Beneficios}}  \\ \hline
   3.1& ¿Pueden las aplicaciones servirse de estos datos para generar servicios, reutilización? & \si  \\ \hline
   3.2& ¿Se pueden generar múltiples vistas de los datos? & \si  \\ \hline
   3.3& ¿Se pueden integrar con otras fuentes de datos? & \si  \\ \hline     
   \multicolumn{2}{|c|}{\textbf{Consumo}}  \\ \hline
   4.1& Uso de anotaciones & \si  \\ \hline        
   4.2& ¿Se provee un \gls{API} o servicio web de consumo? & \si  \\ \hline
   4.3& ¿Se provee algún mecanismo de sindicación para obtener los datos?& \no  \\ \hline
   4.4& ¿Existe algún modelo formal o especificación de los datos publicados? & \si  \\ \hline                                                             
  \hline
  \caption{$T^{4}_1$-Tabla de Validación sobre Características de \opendata.}
  \label{table:validation-t41}
\end{longtable}



\subsection{Tabla de Validación $T^{5}$}
En la Tabla~\ref{table:validation-t5} se recogen los criterios para pertenecer a la nube de datos 
enlazados y abiertos. Estos criterios son complementarios a los señalados en las tablas anteriores 
en el sentido del grado de cumplimiento de \linkeddata y \opendata. Sin embargo, la importancia 
de esta valoración reside en calibrar el grado de reutilización de información y la capacidad 
de difusión del \dataset \gls{RDF}. Es por ello que se fija una valoración de referencia de $5$ criterios 
positivos (\si).

\begin{longtable}[c]{|l|p{7cm}|c|} 
\hline 
  \textbf{ID} & \textbf{Pregunta} &  \textbf{Cumplimiento}  \\\hline
\endhead
   1& ¿Son los recursos \gls{RDF} accesibles mediante \gls{HTTP} o \gls{HTTPS}? & \si  \\ \hline
   2& ¿Se provee negociación de contenido? & \si  \\ \hline
   3& ¿El \dataset contiene más de 1000 tripletas? & \si  \\ \hline
   4& ¿Se provee, al menos, 50 enlaces a \datasets ya disponibles en el diagrama? & \si \\ \hline
   5& ¿Se provee acceso al \dataset completo? & \si  \\ \hline
  \hline
  \caption{$T^{5}$-Tabla de Validación sobre Características para pertenecer a \textit{The Linking Open Data Cloud} de los Anuncios de Licitación.}
  \label{table:validation-t5}
\end{longtable}



\subsection{Tabla de Validación $T^{6}$}
En la Tabla~\ref{table:validation-t6} se recogen los criterios para pertenecer a un registro \gls{CKAN}, complementario 
y obligatorio a la pertenencia a la nube de datos enlazados y abiertos. Este valoración cobra especial relevancia 
si el \dataset \gls{RDF} va a ser publicado para su reutilización por terceros, en el caso de datos enlazados a nivel 
privado no tendría tanta trascendencia. La valoración de un \dataset RDF de referencia se fija 
en $33$ criterios positivos (\si) y $14$ criterios positivos (\na).

\begin{longtable}[c]{|l|p{7cm}|c|} 
\hline 
  \textbf{ID} & \textbf{Pregunta} &  \textbf{Cumplimiento}  \\\hline
\endhead
  1&\multicolumn{2}{|c|}{\textbf{Standard CKAN fields}}  \\ \hline
  1.1&  \textit{Name} & \si  \\ \hline
  1.2 &  \textit{Title} & \si  \\ \hline
  1.3 &  \textit{URL} & \si  \\ \hline
  \multicolumn{3}{|c|}{\textbf{Enhanced CKAN fields}}  \\ \hline
   1.4&  \textit{Version} & \si  \\ \hline
   1.5&  \textit{Notes} & \no  \\ \hline
   1.6&  \textit{Author} & \si  \\ \hline
   1.7&  \textit{Author email} &\si  \\ \hline
   1.8&  \textit{License} &\si  \\ \hline
   \multicolumn{3}{|c|}{\textbf{Custom CKAN fields}}  \\ \hline
   1.9&  \textit{shortname}  \\ \hline
   1.10&  \textit{license\_link} & \si  \\ \hline
   1.11&  \textit{sparql\_graph\_name} & \no  \\ \hline
   1.12&  \textit{namespace} & \si  \\ \hline
   1.13&  \textit{triples} & \si  \\ \hline
   1.14&  \textit{links:xxx} &\si  \\ \hline
  2&\multicolumn{2}{|c|}{\textbf{CKAN tags}}  \\ \hline
  2.1&  \textit{<topic>} &\si \\ \hline
  \multicolumn{3}{|c|}{\textbf{Metainformation CKAN tags}}  \\ \hline  
  2.2&\textit{format-<prefix>}&\si \\ \hline
  2.3&\textit{no-proprietary-vocab}&\na \\ \hline
  2.4 &\textit{deref-vocab}&\si \\ \hline
  2.5&\textit{no-deref-vocab}&\na \\ \hline
  2.6&\textit{vocab-mappings}&\si \\ \hline
  2.7&\textit{no-vocab-mappings}&\na \\ \hline
  2.8&\textit{provenance-metadata}&\si \\ \hline
  2.9&\textit{no-provenance-metadata}&\na \\ \hline
  2.10&\textit{license-metadata}&\si \\ \hline
  2.11&\textit{no-license-metadata}&\na \\ \hline	
  2.12&\textit{published-by-producer}&\na \\ \hline
  2.13&\textit{published-by-third-party}&\si \\ \hline		
  2.14&\textit{limited-sparql-endpoint}&\no \\ \hline
  2.15&\textit{lodcloud.nolinks}&\no \\ \hline		
  2.16&\textit{lodcloud.unconnected} &\no \\ \hline
  2.17&\textit{lodcloud.needsinfo}&\no \\ \hline				
  2.18&\textit{lodcloud.needsfixing}&\no \\ \hline				
  3& \multicolumn{2}{|c|}{\textbf{CKAN resource links}}  \\ \hline
  3.1&  \textit{Download Page} &\si \\ \hline
  3.2&  \textit{meta/sitemap} &\na \\ \hline
  3.3&  \textit{api/sparql} &\si \\ \hline
  3.4&  \textit{meta/void} &\si \\ \hline
  3.5&  \textit{application/rdf+xml} &\si \\ \hline
  3.6&  \textit{text/turtle} &\na \\ \hline
  3.7&  \textit{application/x-ntriples} &\na \\ \hline
  3.8&  \textit{application/x-nquads} &\na \\ \hline
  3.9&  \textit{meta/rdf-schema} &\si \\ \hline
  3.10&  \textit{example/rdf+xml} &\si \\ \hline
  3.11& \textit{example/turtle} &\na \\ \hline
  3.12&  \textit{example/ntriples }&\na \\ \hline
  3.13&  \textit{example/rdfa} &\na \\ \hline
  3.14&  \textit{mapping/\{format\}} &\no \\ \hline
    \hline
  \caption{$T^{6}$-Tabla de Validación para registrar el \dataset en CKAN de los Anuncios de Licitación.}
  \label{table:validation-t6}
\end{longtable}


