%
% cabeceras.tex
%
% Copyright 2003, Diego Berrueta Muñoz
%
% Cabeceras comunes
%

\usepackage[T1]{fontenc}

%Glosario

\usepackage[toc,style=treenoname,order=word,counter=section]{glossaries}

\usepackage{xspace}


\usepackage{tikz,times}
\usetikzlibrary{mindmap,backgrounds}

% cambia algunas fuentes (utilidad dudosa)
\usepackage[scaled=0.92]{helvet}
\usepackage{pifont}
\usepackage{courier}

% cambia algunas fuentes en modo matemático a Palatino
\usepackage{mathpazo}

% españolización
\usepackage[spanish]{babel}
\usepackage[utf8]{inputenc}
%\extrasspanish

% gráficos y colores
\usepackage{rotating}
\usepackage{graphicx}
\usepackage{color}
%\usepackage[all]{xy}
%\usepackage{pstricks}
\usepackage{pst-node}
%\usepackage[dvips,usenames]{pstcol}
%\usepackage{pdftricks}
%\usepackage{pst-uml}  % para hacer diagramas UML
%\usepackage{rail}     % para hacer diagramas de gramáticas

% mejoras visuales
\usepackage{enumerate}
\usepackage{fancyhdr}  % para configurar los encabezados
\usepackage{fancybox}  % para hacer cajitas
\usepackage[normal,oneline,sf,bf]{caption2}
\usepackage{titlesec}  % para configurar los títulos de sección
\usepackage{paralist}


\usepackage{epigraph}

% citas, referencias e índices
\usepackage{cite}
%\usepackage{citesort}   % da errores al compilar
\usepackage{makeidx}

% incrustaciones de código fuente
%\usepackage[norules,nolineno]{lgrind}
\usepackage{verbatim}
\usepackage{listings}
%\usepackage{noweb,a4wide}


%\usepackage{textcomp}
\usepackage[right]{eurosym}

% columnas
\usepackage{multicol}

% tablas
\usepackage{longtable}
%\usepackage{ltxtable}

% impresión elegante de URLs
\usepackage{url}

\makeatletter
\def\url@pfcstyle{%
  \@ifundefined{selectfont}{\def\UrlFont{\sf}}{\def\UrlFont{\small\ttfamily}}}
\makeatother
%% Now actually use the newly defined style.
\urlstyle{pfc}


% márgenes
\usepackage[a4paper, left=30mm, right=20mm, top=25mm, bottom=25mm]{geometry}
%\usepackage[a4paper, left=20mm, right=20mm, top=25mm, bottom=25mm]{geometry}

% 
% \usepackage[a4,center,cam]{crop}
\usepackage{blindtext}

% salida en PDF navegable
%\usepackage{hyperref}
\usepackage[plainpages=false,colorlinks, linkcolor=black]{hyperref}

% quitar en versión final
%\usepackage{showkeys}   % depuración de etiquetas y referencias
%\usepackage{showidx}    % depuración de índice

% configuración del paquete "listings"

\lstset{%
        %language=Java,
	basicstyle=\footnotesize\sffamily,
	keywordstyle=\bfseries, %\color{darkred}
 	stringstyle=\itshape, %\color{violet}
 	commentstyle=\itshape, %\color{blue}
 	showspaces=false,
 	showtabs=false,
 	showstringspaces=false,
 	frame=trBL,
        frameround=tttt,
        %backgroundcolor=\color{lightyellow},
	inputencoding=utf8,
 	extendedchars=true,
 	numbers=none,
        aboveskip=0.5cm,
        belowskip=0.5cm,
        xleftmargin=1cm,
        xrightmargin=1cm,
	breaklines=true
}
\definecolor{darkred}{rgb}{0.5, 0, 0}
\definecolor{violet}{rgb}{1, 0, 1}
\definecolor{lightyellow}{rgb}{1,1,0.8}


%%%%%%%%%%%%%%%%%%%%%%%%%%%%%%%%%%%%%%%%%%%%%%%%%%%%%%%%%%%%%%%%%%%%%%
% cabeceras y pies de página (con el paquete "fancyhdr")
\headheight 15pt
%\addtolength{\headwidth}{\marginparsep}
%\addtolength{\headwidth}{\marginparwidth}
%\renewcommand{\chaptermark}[1]{\markboth{\MakeUppercase{#1}}{}}
%\renewcommand{\sectionmark}[1]{\markright{\thesection\ #1}}
%\fancyhead[LE,RO]{\textbf{\thepage}}
%\fancyhead[RE]{\textit{\leftmark}}
%\fancyhead[LO]{\rightmark}
%\fancyfoot[LCR]{}
\fancyhead{} % Todos los campos en blanco en la cabecera
\fancyfoot{} % Lo mismo al pie
\fancyhead[RO, LE]{\thepage}
\fancyhead[LO, RE]{\slshape\leftmark}
\renewcommand{\headrulewidth}{0.5pt}
\renewcommand{\footrulewidth}{0.5pt}


%%%%%%%%%%%%%%%%%%%%%%%%%%%%%%%%%%%%%%%%%%%%%%%%%%%%%%%%%%%%%%%%%%%%%%
% títulos de secciones (con el paquete "titlesec")
\titleformat{\chapter}[display]
	{\fontfamily{pag}\selectfont\Huge}
	{\LARGE\chaptertitlename\ \thechapter}{20pt}{\bfseries}
\titleformat{\section}
	{\fontfamily{phv}\selectfont\LARGE}
	{\thesection}{1em}{\bfseries}[\titlerule]
\titleformat{\subsection}
	{\fontfamily{phv}\selectfont\Large}
	{\thesubsection}{1em}{\bfseries}
\titleformat{\subsubsection}
	{\fontfamily{phv}\selectfont\large}
	{\thesubsubsection}{1em}{\bfseries}


%%%%%%%%%%%%%%%%%%%%%%%%%%%%%%%%%%%%%%%%%%%%%%%%%%%%%%%%%%%%%%%%%%%%%%
% espaciado entre párrafos
\addtolength{\parskip}{+0.2cm}


%%%%%%%%%%%%%%%%%%%%%%%%%%%%%%%%%%%%%%%%%%%%%%%%%%%%%%%%%%%%%%%%%%%%%%
% salida en PDF navegable (con el paquete "hyperref")
\hypersetup{bookmarks,
	bookmarksnumbered,
%	colorlinks, % quitar en las versiones impresas
	hyperindex,
	%linkcolor=red,
	%anchorcolor=black,
	%citecolor=green,
	citecolor=violet,
	%filecolor=magenta,
	%menucolor=red,
	%pagecolor=red,
	%urlcolor=cyan,
	pdftitle={PhD. Dissertation-Jose María Alvarez Rodríguez},
	pdfauthor={Jose María Alvarez Rodríguez}
	pdffitwindow,
	plainpages=false,
	pageanchor=false,
	pdfstartview={}}


%%%%%%%%%%%%%%%%%%%%%%%%%%%%%%%%%%%%%%%%%%%%%%%%%%%%%%%%%%%%%%%%%%%%%%
% profundidad de secciones y numeración
%\setcounter{tocdepth}{4}
\setcounter{secnumdepth}{3}


%%%%%%%%%%%%%%%%%%%%%%%%%%%%%%%%%%%%%%%%%%%%%%%%%%%%%%%%%%%%%%%%%%%%%%
% división silábica
\hyphenation{pu-bli-ca-ción con-tex-tua-li-za-ción pro-ble-mas Mi-nis-te-rio li-ci-ta-ción si-guien-tes desa-rro-llar con-tra-ta-ción pan-europea fa-ci-li-tar in-ter-opera-bility
 elec-tró-ni-ca li-ci-ta-do-ras he-te-ro-gé-neos res-trin-gi-do cons-cien-te de-sa-rro-lla-da mo-de-lo paí-ses ge-ne-ra-da
 par-ti-ci-pa-ción ad-mi-nis-tra-ti-vos bi-blio-gra-fía re-gis-tra-do-ra ad-mi-nis-tra-ti-va ca-rac-te-rís-ti-cas
 Pa-ra-le-la-men-te des-cri-ben vo-ca-bu-la-rio me-dian-te do-cu-men-to di-fe-ren-tes bi-blio-gra-fía mo-de-lo
 ve-ri-fi-ca rea-li-zar res-tric-ciones Por ejem-plo su-ge-ren-cias mo-de-los rea-li-za si-guien-do he-te-ro-gé-neas ello
 ra-zo-na-ble li-mi-ta-cio-nes reu-ti-li-cen ini-cia-ti-va va-ria-da exis-ten ob-te-ner con-ti-nua-men-te me-dian-te ins-ti-tu-ción
 pro-pues-ta pu-bli-ca-ción ca-tá-lo-go mo-ni-to-ri-za-ción vi-sua-li-za-ción mo-de-los co-rres-pon-dien-tes
 re-fe-ren-cia re-la-ti-vas par-ti-cu-lar  ele-men-to mo-de-la-do se-le-ccio-na-do pro-du-cción Re-con-ci-lia-ción des-crip-cio-nes broader-Transitive 
 de-pen-dien-do Ca-tá-lo-gos vo-ca-bu-la-rios Pro-pie-ta-rio des-plie-gue De-sa-rro-lla-dor mo-de-lar res-pues-ta si-mi-la-res rea-li-za-do SE-RI-MI 
Evol-ving ha-bi-li-tar rea-li-za-ción pu-bli-ca-ción ge-ne-ra-ción rea-li-za-ban ac-tua-li-za-ción ac-ce-so ha-cer es-tán-dar 
vo-ca-bu-la-ry con-si-de-ra-do des-cri-bir pu-bli-car pu-bli-ca-dos ta-xo-nó-mi-ca Pro-duct-on-to-lo-gy Va-lo-res in-terope-ra-bi-li-dad MOL-DEAS 
man-te-ni-mien-to re-pre-sen-tar ta-xo-no-mías des-crip-ción ge-ne-ra cons-truc-ción re-pre-sen-tan-do rea-li-za-das trans-fe-ren-cia 
uti-li-za-ble pu-bli-ca be-ne-fi-cios res-pon-sa-bi-li-da-des rea-li-zan Ad-mi-nis-tra-ción con-ti-nua-ción prio-ri-ta-rio ICT 
in-terope-ra-ble ge-ne-ra-do con-fi-gu-ra-cio-nes rea-li-zan-do rea-lis-ta má-xi-mo Cons-ti-tu-yen-do pro-ble-ma po-si-bi-li-dad ma-yor coo-pe-ra-ción 
ra-zo-na-mien-to Des-crip-tion con-cep-tua-li-za-ción de-sa-rro-lla-dos ope-ra-do-res axio-mas go-bier-no or-ga-nis-mos éxi-to 
si-guien-te do-mi-nios pro-duc-ción Na-tio-nal orien-ta-da ló-gi-ca in-di-vi-duos di-se-ña-do for-ma-li-za-dos ta-xo-no-mía 
auto-má-ti-ca ca-li-dad re-co-men-da-ble ad-mi-nis-tra-ti-vo su-mi-nis-tre error ca-rac-te-rís-ti-ca ele-men-tos he-rra-mien-tas 
eva-lua-ción res-pec-to Ad-mi-nis-tra-cio-nes ex-pe-ri-men-tal orien-ta-das re-gis-tros do-cu-men-tos má-xi-ma je-rar-quía 
sa-tis-fac-to-rios obs-tan-te con-su-mi-dos con-fe-ren-cias SPARQL rea-li-za-dos va-lo-ran eco-no-mi-zar fle-xi-ble ex-pe-dien-te 
pro-ce-di-mien-to usan-do ame-ri-ca-nas fa-mi-lia ni-vel ra-zo-na-do-res re-pre-sen-ta-ción lo-ca-li-za-do Man-ches-ter 
en-ri-que-ci-mien-to he-rra-mien-ta rea-li-zar-se prio-ri-ta-rios Li-ci-ta-cio-nes ne-ce-sa-rios ma-te-ria-li-zan es-pe-cia-li-zán-do-se 
Com-pu-ting pla-ni-fi-ca-ción cua-li-ta-ti-va de-sa-for-tu-na-da-men-te li-de-ra-do ma-te-ria-li-za-do Cons-tructs 
ma-te-ria-li-za WESO en-ten-di-mien-to MOLDEAS ela-bo-ra-ción re-fe-ren-te re-gis-tra-tion uni-la-te-ral po-si-bi-li-tar ope-rar 
in-ter-ope-ra-bles ad-mi-nis-tra-ti-vas fle-xi-bi-li-dad auto-no-mía co-mer-cio procesa-miento de-sa-rro-llan-do meta-in-for-ma-ción 
apro-pia-dos li-mi-ta-do es-pe-cia-li-za-ción ló-gi-cas in-de-pen-dien-te-men-te ge-ne-ral dis-po-ner dis-mi-nu-yen-do co-rrien-te 
rea-pro-ve-char de-sa-rro-llo ge-ne-ral Ins-ti-tu-te com-pu-ting pos-te-rior con-su-mi-da mi-llo-nes Fi-gu-ras cons-tan-te equi-va-len-cia ope-ra-cio-nes 
fac-to-ri-za-ción Res-pon-sa-bles Geo-Linked-Data reu-ti-li-za-ción co-rrec-tos trans-for-ma-ción bien con-si-guien-te des-cu-brien-do 
geo-rre-fe-ren-cia-ción pro-li-fe-ra-ción cum-pli-mien-to Ne-go-tia-ted uti-li-dad ma-nual es-pe-cia-li-za-ción ma-yo-ría mo-de-la 
des-cu-bri-mien-to rea-li-za-da pu-bli-can se-lec-cio-nar me-ca-nis-mos fun-cio-na-li-dad ve-ri-fi-car ines-pe-ra-dos re-gis-tro 
pro-pues-tos trans-pa-ren-te va-ria-bles bi-blio-te-ca bi-blio-te-cas adap-ta-bi-li-dad in-ter-co-ne-xión va-ria-ble vi-sua-li-zar 
ope-ra-ti-vo ge-ne-rar su-mi-nis-tra-dos va-li-da-ción su-mi-nis-tra-dores pro-pie-da-des hi-po-ni-mia re-cu-pe-ra-dos in-de-pen-dien-te-ment-te 
triun-fo reu-ti-li-za-ción ge-ne-ral apro-xi-ma-ción va-ria-bi-li-dad teó-ri-cas con-ti-nuo di-se-ño des-afor-tu-na-da-men-te In-dus-trial prin-ci-pios 
di-fe-ren-cia-das eva-lua-dos re-qui-si-tos co-la-bo-ra-cio-nes co-la-bo-ra-ción in-dus-tria-les Se-venth pon-de-ra-do re-po-si-to-rios 
ex-pe-ri-men-ta-les pu-bli-ci-tar co-la-bo-ra-ción bi-llo-nes Doc-to-ra-do re-so-lu-ción cul-tu-ra-les trans-cur-so Pa-rro-quias Lo-gics co-rrec-to 
sig-ni-fi-ca-do Aña-dien-do ope-ra-ción ma-ne-ra agi-li-zar con-fian-za per-so-na-les in-al-te-ra-dos re-la-ti-vos des-cri-bien-do 
se-ña-la-das vo-ca-bu-la-ries con-tex-tua-li-za-da Ba-rras cir-cuns-cri-bién-do-se for-ma-li-za-ción fa-ci-li-tan-do re-ali-men-ta-ción 
re-fe-ren-cias ma-ni-fes-tar-se CKAN lo-ca-li-dad dia-gra-ma ver-da-de-ra-men-te es-ta-ble-ci-mien-to con-si-de-ra ta-reas co-rrec-ta 
co-he-ren-te pos-te-rior-men-te apro-pia-das exac-tos va-rios pro-pues-to ex-pe-ri-men-to re-fe-ren-tes orien-ta-dos ha-bi-tual con-cre-tas 
con-fi-gu-ran afi-lia-ción Gi-ner va-lo-ra-ción res-pon-sa-bi-li-dad}   

%%Tablas
\addto\captionsspanish{
        \def\listtablename{\'Indice de tablas}%
        \def\tablename{Tabla}} 

%%%Math
\usepackage{latexsym}
\usepackage{amsmath}
\usepackage{amssymb}
\usepackage{amsthm}

\usepackage{algorithm}
\usepackage{algorithmic}
\usepackage{multirow}
\usepackage{rotating}


\newtheorem{theorem}{Theorem}[section]
\newtheorem{proposition}[theorem]{Proposición}
\newtheorem{lemma}[theorem]{Lema}
\newtheorem{definition}[theorem]{Definición}
\newtheorem{examples}[theorem]{Ejemplos}
\newtheorem{remarks}[theorem]{Remarks}
\newtheorem{corollary}[theorem]{Corolario}
\newtheorem{remark}[theorem]{Remark}
\newtheorem{example}[theorem]{Ejemplo}
\newtheorem{conjecture}[theorem]{Conjecture}
\newtheorem{note}[theorem]{Nota}



\newsavebox\FrameBox
\newenvironment{Frame}{%
  \par\setbox\FrameBox\hbox\bgroup\minipage{0.9\textwidth}\parskip\baselineskip\ignorespaces
}{%
  \endminipage\egroup\fbox{\box\FrameBox}\par
}

\newcommand{\si}{$\oplus$\xspace}
\newcommand{\no}{$\ominus$\xspace}
\newcommand{\na}{$\odot$\xspace}


%Extraer
\newcommand{\linkeddata}{\textit{Linked Data}\xspace}
\newcommand{\opendata}{\textit{Open Data}\xspace}
\newcommand{\lod}{\textit{Linking Open Data}\xspace}
\newcommand{\ogd}{\textit{Open Government Data}\xspace}
\newcommand{\datasets}{\textit{datasets}\xspace}
\newcommand{\dataset}{\textit{dataset}\xspace}
\newcommand{\provenance}{\textit{provenance}\xspace}
\newcommand{\trust}{\textit{trust}\xspace}
\newcommand{\egov}{\textit{e-government}\xspace}
\newcommand{\pusi}{\textit{Public Sector Information}\xspace}
\newcommand{\gd}{\textit{Government Data}\xspace}
\newcommand{\wod}{Web de Datos\xspace}
\newcommand{\wode}{\textit{Web of Data}\xspace}
\newcommand{\eproc}{\textit{\gls{e-Procurement}}\xspace}
\newcommand{\gld}{\textit{Government Linked Data}\xspace}

