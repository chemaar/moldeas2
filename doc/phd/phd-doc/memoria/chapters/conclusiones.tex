\section{Cumplimiento de Objetivos}
En primer lugar para realizar la validación del estudio realizado en este documento cabe valorar el grado 
de consecución de los objetivos fijados al inicio de este trabajo y que se han señalado en la 
Sección~\ref{objetivos}:

\begin{description}
\item [Panorámica de la contratación pública electrónica.] En concreto el objetivo número $1$ se 
definía como una tarea en la que se debía ``Estudiar, analizar y valorar las capacidades actuales del dominio de la contratación pública'', 
en el Capítulo~\ref{capitulo:eprocurement} se ha realizado una síntesis de las actuales tendencias en contratación 
pública electrónica, específicamente en el ámbito de la Unión \gls{Europea}, realizando una valoración de las acciones 
estratégicas y proyectos que se están llevando a cabo a nivel europeo y que en muchos casos se ven reflejadas 
en las acciones nacionales y regionales. La importancia de la contratación pública electrónica como proceso 
crítico dentro de las Administraciones Públicas ha quedado patente debido a las posibilidades de generación 
de servicios de negocio y las cantidades presupuestarias manejadas para la realización de servicios y obras 
a la ciudadanía y en consecuencia para el impulso del mercado. Por otra parte, es importante prestar atención 
a la problemática que surge en el momento de facilitar el acceso a la información de las licitaciones a nivel 
europeo como las características multing\"{u}es, la dispersión de información, la multiplicación de fuentes 
de información heterogéneas, etc., que provocan un descenso en la competitividad, ya que los proveedores de la Administración, 
dependiendo de la situación, son reducidos a un ámbito concreto. Con todo ello, se pone de manifiesto la necesidad de 
suministrar un entorno ágil y flexible para la gestión de la información y datos con los objetivos tecnológicos 
de impulsar la interoperabilidad e integración de servicios dentro de la propia Administración y por otra parte, 
mejorar la imagen corporativa de la misma en cuanto a transparencia y servicios a los ciudadanos y empresas. El repaso 
realizado en el capítulo mencionado consolida tanto los problemas como las soluciones y estrategias actuales proporcionando 
un dominio ampliamente motivado para la investigación y la innovación.

\item [Panorámica de \opendata, \linkeddata y Web Semántica.] Uno de los puntos clave de estudio e innovación en este trabajo 
consiste en la aplicación de los principios promulgados por la iniciativa de Web Semántica y en concreto de \linkeddata al 
dominio de la contratación pública electrónica. Además, teniendo en cuenta la tendencia actual de apertura de datos 
impulsada por el movimiento \opendata se ha realizado un conveniente repaso de la tecnología, iniciativas, proyectos, etc., 
relacionados con estos movimientos, a la vez que contrastado los beneficios, ventajas, etc., con el objetivo 
de adquirir una perspectiva realista de las aspiraciones de estas iniciativas y su impacto en los servicios actuales así 
como la trascendencia de su aplicación a un sector como la contratación pública electrónica. Esta panorámica permite 
obtener una visión general de estos enfoques y su posible potencial para la resolución de problemas de gestión e integración 
de información y datos en un espectro temporal desde la actualidad hasta medio/largo plazo.

\item [Ciclo de vida para \linkeddata.] Atendiendo al objetivo definido como ``Definir los métodos basados en semántica para producir, publicar, consumir y validar 
la información de los anuncios de licitación siguiendo las directrices de \opendata y \linkeddata'' y tras la revisión de los trabajos 
en estas áreas de conocimiento, se ha realizado una descripción de un ciclo de vida de datos enlazados que se alinea con los enfoques 
existentes, definiendo un ciclo de vida basado en la abstracción en 3 niveles de los procesos, métodos y tareas a realizar 
para la consecución cuantificable de la promoción de datos siguiendo las iniciativas de \linkeddata y \opendata.

\item [Sistema MOLDEAS.] La denominación de \gls{MOLDEAS} engloba tanto el ciclo de vida como las herramientas que se han desarrollado 
para dar soporte a los procesos contemplados en el ciclo de vida de datos enlazados. De esta manera se ha implementado un conjunto 
de componentes como activos experimentales para el cumplimiento del ciclo de vida y la demostración del consumo de datos enlazados. El 
sistema MOLDEAS se convierte de esta manera en un primer y gran esfuerzo para la provisión de herramientas de carácter genérico 
en el ámbito de \linkeddata, aunando tecnología ya existente y mejorando las mismas como por ejemplo ONTOSPREAD, y particularizando 
su aplicación al dominio de la contratación pública electrónica. Con todo ello, se completa el capítulo propio de ingeniería 
de este estudio y se da respuesta a los objetivos marcados de ``Definir los algoritmos y procesos para dar soporte a la aplicación de los métodos basados en semántica 
a la información de los anuncios de licitación'' e ``Implementar y reutilizar los componentes software necesarios para dar soporte a los métodos semánticos''.

\item [Aplicación de MOLDEAS a \eproc.] Una vez que se disponen de los datos necesarios sobre anuncios de licitación en general y los relativos 
a las clasificaciones de productos y organizaciones, la aplicación de MOLDEAS se ha reflejado para la promoción de datos con un enfoque 
cuantitativo y cualitativo, permitiendo contrastar los procesos definidos en el ciclo de vida. De esta manera se da cumplimiento 
al gran objetivo de ``Aplicar los métodos semánticos definidos al contexto de los anuncios de licitación pública''.  

\item [Experimentación y Validación.] Los experimentos realizados se concretan en 4 acciones diferentes:
\begin{itemize}
 \item Elaboración y diseño de tablas de validación de \linkeddata y \opendata de acuerdo a la documentación 
actual y la propia experiencia del autor y supervisor.
\item Contraste de la elaboración de un demostrador público con capacidad para el consumo de \linkeddata, con el objetivo de 
verificar que la información y datos pueden ser consumidos por nuevos servicios manteniendo y mejorando el comportamiento 
actual y sentando la base para un estudio profundo de un posible sistema experto basado en tecnología semántica 
para el dominio de la contratación pública electrónica.
\item Demostración del aumento de la expresividad mediante el uso de \linkeddata en sistemas de recuperación 
de información basados en vocabularios controlados.
\item Mejora del rendimiento de consultas en \gls{SPARQL} mediante la aplicación de determinadas características obtenidas 
tras la valoración de la documentación existente y la experimentación empírica.
\end{itemize}

Los resultados de los experimentos han sido validados y evaluados con el objetivo de suministrar una demostración 
científica al estudio realizado en este documento y cumplir con la tarea de ``Establecer un conjunto de prueba y validación de los componentes implementados para verificar la corrección, 
 validez y rendimiento de los métodos propuestos''.


\item [Impacto y difusión.] En cualquier proyecto o trabajo de investigación es absolutamente necesario 
dar difusión y crear concienciación de los avances realizados tanto a la comunidad científica como 
industrial. Por ello, esta tarea transversal se ha cubierto satisfactoriamente mediante la realización 
de distintas publicaciones, realizando la componente de investigación en un proyecto como 
``10ders Information Services'' en colaboración con empresas de servicios y tecnología y participando 
activamente en la comunidad investigadora mediante el establecimiento de colaboraciones con personas e 
instituciones de carácter internacional, que ha conllevado la elaboración conjunta de publicaciones y 
la pertenencia a comités de programas y revisión de artículos en diferentes revistas, conferencias 
y talleres. La justificación pormenorizada de estas actividades se ha señalado en los Apéndices~\ref{capitulo:impacto} y~\ref{capitulo:publicaciones}, 
cumpliendo así con el objetivo ``Difundir, formar y transferir la tecnología y conocimiento generado tanto a las comunidades científicas como industriales''.
 
\end{description}

Transversalmente a estos objetivos específicos, todo el trabajo realizado en las definiciones teóricas, desarrollo de activos experimentales 
y diseño de experimentos, se ha enmarcado bajo una apuesta fiel y constante por la aplicación y uso de los estándares 
dando así acogida a la meta de ``Promover el uso de estándares y la reutilización de 
información y modelos de conocimiento compartido''.

\subsection{Consecución de la Hipótesis de Partida}
La hipótesis de partida según la cual se planteaba la mejora en el acceso a la información 
contenida en los anuncios de licitación, tanto en términos cuantitativos como cualitativos, mediante 
métodos semánticos basados en aplicar y cumplir los principios de las iniciativas \opendata y \linkeddata, se ha logrado 
mediante la definición de un ciclo de vida para los datos enlazados basado en procesos, métodos y tareas que permiten 
la gestión avanzada de la información y datos provenientes de los anuncios de licitación.  

De esta forma, las fases de \eproc directamente implicadas en la publicación de información, como \textit{eNotification} y \textit{eAccess}, se 
ven favorecidas por las prácticas basadas en semántica y datos enlazados, suministrando un entorno estándar para el modelado 
de datos y publicación de los mismos. La experimentación y evaluación realizada permiten asegurar, a través de una serie 
de criterios, que se cumplen las directrices de \linkeddata y \opendata con la consiguiente mejora cualitativa en el 
acceso a la información de los anuncios de licitación. De la misma forma, la promoción mediante datos enlazados 
de las clasificaciones de productos, especialmente el CPV 2008, y su posterior reconciliación y enlazado con otros 
esquemas de productos y servicios, permite ampliar el vocabulario de entrada, aumentando la expresividad 
de las consultas de forma cuantitativa respecto al número de términos posibles para acceder a la información publicada. 

Adicionalmente, el uso de datos enlazados y de tecnología semántica aborda la problemática relativa a los siguientes puntos:
\begin{itemize}
 \item Dispersión de la información. La difusión de los anuncios de licitación se realiza por distintos medios, en los que 
se utilizan diferentes formatos de publicación y se publica información de distinto carácter. El enfoque propuesto 
por \linkeddata mitiga enormemente esta situación ya que permite establecer un modelo de información, de datos y de identificación 
único y homogéneo para los recursos. También, cabe destacar que el uso de datos enlazados no es intrusivo con las grandes bases de datos corporativas, permitiendo una 
adopción de estos principios de forma sencilla.
 \item Múltiples fuentes de datos. Uno de los principales problemas de la publicación de información en diferentes fuentes reside 
en la necesidad de identificar unívocamente los recursos de información. En el campo de la contratación pública electrónica esta situación 
se convierte en crítica ya que las oportunidades de negocio presentes en los anuncios de licitación son claves para el sector 
económico y empresarial de un país. Por ello, la utilización de técnicas que permitan identificar y acceder a la información 
y los datos en condiciones de certeza, favorece la gestión y clasificación de información, evitando que los agentes implicados 
en los procesos de contratación pública se vean saturados por información no relevante o duplicada. Los principios propuestos 
por los datos enlazados solventan esta situación mediante la utilización de URIs, permitiendo la identificación única 
de recursos de información. Finalmente, la utilización de técnicas semánticas para la reconciliación de entidades suministra 
el entorno tecnológico necesario para establecer los enlaces entre recursos de información similares, favoreciendo 
la reutilización de información de forma eficiente. 

 \item Heterogeneidad de los formatos de publicación y explotación. Debido a la casuística propia de cada licitador y su entorno 
de trabajo, se genera información y datos en múltiples formatos, tanto para su publicación como para su explotación. La unificación 
de la información y datos en un modelo estándar, como RDF, permite solventar esta diversidad de formatos, siendo no intrusiva con esfuerzos 
anteriores.
 \item Multiling\"{u}ismo y multiculturalidad. Las iniciativas de \linkeddata y Web Semántica se consideran inherentemente internacionales, 
proporcionando así un entorno de gestión de la información y datos basado en el uso de distintos idiomas y adaptado a las condiciones 
particulares de cada región. Por lo tanto, la aplicación de datos enlazados a dominios multiling\"{u}es es un factor 
clave para lograr la internacionalización en el contexto global de ejecución de aplicaciones y servicios.
\end{itemize}


\section{Principales Aportaciones}
\subsection{Aportaciones Científicas}
El estudio desarrollado en este trabajo sobre la aplicación los principios de \opendata y las tecnologías semánticas, concretamente 
mediante la aplicación de la iniciativa \linkeddata al dominio de la contratación pública electrónica, ha servido para la obtención 
de las siguientes aportaciones desde el punto de vista científico:
\begin{description}
 \item [Repaso del estado actual de \eproc y aplicación de tecnologías semánticas.] En múltiples ocasiones la selección de una tecnología 
para ser aplicada sobre un dominio es una decisión crítica ya que en un entorno de producción no es posible en muchas ocasiones dedicar 
esfuerzo y tiempo a comprobar iniciativas basadas en la investigación e innovación. Por ello, el estudio realizado conlleva un valor 
adicional desde un punto de vista estratégico para el campo de la contratación pública electrónica, al haber conjugado el repaso 
de este dominio y enlazarlo con el trabajo relacionado bajo el paradigma de la Web Semántica y \linkeddata. 

\item [Ciclo de vida para \linkeddata.] Actualmente con el éxito y las expectativas generadas en torno a los datos enlazados abiertos, se ha 
puesto de manifiesto la necesidad de gestionar el ciclo de vida de los datos enlazados y no limitarse únicamente a la exposición de 
las bases de datos existentes cumpliendo unas ciertas directrices. Esta situación queda reflejada en las múltiples guías que aparecen 
a lo largo de la geografía de cómo abrir y enlazar datos, en algunos casos utilizando un enfoque \textit{top-down}, como en el Reino 
Unido, y en otros casos como España aplicando un sistema \textit{bottom-up}. El objetivo final es disponer de una serie de guías 
y buenas prácticas para minimizar el esfuerzo en la apertura y enlazado de datos. De la misma forma, en las distintas acciones de 
investigación como proyectos, actividad en universidades y centros tecnológicos, etc., se está suministrando un gran cantidad de información 
sobre cómo abordar esta situación. Sin embargo, la definición de un claro ciclo de vida basado en la abstracción de procesos, métodos 
y tareas no se ha realizado de forma completa y los enfoques existentes si bien se postulan como recetas o guías prácticas, todavía 
no han alcanzado un nivel de normalización a través de alguna institución como el \gls{W3C}. Con la meta fijada en gestionar el ciclo de vida 
de datos enlazados, reaprovechar el conocimiento generado en las propuestas anteriores y la experiencia particular en el desarrollo 
de estas actividades, se ha definido un ciclo de vida que se ha aplicado a la información y datos concernientes a los anuncios 
de licitación públicos.

\item [Sistema MOLDEAS.] La denominación bajo este nombre cubre el desarrollo de activos experimentales y la aplicación de tecnología 
existente, con mejoras en la misma, para la realización de los procesos del ciclo de vida de datos enlazados en el dominio de los 
anuncios de contratación pública. De esta manera se han definido modelos e implementado algoritmos con las siguientes capacidades:
\begin{itemize}
 \item Modelo de información y datos para los anuncios de licitación pública, el catálogo de clasificaciones de productos y las organizaciones basado 
en estándares y vocabularios comunes.
 \item Producción de datos enlazados de acuerdo a los modelos definidos.
 \item Reconciliación de entidades específica para la información y datos anteriores.
 \item Establecimiento de un modelo de publicación de datos enlazados abiertos.
 \item Implementación de un conjunto de componentes para el consumo y explotación de datos 
desde una aplicación externa aplicando tecnología existente y mejorando la propia como ONTOSPREAD, propagación sobre instancias no 
sólo sobre clases.
 \item Diseño de criterios de validación de los datos generados y evaluación semi-automática de los mismos.
 \item Demostración de la mejora cuantitativa y cualitativa en el acceso a la información y datos concernientes 
a los anuncios de licitación.
\end{itemize}
\item [Generación de \textit{know-how}.] La realización de este trabajo ha permitido la generación de nuevo conocimiento sobre 
un campo tan relevante como la administración electrónica y en concreto sobre el proceso de contratación pública. De esta forma, 
es posible abordar problemas similares con el enfoque de \gls{MOLDEAS}, ya que si bien su escenario de aplicación es la contratación 
se puede extrapolar y aplicar a otros dominios. El modelo de trabajo, investigación e innovación realizado ha permitido 
la creación de soluciones genéricas que se han demostrado efectivas en un dominio concreto pero sin menoscabar su relevancia 
y aplicación a otros sectores, esta flexibilidad permite ofrecer información detalle para futuros esfuerzos en otros campos.
\end{description}
\subsection{Aportaciones Tecnológicas}
Desde un punto de vista tecnológico la realización de este trabajo se ha manifestado en la utilización de múltiples herramientas 
provenientes de distintos proveedores para la implementación de diversas tareas. En concreto, las aportaciones tecnológicas 
en este ámbito se sintetizan en las siguientes acciones:

\begin{itemize}
 \item Adaptación de la entrada de la biblioteca Apache \gls{Mahout} para el consumo de datos enlazados.
 \item Adición de nuevas capacidades a la biblioteca ONTOSPREAD.
 \item Realización de algoritmos de reconciliación de entidades basados en Apache \gls{Lucene} y \gls{Solr}.
 \item Creación de un API para el consumo y explotación de datos enlazados mediante la aplicación de diferentes 
métodos de expansión de consultas.
 \item Creación de un sistema de test y validación de datos enlazados semi-automático.
\end{itemize}

\section{Conclusiones Científicas}
El estudio de los datos enlazados abiertos en el campo de las licitaciones públicas ha puesto de manifiesto las necesidades 
de este entorno y la necesidad de realizar una investigación profunda para impulsar la contratación pública electrónica dada 
su relevancia e impacto en la sociedad. Por ello, el enfoque basado en tecnologías semánticas y datos enlazados abiertos 
permite la culminación práctica de ciertas necesidades como la mejora de acceso y la reutilización de la información. Sin embargo, 
la conversión de un entorno tan amplio debe tomarse como una actividad iterativa, en la cual con la aplicación de buenas 
prácticas se consiga dar respuesta y solución a los inconvenientes que presenta. Una vez que la problemática de la contratación 
pública electrónica está motivada convenientemente para que las tecnologías basadas en semántica puedan influir y mejorar 
este entorno, surgen nuevas situaciones que se han reflejado a lo largo del trabajo y que a continuación se detallan.

\begin{itemize}
 \item Todos los procesos implicados en el tratamiento de datos enlazados deben asegurarse suministrando 
mecanismos para su validación con el objetivo de facilitar su reutilización posterior en condiciones 
de ausencia de incertidumbre.
  \item El uso de estándares es clave para minimizar los problemas de integración e interoperabilidad.
 \item El tratamiento de grandes cantidades de datos conlleva prestar especial atención a los algoritmos diseñados y 
a la construcción de consultas sobre los mismos.
 \item La necesidad de modelar la información y datos para su posterior reutilización en un dominio extenso debe realizarse 
con el suficiente grado de especificidad, pero teniendo presente la posibilidad de extensibilidad, por lo que es conveniente 
no realizar grandes modelos poco usables y que tan sólo el autor o autores pueden manejar. En este sentido la estrategia 
a seguir debe basarse en la generación de un marco de trabajo común y la creación sostenible y escalable de conocimiento 
con el objetivo de impulsar su reutilización.
 \item Los principios de \linkeddata y \opendata verdaderamente ayudan a favorecer la reutilización de información si siguen 
unas directrices de producción, publicación y consumo adecuadas. Sin duda, la realización práctica de la Web Semántica 
se ve reflejada bajo la iniciativa de \linkeddata y su éxito actual, queda patente en general y en el caso objeto de estudio 
de este documento.
 \item El maremágnum de tecnología, enfoques, algoritmos, vocabularios, conjuntos de datos, etc., dentro de la iniciativa de 
\linkeddata convierte la toma de ciertas decisiones y la adopción de soluciones en un trabajo tedioso en el cual la experiencia 
cobra una especial relevancia. La necesidad de catalogación y valoración de todos estos trabajos y esfuerzos se hace cada 
vez más patente.
 \item Existen tareas como la reconciliación de entidades que todavía se hayan en una etapa temprana de desarrollo y en las que son 
posibles múltiples aportaciones, ya que en la actualidad se basan principalmente en técnicas de procesamiento de lenguaje natural.
 \item El uso de datos enlazados permite mejorar el acceso a la información facilitando una mayor expresividad en la realización 
de consultas sobre grandes conjuntos de datos.
 \item El rendimiento, la completitud, el tiempo de ejecución finito y la consulta de grandes bases de datos en un entorno distribuido, es una línea de investigación especialmente 
relevante y abierta en el campo de la \wod.
 \item Desde el punto de vista de la gestión de la información y datos, la Web Semántica se ha convertido en una pieza clave pero 
la creación de algoritmos explotando las capacidades de un entorno mejor informado todavía no se ha impulsado convenientemente 
y se reduce a la adaptación a algoritmos y técnicas perfectamente probadas como la minería de datos o el procesamiento de lenguaje 
natural. La consecuencia principal de esta situación supone que el esfuerzo realizado para la mejora de la información 
y datos finalmente no se ve recompensada por algoritmos más avanzados.
 \item La explotación de la información mediante técnicas de razonamiento no siempre es necesaria y dependiendo del dominio, como en el caso 
de la publicación de información y datos de los anuncios de licitación, es conveniente centrar los esfuerzos en el modelo de datos, 
para su posterior reutilización que en construir un base lógica muy sólida.
\item Los procesos administrativos dentro de la administración electrónica deben necesariamente aprovechar las ventajas 
de las tecnologías de información para ser más eficientes, la tendencia en los últimos años y la planificación estratégica 
para los próximos años así lo corrobora, más aún teniendo en cuenta el carácter global de las comunicaciones e intercambio 
de información.
\item Las primeras versiones de datos enlazados abiertos liberados desde las distintas organizaciones, entidades, etc., tendían 
a basar su éxito en el número de tripletas \gls{RDF}, generando en muchos casos datos superfluos. Una vez superada esta fiebre sobre 
la cantidad de tripletas RDF, el esfuerzo se centra en la publicación de los datos necesarios y suficientes bajo condiciones 
de calidad que aseguren su reutilización.
\end{itemize}

Estos puntos clave extraídos del trabajo desarrollado deben unirse necesariamente a los generados tras la realización y evaluación 
de los experimentos realizados que a continuación se relacionan:
\begin{itemize}
 \item El uso de datos enlazados permite aumentar la expresividad de las consultas facilitando el acceso a la información 
desde un punto de vista cuantitativo y cualitativo. Esta situación se refleja tanto en entornos utilizando vocabularios 
controlados, como la contratación pública, como en entornos abiertos, la propia Web de Datos.
\item El beneficio real de los datos enlazados abiertos reside en la representación de información y datos mediante un modelo estándar (\gls{RDF}) que permite el uso 
de lenguajes (\gls{SPARQL}) y protocolos (\gls{HTTP}) estándar para su acceso, facilitando la recuperación de información.
\item El enlazado de datos consta de una variable de incertidumbre que si bien es aceptada por la comunidad, debe ser tenida en cuenta 
para la realización de aplicaciones de carácter crítico.
\item En la actualidad tanto la información como los datos disponibles son abiertos en su mayor parte e incluso son utilizados 
para la construcción de servicios comerciales pero el esfuerzo requerido para llevar a cabo estos productos 
se ve reflejado en el precio de los mismos, por lo que el uso de datos enlazados simplifica y favorece la reutilización de información.
\item En muchos casos no es necesario llegar a un modelo de 5 $\star$, ya que esto provoca la necesidad de adaptación de los 
desarrolladores a estas prácticas, por lo que la tendencia actual es por una parte generar los datos con el mayor nivel posible 
de estrellas pero facilitando la labor para su consumo, por ejemplo utilizando como formato de representación JSON.
\item La aplicación de patrones de diseño en la planificación de los datos a transformar permite dar una respuesta 
homogénea a problemas recurrentes de representación de información y datos.
\item La creación de un sistema experto para la recuperación de información de los anuncios de licitación es un dominio 
en el cual se puede profundizar variando una gran cantidad de parámetros, tanto de configuración de los algoritmos como 
valorando la implicación de introducir nuevas variables de información.
\item La división de consultas en SPARQL sobre grandes conjuntos de datos en otras más pequeñas y su distribución, permite 
la mejora del rendimiento ya que se simplifican dos factores importantes: 1) el número de tripletas encajadas y 2) el 
tamaño del conjunto de datos sobre los que la consulta es ejecutada.
\item El trabajo desarrollado en la nueva especificación de SPARQL da respuesta a algunos de los problemas o ausencias de funcionalidad encontradas 
en la especificación inicial, como el uso de agregados, \textit{subqueries} o federación de consultas. Claramente la nueva especificación 
se acerca a lenguajes de consulta ya maduros como \gls{SQL}, este esfuerzo sin duda es clave para la adopción de datos enlazados.
\item Existen características en SPARQL deseables como la priorización del encaje de tripletas para la ordenación de resultados 
que todavía no se han incluido en las especificaciones, pero si están presentes en productos como Virtuoso y que resultan 
de gran valor para la construcción de aplicaciones.
\end{itemize}

Desde un punto de vista de la administración electrónica, la elaboración de este trabajo constata los siguientes puntos clave en 
el dominio de la contratación pública electrónica:
\begin{itemize}
 \item La información y datos de carácter público sirven como impulso para la generación de servicios y oportunidades 
de negocio que permiten promover el tejido empresarial de una región. Su apertura y reutilización es clave para la actual sociedad 
en la información, en la que los servicios electrónicos y de valor añadido permiten una interacción eficiente 
entre los agentes implicados, ya sean personas o empresas.
\item La administración electrónica y los servicios en el campo de \textit{eGovernment} son claves para la modernización 
de la Administración, facilitando un uso eficiente de los recursos.
\item La apertura de datos permite dar respuesta a la emergente legislación sobre transparencia en el sector 
público, favoreciendo un entorno de confianza entre los ciudadanos y los gestores de los recursos públicos.
\item La contratación pública es uno de los procesos administrativos más relevantes debido 
a su impacto en la sociedad y en el tejido empresarial de una región. Por ello, las mejoras aplicables 
en este contexto, como el uso de medios electrónicos, son claves para la mejora en la eficiencia de este proceso, 
generando así un entorno competitivo transfronterizo con ventajas para todos los agentes implicados.
\end{itemize}

La extracción de estas conclusiones desde un punto de vista científico constata la conveniencia de la aplicación de los datos enlazados 
abiertos al dominio de la contratación pública electrónica, así como abre diversas líneas de investigación tanto para la 
propia administración electrónica como para la investigación a realizar en este campo.

\section{Conclusiones Tecnológicas}
El desarrollo de activos experimentales para dar soporte a los procesos indicados en el ciclo de vida de datos enlazados 
ha puesto de manifiesto tanto ventajas como inconvenientes en la tecnología actual para afrontar estas tareas, entre las cuales 
se destacan:

\begin{itemize}

 \item De la misma forma que el número de vocabularios, catálogos de datos, etc., constituyen una fuente de reutilización 
relevante pero cuya clasificación no permite a usuarios no técnicos la selección de los mismos, también queda patente en la tecnología 
asociada a los datos enlazados, ya que la diversidad de herramientas constituye un amplio abanico de oportunidades en las que 
la incertidumbre sobre cuál utilizar, da lugar simplemente a indagar las aplicadas en los proyectos de éxito sin tener en cuenta 
más factores. Esto implica un cierto grado de monopolio ya que la selección de tecnología se convierte en una simple consulta, 
sin realizar una verdadera evaluación pero por otro lado la aplicación de este tipo de tecnología y herramientas justifica 
cualquier juicio posterior por terceros.

\item Durante mucho tiempo el uso de tecnología relacionada con semántica se convertía en un proceso de depuración, esta cuestión 
ha sido parcialmente solucionada debida a la inversión realizada a través de proyectos de investigación y a la participación de grandes 
proveedores de software en su desarrollo.

\item La proliferación de múltiples utilidades para la realización de diversas tareas conlleva una falta de integración de las mismas 
por lo que la intervención manual es inevitables. Plataformas integradas como la propuesta en el proyecto LOD2 o bien el 
\textit{Linked Media Framework} permiten mejorar la adopción de la tecnología.

\item Las actuales herramientas para la gestión y explotación de información no están preparadas para trabajar intrínsecamente 
con datos enlazados o con semántica, por lo que el esfuerzo para el consumo de datos enlazados y la creación de nuevos 
servicios y aplicaciones reutilizando soluciones existentes conlleva un esfuerzo adicional en el cual, en la mayoría de los casos, 
se pierde parcialmente las ventajas de reutilizar información y datos provenientes de modelos compartidos.

\end{itemize}

\section{Futuras Líneas de Investigación y Trabajo}
Las posibilidades de abrir nuevas líneas de investigación que tomen como semilla el trabajo realizado en este estudio son múltiples 
y se pueden abordar desde distintos puntos de vista:
\begin{itemize}
 \item Las concernientes a la Web Semántica y \linkeddata en general.
 \item Las referentes a la administración electrónica y en particular al proceso administrativo de contratación pública electrónica.
\end{itemize}

\subsection{Visión Científica}
La posibilidad de emprender acciones en una línea de investigación es múltiple, la selección de las más interesantes no es una tarea sencilla y puede estar
condicionada por diferentes factores: dificultad, ventaja estratégica o disponer de un escenario de aplicación correcto. Entre las más relevantes 
surgidas tras la realización de este trabajo se encuentran:

\begin{itemize}
 \item Realización de un sistema de catalogación de vocabulario y conjuntos de datos basado en diferentes métricas que permitan 
tanto a un usuario de dominio como a un desarrollador seleccionar áquel más conveniente a sus necesidades.
 \item Mejora de los algoritmos de reconciliación de entidades para llevar a efecto uno de los puntos clave de \linkeddata como 
la identificación de recursos iguales, tanto para la realización de consultas como para su procesamiento mediante procesos 
de razonamiento.
 \item Establecimiento de un conjunto de métricas que permitan establecer y asegurar la calidad y procedencia de los datos.
 \item Mejora del rendimiento de las consultas federadas sobre grandes bases de datos en continuo crecimiento y evolución. Creación 
de un \textit{benchmark} basado en el conjunto de datos concerniente a los anuncios de licitación.
 \item Adaptación y mejora de las técnicas y algoritmos basados en semántica a entornos con características de tiempo real.
 \item Estudio de la aplicación de la investigación e innovación a otras etapas de la contratación pública electrónica, como la 
decisión de ofertas, y en general a la administración electrónica.
 \item Estudio de las posibilidades de realización de un sistema experto específico para la recuperación de información 
de anuncios de licitación teniendo en cuenta la mayor cantidad de descriptores posibles.
 \item Estudio de los operadores de agregación que permiten establecer un orden en la recuperación de documentos.
 \item Mejora de la capitalización del conocimiento experto en el dominio de contratación pública electrónica.
 \item Estudio de las posibilidades de creación de un sistema predictivo de anuncios de licitación de acuerdo a 
la información histórica de los anuncios de licitación.
 \item Investigación en aspectos relativos al consumo de datos como base para el impulso de los datos enlazados:
\begin{itemize}
 \item Mejora de la escalabilidad, tanto para el indexado como para la extracción de datos a gran escala.
 \item Procesamiento de consultas federadas de forma eficiente.
 \item Búsqueda sobre fuentes de datos heterogéneas.
 \item Descubrimiento automático de \datasets.
 \item Gestión de \datasets dinámicos como los provenientes de sensores, sistemas reactivos, etc.
 \item Calidad de los datos: procedencia, valores, etc.
 \item Usabilidad en la interacción con datos enlazados.
\end{itemize}
\item Comunicación de los resultados obtenidos tanto a la comunidad científica como a los principales organismos implicados 
en la gestión de la compra pública a nivel regional, nacional y europeo, suscitando en consecuencia, la aplicación del conocimiento generado 
a otros dominios estratégicos en el sector público.
\end{itemize}

\subsection{Visión Tecnológica}
Desde un punto de vista tecnológico, las mejoras y líneas de actuación futuras se centran en dos grandes grupos: 1) las referentes 
al sistema \gls{MOLDEAS} como demostrador público y 2) las concernientes a tecnología de base para la Web Semántica y \linkeddata. Es conveniente, 
por tanto, destacar las siguientes:
\begin{itemize}
 \item Mejora del sistema de consumo de datos enlazados de MOLDEAS, personalización y prueba intensiva de los algoritmos disponibles.
 \item Mejora del sistema de visualización y consumo de datos enlazados desde el punto de vista del usuario final.
 \item Continuación del desarrollo del sistema de validación de datos enlazados. En este sentido se ha planificado la realización de 
acciones determinadas, como un proyecto fin de carrera para impulsar su desarrollo a corto/medio plazo.
 \item Contribuir con nuevas herramientas a la comunidad de \linkeddata para la mejora del consumo de datos enlazados, desde 
lenguajes de programación y reutilización sencilla de información y datos. Por ejemplo, facilitando la traslación de los datos 
provenientes de un repositorio \gls{RDF} a objetos de un modelo de negocio y viceversa.
 \item \ldots
\end{itemize}

Como se ha podido comprobar las líneas de actividad son múltiples tanto desde un punto de vista científico como tecnológico. Evidentemente 
también es necesario considerar dos factores importantes tras la realización de una investigación profunda en un dominio concreto:
\begin{itemize}
 \item Capitalización del conocimiento y de la propiedad industrial e intelectual mediante la realización de patentes y de publicaciones.
 \item Creación de nuevas oportunidades de negocio y servicios a través de la constitución de \textit{spin-off}s y \textit{startup}s que 
culminen la cadena de valor de la investigación y desarrollo, mediante la transferencia de tecnología a empresas de base tecnológica 
y la generación de nuevos servicios de negocio. En este sentido, se tiene presente tanto la transferencia de tecnología semántica, ya han surgido 
múltiples empresas tanto a nivel nacional como internacional que hacen uso comercial de esta tecnología, como la implementación de servicios 
de valor añadido explotando los datos abiertos enlazados, por ejemplo capitalizando la información sobre organizaciones. Si bien la competencia 
en el mercado constituye una barrera de entrada, la adición de semántica supone un avance respecto a los actuales servicios, por lo que 
es posible crear mejores aplicaciones, más informadas, e impulsar la integración e interoperabilidad de las mismas.
 \item Continuación de la investigación, desarrollo e innovación mediante la realización de propuestas a los principales 
programas de investigación competitivos para que la aplicación y extensión del conocimiento generado siga evolucionando, 
con posibilidad de resolver la problemática presente en distintos dominios.
\end{itemize}

Finalmente, con el emergente flujo de información y datos constante y heterogéneo, las técnicas de procesamiento eficiente y explotación de la información 
y datos para la creación de nuevos servicios, constituye una línea de actividad desde diferentes puntos de vista, que la semántica y los datos 
enlazados son capaces de abordar en condiciones de éxito.
