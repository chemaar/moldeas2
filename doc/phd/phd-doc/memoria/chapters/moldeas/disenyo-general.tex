En un sentido amplio, dado un problema y un dispositivo en el cual resolverlo, es necesario suministrar un método preciso de respuesta a la casuística 
planteada en el problema, de acuerdo al dispositivo objetivo, tiene la denominación de \textit{algoritmo}. El diseño de algoritmos\cite{Vallecillo} 
tiene dos componentes importantes:

\begin{enumerate}
  \item El primero se refiere a la búsqueda de métodos o procedimientos, secuencias
finitas de instrucciones adecuadas al dispositivo establecido, que permita resolver el problema.  
\item  El segundo permite medir de alguna forma el coste (en tiempo y recursos) de consumo de un algoritmo con el fin de encontrar la
solución, ofreciendo la posibilidad de comparar distintos algoritmos que resuelven un mismo problema.
\end{enumerate}

Una vez que se dispone de un algoritmo que funciona correctamente, es necesario
definir criterios con el objetivo de medir su rendimiento o comportamiento. Estos criterios se centran principalmente en su 
simplicidad y en el uso eficiente de los recursos. La sencillez es una característica sensiblemente interesante en el diseño 
de algoritmos, facilitando su verificación, el estudio de su eficiencia y el mantenimiento. De ahí que muchas ocasiones se priorice la simplicidad y legibilidad 
del código frente a alternativas más crípticas y eficientes del algoritmo. Por otra parte, el uso eficiente de los recursos suele medirse en función de dos
parámetros: el \textit{espacio}, es decir, memoria utilizada, y el \textit{tiempo}, unidades de tiempo de ejecución. En ambos casos, 
se hace referencia a los costes que supone la búsqueda de la solución al problema planteado mediante un algoritmo, además estos dos parámetros 
son utilizados para una posible comparación ulterior de los algoritmos entre sí.

El tiempo de ejecución de un algoritmo dependerá de diversos factores como los datos de entrada suministrados, la calidad del código
generado por el compilador para crear el programa objeto, la naturaleza y rapidez de las instrucciones máquina del procesador concreto que ejecute el programa, y la
complejidad intrínseca del algoritmo.  Existen dos tipos posibles de estudio sobre el tiempo:

\begin{enumerate}
\item  Áquel que proporciona una medida teórica (a priori), consistiendo en la obtención de una función que acote 
(inferior o superiormente) el tiempo de ejecución del algoritmo para unos valores de entrada determinados.
\item  Áquel que ofrece una medida real (a posteriori), consistiendo la medición del tiempo de ejecución del algoritmo para unos valores de entrada determinados 
y un entorno de ejecución particular.
\end{enumerate}

Ambas medidas son importantes puesto que si bien la primera ofrece estimaciones
del comportamiento de los algoritmos de forma independiente del ordenador en donde serán implementados y sin necesidad de ejecutarlos, 
la segunda representa las medidas reales del comportamiento del algoritmo. 

\subsection{Consideraciones sobre Diseño de Programas}\label{consideraciones-diseno}
El objetivo de implementación del sistema \gls{MOLDEAS} recae en suministrar parcialmente soporte 
a los distintos procesos implicados en el ciclo de vida de datos enlazados, 
si bien algunas tareas se realizan mediante el uso de ciertas herramientas, existen otras que deben ser parametrizadas e implementadas 
para ofrecer un entorno homogéneo para el manejo de la información y datos 
provenientes de los anuncios de licitación. La separación de responsabilidades 
entre los distintos componentes se realiza de acuerdo a su funcionalidad, de esta manera 
es posible efectuar cambios transparentes en los componentes sin que los otros sean 
involucrados en el proceso de cambio: implementación, prueba y empaquetamiento. Por ello, 
se han definido los interfaces de comunicación entre los mismos como un API para que 
cualquier futuro desarrollo se apoye en este \textit{framework} para añadir nuevas 
funcionalidades. Las claves para un diseño abierto de un \gls{API} 
coinciden en muchos sentidos con los de un lenguaje de programación~\cite{Interpretes}:

\begin{description}
\item[Concisión notacional:] el API deberá proporcionar un entorno con un nivel
de detalle adecuado: interfaces claras, simples, unificadas etc. Las posibles 
ampliaciones sobre el \textit{framework} de \linebreak \textit{MOLDEAS} deben resultar sencillas
y no presentar inconvenientes para que el programador comprenda y extienda su diseño.

\item[Ortogonalidad:] la funcionalidad del API debe suministrar el mecanismo
adecuado para combinar nuevos componentes y rechazar algunos de los ya
presentes. Por ejemplo la adición de nuevas restricciones no debe suponer una
recodificación del código del algoritmo.

\item[Abstracción:] el diseño del API debe basarse en el uso de técnicas como
los patrones de diseño e interfaces, favoreciendo la abstracción de algoritmos.

\item[Seguridad:] los algoritmos implementados deben tener puntos obligatorios de restricciones
para verificar por ejemplo su terminación aunque se añadan nuevas
restricciones. 

\item[Expresividad:] el API debe ser lo suficientemente ``rica'' como para que
nuevas ampliaciones puedan ser formuladas de forma sencilla de acuerdo a la información 
y datos presentes en los anuncios de licitación y en su modelo formal. 

\item[Extensiblidad:] el API debe basarse en técnicas de programación que
favorezcan la adición de nuevas características y su adaptación para nuevas
configuraciones del algoritmo.


\item[Portabilidad:] el lenguaje seleccionado para proporcionar estas técnicas
debe poseer esta característica.

\item[Eficiencia:] en cualquier implementación de un algoritmo o conjunto de los mismos esta propiedad es fundamental y
aunque el API diseñado, atendiendo a las características anteriores pueda
sobrecargar la ejecución básica de los procesos, su penalización en tiempo no es tan alta 
como para descartar los principios de diseño anteriores.

\item[Librerías e interacción con el exterior:] la ejecución de las funcionalidades 
provistas deberá ser independiente del lenguaje de ontologías utilizado, así se aisla el
conjunto de técnicas de la fuente de conocimiento favoreciendo el uso del
cualquier lenguaje de formalización.

\item[Entorno:] para facilitar la depuración de los algoritmos realizados se
proveerá un entorno gráfico en el cual poder visualizar la ejecución.
\end{description}

Además de estas principios generales para el diseño del API, hay que tener en
cuenta: 

\begin{itemize}
  \item El entorno de ejecución puede ser una aplicación web, por lo que se
  deberá tener en cuenta en el diseño para que se pueda integrar fácilmente con
  \textit{frameworks} como Spring.
  \item La mayoría de las aplicaciones utilizando Web Semántica utilizan el
  lenguaje Java, por lo tanto, este será el lenguaje seleccionado en su última versión.
\end{itemize}
 
\subsection{Patrones de Diseño}\label{patrones}
Con el objetivo de cumplir los criterios mencionados anteriormente, el sistema \gls{MOLDEAS} se basará en la aplicación 
intensiva de patrones de diseño\cite{Gamma,CoreJ2EEPatterns} como buena práctica 
de programación. Normalmente estos patrones indican la interacción que ha de realizarse entre los distintos elementos 
que participan en el problema, así para cada uno se cuenta con un conjunto de objetos, cada uno de los cuales realizará una función proveyendo servicios a 
los demás objetos implicados. Los patrones de diseño proponen soluciones con distintas características: 
elegantes, modulares, escalables y flexibles. Una posible definición de los mismos 
se presenta a continuación:

\begin{Frame}
\textit{Los patrones de diseño representan soluciones para problemas recurrentes en la ingeniería del software}.
\end{Frame}

En general, se trata de soluciones estándar para un problema habitual en programación que utiliza 
distintas técnicas para la flexibilización del código, tratando al mismo tiempo de satisfacer 
ciertos criterios no funcionales. Se suelen asimilar a una estructura determinada de implementación 
que cumple una finalidad concreta y permite describir ciertos aspectos de un programa.

No obstante, pese a que los patrones ofrecen buenas soluciones puede que en algún caso 
resulte contraproducente el uso de los mismos, el empleo de estas técnicas debe realizarse, 
por tanto, en el momento justo. La problemática reside en establecer cuándo es adecuada su aplicación, así se pueden citar varios criterios:
\begin{itemize}
    \item El código del programa ha crecido exponencialmente.
    \item Las clases de un programa aglutinan código semánticamente que no corresponde con su funcionalidad.
    \item Se deben realizar pruebas unitarias de clases.
    \item El diseño del programa es altamente complejo y las relaciones entre las distintas 
clases es opaca.
   \item La comunicación con otros desarrolladores ha decaído debido a la complejidad del código.
\end{itemize}

El libro \textit{The Ganf of Four (Gof)}~\cite{Gamma} fue pionero en introducir estas técnicas de 
programación caracterizando las mismas en tres niveles:
\begin{description}
    \item [Patrones creacionales:]se utilizan para la creación e inicialización, se pueden destacar: \textit{Abstract Factory}, \textit{Builder}, \textit{Factory Method}, \textit{Prototype} o
    \textit{Singleton}.
    \item [Patrones estructurales:]su objetivo es separar la interfaz de
    operaciones de la implementación. Tratan de organizar cómo las clases y
    objetos se agrupan para generar estructuras y organizaciones más grandes.
    Por ejemplo: \textit{Adapter}, \textit{Bridge}, \textit{Composite}, \textit{Decorator} o \textit{Facade}.
    \item [Patrones de comportamiento:] describen la comunicación entre los
    objetos.  \textit{Chain of Responsability}, \textit{Command}, \textit{State}, \textit{Strategy} o \textit{Visitor} son
    ejemplos de este conjunto.
\end{description}

En combinación con los patrones de diseño, se encuentra la técnica de \textit{refactoring}~\cite{Fowler1999}, 
utilizada para reestructurar y refinar el código fuente (en muchos casos aplicando un determinado patrón) 
sin alterar la funcionalidad o comportamiento externo del mismo.



