La metodología de la Programación Extrema~\cite{XP} (\gls{XP}), que ha servido de guía para el desarrollo 
del sistema MOLDEAS, hace hincapié en la importancia de las pruebas. La realización de ciclos de desarrollo breves
(incorporando la funcionalidad perseguida en pequeñas versiones totalmente funcionales) habilita la posibilidad de 
concentrarse en la validación, en cada momento, de un segmento reducido del código o de la funcionalidad, 
dependiendo del nivel de la prueba. Pero también exige la existencia de un mecanismo que permita 
realizar los ensayos de forma automática; de lo contrario, la obligación de repetirlos 
frecuentemente los convertiría en extremadamente tediosos.

\subsection{Pruebas Caja Blanca}
Son pruebas que se realizan para verificar la validez de las operaciones internas del programa, 
intentando garantizar que todos los caminos de ejecución del programa han sido probados. Para realizar las pruebas de caja blanca se han 
seleccionado las pruebas unitarias por su importancia dentro la metodología de Programación Extrema. 
La descripción de las pruebas unitarias se amplia a través de los siguientes puntos.

\begin{description}

\item [Validación con pruebas unitarias:] representan el nivel más bajo de prueba del \textit{software}, 
permitiendo al desarrollador obtener retroalimentación instántanea del correcto
comportamiento (o no) de un componente determinado.

\item [Necesidad de las pruebas unitarias:] como buena práctica de programación
el desarrollador debería codificar la prueba antes que la propia funcionalidad,
para así pensar los resultados antes de obtenerlos. Como norma general, un buen
test siempre debería fallar la primera vez. Por otra parte, facilitan la
depuración, si un test falla se dispone de un ámbito muy reducido en el cual localizar
el error, incrementándose la productividad de los desarrolladores. 

Además, desde el punto de vista de la gestión de un proyecto un buen conjunto de
pruebas unitarias se pueden utilizar como documentación del código fuente, así,
por ejemplo si se introducen nuevos desarrolladores éstos pueden comprender
fácilmente la funcionalidad que se está realizando y probando.

\item [Codificación de Pruebas:] las pruebas unitarias deben seguir una serie de
criterios:
\begin{itemize}
  \item Prueban funcionalidad determinada, un método.
  \item Ejecución rápida, si no fuera así los test sólo se ejecutarían
  cada cierto tiempo repercutiendo en la integración del código.
  \item Se debe determinar sobre qué clases se van a generar casos de prueba. No
  todas las clases y métodos son candidatos a ser probados.
  \item Son independientes de otras pruebas.
  \item Deben ser independientes de terceros, por ejemplo de una base de datos.
\end{itemize}

\item [\textit{Refactoring}:] despues de que un componente haya pasado un test unitario,
ya pueden aplicarse las técnicas de \textit{refactoring} para mejorar la calidad
de código. Este proceso, conjunto de prueba y recodificación, ya puede realizarse 
porque se dispone de una manera de probar la funcionalidad de forma automática. En las
primeras etapas de desarrollo será cuando se haga más intenso el uso de pruebas
unitarias ya que se está codificando una nueva funcionalidad en cada momento.
\end{description}

\subsubsection{Junit}\label{junit}
Herramienta Java para la realización de pruebas unitarias de
programas, ejecuta test de pruebas de forma automática realizando un registro de
los resultados obtenidos. El gran éxito de esta herramienta se debe a su
facilidad de uso en comparación con el trabajo que realiza.

La realización de las pruebas de los programas normalmente resultan costosas
temporalmente y muchas veces no se almacenan ni se documentan. Con esta
herramienta, se consigue que la prueba de programas rebaje su coste, además de
liberarnos de una parte de la construcción de programas que en muchos casos
resulta pesada y poco eficiente.

Por tanto, con Junit se aumenta el nivel de calidad en el test de los programas,
entendiendo por calidad en el test una mejora en la realización de
las pruebas en cuanto a casos contemplados, número y registro de las mismas y facilidad de uso.

Las ventajas de uso de Junit se pueden resumir en los siguientes puntos:
\begin{itemize}
\item Facilidad de uso.
\item  Creación de conjuntos de pruebas para una clase.
\item  Combinación con otras herramientas de gestión de proyectos como Maven.
\item  Código libre.
\item  Bien documentado.
\item  Pruebas a diferentes niveles y capas.
\item  Extensible.
\end{itemize}

Cabe señalar que existen extensiones de Junit que permiten realizar pruebas en
distintos ámbitos, siempre y cuando se haya seguido un diseño apropiado. 
De esta manera existen \textit{frameworks} basados en Junit para probar:
\begin{itemize}
\item  XmlUnit, para documentos XML.
\item  DbUnit, para probar el estado de la base de datos.
\item  EasyMock, para generar las clases que basadas en las interfaces de
negocio permitan probar los servicios y funcionalidad de la aplicación.
\item  HttpUnit, JWebUnit o Cactus para las distintas capas de
una arquitectura cliente/servidor.
\end{itemize}

El contexto del sistema \gls{MOLDEAS} se han realizado $140$ \textit{tests} enclavados 
dentro de $48$ clases Java específicas para Junit.

\subsection{Pruebas de Caja Negra}
Son pruebas que se centran en los requisitos funcionales del
\textit{software}. Existen diferentes tipos como, la prueba de los valores
límites o partición equivalente. Para realizar este tipo de pruebas se utilizan
posibles entradas al sistema y se verifica que cumple con los resultados
esperados. Las pruebas realizadas para la validación del sistema de recuperación 
de anuncios de licitación se pueden considerar incluidas dentro de este tipo de pruebas de caja negra y teniendo en cuenta
que el sistema \gls{MOLDEAS} trabaja con una configuración externa, su funcionamiento
dependerá de la misma y no del programa en sí, es decir, desde un punto de vista
de software la implementación realizada representa un conjunto de 
configuraciones a través de Spring. Considerando el carácter investigador 
de este estudio, resultan más relevantes las pruebas de
validación científica que las propias de caja negra, más apropiadas 
desde un punto de vista de aplicación de \textit{software}.


\subsection{Aportaciones Finales sobre las pruebas}
La codificación de pruebas unitarias antes de implementar cierta funcionalidad
puede resultar tediosa e incluso se considera como un retraso en el desarrollo, sin embargo disponer de un 
método eficaz para realizar pruebas ayuda en
posteriores etapas a la codificación de determinada función y asegura una cierta
calidad en el \textit{software}. 

Por ejemplo, se realiza un determinado método para una cierta tarea, para ejecutar la prueba se utiliza 
un método \textit{main} en la propia clase donde se ha definido y una vez que se comprueba que todo
funciona correctamente, se elimina ese método \textit{main}, otorgando a esa
funcionalidad la etiqueta de éxito. Si con posterioridad, nuevas situaciones obligan a 
modificar el código de esa clase porque no se habían contemplado todos los
requisitos funcionales, pueden surgir las siguientes cuestiones: ¿quién está dispuesto a recodificar esa función?, ¿cuál era su comportamiento?, 
¿para qué utilizaba esa variable? y ¿cómo se prueba que mantiene su funcionalidad anterior (necesario para otros módulos) y 
que además incorpora la nueva funcionalidad?. La respuesta es sencilla, se volverá a codificar un método
\textit{main}, creando un nuevo caso de prueba y prácticamente se delega en la ``suerte'' la entrega de 
un comportamiento correcto tanto para la versión anterior como para las nuevas funcionalidades. En 
conclusión, una prueba unitaria hubiera facilitado el desarrollo, ya que tan sólo se debería generar un
caso de prueba para la nueva funcionalidad. Pero en cambio, se ha perdido tiempo: codificando métodos que posteriormente 
serán eliminados, pensando en qué hacía la función, para qué se utiliza cierta variable, etc. 

En definitiva, las pruebas unitarias facilitan el desarrollo de \textit{software} de
calidad, en contraposición de los engendros (no desarrollos) de \textit{software} creados
al margen de la realización de pruebas.

\subsection{Métricas de código fuente}
La calidad del \textit{software}, en cuanto a código fuente, es uno de los requisitos que se
debe exigir a cualquier aplicación. Normalmente, no se plantea como uno de los
requisitos no funcionales prioritarios y esto provoca que aunque se haya
realizado un buen análisis e incluso un buen diseño, el código generado no es lo suficientemente bueno y es el
origen de todos los posibles errores y en consecuencia no se cumple con los
requisitos funcionales establecidos. La definición de calidad de \textit{software}, de
forma genérica, es mucho más extensa:  

\begin{Frame}
\textit{Concordancia con los requisitos funcionales y de
rendimiento explícitamente establecidos con los estándares de desarrollo
explícitamente documentados y con las características implícitas
que se espera de todo software desarrollado profesionalmente}. R.S. Pressman (1992).
\end{Frame}

\begin{Frame}
\textit{El conjunto de características de una entidad que le confieren su
aptitud para satisfacer las necesidades expresadas y las implícitas}. \gls{ISO} 8402
(\gls{UNE} 66-001-92).
\end{Frame}

Estas definiciones de calidad se refieren sobre todo a la metodología utilizada y
al grado de satisfacción de los requisitos funcionales del producto \textit{software}. 

Las métricas que aquí se presentan constituyen medidas de calidad del código fuente, ya
que se estima que los requisitos funcionales han sido cubiertos al obtener un
prototipo experimental que ha sido probado a través de distintos conjuntos de pruebas.

A continuación, se presentan una serie de métricas de calidad de código fuente
(sólo del componente \texttt{moldes-api}), obtenidas mediante el \textit{plugin} Metrics de Eclipse, que 
permite generar una serie de medidas, ver Tabla \ref{tabla:metricas}, significativas del código fuente, basadas en 
las definiciones realizadas en el libro \textit{Object-Oriented Metrics, measures of Complexity} de \textit{Brian Henderson-Sellers}, Prentice Hall, 1996. 

Estas métricas miden, principalmente, valores de cohesión y acoplamiento que permiten establecer si el código está bien estructurado. 
Sin valorar detalles precisos de los tipos y niveles de cohesión y acoplamiento, estas dos características se pueden definir como:

\begin{Frame}
\textit{Cohesión, consecuencia del ocultamiento de la información. Un módulo con
cohesión (Pressman en ``Ingeniería del Software: Un enfoque práctico'')
realiza solamente una tarea, requiriendo poca
interacción con el resto de los procedimientos que se realizan en el resto del
sistema de software}. 
\end{Frame}

Según \textit{Sommerville} en ``Requirements Engineering. Processes and Techniques'' la cohesión es deseable
debido a que una unidad (componente) representa una parte
simple de solución. Si es necesario cambiar el sistema, la parte
correspondiente está en un solo lugar y lo que se desee hacer con
el mismo estará encapsulado en él. La meta, según \textit{Lawrence} en ``Ingeniería del \textit{software}: Teoría y práctica'', es lograr 
que los componentes sean lo más cohesivos posible.


\begin{Frame}
\textit{Acoplamiento, está relacionado con la cohesión. Es un
indicador de la fuerza de interconexión entre los componentes o
elementos de la arquitectura }
\end{Frame}

Los sistemas altamente acoplados tienen una fuerte interconexión, lo que se
refleja en una gran dependencia entre los componentes. Los
sistemas poco acoplados, por otro lado, tienen poca relación entre
sus componentes o elementos. El objetivo, según (\textit{Lawrence}), es mantener el
acoplamiento en el nivel más bajo posible; la conectividad sencilla entre
módulos da como resultado un \textit{software} que es más fácil de comprender y menos
propenso al ``efecto onda'' producido cuando los errores aparecen en una
posición y se propagan a lo largo del sistema. 


   \begin{longtable}{|p{2cm}|p{3cm}|p{8cm}|}

      \hline 
	    \multicolumn{3}{|c|}{\textbf{Métricas de código fuente}}\\ \hline
	     \textbf{Tipo} &  \textbf{Descripción (EN)} &  \textbf{Descripción (ES)} \\ \hline
      \endhead
			NSM&\textit{Number of Static Methods} & Número total de métodos estáticos.\\ \hline
			TLOC&\textit{Total Lines of Code}&Número total de líneas de código. No incluye
			comentarios ni líneas en blanco.\\ \hline 
			CA&\textit{Afferent Coupling} &Número de clases fuera del paquete que dependen de
			clases dentro del paquete.\\ \hline 
			RMD&\textit{Normalized Distance} & $|RMA + RMI - 1 |$, este número debe ser pequeño.
			Cercano a 0 indica un buen diseño de paquetes.\\ \hline
			NOC&\textit{Number of Classes}& Número total de clases.\\ \hline
			SIX&S\textit{pecialization Index} &Indice de especialización :$NORM * DIT / NOM$\\			\hline 
			RMI&\textit{Instability} & $CE / (CA + CE)$\\ \hline
			NOF&\textit{Number of Atributtes} &Número de atributos definidos en un ámbito.\\ \hline
			NOP&\textit{Number of Package}s& Número  total de paquetes.\\ \hline
			MLOC&\textit{Method Lines of Code} & Número de líneas de código por método. No incluye
			comentarios ni espacios en blanco.\\ \hline
			WMC&\textit{Weighted methods per Class}& Suma de las complejidades ciclomáticas de
			McCabe de todos los métodos de una clase. \\ \hline 
			NORM&\textit{Number of Overriden Methods}&Número total de métodos sobreescritos. No
			incluye los de la clase ``Object''. \\ \hline 
			NSF&\textit{Number of Static Attributes} & Número total de atributos estáticos.\\ \hline
			NBD&\textit{Nested Block Depth} & Número de bloques anidados.\\ \hline
			NOM&\textit{Number of Methods} & Número de métodos definidos en un ámbito.\\ \hline
			LCOM&\textit{Lack of Cohesion of Methods}& Cohesión de una clase. Si $m(A)$ es el
			número de métodos que acceden a un atributo $A$, se calcula la media de $m(A)$ para
			 todos los atributos, se resta el número de métodos $m$ y se divide el resultado
			 entre $(1-m)$. Un valor bajo indica clase con alta cohesión (preferible) y alto, baja
			 cohesión y se debería dividir en $n$ subclases. (\textit{Henderson-Sellers})\\ \hline
			VG&M\textit{cCabe Cyclomatic Complexity} &Complejidad ciclomática.\\ \hline
			PAR&\textit{Number of Parameters} & Número de parámetros.\\ \hline
			RMA&\textit{Abstractness}& Número de clases e interfaces abstractas en un paquete divido
			entre el total de tipos en el paquete.\\ \hline 
			NOI&\textit{Number of Interfaces}&Número de interfaces.\\ \hline
			CE&\textit{Efferent Coupling}& Número de clases dentro del paquete que dependen de
			clases fuera del paquete.\\ \hline 
			NSC&\textit{Number of Children}& Número de hijos de un tipo o clase.\\ \hline
			DIT&\textit{Depth of Inheritance Tree}&Distancia de una clase en la jerarquía de herencia.\\
			\hline \hline
        \caption{Métricas del código fuente} \label{tabla:metricas}
        \end{longtable}      
        
   \begin{longtable}{|p{1.5cm}|p{1.5cm}|p{1.5cm}|p{1.5cm}|p{1cm}|p{2cm}|p{2.5cm}|}

      \hline 
	    \multicolumn{7}{|c|}{\textbf{Valores Métricas de código fuente}}\\ \hline
	    \textit{ID} &  \textit{Total} &  \textit{Media}&  \textit{Desv. est.} & \textit{Max}& \textit{Rango}& \textit{Ámbito} \\ \hline
      \endhead
		NSM&103&$0.656$&$1.967$&16& \na &Tipo. \\ \hline
		TLOC&8090&\na&\na&\na&\na&\na \\ \hline
		CA&\na&$6.545$&$15.455$&86&\na&Paquete.\\ \hline
		RMD&\na&$0.331$&$0.33$&1&\na&Paquete. \\ \hline
		NOC&157&$4.758$&$4.164$&23&\na&Paquete. \\ \hline 
		SIX&\na&$0.074$&$0.329$&3&\na&Tipo. \\		\hline 
		RMI&\na&$0.648$&$0.351$&1&\na&Paquete. \\ \hline
		NOF&158&$1.006$&$1.684$&9&\na&Tipo. \\ \hline
		NOP&33&\na&\na&\na&\na&\na \\ \hline
		MLOC&4235&$5.639$&$7.025$&47&\na&Método \\ \hline
		WMC&1167&$7.433$&$9.055$&52&\na&Tipo. \\ \hline
		NORM&39&$0.248$&$0.711$&3&\na&Tipo. \\ \hline
		NSF&91&$0.58$&$1.506$&$15$&\na&Tipo. \\ \hline
		NBD&\na&$1.304$&$0.723$&7&\na&Método. \\ \hline
		NOM&648&$4.127$&$4.403$&22&\na&Tipo. \\		\hline 
		LCOM&\na&$0.129$&$0.26$&1&\na&Tipo. \\ \hline 
		VG&\na&$1.554$&$1.736$&25&\na&Método autogenerado por Eclipse. \\ \hline
		PAR&\na&$0.606$&$0.756$&4&5&Método.\\ \hline 
		RMA&\na&$0.082$&$0.176$&$0.714$&\na&Paquete. \\ \hline
		NOI&14&$0.424$&$1.016$&5&\na&Paquete. \\ \hline
		CE&\na&$3.576$&$2.535$&11&\na&Paquete. \\ \hline
		NSC&20&$0.127$&$1.144$&14&\na&Media y máximo por tipo. \\ \hline
		DIT&\na&$1.242$&$0.569$&4&\na&Tipo. \\ \hline
    	 \hline
        \caption{Valores de Métricas del código fuente de \texttt{moldeas-api}} \label{tabla:metricas-valores}\\
        \end{longtable}    
  
\subsection{Aportaciones Finales sobre Métricas de Código Fuente}
Todas las medidas obtenidas, ver Tabla \ref{tabla:metricas-valores}, están dentro de
los rangos fijados por la herramienta, aquellos valores que no son aplicables 
se señalan con el símbolo \na. Sobre los valores obtenidos hay que destacar 
los siguientes puntos:

\begin{itemize}
  \item La complejidad ciclomática de McCabe es siempre aceptable, obteniendo
  una media de $1.554$ por debajo de 10, y con un máximo de 25 en un método generado 
por Eclipse (\texttt{equals}).

  \item En LCOM o medida de cohesión de una clase se obtiene una media $0.129$.
  En esta medida el valor adecuado es cercano a 0, por lo que se puede concluir que
  las clases tienen una excelente valoración en cohesión. El valor máximo
  es de $0.26$ que resulta de nuevo un excelente valor.

  \item Las medidas de acoplamiento CA y CE de nuevo proporcionan buenos valores para
  todas las clases. 

\end{itemize}

Con estas medidas, se ha creado un producto cuyo código fuente posee las
características deseadas de alta cohesión y bajo acoplamiento. Por otra parte,
esta situación no es extraordinaria ya que el uso de patrones de diseño y de buenas
prácticas de programación, como el binomio de pruebas unitarias y
\textit{refactoring}, aseguran calidad en el código fuente implementado, no obstante, 
toda aplicación y por extensión su código fuente son siempre candidatos a ser mejorados.




