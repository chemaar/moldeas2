El término ``Web Semántica'' se puede interpretar como una evolución de la web
actual. El uso del adjetivo ``semántica'' aporta un atractivo extra a la Web, dotando a la misma de características cercanas a lo que se entiende 
habitualmente por inteligencia artificial y que tan utilizadas son en películas de ciencia-ficción.
No se debe confundir un avance tecnológico con una revolución en la concepción
de la web actual y es preciso ser cautos con el campo de aplicación de la
``semántica'' en la red y seleccionar aquellos escenarios en los que realmente
la semántica enriquezca el desarrollo actual.

Siguiendo las directrices que marcan las grandes empresas y que
corresponden en muchos casos a la opinión de \textit{Gartner} (``Semantic Web Technologies
Take Middleware to Next Level''~\cite{gartner}), ya se vaticinaba que en el año 2005 la integración de 
aplicaciones empresariales estaría guiada por ontologías y por tanto impulsaba, la Web Semántica.

El despliegue de tecnología semántica, confiere una visión más realista, en
la cual existen grandes discusiones abiertas sobre temas diversos, por ejemplo
reglas, razonamiento distribuido, procesamiento de \textit{\gls{Big Data}}, lenguajes de formalización creados por distintas instituciones 
con distintos niveles de expresividad, herramientas en continuo desarrollo no
estables, aplicaciones que se venden como Web Semántica cuando no lo son
estrictamente o bien simplemente se han unido a la ``iniciativa'' de una tecnología, con el único objetivo de disponer 
de este sello en sus aplicaciones. 

Las expectativas se presentan ambiciosas y tentadoras, pero no menos que en la década de los años setenta con la eclosión 
de la inteligencia artificial. Por todo ello, es necesario un análisis riguroso y exhaustivo para dotar a este avance
tecnológico de un planteamiento teórico y práctico adecuado donde realmente sea aplicable.  

Esta introducción aunque un tanto pesimista, trata simplemente de otorgar un enfoque
realista a esta tecnología de vanguardia, ciertamente aunque las barreras a
superar son importantes, también hay que resaltar que el impulso realizado por
todas las partes implicadas: empresa, universidad, administración, desarrolladores,
etc., está siendo considerable y los resultados estables no se dilatarán excesivamente en el tiempo pero contando 
siempre con una base firme. 