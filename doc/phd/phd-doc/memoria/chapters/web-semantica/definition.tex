La Web Semántica siguiendo la definición propuesta por el W3C se presenta como: 
\begin{Frame}
\textit{Una web extendida, dotada de mayor significado, en la que cualquier
usuario en Internet podrá encontrar respuestas a sus preguntas de forma más rápida y
sencilla gracias a una información mejor definida.}
\end{Frame}

Surge como respuesta a las dificultades que aparecen al tratar de 
automatizar muchas tareas en la web actual. Hoy en día, los contenidos de la web son
creados considerando que van a ser consumidos principalmente por personas,
lo que hace difícil su interpretación por parte de agentes \textit{software}.
En consecuencia, algunas tareas comunes como la búsqueda de información en la web actual,
son notoriamente mejorables; otras tareas aparentemente sencillas son casi imposibles de implementar.
En el origen de este problema se encuentra la incapacidad de los agentes \textit{software} (las máquinas)
para encontrar, interpretar, extraer y combinar la información ya disponible en la web.


Para tratar de paliar esta situación, \textit{Tim Berners-Lee}, reconocido como ``padre de la web'',
impulsa a través del \gls{W3C} el desarrollo de la Web Semántica ~\cite{WeavingTim,berners-lee06a}.
A través de esta iniciativa, se desarrollan los lenguajes y formalismos que permiten
extender la web actual, de tal manera que sus contenidos sean accesibles tanto a
personas como a máquinas. Esta tecnología abre la puerta a una nueva generación
de aplicaciones informáticas capaces de encontrar, seleccionar y combinar la
información dispersa en la web para realizar tareas que actualmente se ejecutan de forma manual: seleccionar los resultados relevantes en una búsqueda, agregar
información procedente de distintas fuentes, etc.


Una de las fortalezas de la web actual es la ingente cantidad de información que se encuentra
publicada en ella y la infinidad de servicios a los que se puede acceder.
Sin embargo, si la explotación de estos recursos requiriese necesariamente la intervención humana,
como sucede ahora, su utilidad estaría limitada, de ahí que el W3C pretenda
``guiar a la web hacia su máximo potencial'' como herramienta universal y multipropósito.
La Web Semántica proporciona una infraestructura para explotar eficientemente el potencial de
la web~\cite{Berendt02,decker00knowledge}, se encuentra, sin embargo en continuo desarrollo, gestándose gradualmente su progresión.

En el contexto de la iniciativa de la Web Semántica, se han desarrollado y se continúan
desarrollando estándares que son el soporte necesario para hacer realidad la misma.
En la base se utilizan tecnologías estándar ya asentadas, como \gls{XML}~\cite{XML11}, \gls{HTTP}~\cite{http-rfc} y \gls{URI}s~\cite{uri-rfc},
que son también la base de la web actual. Pero además se han creado nuevos mecanismos
para describir semántica y formalmente la información, se trata, entre otros, de \gls{RDF}~\cite{RDF}, \gls{RDF Schema}~\cite{RDFS} y \gls{OWL}~\cite{OWL11,owl2-primer}. Con estos lenguajes y modelos se puede describir la información de manera precisa y
y carente de ambig\"{u}edad basándose en teorías lógicas como las \textit{Description Logics~\cite{baader03description} (\gls{DL}s)}.
En este sentido juegan un papel fundamental las taxonomías, los tesauros y las ontologías,
como estructuras capaces de dotar de significado a los datos 
~\cite{Benjamins98knowledgemanagement}.

\subsubsection{No es Web Semántica}
En muchas ocasiones, no existe manera más conveniente para definir un concepto que encontrar
un contraejemplo. Así, \textit{Tim Berners-Lee} insiste en que la Web Semántica \textbf{No} es inteligencia artificial: 
\begin{Frame}
\textit{El concepto de documento entendible por una máquina no implica algún tipo de
inteligencia artificial mágica que permita a las máquinas comprender el
farfullar de los humanos. Sólo indica una habilidad de la máquina para resolver
un problema bien definido a base de realizar operaciones bien definidas sobre
unos datos bien definidos. En vez de pedir a las máquinas que entiendan nuestro
lenguaje, se le pedirá a la gente que haga un esfuerzo extra.} \textit{A roadmap
to the Semantic Web.  What the semantic Web isn't but can represent}. 1998. 

Fuente: \url{http://www.w3.org/DesignIssues/RDFnot.html}
\end{Frame}

Por tanto, desposeída de su aura mágica al no constar entre sus objetivos la
conquista del lenguaje natural, la Web Semántica queda reducida a un intercambio
de información eficiente entre los agentes.
