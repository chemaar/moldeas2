Las Administraciones Públicas son uno de los mayores compradores de la Unión \gls{Europea} (UE), ya que
sus adquisiciones en conjunto representan un $17$\%~\cite{europeanStrategy} del Producto Interior Bruto (\gls{PIB}). La tendencia se dirige a intentar 
agilizar este gran mercado de oportunidades, que asciende a unos $2,155$ billones de euros, para 
facilitar el acceso a las empresas, especialmente a las pequeñas y medianas empresas (\gls{PYME}s), posibilitando 
el acceso a la información presente en los contratos públicos y permitiendo así la construcción de un entorno de mercado competitivo. Por otra parte, 
un punto clave reside en el impulso de la transparencia y la eficiencia en los propios procesos administrativos asociados a la contratación pública.
Esta situación ha supuesto la germinación de una voluntad en las Administraciones Públicas para tratar de simplificar 
el proceso de contratación pública tanto desde un punto vista legal, como operativo. Las acciones de cambio vienen impulsadas desde la Unión Europea 
a través de distintas directivas que son transpuestas en los distintos \gls{Estados} Miembros, dando soporte a estos procesos de adquisición de bienes y servicios. 
No obstante, las oportunidades comerciales que representan estos contratos públicos están fragmentadas en 
cientos de fuentes, en general de carácter local, representando una dilatada sucesión de anuncios de 
licitación que no llegan a tener una difusión adecuada, lo cuál es inaudito en la era de las comunicaciones instantáneas 
y globales. 

Por otra parte, teniendo en cuenta que la información incluida en los anuncios de licitación
tiene la consideración de Información del Sector Público (\gls{PSI}), se deben proveer los mecanismos
necesarios de acceso a esta información para facilitar la consulta de los datos contenidos
en los anuncios de licitación. Actualmente la corriente \opendata dentro del seno de las organizaciones e individuos en general y de las entidades públicas en particular, ha generado un entorno
de datos abiertos en el cual deben formar parte necesariamente los contratos públicos. 
Esta sensibilidad emergente respecto a la apertura de datos implica una necesaria relación con la 
contratación pública debido al valor de la información que contienen, tanto para crear nuevos servicios 
de valor añadido agregando datos de distintas fuentes, como medio de información para 
las empresas interesadas en atender a procesos de contratación pública.

%Linked Data y semantic web
Desde otro punto de vista, el creciente uso de Internet durante los últimos años ha puesto de manifiesto un
nuevo entorno de ejecución para las aplicaciones, utilizando como nueva
plataforma la Web, el gran sistema distribuido. Nuevas tecnologías y paradigmas
están emergiendo para dar soporte al desarrollo y despliegue de aplicaciones y
servicios, así como para la publicación de datos e información. Los modelos de
desarrollo están evolucionando hacia un estilo más colaborativo en el cual las empresas
ofrecen su software como servicios (\textit{Software as a Service}-\gls{SaaS}), materializado a
través del paradigma de \textit{Cloud Computing}, implementado con tecnología de
servicios con el objetivo de que terceros puedan utilizar estos nuevos servicios y datos para
la construcción de aplicaciones agregadas con valor añadido. 

En este sentido, iniciativas como la Web Semántica~\cite{Berners-Lee2001} dentro de la nueva ``Web de Datos'' o \wode a través de modelos y
formatos de datos de conocimiento compartido unificados, intentan elevar el
significado de los elementos y recursos que están disponibles en la web con el
objetivo de mejorar la integración e interoperabilidad entre aplicaciones, impulsando la implantación de este enfoque. Dentro de la iniciativa de Web
Semántica hay que destacar principalmente dos esfuerzos: 
\begin{enumerate}
 \item La iniciativa \linkeddata~\cite{Berners-Lee-2006}, propone la publicación de datos
enlazados siguiendo el modelo \gls{RDF}~\cite{rdf-syntax} para facilitar la creación de una web de
datos en la que éstos se puedan mostrar, intercambiar y conectar a través de
URIs. La tendencia actual de publicación de datos enlazados está marcando una
evolución en la interoperabilidad de aplicaciones con el consiguiente efecto que
conlleva para las relaciones \textit{Business to Business} (\gls{B2B}), \textit{Business to
Client}(\gls{B2C}) o \textit{Administration to Administration} (\gls{A2A}). Entre los casos de éxito
se podrían destacar: administración electrónica (iniciativa de \textit{Open Government
Data}~\cite{8-principles}-\gls{OGD}), contratación pública electrónica de bienes y servicios (\textit{\gls{e-Procurement}}), oferta
formacional, contextualización de aplicaciones, etc. 

\item El desarrollo de lenguajes y formalismos lógicos para representar el
conocimiento sobre un universo de discurso que permita la inferencia de nuevos
datos a partir de los datos ya publicados. En este contexto se han reimpulsado
el uso de técnicas de razonamiento y de sistemas basados en conocimiento, 
como pueden ser las ontologías y los sistemas basados en reglas lógicas (Ontobroker,
XSB, etc.) o de producción (Drools, JRules, etc.). La aplicación de estos
sistemas está ampliamente asentada en la resolución de diversos problemas
(diagnóstico, planificación, reglas de negocio, etc.) pero siempre utilizando un
enfoque para la representación del conocimiento y de los datos, en muchos casos
específico y no estandarizado, pero que con la aplicación de los principios
de la Web Semántica se ve favorecido por un nuevo contexto de estandarización. 
\end{enumerate}

Por lo tanto, teniendo en cuenta el contexto estratégico que supone la contratación pública
para el mercado común y la sostenibilidad económica y considerando iniciativas
como la Web Semántica, que animan y potencian la reutilización de datos, información
y modelos de conocimiento compartido, queda patente que la inclusión de la tecnología
semántica en el contexto de la administración electrónica y en concreto en los
procesos de contratación pública electrónica, mejora e impulsa este dominio con un 
nuevo enfoque. El estudio e implantación de la semántica conlleva una mejora
cualitativa en cuanto a la provisión de un modelo estandarizado, cooperativo y con
nuevas capacidades, para mejorar e instrumentar este entorno desde un punto
de vista tecnológico y realizar la visión estratégica que desde la política
se propugna. Si se aplica el principio de evolución de las especies
de \textit{Charles Darwin}, se diría que la semántica aplicada a la contratación pública
supone una evolución natural a la forma de entender este proceso administrativo, favoreciendo
su supervivencia en el nuevo contexto de la \wode.

\textit{It is not the strongest of the species that survive, nor the most
intelligent, but the ones most responsive to change.}

\section{Motivación}
La aplicación de las tecnologías semánticas e iniciativas como \opendata y
\linkeddata, están de rigurosa actualidad, convirtiéndose en pieza clave
para el desarrollo de nuevas aplicaciones y servicios de valor añadido,
reutilizando datos provenientes de distintas fuentes: redes sociales, empresas,
personas o instituciones públicas. Por otra parte, la investigación desarrollada
en estas áreas se está materializando a través de múltiples publicaciones de
alto valor científico, realización de conferencias internacionales, desarrollo
de herramientas y transferencia tecnológica a empresas de distintos
dominios, en las cuales se han de integrar fuentes de datos heterogéneas y generar
servicios inteligentes contextualizados para los usuarios, así, se pueden 
encontrar soluciones en dominios tan dispares como salud, control de procesos
industriales, administración electrónica, gestión de recursos humanos, etc.

En concreto, en el caso de la administración electrónica, uno de los procesos
administrativos más influyentes por su impacto económico en la sociedad es la compra y adquisición
de bienes y servicios, ya que impulsa la actividad económica de las empresas como proveedores
de la Administración Pública y permite disponer nuevos servicios a los ciudadanos que, en último término, son los clientes
y contribuyentes de la Administración. Es por ello, que la aplicación de
las últimas tendencias de investigación y tecnológicas en este contexto se plantean
como clave para mantener a la Administración Pública como paradigma de funcionamiento,
organización y actualización.

\subsection{Motivación Investigadora}
La investigación en el campo de la Web Semántica se materializa a través de distintas
líneas de estudio, que van desde la definición de modelos lógicos formales y 
su combinación con los ya existentes, hasta actividades más orientadas a la innovación para dar un nuevo
enfoque a la resolución de problemas subyacentes a la informática, como es la integración
de fuentes de datos heterogéneas o gestión de la información. Este amplio espectro
de aplicación permite iniciar actividades de investigación en múltiples líneas
y con distinto carácter: básica, aplicada, innovadora, etc. En el caso objeto de estudio, la investigación
realizada se centra en aplicar los principios de la Web Semántica y \linkeddata al dominio
de la contratación pública electrónica, con el objetivo de facilitar el acceso a la información que estos
documentos contienen. Por lo tanto, la actividad de investigación a realizar, centra su foco
en cómo puede la semántica facilitar el acceso a la información de los anuncios
de licitación pública. 

En cualquier actividad de investigación a partir de una serie de hipótesis iniciales se intenta demostrar la validez 
de ciertos supuestos siguiendo una metodología que permita caracterizar y cuantificar la veracidad de los principios marcados. En este sentido, la aplicación
de semántica a los contratos públicos provee un escenario extremadamente rico para realizar la 
experimentación necesaria que evalúe la mejora de la conjunción de la semántica como modelo
para la gestión de la información y datos contenidos en los anuncios de licitación públicos, en contraposición
con los actuales procesos tradicionales. En concreto, la investigación a realizar posee un objetivo
muy específico pero de gran calado para la innovación en el ámbito de la administración pública electrónica
y de los procesos de contratación pública.

Por otra parte y en cuanto a los resultados esperados, la investigación realizada debe demostrar
las ventajas de la semántica en este dominio, ya probada en muchos casos para la resolución de problemas en otros contextos, 
para que de este forma se pueda promocionar el \textit{know-how} tanto verticalmente, abordando otras
situaciones sensibles en la contratación pública, como horizontalmente en otros procesos
administrativos. 

En resumen, la investigación realizada debe proveer los resultados y mecanismos
adecuados para crear concienciación, tanto en la comunidad científica, como en la industrial, de la bondad
de la aplicación de la semántica al dominio de la contratación pública electrónica.

\subsection{Motivación Técnica}
Los procesos de contratación pública conllevan una serie de actividades que hacen uso
de las más avanzadas técnicas para la gestión de la información y de datos, así como para
el soporte a la decisión. La relevancia de la información que se maneja
en estos procesos sumada a la trascendencia económica, implica que la tecnología que da
soporte a todo el proceso debe asegurar la fiabilidad y corrección del mismo, impulsando
y mejorando la cadena de valor tanto para la propia Administración como para los proveedores.

En general, la Administración Pública suele ser referente tecnológico por la inversión
en infraestructuras que realiza. Es por ello, que la aplicación de la semántica
en este dominio debe, no sólo ser capaz de proveer las mismas capacidades para el 
despliegue de servicios, sino mejorar la calidad de los mismos. Este reto
tecnológico implica, en muchos casos, el desarrollo de herramientas que faciliten
la adición de semántica aplicando un enfoque de ingeniería cuantificable. También, y
desafortunadamente el uso de tecnología semántica implica, más veces de las deseadas, 
la depuración de la misma, por ello dentro de esta investigación la tecnología semántica juega un papel
relevante tanto para probar su valía en este dominio, como para mejorar y ampliar sus capacidades.

\subsection{Motivación Personal}
La carrera en investigación propone desafíos para la resolución de problemas
de diversos ámbitos aplicando diferentes técnicas. A lo largo de mi trayectoria profesional
en el campo de la investigación, he tenido la oportunidad de participar
en diversas actividades en este entorno, tales como proyectos, conferencias, redacción de artículos o formación, en los cuales he podido conocer y relacionarme con personas
y organizaciones provenientes de distintos ámbitos así, sector privado, público y académico. Cada uno 
de ellos, con distintos objetivos y perfiles, han contribuido al enriquecimiento de mi propia experiencia, tanto
a nivel personal, como evidentemente profesional, el cual queda materializado en la realización
de esta tesis, fiel reflejo de la contribución de la misma, ya que afirmar que se trata de un trabajo personal aislado sería
entender al individuo como un ente apartado del mundo, lo que implicaría negar el factor
social inherente al mismo. Por ello, a nivel personal, la realización de la tesis constituye
un punto de inflexión en mi carrera de investigación, supone no sólo un compendio de mi historia
personal hasta el momento presente, sino también y con seguridad un elemento clave para mi devenir profesional y personal
tanto a corto, como a medio y largo plazo. 

Por otra parte, se plantean también dos retos: 1) investigador, que reside en el estudio,
análisis, aplicación y prueba de la validez de uso de 
las tecnologías semánticas en el campo de la contratación pública electrónica y 2) tecnológico, al aplicar
conceptos de ingeniería a la resolución de un problema, mediante una serie de métodos y herramientas
cuantificables permitiendo así desarrollar tanto la motivación intrínseca (desafío de resolver
un problema concreto) como extrínseca (difusión del conocimiento y solución generadas).

Finalmente, actúa como incentivo la obtención del propio título de Doctor, que supone
la mayor distinción a nivel académico y formativo que se puede lograr en el actual sistema
de educación, que se trasluce igualmente en gran satisfacción y orgullo no sólo a nivel personal, familiar y social sino también 
profesional. 

\subsection{El proyecto \textit{10ders Information Services}}\label{10ders}
\textit{\gls{10ders} Information Services}~\cite{10ders} es un proyecto de investigación
cofinanciado por el Ministerio de Industria, Turismo y Comercio dentro del plan Avanza 2 
con código TSI-020100-2010-919, liderado por Gateway S.C.S.~\cite{gateway} (creadores de la plataforma
de alertas de anuncios de licitación Euroalert.net~\cite{euroalert}) y desarrollado en colaboración con la empresa Exis TI~\cite{exis} y el equipo
de investigación \gls{WESO} de la Universidad de Oviedo. 

El objetivo principal de este proyecto es el siguiente:

\textit{Construir una plataforma pan-europea y unificada que agregue todas
las licitaciones públicas de la Unión \gls{Europea} con el fin de diseñar productos y
servicios de información baratos y accesibles que ayuden a las \gls{PYME} a
ser más competitivas en este mercado.}

De la misma forma, este objetivo se desarrolla dentro de un proceso de ingeniería
y tecnología cuya meta es:

\textit{Diseñar una arquitectura escalable capaz de localizar e interoperar con miles
de fuentes de información públicas completamente heterogéneas para agregar
la información de los cientos de miles de anuncios de licitación publicados
diariamente en la Unión Europea en múltiples idiomas y culturas, con el fin de
comprender y extraer conocimiento para el diseño de nuevos productos de
información para PYMES.}

Teniendo en cuenta este objetivo y el entorno tecnológico predefinido, los siguientes
problemas que atañen a la contratación pública deben ser abordados y resueltos con la
aplicación de nuevos métodos.

\begin{description}
 \item  [Dispersión de la información.] Los anuncios de licitación se difunden en multitud de fuentes
públicas de ámbito europeo, nacional, regional y local. Adicionalmente, las licitaciones de menor
valor correspondientes a procedimientos con publicidad restringida y que suponen una larga
serie de oportunidades comerciales, se publican en los perfiles del contratante de cada
entidad, agencia o autoridad pública. Conseguir el diseño de un sistema eficiente de rastreo y
monitorización para miles (cifra estimada) de fuentes de información, capaz de detectar las
nuevas oportunidades publicadas diariamente, y que además pueda agregar nuevas fuentes a
medida que sean descubiertas, constituye el primer problema tecnológico.

\item [Mismo anuncio en más de una fuente.] Dependiendo de la diferente normativa legal local,
regional o nacional, el mismo anuncio de licitación se publica en más de una fuente al
mismo tiempo. Como dificultad adicional, lo más habitual es que no se utilice ningún tipo de
código o referencia que identifique el anuncio de forma coherente entre las diferentes
fuentes. En ocasiones, entre los datos suministrados por cada una de las fuentes existirá más
detalle en determinados aspectos del mismo anuncio de licitación; así por ejemplo, una fuente
puede disponer de mejores datos sobre la autoridad contratante, mientras que otra puede que los suministre
más completos sobre el objeto del contrato. En ese contexto, existe un gran interés en poder
identificar de forma algorítmica qué datos provenientes de distintas fuentes se refieren al
mismo anuncio, y al mismo tiempo ser capaces de utilizar los de mayor calidad en cada caso,
no sólo para evitar tener información duplicada, sino para disponer de la calidad más elevada posible
en los datos a explotar, agregando la mejor información existente en cada publicación.

\item [Heterogeneidad de los formatos de los anuncios.] Aunque los datos deben ser públicos, no
existe un formato de anuncio de licitación unificado, ni tan siquiera un número limitado de
posibilidades. Descargar y procesar la información de cada fuente para que sea utilizable de
forma agregada, supone resolver diversos problemas complejos para interpretar la información
estructurada o no, de los anuncios y extraer datos que permitan
homogeneizar todas estas fuentes. Además los datos disponibles y su formato varían a lo largo
del tiempo, ya que ninguna de las fuentes públicas ofrece una garantía de estabilidad o
compatibilidad futura de los formatos utilizados para la publicación.

\item [Almacenamiento.] La cantidad de información que deberá procesar el sistema (solo el Diario
Oficial de la Unión Europea-\gls{DOUE} supone más de $20,000$ documentos cada día) hace que sea necesario resolver la forma en que
deba almacenarse para que pueda explotarse de forma eficiente. El sistema de
almacenamiento diseñado, tiene el reto adicional de ser escalable para incluir nuevas
fuentes de información.

\item [Diversidad de formatos de explotación.] Un almacén de información unificado con los
anuncios de compras públicas ofrece múltiples posibilidades de explotación en diferentes
productos, algunos de los cuales surgirán para responder a necesidades identificadas mucho
después de cerrar el diseño del sistema. El último gran problema tecnológico es diseñar la
arquitectura, en la que puedan convivir múltiples subconjuntos de información optimizados
para requisitos de explotación diversos. En el supuesto más extremo, determinados casos de
minería de datos estarán interesados en explotar aquellos datos que no tienen utilidad al margen de ese
caso concreto, luego la arquitectura debe reducir la barrera del re-procesado en tiempo real de
la información disponible, para extraer datos adiciones sin penalizar con ello otros servicios en
explotación.

\item [Multiling\"{u}ismo y multiculturalidad.] Como reto añadido se advierte que los documentos de
licitación pueden publicarse en una o varias de las $23$ lenguas oficiales de la Unión Europea.
Adicionalmente cada país, en ocasiones utiliza otras lenguas también oficiales en sus respectivos territorios. Todos
los algoritmos y sistemas que se incorporen a la arquitectura, deberán procesar, agregar e
interpretar información que estará escrita en un número elevado de lenguas, que además
estará expresada incorporando particularidades culturales que van desde la moneda, hasta el
sistema impositivo pasando por la forma en que se codifican las direcciones postales.
Suplementariamente, cuando el mismo anuncio se publica en más de un idioma, no siempre bastará
con centrarse en uno de los idiomas, sino que en ocasiones habrá que analizar todos y cada uno de ellos
utilizando las partes con datos de mayor calidad obtenidos entre las distintas versiones del mismo
documento.
\end{description}

Una vez contextualizado el proyecto ``10ders Information Services'', las actividades en las que 
participa la Universidad de Oviedo, se enclavan en la aplicación de las tecnologías semánticas
e iniciativas como \linkeddata, añadiendo una capa de conocimiento basada en semántica para proporcionar
una visión de los datos que sea compatible con los principios y directrices de las 
actuales corrientes de \opendata y \linkeddata. Con ello se pretende dar respuesta a los retos
que presenta la contratación pública a nivel europeo, y como se ha reseñado en los puntos anteriores las tecnologías
basadas en semántica, aportan una solución coherente y sostenible a la identificación y gestión de la información 
(\textbf{Dispersión de la información} y \textbf{Mismo anuncio en más de una fuente}) mediante modelos de datos y formatos
compartidos y consensuados (\textbf{Heterogeneidad de los formatos de los anuncios} y \textbf{Diversidad de formatos de explotación}) en un entorno
intrínsecamente para dar soporte a la internacionalización y acceso ubicuo a la información de forma
contextualizada (\textbf{Multilingüismo y multiculturalidad}).

\section{Planteamiento y Definición Inicial del Problema}
Una vez que se ha repasado y presentado en las anteriores secciones una introducción a la casuística de la contratación
pública y tecnología semántica, se han vislumbrado los distintos orígenes de la motivación y activado el contexto
de ejecución dentro del proyecto ``10ders Information Services'', es el momento adecuado para plantear
el problema a resolver y definir la hipótesis de partida objeto de estudio en esta tesis y de redacción
en este documento.

Se parte de un dominio en el cual la publicación de la información de forma accesible, multiformato, multicanal, multiling\"{u}e, etc., 
resulta clave para facilitar las oportunidades de participación en los procesos de contratación pública y favorecer la publicidad
de los anuncios de licitación mejorando la eficiencia de la propia Administración al generar un entorno
competitivo. Por otra parte, la iniciativa de la Web Semántica y una parte de su realización como es el enfoque
de \linkeddata, proporciona la base para el modelado de datos e información de forma estándar y consensuada. Finalmente, el tercer
componente a considerar reside en la tendencia actual de apertura de datos por parte de las instituciones públicas, gracias
a la sensibilización generada por el movimiento de \opendata. A partir de la conjugación de estos tres puntos clave se formula
la siguiente hipótesis, vertebradora de nuestra investigación.

\begin{Frame}
\textbf{Es posible mejorar el acceso a la información contenida en los anuncios de licitación de 
las distintas instituciones públicas europeas, tanto en términos cuantitativos como cualitativos, mediante métodos semánticos basados 
en aplicar y cumplir los principios de la iniciativa \linkeddata y de la misma forma mantener y favorecer
los principios de la corriente \opendata.}
\end{Frame}


\section{Objetivos Científico/Técnicos}\label{objetivos}
La cooperación de las iniciativas basadas en semántica como \linkeddata, unido al movimiento emergente
de \opendata en el contexto de la licitación pública, presenta una serie de beneficios que permiten
dar respuesta a problemas de gran calado que dificultan el despegue del mercado económico común. El estudio 
de la aplicación de la semántica en este contexto, junto con el despliegue tecnológico que soporte
las capacidades necesarias que se han subrayado dentro del proyecto ``10ders Information Services'', se
materializa en los siguientes objetivos científico-técnicos:

\begin{enumerate}
\item Estudiar, analizar y valorar las capacidades actuales del dominio de la contratación pública.  %1
\item Estudiar, analizar y valorar las corrientes de \opendata y Web Semántica, más en concreto \linkeddata, para su
aplicación en el dominio de la contratación pública.%2
 \item Definir los métodos basados en semántica para producir, publicar, consumir y validar 
la información de los anuncios de licitación siguiendo las directrices de \opendata y \linkeddata. %3
 \item Definir los algoritmos y procesos para dar soporte a la aplicación de los métodos basados en semántica 
a la información de los anuncios de licitación. %4
\item Implementar y reutilizar los componentes software necesarios para dar soporte a los métodos semánticos.%5
\item Promover el uso de estándares y la reutilización de información y modelos de conocimiento compartido.%6
\item Aplicar los métodos semánticos definidos al contexto de los anuncios de licitación pública.%7
\item Establecer un conjunto de prueba y validación de los componentes implementados para verificar la corrección, 
 validez y rendimiento de los métodos propuestos.%8
\item Difundir, formar y transferir la tecnología y conocimiento generado tanto a las comunidades científicas como industriales.%9
\end{enumerate}

Cada uno de estos objetivos se lleva a cabo en el conjunto de tesis y se presentan a lo largo de los capítulos que 
se desarrollan en este documento, según se desglosa en la Tabla~\ref{tabla:objetivos}.


\begin{longtable}[2]{|p{4cm}|p{8cm}|} 

\hline

  \textbf{Objetivo(s) } &  \textbf{Capítulo(s) } \\\hline
  
\endhead
  1 & Estudio, análisis y valoración  de la panorámica de la contratación pública en el Capítulo~\ref{capitulo:eprocurement}. \\ \hline
  2 & Estudio, análisis y valoración de la panorámica de la Web Semántica y \linkeddata en el Capítulo~\ref{capitulo:semantica}. \\ \hline
  3 y 4 & Definición de los métodos semánticos necesarios para dar soporte a la contratación pública mediante \linkeddata en los 
  Capítulos~\ref{capitulo:metodos} y~\ref{capitulo:metodos-separados}. \\ \hline
  5 y 6 & Implementación y reutilización de los componentes software necesarios para aplicar los puntos anteriores en el Capítulo~\ref{capitulo:moldeas}.\\ \hline
  7 y 8& Experimentación y validación de los métodos semánticos aplicados al contexto de la licitación pública en los Capítulos~\ref{capitulo:validacion} y~\ref{capitulo:conclusiones}. \\ \hline
  9 & Difusión y publicaciones realizadas tras el proceso de estudio, investigación y desarrollo en el Apéndice~\ref{capitulo:publicaciones}. \\ \hline

\hline

 \caption{Alineación de Objetivos y Capítulos.}
  \label{tabla:objetivos}
\end{longtable}


\section{Metodología de la Investigación}
La realización de la tesis debe plantearse como un proceso de trabajo, evolución y realimentación, en el cual
a partir de una hipótesis inicial, unos objetivos concretos y un plan de trabajo se puedan alcanzar 
los objetivos planteados de una forma sistemática. Con el objetivo de facilitar este proceso y asegurar la calidad del trabajo, 
es conveniente plantear el siguiente esquema para identificar correctamente el ``Problema a Resolver''.

\begin{itemize}
    \item ¿Por qué no se ha resuelto todavía?
    \item ¿Cómo se puede resolver?, ¿Resolución total, parcial, etc.?
    \item ¿Qué enfoques se pueden utilizar para el problema propuesto?, ¿Se puede mejorar alguno?
    \item ¿Cuáles son los factores críticos de éxito?, ¿Se pueden minimizar riesgos?, ¿Cuál sería el caso peor?
    \item ¿Es un problema actual?
    \item ¿Cuál sería el impacto de su resolución?
    \item ¿Cuáles son los escenarios relevantes para su aplicación posterior? 
\end{itemize}

Teniendo presentes las posibles respuestas a estas preguntas cabe definir claramente el proyecto de tesis describiendo pormenorizadamente
el problema o problemas a resolver, por ello es conveniente fijar un enfoque general, repasar el trabajo realizado y proponer
una planificación para la consecución en tiempo y esfuerzo de los objetivos prefijados. En este sentido, existen 
métodos~\cite{Hevner:2010:DRI:1859261} para dar soporte a la formulación, evaluación y validación del problema.

\begin{description}
 \item [\textit{Constructs}.] Proveen lenguaje, terminología y espacio, en la que un problema es definido y explicado.
 \item [\textit{Models}.] Cubren los hechos y conceptos en un dominio de interés o tipo de situaciones. Usan \textit{Constructs} 
  como lenguaje de descripción del espacio del problema.
 \item [\textit{Methods}.] Describen procesos y guían a los usuarios sobre cómo identificar soluciones aplicables a un problema investigación. 
  Abarcan desde la parte teórica de algoritmos matemáticos a su realización. Desarrollo de una metodología, técnica o algoritmo.
 \item [\textit{Implementations}.] Implementan los anteriores con el objeto de demostrar su viabilidad. Implementaciones de referencia.
\end{description}

Como resultado de su aplicación se puede, por ejemplo, generar una nueva área de investigación, desarrollar un \textit{framework} de trabajo, 
resolver problemas estancados, explorar minuciosamente un área, refutar los resultados actuales, crear nuevas metodologías o 
algoritmos más genéricos. 

Esta metodología de trabajo sólo pretende ser una pequeña guía para centrar los esfuerzos y 
economizar el tiempo de desarrollo de la tesis. Las tareas aquí mencionadas serán 
responsabilidades tanto del Director de la tesis como del Doctorando. El objetivo final, será obtener un 
resultado relevante para la comunidad científica que pueda ser proyectado posteriormente 
en un entorno industrial.

\subsection{Plan de Trabajo}
El plan de trabajo que da soporte a la realización de la investigación, se divide en las siguientes fases:
\begin{enumerate}
 \item Trabajo de Investigación, resumen y evaluación de la investigación realizada identificando problemas, métodos y propuesta de tema objeto
de estudio. El resultado de este trabajo se materializa en:
  \begin{itemize}
   \item Entrega del Trabajo de Investigación en el Departamento de Informática y obtención del Diploma de Estudios Avanzados.
   \item Realización, al menos, de dos publicaciones en congresos. En este sentido, se debe adecuar el lugar de publicación al 
    avance del trabajo realizado. Se fijan como lugares objetivos: \textit{PhD Symposium}, \textit{workshops} o conferencias internacionales 
    de un nivel asequible. El objetivo principal será obtener realimentación de la línea de investigación marcada.
  \item El tiempo para cubrir esta primera etapa será de doce meses.
  \end{itemize}
\item Elaboración y evaluación de la solución propuesta. En esta etapa, se deben resolver las cuestiones principales de la tesis. 
  Encontrar posibles soluciones, formulación, evaluación y validación. El resultado de este trabajo se materializa en:
 \begin{itemize}
  \item Documento con la definición formal, evaluación y validación.
  \item Realización, al menos, de dos publicaciones en congresos o revistas (JCR) de carácter y prestigio internacional. 
  Retroalimentación de  expertos: colaboración en listas de correos, proyectos, etc.
  \item El tiempo para cubrir esta segunda etapa será de doce meses.
 \end{itemize}
  \item Escritura del documento de tesis. Finalización, redacción de toda la documentación necesaria.
  El resultado de este trabajo se materializa en:
  \begin{itemize}
    \item Documento final de la tesis para la obtención del título de Doctor.
   \item Realización, al menos, de dos publicaciones en congresos de carácter y prestigio internacional y dos revistas (JCR).  
   \item Realimentación de expertos: colaboración en listas de correos, proyectos, etc. 
   \item El tiempo para cubrir esta última etapa será de doce meses.
  \end{itemize}

\end{enumerate}

Este proceso de elaboración de la tesis es ideal, y ha sido cubierto teniendo en cuenta
las fluctuaciones habituales procedentes del ejercicio de otras tareas. Para minimizar
los retrasos, se ha establecido un método de monitorización consistente en revisiones periódicas entre el Director y el Doctorando. 

\subsection{Metodología de Trabajo}
En esta sección se presenta un resumen de la metodología de trabajo que se ha llevado a cabo durante la duración 
del trabajo en el proceso para la consecución de la tesis. Para alcanzar los objetivos científico-técnicos planteados, se han definido
tareas de ``investigación y prototipado'' y otras de carácter transversal que cubren las necesidades generales de la 
realización de la tesis, como pueden ser la ejecución de pruebas o la redacción de documentación y publicaciones.

El estudio realizado en cuanto a contratación pública, como proceso administrativo y dentro del marco de la administración
electrónica, así como la panorámica de uso de la Web Semántica, \linkeddata y \opendata afectan de forma crucial
en las tareas propias de ``investigación y prototipado'', ya que se produce un proceso de realimentación que permite
acotar las referencias a consultar y focalizar el proceso de investigación. Esto conlleva que aquellas tareas
propias de la ingeniería, sirvan como retroalimentación para el estudio realizado en las distintas líneas
de investigación. De esta forma, la validación y las pruebas sirven para definir y realimentar los activos
experimentales de los componentes de software.

La investigación sigue un modelo diferente a los clásicos modelos de ciclo de vida de ingeniería de software,
 más orientado al problema científico a resolver. Con ello, se realiza un enfoque de ``investigación y experimentación concurrente'', 
en el que las tareas de investigación y experimentación se realizan en paralelo obteniendo 
así un ciclo de vida en ``espiral'', que se adapta mejor al dinamismo de la investigación. De esta forma se sientan las bases para un modelo
de estudio y desarrollo con realimentación continua de la experimentación, se orienta a la consecución de investigación y prototipos iterativos.

En contraste con las aproximaciones donde los proveedores de software tradicionales evalúan el resultado del proyecto durante la fase final del mismo,
se realiza una aproximación de ingeniería concurrente dónde la investigación, el prototipado de software, 
la experimentación y la validación se realizan en paralelo. De esta manera, las actividades de investigación reciben realimentación de su
relevancia en una fase temprana de la ejecución.

Finalmente, el \textit{software} realizado durante la tesis se desarrolla mediante un modelo basado en comunidad
de software libre, siendo el que mejor se adapta para obtener una
colaboración y coordinación eficiente de las diferentes partes involucradas en la ejecución
del proceso: validación por parte del Director, publicación de resultados, etc.
 Este modelo es fácil de gestionar y permite incorporar nuevos investigadores y contribuciones 
externas sin esfuerzo. Para dar soporte a este modelo se usa la infraestructura proporcionada por Google Code, adicionalmente, 
se emplean otras herramientas de software libre disponibles como:
\begin{itemize}
 \item Herramientas de desarrollo software: \gls{Eclipse} IDE y \textit{plugins}, lenguajes de
programación: Java, Python, etc.
 \item Herramientas de razonamiento y distribución: razonadores como Pellet, motores de
inferencia como Drools, Prolog, Apache Hadoop, Apache \gls{Mahout}, etc.
\item Herramientas de integración de software: \gls{ANT}, \gls{Maven}, Continuum, etc.
 \item Herramientas de edición de documentos: Open-Office y LaTeX.
\item Otras herramientas: Google \gls{Refine}, Protégé, \gls{SNORQL}, \gls{Pubby}, etc.
\end{itemize}

\section{El Camino hacia la Tesis}
En el verano del año 2002 tras superar el primer curso en la Escuela de Ingeniería Técnica Informática de Oviedo
y adquirir los primeros conocimientos de programación, algorítmica y matemáticas aplicados a la informática
decidí, \textit{motu proprio}, introducirme en el mundo de Internet. Recuerdo, el portal de \textit{Geocities} en el 
cual se proporcionaba una herramienta tipo \textit{Microsoft Front Page}, para la creación de sencillas páginas
web, incluyendo formularios, encuestas, etc., en aquel momento, aunque no entendía el código \gls{HTML} ni la forma de publicar páginas
web, cree mi primera página, que incluía con una encuesta con una votación sobre los participantes de un programa
de talentos, que en ese momento se emitía en televisión. Una vez concluida la edición de la página, cuando pulse el botón ``Publicar'' 
rápidamente avisé a mi hermana para mostrarle, orgulloso, que mis conocimientos estaban en aumento, 
sin embargo su reacción no fue ni mucho menos la esperada, con un gesto que denotaba incredulidad y asombro comentó: ``¿qué es esto?''. 
Esta anécdota que suele recordar cada cierto tiempo, o más bien cuando le muestro algo nuevo que he hecho, me recuerda 
la evolución que he experimentado en el transcurso de estos últimos años en el campo de la informática y de la tecnología 
web en concreto.

En el camino hacia la Tesis, existe otro punto de inflexión que tiene lugar en el invierno del año 2004 tras finalizar y 
obtener el título de Ingeniero Técnico en Informática de Sistemas, cuando surgió la posibilidad
de formar parte de un nuevo equipo multidisciplinar, en el cual se iba a iniciar actividad de investigación en el área
de Web Semántica. Este equipo comandado por expertos de la Universidad de Oviedo como José Emilio Labra, Enrique Del Teso (Filología),
 Guillermo Lorenzo (Filología) o Roger Bosch (Filosofía) y ubicado en la Fundación CTIC bajo la 
supervisión de Antonio Campos, comenzó su andadura con la tecnología semántica para desarrollar un sistema de búsqueda para
el Boletín Oficial del Principado de Asturias (\gls{BOPA}). Con aquel equipo, con el cual compartí casi 5 años y formado
por Diego Berrueta, Luis Polo, Emilio Rubiera, Manuel Cañon e Iván Frade se desarrolló un buscador ``semántico''~\cite{bopaEstonia} para
las disposiciones que se publicaban en el BOPA. Se basada en la utilización en una técnica de ``Concept Expansion'', utilizando
ontologías que después se transformaba a una consulta en \textit{Apache Lucene}, los resultados no pudieron ser mejores y fue un gran éxito a nivel
estratégico, de tal forma que actualmente todavía está en producción (la parte de búsqueda sintáctica) en la Administración del Principado de 
Asturias y, siendo seleccionado como un caso de uso~\cite{bopa-use-case} por el \gls{W3C}. 

Al presente valorando y analizando el desarrollo de aquel proyecto en el que se abordaron varios problemas, tales como:\textit{screen scrapping} de los datos de las disposiciones, reconciliación de entidades, transformación de clasificaciones de productos o \textit{mapeo} de palabras a conceptos, 
que atañen hoy al mundo de la Web Semántica y de la iniciativa \linkeddata y también casi accidentalmente, se proporcionaron 
servicios como suscripción \gls{RSS} a un consulta, navegación gráfica sobre las ontologías de dominio, búsqueda contextualizada por varios criterios
y múltiples formatos para visualizar la información de las disposiciones (\gls{HTML} y \gls{RDF}). Hoy en día este proyecto estaría
de actualidad en cualquier Administración, probablemente bajo un nombre tipo: ``Portal multicanal de \opendata y \linkeddata para el acceso a las disposiciones
del boletín x'', la cuestión reside en que de una manera intuitiva, se desarrolló investigación y tecnología a la cual en nuestros días
se le designa con un nombre concreto y que además se puede comercializar. Este antecedente simplemente trata de revelar la evolución 
que en este área se ha producido. 

Durante los siguientes años, seguí mi andadura en CTIC~\cite{ctic} participando en proyectos de investigación relacionados 
siempre con semántica y sobretodo con servicios web, en los cuales me gustaría destacar la actividad desarrollada 
en el proyecto PRAVIA~\cite{pravia} y su sucesor a nivel nacional el proyecto PRIMA~\cite{prima}, en ambos casos se trataba de utilizar servicios web semánticos para facilitar la interacción con los proveedores de servicios, en el sector 
asegurador, si bien se consiguieron realizar distintos prototipos, la gran lección aprendida fue
la relativa a la notable complejidad de esta iniciativa, que por entonces, y probablemente de momento no se puede resolver de forma
sencilla. Este trabajo me facilitó la posibilidad de participar en otros proyectos como EzWeb~\cite{ezweb}, en la parte de recomendación
y descubrimiento de servicios, MyMobileWeb~\cite{mymobileweb}, adición de semántica para gestionar el contexto en 
aplicaciones móviles, SAITA~\cite{saita}, creación de una pasarela \gls{REST}-\gls{SOAP}, etc. Es en esta época cuando surge mi trabajo de investigación,
 cuyo tema principal era la interoperabilidad e integración en arquitecturas orientadas a servicios usando tecnologías semánticas. 
Finalmente, en mi última etapa en CTIC participé en el proyecto europeo ONTORULE~\cite{ontorule}, en el cual realicé actividades 
relacionadas con la gestión del proyecto y desde el punto de vista de la investigación, con la esfera de las reglas de negocio y ontologías, 
específicamente en tareas relacionadas con \gls{RIF}~\cite{rif-core}(\textit{Rule Interchange Format}).

\subsection{Elaboración de la Tesis en el contexto del equipo de investigación WESO}
Desde noviembre del año 2010 y como miembro del equipo de investigación de Web Semántica Oviedo dirigido por José 
Emilio Labra Gayo se disfruta de un entorno dinámico, colaborativo y flexible para la realización de trabajos 
de investigación e innovación mediante la realización de tareas y actividades como participación en proyectos, 
viajes, redacción de artículos, asistencia a conferencias, etc. 

La metodología de trabajo utilizada en este equipo favorece la interacción entre las personas y 
el impulso de la investigación para conseguir resultados tanto desde punto de vista cualitativo 
como cuantitativo. 

Mi llegada al equipo \gls{WESO} ha supuesto tanto un nuevo reto como un impulso a mi carrera profesional e investigadora 
en la cual he podido perfeccionar y aumentar mi conocimiento aplicándolo a nuevos desafíos y trabajos 
en los cuales he tenido la suerte de participar con excelentes profesionales que de una forma 
u otra también forman parte de este trabajo.

\section{Convención utilizada en este documento}
A lo largo de este documento se utilizan distintos tipos de letra y puntuación para señalar partes destacadas
del mismo y centrar la lectura del lector. La convención seguida en todo el documento es la siguiente:
\begin{description}
 \item [Letra en \textit{cursiva}.] Se utiliza para destacar las palabras procedentes de otro idioma, por ejemplo en inglés \linkeddata, o para señalar
tecnicismos, por ejemplo \textit{cloud computing}.
 \item [Letra en \textit{cursiva} en un cuadro.] Se utiliza para referenciar partes de texto que han sido tomadas de otros documentos en las que 
es importante mantener exactamente la definición que realizan. Por ejemplo, algunos textos tomados de directivas de la Unión Europea.
\item [Letra en negrita.] Se utiliza para destacar palabras o párrafos de alto valor para el correcto entendimiento del documento. Por ejemplo,
en la formulación de la hipótesis.
\item [Texto ``entrecomillado''.] Se utiliza para destacar palabras o párrafos en los cuales se cita alguna frase u oración
de un autor conocido o bien para simular la primera persona en la expresión escrita.
\item [Figuras.] Sirven para designar: diagramas, imágenes o trozos de ejemplo de código fuente. Por ejemplo, se utilizan para formalizar
de forma gráfica el diseño del sistema \gls{MOLDEAS}.
\item [Tablas.] Se utilizan tablas para indicar una serie de características y su evaluación. Por ejemplo, durante la evaluación,
se usa este formato de presentación para condensar gran cantidad de información facilitando así su lectura.
\end{description}

El principal objetivo del empleo de esta notación, es amenizar y hacer más atractiva
la lectura y comprensión del texto.

Igualmente, se utiliza indistintamente el concepto de licitaciones, anuncio de licitación o contrato público, para
designar al documento que contiene la información y los datos de un contrato público. También se utilizan tecnicismos 
provenientes del idioma inglés, cuyo uso está perfectamente asentado e integrado en la prosa habitual
en lo concerniente a temas como Web Semántica, \linkeddata, \opendata, \lod o \wod. Entre ellos, se puede destacar \gls{URI}, \gls{IRI},
recurso RDF, etc., para evitar la cacofonía y aunque la traducción al español implica el uso del género masculino, en 
muchos casos se opta por adaptar la expresión escrita a su uso habitual. De esta manera, se habla
de ``la URI de un recurso RDF'', en lugar de ``el URI de un recurso RDF'' y en general, para ciertas
traducciones se emplea el género femenino en lugar del masculino con el objetivo de facilitar la lectura
y adaptar la forma escrita al modo habitual de expresión.

