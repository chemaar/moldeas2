Como ya se ha introducido en el Apéndice~\ref{impacto}, la difusión de los resultados
de investigación es una pieza clave para validar y continuar el trabajo en una
determinada línea de investigación. Es por ello que la realización de publicaciones
científicas y de otras más orientadas a la industria, deben ser contempladas
desde el principio para conseguir un impacto real y de calidad en las distintas
comunidades. En las siguientes secciones se detallan algunas de las publicaciones
más destacadas realizadas en colaboración con distintas instituciones y empresas 
que han resultado del trabajo de esta tesis.

Para facilitar la lectura de las publicaciones realizadas a continuación 
se dispone de una lista, ver Tabla~\ref{table:publications}, de las distintas personas con las que se ha
colaborado ordenadas por afiliación.

\begin{longtable}[c]{|p{8cm}|p{6cm}|} 
\hline
  \textbf{Autor} & \textbf{Afiliación} \\\hline
\endhead
 José Emilio Labra Gayo & Universidad de Oviedo \\\hline
 Patricia Ordoñez De Pablos & Universidad de Oviedo \\\hline
 Francisco Cifuentes Silva & Universidad de Oviedo \\\hline
 Jorge González Lorenzo & Universidad de Oviedo \\\hline
 Pablo Abella Vallina & Universidad de Oviedo \\\hline
 Weena Jiménez Nácero & Universidad de Oviedo \\\hline
 Miguel García Rodríguez & Universidad de Oviedo y Fundación CTIC\\\hline
 Antonio Campos López & Fundación CTIC\\\hline
 Diego Berrueta Muñoz & Fundación CTIC\\\hline
 Luis Polo Paredes & Fundación CTIC\\\hline
 Emilio Rubiera Azcona & Fundación CTIC\\\hline
 Jose Luis Marín & Gateway S.C.S\\\hline
 Ángel Marín & Gateway S.C.S\\\hline
 Mai Rodríguez & Gateway S.C.S\\\hline
 Ramón Calmeau & EXIX TI\\\hline
 Giner Alor-Hernández & Instituto Tecnológico de Orizaba en México\\\hline
 Cauthemoc Sánchez & Instituto Tecnológico de Orizaba en México\\\hline
 Jaime Alberto Guzman Luna & Universidad Nacional de Colombia\\\hline
 Cristina Casado Lumbreras & Universidad Carlos III de Madrid\\\hline
 Alejandro Rodríguez González & Universidad Carlos III de Madrid\\\hline
 Ricardo Colomo Palacios & Universidad Carlos III de Madrid\\\hline
\hline
\caption{Colaboración con autores en las publicaciones.}\label{table:publications}\\    
\end{longtable}

\subsection{Revistas Internacionales con Índice de Impacto}

\begin{enumerate}
 \item Jose María Alvarez Rodríguez, José Emilio Labra Gayo, Francisco Cifuentes Silva, Giner Alor-Hernández, Cauthemoc Sánchez y Jaime
Alberto Guzman Luna. \textit{Towards a Pan-European E-Procurement platform to Aggregate, Publish and Search Public Procurement Notices 
powered by Linked Open Data: The MOLDEAS Approach}. International Journal of Software Engineering and Knowledge Engineering (IJSEKE ) 
Focused Topic Issue on Consuming and Producing Linked Data on Real World Applications. 2011. Factor de Impacto: $0.262$.

\item Jose María Alvarez Rodríguez, José Emilio Labra Gayo, Patricia Ordoñez De Pablos. \textit{Survey of New Trends on \eproc Applying Semantics}.
International Journal of Computers in Industry Focused Topic Issue on New Trends on \eproc Applying Semantics. 2014. Factor de Impacto: $1.620$. Como
parte de la aceptación de esta revista para ser editores invitados del \textit{Special Issue} titulado como \textit{New Trends on \eproc Applying Semantic}
se nos emplaza por parte de los editores a realizar esta artículo. La semilla del mismo es el propio documento de la tesis.

\end{enumerate}

\subsection{Editor Invitado de \textit{Special Issues} en Revistas Internacionales}

\begin{enumerate}
\item Jose María Alvarez Rodríguez, José Emilio Labra Gayo, Patricia Ordoñez De Pablos. \textit{New Trends on \eproc Applying Semantics}.
International Journal of Computers in Industry Focused Topic Issue on New Trends on \eproc Applying Semantics. 2014. Factor de Impacto: $1.620$.
\end{enumerate}

\subsection{Revistas Internacionales}
\begin{enumerate}
\item Jose María Alvarez Rodríguez, José Emilio Labra Gayo, Patricia Ordoñez De Pablos.\textit{ An Extensible Framework to Sort Out 
Nodes in Graph-based Structures Powered by the Spreading Activation Technique: The ONTOSPREAD approach}. 
International Journal of Knowledge Society Research (IJKSR). 2011. Aceptado para publicación.
\item Jorge González Lorenzo, José Emilio Labra Gayo y Jose María Alvarez Rodríguez. \textit{A MapReduce implementation of the Spreading Activation 
algorithm for processing large knowledge bases based on semantic networks}. International Journal of Knowledge Society Research (IJKSR). Aceptado para publicación. 
\item Jose María Alvarez Rodríguez, José Emilio Labra Gayo, Ramón Calmeau, Ángel Marín y Jose Luis Marín. \textit{Query Expansion Methods and Performance Evaluation for Reusing Linking 
Open Data of the European Public Procurement Notices}. Current Topics in Artificial Intelligence. 14th Conference of 
the Spanish Association for Artificial Intelligence, CAEPIA 2011, La Laguna, Spain, November 8-11, 2011, Selected Papers.
\item Jose María Alvarez Rodríguez, José Emilio Labra Gayo, Ramón Calmeau, Ángel Marín y Jose Luis Marín.\textit{Innovative Services to ease the Access to the Public Procurement Notices 
using Linking Open Data and Advanced Methods based on Semantics}. International Journal of Electronic Government. Aceptada para publicación, marzo 2012.
\end{enumerate}


\subsection{Capítulos de Libros}
\begin{enumerate}
\item Jose Luis Marín, Mai Rodríguez, Ramón Calmeau, Ángel Marín, Jose María Alvarez Rodríguez y José Emilio Labra Gayo.
 \textit{Euroalert.net: aggregating public procurement data to deliver commercial services to SMEs}. 
``E-Procurement Management for Successful Electronic Government System''. IGI Global. 2012.

\item Jose María Alvarez Rodríguez, Luis Polo Paredes, Emilio Rubiera Azcona, José Emilio Labra Gayo y Patricia Ordoñez De Pablos.
 \textit{Enhancing the Access to Public Procurement Notices by Promoting Product Scheme Classifications to the 
Linked Open Data Initiative}. ``Cases on Open-Linked Data and Semantic Web Applications''. IGI Global. 2012.
\end{enumerate}
\subsection{Conferencias Internacionales}

\begin{enumerate}
\item Jose María Alvarez Rodríguez, José Emilio Labra Gayo, Patricia Ordoñez De Pablos.\textit{ An Extensible Framework to Sort Out 
Nodes in Graph-based Structures Powered by the Spreading Activation Technique: The ONTOSPREAD approach}. The 4th World Summit on the 
Knowledge Society (WSKS 2011). 2011.
\item Jorge González Lorenzo, José Emilio Labra Gayo y Jose María Alvarez Rodríguez. \textit{A MapReduce implementation of the Spreading Activation 
algorithm for processing large knowledge bases based on semantic networks}.The 4th World Summit on the 
Knowledge Society (WSKS 2011). 2011.
\item Jose María Alvarez Rodríguez, José Emilio Labra Gayo, Ramón Calmeau, Ángel Marín y Jose Luis Marín.\textit{ Innovative Services to ease the Access to the Public Procurement Notices 
using Linking Open Data and Advanced Methods based on Semantics}. 5th International Conference on Methodologies \& Tools enabling e Government (MeTTeG 2011). Camerino-Italia 2011.
\end{enumerate}

\subsection{\textit{Workshops} Internacionales}

\begin{enumerate}
\item Jose María Alvarez Rodríguez, Luis Polo Paredes, Pablo Abella Vallina, Weena Jiménez Nácero y José Emilio Labra Gayo. 
\textit{Application of the Spreading Activation Technique for Recommending Concepts of well-known ontologies in Medical Systems}. 
SATBI 2011. ACM BCB (Chicago IL, USA).
\item Jose Luis Marín, Mai Rodríguez, Ramón Calmeau, Ángel Marín, Jose María Alvarez Rodríguez y José Emilio Labra Gayo.
\textit{Euroalert.net: Building a pan-European platform to aggregate public procurement data and deliver commercial services for SMEs powered by open data}.
in Workshop Share-PSI.eu. 2011.
\item Jose María Alvarez Rodríguez, Emilio Rubiera Azcona y Luis Polo Paredes. \textit{Promoting Government Controlled Vocabularies 
for the Semantic Web: the EUROVOC Thesaurus and the CPV Product Classification System}. SIEDL 2008. ESWC 2008. 
\end{enumerate}

\subsection{\textit{Posters} Internacionales}

\begin{enumerate}
\item Jose María Alvarez Rodríguez, José Emilio Labra Gayo, Ramón Calmeau, Ángel Marín y Jose Luis Marín. 
\textit{MOLDEAS-Methods On Linked Data for E-procurement Applying Semantics}. The Seventh Reasoning Web Summer School, Galway, 
Ireland at the Digital Research Enterprise Institute (DERI) from August 23-27, 2011.

\item Jose María Alvarez Rodríguez y José Emilio Labra Gayo. \textit{Semantic Methods for Reusing Linking Open Data of the European Public Procurement Notices}.
PhD Symposium ESCW 2011. Se aceptó como póster pero finalmente se declinó ir a presentar este trabajo.

\end{enumerate}


\subsection{\textit{Workshops} Nacionales}

\begin{enumerate}
\item Jose María Alvarez Rodríguez, José Emilio Labra Gayo, Ramón Calmeau, Ángel Marín y Jose Luis Marín. \textit{Query Expansion Methods and Performance Evaluation for Reusing Linking 
Open Data of the European Public Procurement Notices}. Workshop Tecnologías Linked Data y sus Aplicaciones en España, 
CAEPIA 2011, La Laguna, Spain, November 8-11, 2011.

\item Francisco Cifuentes Silva, José Emilio Labra Gayo y Jose María Alvarez Rodríguez. \textit{An architecture and process of 
implantation for Linked Data environments}. Workshop Tecnologías Linked Data y sus Aplicaciones en España, CAEPIA 2011, 
La Laguna, Spain, November 8-11, 2011.

\item Jose María Alvarez Rodríguez, Emilio Rubiera Azcona y Luis Polo Paredes. \textit{Generación automática de ontologías en SKOS de 
clasificaciones estándar de productos: Common Procurement Vocabulary (CPV)}. CEDI 2007. CAEPIA 2007.

\end{enumerate}


\subsection{Relacionados con Web Semántica y \linkeddata}

\begin{enumerate}

\item Cristina Casado Lumbreras, Alejandro Rodríguez González, Jose María Alvarez Rodríguez y Ricardo Colomo Palacios. 
\textit{PsyDis: towards a diagnosis support system for psychological disorders.}. International Journal Expert Systems With Applications. 2012. 
Factor de Impacto: $1.926$. Aceptado para publicación.

\item Jose María Alvarez Rodríguez y Antonio Campos López. 
\textit{Integration and Interoperability on Service Oriented Architectures using Semantics}. PhD Symposium. ICWE 2009.
San Sebastián, Spain.

\item Miguel García Rodríguez, Jose María Alvarez Rodríguez, Diego Berrueta Muñoz y Luis Polo Paredes. 
\textit{Declarative Data Grounding Using a Mapping Language}. Communications of SIWN. ISSN 1757-4439. April 2009.


\item Miguel García Rodríguez, Jose María Alvarez Rodríguez, Diego Berrueta Muñoz y Luis Polo Paredes. 
\textit{Declarative Data Grounding Using a Mapping Language}. 3rd Complex Distributed Systems (CODS 2009). Leipzig, Germany International Conference.

\end{enumerate}

Como resumen de estas publicaciones, cabe señalar que la mayoría se han realizado en la fase final
de la tesis debido a la consolidación de la investigación realizada y a la disposición de fondos para su
difusión. De la misma forma, la publicación de los artículos ha sido acompañada de las presentaciones
consiguientes en los lugares de aceptación y en la mayoría de los casos los artículos presentados
han recibido la correspondiente invitación para su posterior publicación en revistas bajo las condiciones
establecidas por los organizadores sobre extensión y contenido.

Finalmente, estos trabajos sirven como punto de partida y experiencia para la realización de nuevas publicaciones, teniendo
como objetivo la mejora continua en calidad y contemplando aspectos de investigación de la tesis que se 
consideran de gran valor para este trabajo y de los cuales se intentará obtener la mayor realimentación posible a 
través de su divulgación en la comunidad científica.


\subsection{Otros}
Debido a la participación del equipo \gls{WESO} de la Universidad de Oviedo en la ``Red Temática Española
de Linked Data'' se ha presentado este trabajo en la reunión plenaria celebrada en junio de 2011 en las
instalaciones de la Universidad de Politécnica de Madrid. 

Por otra parte, se han redactado varios entregables~\cite{web-personal} de los distintos proyectos de investigación
en los que se ha participado y dirigido proyectos fin de carrera como ``Sistema Dinámico y Automático de Generación de Preguntas desde Fuentes Heterogéneas basada en Formatos de Publicación de la Web Semántica''
presentado por Lucía Otero García en marzo de 2010 para la obtención del título de Ingeniería Técnica en Informática
en la Escuela Politécnica de Ingeniería de Gijón.

Además, como parte de la actividad institucional y profesional desarrollada por los distintos miembros del grupo
se han realizado demostraciones a distintas empresas y organismos públicos con el objetivo de concienciar
de las posibilidades del enfoque de \linkeddata para la resolución de ciertos problemas, como la integración 
de fuentes datos heterogéneas o la mejora de los sistemas de búsqueda.
