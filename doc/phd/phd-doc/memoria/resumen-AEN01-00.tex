\chapter*{Resumen Ejecutivo}
\thispagestyle{empty}
Este subproyecto consiste en finalizar la generación y documentación de datos RDF de
entidades que serán utilizadas para el marcado automático de documentos en el proyecto
Historia de la Ley. Las entidades sobre las que se trabajará son las siguientes:
\begin{itemize}
 \item Personas: Como parlamentarios, políticos, y personas naturales.
 \item Localidades: Como divisiones político administrativas (países, regiones, comunas, provincias) o divisiones electorales (distritos y circunscripciones).
 \item Documentos: Como diarios de sesión parlamentaria, informes de comisión y otros.
 \item Organizaciones: Como ministerios, cámaras, ONG’s, partidos políticos y otros.
 \item Roles: Diferentes roles que pueda tomar una persona en cualquiera de sus apariciones en diferentes documentos, ya sea como participante activo o como
alguien mencionado dentro del documento.
 \item Eventos: Como sesiones parlamentarias, reuniones de comisión, nombramientos, cuentas públicas u otros.
 \item Proyectos de Ley: Este tipo de entidad se modela separada dada su naturaleza evolutiva en el tiempo.
\end{itemize}


Hay que considerar que los modelos a utilizar deben permitir una fácil integración con la
herramienta de mediación a generar en el siguiente subproyecto. Para cada uno de estos tipos de entidades se debe generar documentación que
posteriormente será publicada en la Web de datos.bcn.cl para habilitar el consumo de los
datos. Esta documentación contempla lo siguiente:
\begin{itemize}
 \item Modificaciones a las ontologías documentadas mediante.
 \item Diseño de URIs por cada tipo de instancia.
 \item Diseño del RDF de ejemplo por cada acceso a URI.
\end{itemize}


