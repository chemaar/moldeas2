%sobre
\chapter*{Sobre este documento}

Este documento recoge el estudio realizado sobre la aplicación
de métodos semánticos para la producción, publicación y consumo 
de datos enlazados abiertos en el contexto de la administración pública
electrónica y concretamente en el campo de \eproc o contratación pública
electrónica. Esta memoria de investigación titulada como
``Métodos Semánticos de Reutilización de Datos Abiertos Enlazados en las
Licitaciones Públicas'' se ha elaborado para la obtención del título
de Doctor por la Universidad de Oviedo.

El documento se encuentra organizado de la siguiente forma:
\begin{itemize}
\item ``Introducción'' (ver Capítulo~\ref{capitulo:introduccion}): se realiza
una motivación del problema de investigación a resolver identificando el plan
y la metodología de trabajo seguida durante el proceso de elaboración
de la tesis doctoral.

\item ``Contratación Pública y \eproc'' (ver Capítulo~\ref{capitulo:eprocurement}): se 
presenta la casuística del dominio del problema a estudiar, así como las soluciones
actuales más cercanas al enfoque desarrollado durante la investigación en el dominio
de la administración electrónica centrando los esfuerzos en la valoración
de los procesos de contratación pública electrónica.

\item ``Panorámica de uso de la Web Semántica y \linkeddata'' (ver Capítulo~\ref{capitulo:semantica}): se
repasan desde un punto de vista general el estado del arte de estas iniciativas, así como
su aplicación en los distintos dominios.

\item ``Definición de Métodos Semánticos'' (ver Capítulo~\ref{capitulo:metodos}): se realiza
la definición teórica de una serie de procesos, métodos y tareas a realizar para la promoción
de datos siguiendo las directrices de \linkeddata y atendiendo a los principios
de \opendata.

\item ``Métodos Semánticos en el ámbito de las Licitaciones Públicas'' (ver Capítulo~\ref{capitulo:metodos-separados}): se materializan 
los procesos, métodos y tareas definidos en el capítulo anterior para su aplicación en el campo de los anuncios de licitación públicos.


\item ``Sistema MOLDEAS'' (ver Capítulo~\ref{capitulo:moldeas}): se describe la implementación
de los componentes realizados para dar soporte a los procesos, métodos y tareas de carácter
semántica en el dominio de las licitaciones públicas.


\item ``Experimentación y Validación'' (ver Capítulo~\ref{capitulo:validacion}): se diseña
el experimento y se evalúan y valoran los resultados obtenidos de la investigación realizada.

% \item ``Evaluación'' (ver Capítulo~\ref{capitulo:evaluacion}): se evalúan los
% resultados obtenidos durante la experimentación realizada.

\item ``Conclusiones y Trabajo Futuro''(ver Capítulo~\ref{capitulo:conclusiones}): se comentan las conclusiones obtenidas
tras la realización de esta memoria de investigación. También se marcan las líneas de evolución futura.

\item ``Impacto y Difusión'' (ver Apéndice~\ref{capitulo:impacto}): se detallan las actividades
de difusión realizadas a través de los distintos canales durante la elaboración de la tesis doctoral valorando su impacto
en la comunidad científica e industrial.


\item ``Trabajos publicados'' (ver Apéndice~\ref{capitulo:publicaciones}): se enumeran
las publicaciones científicas realizadas que han surgido como parte del trabajo realizado
durante la elaboración de la investigación.


\item ``Tablas de Validación'' (ver Apéndice~\ref{capitulo:tablas}): se presentan
las tablas de validación pormenorizadas atendiendo a todos los criterios definidos
en la experimentación de la investigación.


\end{itemize}

\chapter*{Sobre el autor}

Jose María Alvarez Rodríguez es Ingeniero en Informática (2007) e Ingeniero Técnico en Informática de Sistemas (2005) por la Universidad de Oviedo. 
En junio de 2008 recibió el Premio al Mejor Proyecto Fin de Carrera de Ingeniería Informática por su proyecto ``Activación de Conceptos 
en Ontologías mediante el algoritmo de Spreading Activation'' otorgado por el Colegio Oficial del Principado de Asturias. 


Desde abril del año 2005 a enero del año 2010 desempeña su actividad investigadora en el área de tecnologías semánticas 
del departamento de I+D+i de la Fundación CTIC de Asturias, especializándose en servicios web semánticos, sistemas basados en reglas, sistemas de búsqueda y 
datos enlazados. Durante esta etapa participa activamente en la redacción y ejecución de proyectos de investigación en distintos ámbitos 
destacando: regional (proyecto PRAVIA-PCTI Asturias cod. IE05-172), nacional (proyecto PRIMA-Plan Avanza-cod. TSI-020302-2008-32) y 
europeo (proyecto ONTORULE-FP7 cod. 231875). Como fruto de esta actividad es autor de diversas publicaciones en el área de semántica 
de carácter internacional en conferencias y \textit{workshops} realizando su trabajo de investigación en 2009 bajo 
el título de ``Interoperabilidad e Integración en Arquitecturas Orientadas a Servicios basadas en Semántica'' para el Programa de 
Doctorado 34.1-``Sistemas y servicios informáticos para internet'' del Departamento de Informática de la Universidad de Oviedo. Durante el curso 2008/2009 obtiene una plaza de profesor asociado (ref. Código: F036-75-DL0X041-AL6H. Bopa Nº 181, 4 de Agosto de 2008) 
impartiendo docencia en el área de Ciencias de la Computación e Inteligencia Artificial del Departamento de Informática de la Universidad de Oviedo. 


Actualmente se encuentra contratado por el grupo WESO del Departamento de Informática de la Universidad de Oviedo dirigido por 
el profesor Dr. José Emilio Labra Gayo para la ejecución de distintos proyectos de investigación a nivel regional y nacional, destacando  
el proyecto \textit{10ders Information Services} (Plan Avanza-cod. TSI-020100-2010-919) y \textit{ROCAS-Reasoning On the Cloud Applying Semantics} 
(Ministerio de Ciencia e Innovación-TIN2011-27871) lo que le ha habilitado para la consecución de becas de carácter competitivo internacional, 
como la obtenida en marzo de 2012 a través de la red europea: \textit{HPC-Europa2: Pan-European Research Infrastructure for High Performance Computing}.
 También ha compaginado su actividad investigadora con una plaza de profesor asociado en el área de Lenguajes y Sistemas Informáticos del Departamento de Informática de la Universidad de Oviedo durante el curso académico 
2011/2012 participando en la co-dirección de proyectos fin de carrera. Finalmente y como parte esencial de la actividad investigadora 
es autor y revisor de varias conferencias y revistas de carácter internacional en el área de tecnologías semánticas.



