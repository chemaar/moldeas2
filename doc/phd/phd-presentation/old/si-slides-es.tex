%
% main.tex
%

% notes = hide | show | only
\documentclass[xcolor=dvipsnames,dvip,notes=hide, table]{beamer}

% Para crear una versión 'handout' (impresa)
%\documentclass[xcolor=pst,dvips,handout,notes=show]{beamer}

%
% cabeceras.tex
%
% Copyright 2003, Diego Berrueta Muñoz
%
% Cabeceras comunes
%

\usepackage[T1]{fontenc}

%Glosario

\usepackage[toc,style=treenoname,order=word,counter=section]{glossaries}

\usepackage{xspace}


\usepackage{tikz,times}
\usetikzlibrary{mindmap,backgrounds}

% cambia algunas fuentes (utilidad dudosa)
\usepackage[scaled=0.92]{helvet}
\usepackage{pifont}
\usepackage{courier}

% cambia algunas fuentes en modo matemático a Palatino
\usepackage{mathpazo}

% españolización
\usepackage[spanish]{babel}
\usepackage[utf8]{inputenc}
%\extrasspanish

% gráficos y colores
\usepackage{rotating}
\usepackage{graphicx}
\usepackage{color}
%\usepackage[all]{xy}
%\usepackage{pstricks}
\usepackage{pst-node}
%\usepackage[dvips,usenames]{pstcol}
%\usepackage{pdftricks}
%\usepackage{pst-uml}  % para hacer diagramas UML
%\usepackage{rail}     % para hacer diagramas de gramáticas

% mejoras visuales
\usepackage{enumerate}
\usepackage{fancyhdr}  % para configurar los encabezados
\usepackage{fancybox}  % para hacer cajitas
\usepackage[normal,oneline,sf,bf]{caption2}
\usepackage{titlesec}  % para configurar los títulos de sección
\usepackage{paralist}


\usepackage{epigraph}

% citas, referencias e índices
\usepackage{cite}
%\usepackage{citesort}   % da errores al compilar
\usepackage{makeidx}

% incrustaciones de código fuente
%\usepackage[norules,nolineno]{lgrind}
\usepackage{verbatim}
\usepackage{listings}
%\usepackage{noweb,a4wide}


%\usepackage{textcomp}
\usepackage[right]{eurosym}

% columnas
\usepackage{multicol}

% tablas
\usepackage{longtable}
%\usepackage{ltxtable}

% impresión elegante de URLs
\usepackage{url}

\makeatletter
\def\url@pfcstyle{%
  \@ifundefined{selectfont}{\def\UrlFont{\sf}}{\def\UrlFont{\small\ttfamily}}}
\makeatother
%% Now actually use the newly defined style.
\urlstyle{pfc}


% márgenes
\usepackage[a4paper, left=30mm, right=20mm, top=25mm, bottom=25mm]{geometry}
%\usepackage[a4paper, left=20mm, right=20mm, top=25mm, bottom=25mm]{geometry}

% 
% \usepackage[a4,center,cam]{crop}
\usepackage{blindtext}

% salida en PDF navegable
%\usepackage{hyperref}
\usepackage[plainpages=false,colorlinks, linkcolor=black]{hyperref}

% quitar en versión final
%\usepackage{showkeys}   % depuración de etiquetas y referencias
%\usepackage{showidx}    % depuración de índice

% configuración del paquete "listings"

\lstset{%
        %language=Java,
	basicstyle=\footnotesize\sffamily,
	keywordstyle=\bfseries, %\color{darkred}
 	stringstyle=\itshape, %\color{violet}
 	commentstyle=\itshape, %\color{blue}
 	showspaces=false,
 	showtabs=false,
 	showstringspaces=false,
 	frame=trBL,
        frameround=tttt,
        %backgroundcolor=\color{lightyellow},
	inputencoding=utf8,
 	extendedchars=true,
 	numbers=none,
        aboveskip=0.5cm,
        belowskip=0.5cm,
        xleftmargin=1cm,
        xrightmargin=1cm,
	breaklines=true
}
\definecolor{darkred}{rgb}{0.5, 0, 0}
\definecolor{violet}{rgb}{1, 0, 1}
\definecolor{lightyellow}{rgb}{1,1,0.8}


%%%%%%%%%%%%%%%%%%%%%%%%%%%%%%%%%%%%%%%%%%%%%%%%%%%%%%%%%%%%%%%%%%%%%%
% cabeceras y pies de página (con el paquete "fancyhdr")
\headheight 15pt
%\addtolength{\headwidth}{\marginparsep}
%\addtolength{\headwidth}{\marginparwidth}
%\renewcommand{\chaptermark}[1]{\markboth{\MakeUppercase{#1}}{}}
%\renewcommand{\sectionmark}[1]{\markright{\thesection\ #1}}
%\fancyhead[LE,RO]{\textbf{\thepage}}
%\fancyhead[RE]{\textit{\leftmark}}
%\fancyhead[LO]{\rightmark}
%\fancyfoot[LCR]{}
\fancyhead{} % Todos los campos en blanco en la cabecera
\fancyfoot{} % Lo mismo al pie
\fancyhead[RO, LE]{\thepage}
\fancyhead[LO, RE]{\slshape\leftmark}
\renewcommand{\headrulewidth}{0.5pt}
\renewcommand{\footrulewidth}{0.5pt}


%%%%%%%%%%%%%%%%%%%%%%%%%%%%%%%%%%%%%%%%%%%%%%%%%%%%%%%%%%%%%%%%%%%%%%
% títulos de secciones (con el paquete "titlesec")
\titleformat{\chapter}[display]
	{\fontfamily{pag}\selectfont\Huge}
	{\LARGE\chaptertitlename\ \thechapter}{20pt}{\bfseries}
\titleformat{\section}
	{\fontfamily{phv}\selectfont\LARGE}
	{\thesection}{1em}{\bfseries}[\titlerule]
\titleformat{\subsection}
	{\fontfamily{phv}\selectfont\Large}
	{\thesubsection}{1em}{\bfseries}
\titleformat{\subsubsection}
	{\fontfamily{phv}\selectfont\large}
	{\thesubsubsection}{1em}{\bfseries}


%%%%%%%%%%%%%%%%%%%%%%%%%%%%%%%%%%%%%%%%%%%%%%%%%%%%%%%%%%%%%%%%%%%%%%
% espaciado entre párrafos
\addtolength{\parskip}{+0.2cm}


%%%%%%%%%%%%%%%%%%%%%%%%%%%%%%%%%%%%%%%%%%%%%%%%%%%%%%%%%%%%%%%%%%%%%%
% salida en PDF navegable (con el paquete "hyperref")
\hypersetup{bookmarks,
	bookmarksnumbered,
%	colorlinks, % quitar en las versiones impresas
	hyperindex,
	%linkcolor=red,
	%anchorcolor=black,
	%citecolor=green,
	citecolor=violet,
	%filecolor=magenta,
	%menucolor=red,
	%pagecolor=red,
	%urlcolor=cyan,
	pdftitle={PhD. Dissertation-Jose María Alvarez Rodríguez},
	pdfauthor={Jose María Alvarez Rodríguez}
	pdffitwindow,
	plainpages=false,
	pageanchor=false,
	pdfstartview={}}


%%%%%%%%%%%%%%%%%%%%%%%%%%%%%%%%%%%%%%%%%%%%%%%%%%%%%%%%%%%%%%%%%%%%%%
% profundidad de secciones y numeración
%\setcounter{tocdepth}{4}
\setcounter{secnumdepth}{3}


%%%%%%%%%%%%%%%%%%%%%%%%%%%%%%%%%%%%%%%%%%%%%%%%%%%%%%%%%%%%%%%%%%%%%%
% división silábica
\hyphenation{pu-bli-ca-ción con-tex-tua-li-za-ción pro-ble-mas Mi-nis-te-rio li-ci-ta-ción si-guien-tes desa-rro-llar con-tra-ta-ción pan-europea fa-ci-li-tar in-ter-opera-bility
 elec-tró-ni-ca li-ci-ta-do-ras he-te-ro-gé-neos res-trin-gi-do cons-cien-te de-sa-rro-lla-da mo-de-lo paí-ses ge-ne-ra-da
 par-ti-ci-pa-ción ad-mi-nis-tra-ti-vos bi-blio-gra-fía re-gis-tra-do-ra ad-mi-nis-tra-ti-va ca-rac-te-rís-ti-cas
 Pa-ra-le-la-men-te des-cri-ben vo-ca-bu-la-rio me-dian-te do-cu-men-to di-fe-ren-tes bi-blio-gra-fía mo-de-lo
 ve-ri-fi-ca rea-li-zar res-tric-ciones Por ejem-plo su-ge-ren-cias mo-de-los rea-li-za si-guien-do he-te-ro-gé-neas ello
 ra-zo-na-ble li-mi-ta-cio-nes reu-ti-li-cen ini-cia-ti-va va-ria-da exis-ten ob-te-ner con-ti-nua-men-te me-dian-te ins-ti-tu-ción
 pro-pues-ta pu-bli-ca-ción ca-tá-lo-go mo-ni-to-ri-za-ción vi-sua-li-za-ción mo-de-los co-rres-pon-dien-tes
 re-fe-ren-cia re-la-ti-vas par-ti-cu-lar  ele-men-to mo-de-la-do se-le-ccio-na-do pro-du-cción Re-con-ci-lia-ción des-crip-cio-nes broader-Transitive 
 de-pen-dien-do Ca-tá-lo-gos vo-ca-bu-la-rios Pro-pie-ta-rio des-plie-gue De-sa-rro-lla-dor mo-de-lar res-pues-ta si-mi-la-res rea-li-za-do SE-RI-MI 
Evol-ving ha-bi-li-tar rea-li-za-ción pu-bli-ca-ción ge-ne-ra-ción rea-li-za-ban ac-tua-li-za-ción ac-ce-so ha-cer es-tán-dar 
vo-ca-bu-la-ry con-si-de-ra-do des-cri-bir pu-bli-car pu-bli-ca-dos ta-xo-nó-mi-ca Pro-duct-on-to-lo-gy Va-lo-res in-terope-ra-bi-li-dad MOL-DEAS 
man-te-ni-mien-to re-pre-sen-tar ta-xo-no-mías des-crip-ción ge-ne-ra cons-truc-ción re-pre-sen-tan-do rea-li-za-das trans-fe-ren-cia 
uti-li-za-ble pu-bli-ca be-ne-fi-cios res-pon-sa-bi-li-da-des rea-li-zan Ad-mi-nis-tra-ción con-ti-nua-ción prio-ri-ta-rio ICT 
in-terope-ra-ble ge-ne-ra-do con-fi-gu-ra-cio-nes rea-li-zan-do rea-lis-ta má-xi-mo Cons-ti-tu-yen-do pro-ble-ma po-si-bi-li-dad ma-yor coo-pe-ra-ción 
ra-zo-na-mien-to Des-crip-tion con-cep-tua-li-za-ción de-sa-rro-lla-dos ope-ra-do-res axio-mas go-bier-no or-ga-nis-mos éxi-to 
si-guien-te do-mi-nios pro-duc-ción Na-tio-nal orien-ta-da ló-gi-ca in-di-vi-duos di-se-ña-do for-ma-li-za-dos ta-xo-no-mía 
auto-má-ti-ca ca-li-dad re-co-men-da-ble ad-mi-nis-tra-ti-vo su-mi-nis-tre error ca-rac-te-rís-ti-ca ele-men-tos he-rra-mien-tas 
eva-lua-ción res-pec-to Ad-mi-nis-tra-cio-nes ex-pe-ri-men-tal orien-ta-das re-gis-tros do-cu-men-tos má-xi-ma je-rar-quía 
sa-tis-fac-to-rios obs-tan-te con-su-mi-dos con-fe-ren-cias SPARQL rea-li-za-dos va-lo-ran eco-no-mi-zar fle-xi-ble ex-pe-dien-te 
pro-ce-di-mien-to usan-do ame-ri-ca-nas fa-mi-lia ni-vel ra-zo-na-do-res re-pre-sen-ta-ción lo-ca-li-za-do Man-ches-ter 
en-ri-que-ci-mien-to he-rra-mien-ta rea-li-zar-se prio-ri-ta-rios Li-ci-ta-cio-nes ne-ce-sa-rios ma-te-ria-li-zan es-pe-cia-li-zán-do-se 
Com-pu-ting pla-ni-fi-ca-ción cua-li-ta-ti-va de-sa-for-tu-na-da-men-te li-de-ra-do ma-te-ria-li-za-do Cons-tructs 
ma-te-ria-li-za WESO en-ten-di-mien-to MOLDEAS ela-bo-ra-ción re-fe-ren-te re-gis-tra-tion uni-la-te-ral po-si-bi-li-tar ope-rar 
in-ter-ope-ra-bles ad-mi-nis-tra-ti-vas fle-xi-bi-li-dad auto-no-mía co-mer-cio procesa-miento de-sa-rro-llan-do meta-in-for-ma-ción 
apro-pia-dos li-mi-ta-do es-pe-cia-li-za-ción ló-gi-cas in-de-pen-dien-te-men-te ge-ne-ral dis-po-ner dis-mi-nu-yen-do co-rrien-te 
rea-pro-ve-char de-sa-rro-llo ge-ne-ral Ins-ti-tu-te com-pu-ting pos-te-rior con-su-mi-da mi-llo-nes Fi-gu-ras cons-tan-te equi-va-len-cia ope-ra-cio-nes 
fac-to-ri-za-ción Res-pon-sa-bles Geo-Linked-Data reu-ti-li-za-ción co-rrec-tos trans-for-ma-ción bien con-si-guien-te des-cu-brien-do 
geo-rre-fe-ren-cia-ción pro-li-fe-ra-ción cum-pli-mien-to Ne-go-tia-ted uti-li-dad ma-nual es-pe-cia-li-za-ción ma-yo-ría mo-de-la 
des-cu-bri-mien-to rea-li-za-da pu-bli-can se-lec-cio-nar me-ca-nis-mos fun-cio-na-li-dad ve-ri-fi-car ines-pe-ra-dos re-gis-tro 
pro-pues-tos trans-pa-ren-te va-ria-bles bi-blio-te-ca bi-blio-te-cas adap-ta-bi-li-dad in-ter-co-ne-xión va-ria-ble vi-sua-li-zar 
ope-ra-ti-vo ge-ne-rar su-mi-nis-tra-dos va-li-da-ción su-mi-nis-tra-dores pro-pie-da-des hi-po-ni-mia re-cu-pe-ra-dos in-de-pen-dien-te-ment-te 
triun-fo reu-ti-li-za-ción ge-ne-ral apro-xi-ma-ción va-ria-bi-li-dad teó-ri-cas con-ti-nuo di-se-ño des-afor-tu-na-da-men-te In-dus-trial prin-ci-pios 
di-fe-ren-cia-das eva-lua-dos re-qui-si-tos co-la-bo-ra-cio-nes co-la-bo-ra-ción in-dus-tria-les Se-venth pon-de-ra-do re-po-si-to-rios 
ex-pe-ri-men-ta-les pu-bli-ci-tar co-la-bo-ra-ción bi-llo-nes Doc-to-ra-do re-so-lu-ción cul-tu-ra-les trans-cur-so Pa-rro-quias Lo-gics co-rrec-to 
sig-ni-fi-ca-do Aña-dien-do ope-ra-ción ma-ne-ra agi-li-zar con-fian-za per-so-na-les in-al-te-ra-dos re-la-ti-vos des-cri-bien-do 
se-ña-la-das vo-ca-bu-la-ries con-tex-tua-li-za-da Ba-rras cir-cuns-cri-bién-do-se for-ma-li-za-ción fa-ci-li-tan-do re-ali-men-ta-ción 
re-fe-ren-cias ma-ni-fes-tar-se CKAN lo-ca-li-dad dia-gra-ma ver-da-de-ra-men-te es-ta-ble-ci-mien-to con-si-de-ra ta-reas co-rrec-ta 
co-he-ren-te pos-te-rior-men-te apro-pia-das exac-tos va-rios pro-pues-to ex-pe-ri-men-to re-fe-ren-tes orien-ta-dos ha-bi-tual con-cre-tas 
con-fi-gu-ran afi-lia-ción Gi-ner va-lo-ra-ción res-pon-sa-bi-li-dad}   

%%Tablas
\addto\captionsspanish{
        \def\listtablename{\'Indice de tablas}%
        \def\tablename{Tabla}} 

%%%Math
\usepackage{latexsym}
\usepackage{amsmath}
\usepackage{amssymb}
\usepackage{amsthm}

\usepackage{algorithm}
\usepackage{algorithmic}
\usepackage{multirow}
\usepackage{rotating}


\newtheorem{theorem}{Theorem}[section]
\newtheorem{proposition}[theorem]{Proposición}
\newtheorem{lemma}[theorem]{Lema}
\newtheorem{definition}[theorem]{Definición}
\newtheorem{examples}[theorem]{Ejemplos}
\newtheorem{remarks}[theorem]{Remarks}
\newtheorem{corollary}[theorem]{Corolario}
\newtheorem{remark}[theorem]{Remark}
\newtheorem{example}[theorem]{Ejemplo}
\newtheorem{conjecture}[theorem]{Conjecture}
\newtheorem{note}[theorem]{Nota}



\newsavebox\FrameBox
\newenvironment{Frame}{%
  \par\setbox\FrameBox\hbox\bgroup\minipage{0.9\textwidth}\parskip\baselineskip\ignorespaces
}{%
  \endminipage\egroup\fbox{\box\FrameBox}\par
}

\newcommand{\si}{$\oplus$\xspace}
\newcommand{\no}{$\ominus$\xspace}
\newcommand{\na}{$\odot$\xspace}


%Extraer
\newcommand{\linkeddata}{\textit{Linked Data}\xspace}
\newcommand{\opendata}{\textit{Open Data}\xspace}
\newcommand{\lod}{\textit{Linking Open Data}\xspace}
\newcommand{\ogd}{\textit{Open Government Data}\xspace}
\newcommand{\datasets}{\textit{datasets}\xspace}
\newcommand{\dataset}{\textit{dataset}\xspace}
\newcommand{\provenance}{\textit{provenance}\xspace}
\newcommand{\trust}{\textit{trust}\xspace}
\newcommand{\egov}{\textit{e-government}\xspace}
\newcommand{\pusi}{\textit{Public Sector Information}\xspace}
\newcommand{\gd}{\textit{Government Data}\xspace}
\newcommand{\wod}{Web de Datos\xspace}
\newcommand{\wode}{\textit{Web of Data}\xspace}
\newcommand{\eproc}{\textit{\gls{e-Procurement}}\xspace}
\newcommand{\gld}{\textit{Government Linked Data}\xspace}





%%%%%%%%%%%%%%%%%%%%%%%%%%%%%%%%%%%%%%%%%%%%%%%%%%%%%%%%%%%%%%%%%%%%%%

\title[Suficiencia Investigadora]{Suficiencia Investigadora}
\subtitle{Programa de Doctorado 34.1: Sistemas y servicios
informáticos para internet}
\author{Jose María Alvarez Rodríguez \\ Bienio 2007-2009}


\institute{ Departamento de Informática \\ Universidad de Oviedo}


\date{11 de Septiembre de 2009}

\begin{document}

\frame{
\titlepage

}

%%%%%%%%%%%%%%%%%%%%%%%%%%%%%%%%%%%%%%%%%%%%%%%%%%%%%%%%%%%%%%%%%%%%%%

%\section*{Prologo}
%
% contenidos.tex
%

\frame{
  \frametitle{Índice}
  \tableofcontents[subsectionstyle=hide]

}

\section{Introducción}

\section{Datos Personales}
\frame{
\frametitle{¿Quién soy?}
\begin{exampleblock}{}
 
\begin{description}
 \item [Nombre:] Jose María.
 \item [Apellidos:] Álvarez Rodríguez.
 \item [DNI:] 71637566-H.
 \item [E-mail:] \url{josem.alvarez@josemalvarez.es}
 \item [Titulación:] Ingeniero en Informática por la Universidad de Oviedo,
2007. 
\end{description}

\end{exampleblock}

}


\section{Período de Docencia}


\frame{

\frametitle{Cursos de Doctorado}
\begin{block}{Cursos}
\begin{itemize}
 \item Metodología de la Investigación en Informática ($3$ cr.).
 \item Ingeniería de la Usabilidad ($3$ cr.).
 \item Web Semántica ($4$ cr.).
 \item Distribución de Contenidos en la Red de Internet  ($4$ cr.).
 \item Sistemas Adaptables, Reflectivos y Separación de Aspectos ($3$ cr.).
 \item Lenguajes Dinámicos para el Desarrollo Ágil de Aplicaciones ($3$ cr.).
\end{itemize}
\end{block}


}


\frame{

\frametitle{Trabajos de los Cursos de Doctorado}

\begin{exampleblock}{Trabajos}

\begin{itemize}
 \item Lectura y comentario de dos artículos sobre el arte de investigar.
 \item ``Redes Sociales: ¿éxito encontrado o buscado? \textit{FIXME}:
Usabilidad y Accesibilidad''.
\item ``Sistema Dinámico y Automático de Generación de Preguntas desde Fuentes
Heterogéneas basada en Formatos de Publicación de la Web Semántica''. En curso
a través de un PFC en INFORG.
\item ``Joost: plataforma de P2P TV''.
\item ``DAMA (Data Mediation Adaptative):  un enfoque para la mediación de
datos en un entorno adaptativo''. Componente utilizado en el TI.
\end{itemize}

\end{exampleblock}
}


\frame{
  \frametitle{Cursos de Doctorado} 

...algunas conclusiones...

\begin{figure}[htb]
\centering
	\includegraphics[width=4cm]{images/conclusion}

\end{figure}


}


\frame{

\frametitle{Cursos de Doctorado}

\begin{exampleblock}{Conclusiones}
 \begin{itemize}
\item Promueven la redacción de trabajos como artículos de investigación.
\item Dan a conocer de las áreas del Departamento de Informática
\item Tratan temas verticales más allá de la tecnología.
\item Promueven la importancia de la investigación en las diferentes áreas.
\item Promueven el uso del inglés.
\end{itemize}
\end{exampleblock}

}


\frame{
  \frametitle{Cursos de Doctorado-Evaluación} 

Motivación.

\begin{figure}[htb]
\centering
	\includegraphics[width=4cm]{images/evaluation}
\end{figure}


}


\section{Período de Investigación}

\subsection{Trabajo de Investigación}

\frame{
  \frametitle{Trabajo de Investigación}

\begin{exampleblock}{Sobre el trabajo...}
\begin{itemize}[<+-| alert@+>]
 \item ``Interoperabilidad e Integración en Arquitecturas Orientadas a Servicios
basadas en Semántica''.
\item Propuesta no intrusiva para añadir
semántica (siguiendo los estándares desarrollados bajo la iniciativa de Web
Semántica) a las arquitecturas orientadas a servicios (SOA).
\item Calificación: \textit{Sobresaliente}.

\end{itemize}
\end{exampleblock}

}

\frame{
  \frametitle{Objetivos}

Generales, realistas y teniendo en cuenta el contexto.

\begin{figure}[htb]
\centering
	\includegraphics[width=4cm]{images/target}
\end{figure}

}

\frame{
  \frametitle{Objetivos}

\begin{exampleblock}{Lista de Objetivos}
\begin{enumerate}[<+->]
\item Estudiar y aplicar la semántica para formalizar modelos de conocimiento
compartido de un sector de negocio. 
 \item Estudiar el uso de modelos de conocimiento compartido para unificar
modelos de datos divergentes en un mismo sector de negocio.
 \item Proponer y definir una arquitectura SOA utilizando un modelo de
conocimiento de dominio (MCD) de un sector de negocio unificando modelos
de datos y operaciones. 
 \item Validar la propuesta realizada. 
\end{enumerate}
\end{exampleblock}

}


\frame{
  \frametitle{Conclusiones}

\begin{exampleblock}{Sobre nuestro trabajo...}
\begin{itemize}
 \item El uso de estándares (BPEL, OWL, RDF, etc.) es clave. 
\item SOA, un paradigma múltiples implementaciones.
\item El uso de semántica en SOA debe ser no intrusivo.
\item La ejecución de los procesos de negocio creados a partir
de semántica debe ser delegada a herramientas comerciales como un ESB.
\item El uso de un MCD para un sector de negocio tiene implicaciones positivas
para todos los usuarios.
\item La generación del código BPEL guiado por semántica es la  mayor
aportación de esta propuesta.
\item La semántica en SOA mejora su flexibilidad.
\end{itemize}
\end{exampleblock}

}

\subsection{Actividad Docente}



\frame{
  \frametitle{Profesor}

Actividades docentes...

\begin{figure}[htb]
\centering
	\includegraphics[width=4cm]{images/profesor}
\end{figure}


}

\frame{
  \frametitle{Profesor Asociado de CCIA}

\begin{exampleblock}{Concurso}<1->
\begin{itemize}
 \item  Código: F036-75-DL0X041-AL6H. BOPA de: Boletín Nº 181 del lunes 4 de
agosto de 2008.
\item Departamento de Informática. Universidad de Oviedo.
\end{itemize}


\end{exampleblock}

\begin{block}{Asignaturas en INFORG-Curso 2008-2009}<2->
\begin{itemize}
\item Metodología de la Programación I (6 cr.). Prácticas.
\item Sistemas Operativos II (3 cr.). Teoría y prácticas.
\item Modelos de Componentes (1.5 cr.) Prácticas
\end{itemize}
\end{block}

}


\frame{

\frametitle{Evaluación}

\begin{alertblock}{Valoración}
\begin{itemize}
 \item Ilusión.
 \item Motivación.
 \item Reto de la enseñanza.
\end{itemize}

\end{alertblock}

}

\subsection{Proyectos}

\frame{
  \frametitle{Participación en:}
\begin{alertblock}{Trabajo en:}
 Desde el 2005 en el área de I+D+i, línea de Tecnologías Semánticas. Fundación
CTIC.
\end{alertblock}


\begin{exampleblock}{Redacción de propuestas}<1->
\begin{itemize}
\item \textit{Seventh
Framework Programme}-FP7~\footnote{\url{http://cordis.europa.eu/fp7/}}.
\item Plan Avanza~\footnote{\url{http://www.planavanza.es/}}
\item Plan Regional de Ciencia Tecnología
e Innovación-PCTI~\footnote{\url{http://www.ficyt.es/pcti/}}.

\end{itemize}

\end{exampleblock}


}



\frame{
  \frametitle{Ejecución de:}

\begin{exampleblock}{Proyectos}<1->
\begin{itemize}

\item ONTORULE:\textit{ ONTOlogies meet business RULEs}.
código. FP7-ICT-2007.4.4, ref. 231875.

\item PRIMA: Plataforma de Recursos de Información
y Movilidad para el
sector Asegurador. Plan Avanza I+D. PROFIT Tractor, código TSI-020302-2008-32.

\item PRAVIA: Plataforma de Recursos de Acceso Virtual a la Información del
Sector Asegurador. PCTI, código IE05-172.

\item Otros: EZWeb (Avanza) y SAITA (PCTI).
\end{itemize}

\end{exampleblock}


}

\subsection{Otros}

\frame{
  \frametitle{Otros Méritos}

\begin{block}{Premio del COIIPA}
\begin{itemize}
 \item Primer Premio ``Douglas Engelbart'', Premios Proyecto Fin de Carrera de
Ingeniería Informática.
\item Concede: Colegio de Ingenieros Informáticos de Asturias.
\item Título del proyecto: ``Activación de Conceptos en Ontologías mediante el
Algoritmo de Spreading Activation''.
\item Fecha: Junio 2008.
\item Director: José Emilio Labra Gayo.
\item Tipo: Investigación.
\item Calificación: 10.
\end{itemize}

\end{block}


}

\frame{
  \frametitle{Otros Méritos}

\begin{block}{Actividades}
\begin{itemize}
\item Curso de ``Redacción de propuestas de proyectos para el VII
Programa Marco de I+D''. 20h. FADE. Febrero 2009.
\item Grupo de Trabajo: miembro de  SEA:
\textit{Service Engineering} \& (\textit{service-oriented})
\textit{Architectures} de la iniciativa
INES desde el 2007.
\item Revisión de artículos: \textit{Social Data on the Web 2008}.
(SDoW 2008). Co-located with ISWC 2008. October 27, 2008 Karlsruhe (Germany).
\end{itemize}

\end{block}


}

\subsection{Difusión e Impacto}

\frame{
  \frametitle{Difusión e impacto}

Parte de esta actividad está plasmada en...

\begin{figure}[htb]
\centering
	\includegraphics[width=4cm]{images/research-papers}
\end{figure}


}


\frame{
  \frametitle{Difusión e impacto}

\begin{exampleblock}{Artículos y presentaciones en Conferencias Internacionales}
\begin{itemize}
 \item Jose María Álvarez y Antonio Campos. "Integration and Interoperability on
Service Oriented Architectures using Semantics". In conjunction with ICWE'2009,
San Sebastián, Spain, June 22, 2009. Proceedings of the Doctoral Consortium of
the International Conference on Web Engineering~\footnote{\url{
http://sunsite.informatik.rwth-aachen.de/Publications/CEUR-WS/Vol-484/}}.

\item Miguel García Rodríguez, Jose María Álvarez, Diego Berrueta and Luis
Polo. "Declarative Data Grounding Using a Mapping Language". The 3rd
International Conference on Complex Distributed Systems (CODS 2009). Leipzig,
Germany, 23-25 March 2009.
\end{itemize}
\end{exampleblock}



}



\frame{
  \frametitle{Difusión e impacto}

\begin{block}{Revistas}
\begin{itemize}
\item Miguel García Rodríguez, Jose María Álvarez, Diego Berrueta and Luis
Polo. "Declarative Data Grounding Using a Mapping Language". Communications of
SIWN. ISSN  1757-4439. April 2009. 
\end{itemize}
\end{block}

}


\frame{
  \frametitle{Difusión e impacto}

\begin{exampleblock}{Workshops Internacionales}
\begin{itemize}
\item Jose María Álvarez Rodríguez, Emilio Rubiera Azcona y Luis Polo
 Paredes. ``Promoting Government Controlled Vocabularies for the Semantic Web:
the EUROVOC Thesaurus and the CPV Product Classification System''. ESWC 2008,
The 5th European Semantic Web Conference. Semantic Interoperability in the
European Digital Library (Workshop)-SIEDL 2008. Tenerife, Spain, June 2008.
\end{itemize}
\end{exampleblock}

}



\frame{
  \frametitle{Difusión e impacto}

\begin{block}{Entregables de Proyectos}
\begin{itemize}

\item Jose María Álvarez. (editor). `` D7.2 Dissemination Strategy. V1''.
Proyecto ONTORULE, entregable D7.2. Junio 2009. 

\item Jose María Álvarez. (editor). `` Especificación de la arquitectura de la
plataforma genérica SILO''. Proyecto PRIMA, entregable D.3.1. Abril 2009. 


\item Jose María Álvarez. (editor). `` Especificación de requisitos y
arquitectura de alto nivel, v1.0''. Proyecto PRIMA, entregable D.1.1.
Diciembre 2008. 

\item Jose María Álvarez. (co-editor). ``Memoria Técnica''. Proyecto PRAVIA,
Diciembre 2007. 


\item Jose María Álvarez. (editor). ``Evaluación de plataformas de
servicios web semánticos''. Proyecto PRAVIA , entregable D.8.1. Agosto 2007. 


\end{itemize}
\end{block}



}


\frame{
  \frametitle{Futuro de Tesis}

\begin{figure}[htb]
\centering
	\includegraphics[width=4cm]{images/futuro}
\end{figure}


}

\section{Futuro}
\frame{

\frametitle{Futuro de Tesis}



\begin{alertblock}{Tendencia...}<1->
 ``Trends the Internet of
Services''~\url{http://cordis.europa.eu/fp7/ict/ssai/home\_en.html}
\end{alertblock}


\begin{exampleblock}{Nuestra tendencia...}<2->
 \begin{itemize}
  \item Aplicar semántica como una capa superior de los servicios capaz de
generar código siguiendo un estándar, BPEL.
\item Plataforma de ejecución basada en estándares delegación en
productos ESB+BPEL \textit{engine}.
 \end{itemize}
\end{exampleblock}

\begin{alertblock}{Finalmente...}<3->
 Queda mucho por hacer...investigar, trabajar, probar, etc.
\end{alertblock}

}


%%%%%%%%%%%%%%%%%%%%%%%%%%%%%%%%%%%%%%%%%%%%%%%%%%%%%%%%%%%%%%%%%%%%%%
\appendix


%
% ultima.tex
%

\frame{

  \note[item]{Agradecer la atenci�n prestada.}
  \note[item]{Quedar a disposici�n del tribunal para contestar a las
    preguntas o ampliar cualquiera de los temas expuestos.}

\titlepage
}




\end{document}

