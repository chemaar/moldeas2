\subsection{Conclusiones}

\frame{
  \frametitle{Conclusiones} 

...y después de todo podemos concluir...

\begin{figure}[htb]
\centering
	\includegraphics[width=4cm]{images/conclusion}
\caption{Conclusión.}

\end{figure}


}


\frame{
  \frametitle{Conclusiones}

\begin{alertblock}{Sobre el estado del arte...}
\begin{itemize}
 \item SOA como paradigma es enorme y variado.
\item El ``ruido'' relativo a SOA es muy importante.
\item La semántica no debe ser intrusiva con la tecnología actual.
\item SOA+semántica no se han conseguido resultados realmente funcionales. Pero
podemos reaprovechar el efecto experiencia
\end{itemize}
\end{alertblock}

}


\frame{
  \frametitle{Conclusiones}

\begin{exampleblock}{Sobre nuestro trabajo...}
\begin{itemize}
 \item El uso de estándares (BPEL, OWL, RDF, etc.) es clave. 
\item SOA, un paradigma múltiples implementaciones.
\item El uso de semántica en SOA debe ser no intrusivo.
\item La ejecución de los procesos de negocio creados a partir
de semántica debe ser delegada a herramientas comerciales como un ESB.
\item El uso de un MCD para un sector de negocio tiene implicaciones positivas
para todos los usuarios.
\item La generación del código BPEL guiado por semántica es la  mayor
aportación de esta propuesta.
\item La semántica en SOA mejora su flexibilidad.
\end{itemize}
\end{exampleblock}

}

\subsection{Trabajo Futuro}

\frame{
  \frametitle{Trabajo Futuro} 

Quedan muchas acciones abiertas...

\begin{figure}[htb]
\centering
	\includegraphics[width=4cm]{images/futuro}
\caption{Futuro.}

\end{figure}


}


\frame{
  \frametitle{Trabajo Futuro}

\begin{block}{...corto plazo...}
\begin{itemize}
 \item Desarrollar por completo los componentes especificados.
\item Recrear el escenario del entorno asegurar.
\item Fijar una metodología de evaluación de resultados.
\item Mejorar la arquitectura en base a las pruebas realizadas.
\item Continuar en alerta tecnológica e investigadora de los proyectos de
investigación relacionados.
\item Buscar sinergias con otros proyectos o investigaciones similares.
\item Alinear nuestro enfoque a los estándares en la mayor medida de lo posible.
\item Contribuir al ámbito científico mediante publicaciones.
\end{itemize}
\end{block}

}


\frame{
  \frametitle{Trabajo Futuro}

\begin{block}{...medio/largo plazo...}
\begin{itemize}
 \item Enlazar nuestra propuesta con algún entorno de desarrollo comercial.
 \item Avanzar en la relación entre los usuarios de negocio: BPMN, SBVR,
o reglas de negocio.
\item Alinear nuestra propuesta en el contexto de MDA.
\item Promover el uso de esta tecnología en las empresas.
\end{itemize}
\end{block}

}


\subsection{Difusión e Impacto}

\frame{
  \frametitle{Difusión e impacto}

Parte de este trabajo está plasmado en...

\begin{figure}[htb]
\centering
	\includegraphics[width=4cm]{images/research-papers}
\caption{Publicaciones.}

\end{figure}


}


\frame{
  \frametitle{Difusión e impacto}

\begin{exampleblock}{Artículos y presentaciones en Conferencias Internacionales}
\begin{itemize}
 \item Jose María Álvarez y Antonio Campos. "Integration and Interoperability on
Service Oriented Architectures using Semantics". In conjunction with ICWE'2009,
San Sebastián, Spain, June 22, 2009. Proceedings of the Doctoral Consortium of
the International Conference on Web Engineering~\footnote{\url{
http://sunsite.informatik.rwth-aachen.de/Publications/CEUR-WS/Vol-484/}}.

\item Miguel García Rodríguez, Jose María Álvarez, Diego Berrueta and Luis
Polo. "Declarative Data Grounding Using a Mapping Language". The 3rd
International Conference on Complex Distributed Systems (CODS 2009). Leipzig,
Germany, 23-25 March 2009. 
\end{itemize}
\end{exampleblock}



}


\frame{
  \frametitle{Difusión e impacto}

\begin{block}{Revistas}
\begin{itemize}
\item Miguel García Rodríguez, Jose María Álvarez, Diego Berrueta and Luis
Polo. "Declarative Data Grounding Using a Mapping Language". Communications of
SIWN. ISSN  1757-4439. April 2009. 
\end{itemize}
\end{block}

}


\frame{
  \frametitle{Difusión e impacto}

\begin{block}{Entregables de Proyectos}
\begin{itemize}

\item Jose María Álvarez. (editor). `` Especificación de la arquitectura de la
plataforma genérica SILO''. Proyecto PRIMA, entregable D.3.1. Abril 2009. 


\item Jose María Álvarez. (editor). `` Especificación de requisitos y
arquitectura de alto nivel, v1.0''. Proyecto PRIMA, entregable D.1.1.
Diciembre 2008. 

\item Jose María Álvarez. (co-editor). ``Memoria Técnica''. Proyecto PRAVIA,
Diciembre 2007. 


\item Jose María Álvarez. (editor). ``Evaluación de plataformas de
servicios web semánticos''. Proyecto PRAVIA , entregable D.8.1. Agosto 2007. 


\end{itemize}
\end{block}



}
