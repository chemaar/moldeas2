\frame{
\begin{exampleblock}{Sistema MOLDEAS}
Consumo de Datos Enlazados Abiertos. 
\end{exampleblock}
}


\frame{
  \frametitle{Consumo de Datos Enlazados Abiertos}

\begin{exampleblock}{Objetivos Generales}<1->
 \begin{itemize}
 \item Consumir los datos enlazados desde un lenguaje de programación.
 \item Crear un sistema de recuperación de información.
\end{itemize}
\end{exampleblock}


\begin{block}{1-Definición de los objetivos del experimento}<2->
\begin{enumerate}
 \item ¿Es posible implementar un sistema de recuperación de información utilizando datos enlazados?
 \item ¿Es posible explotar las relaciones semánticas establecidas para mejorar la recuperación de información?
 \item ¿Cuál es el mejor enfoque para la recuperación de información en los anuncios de licitación? 
 \item ¿Cómo afectan los resultados en la implementación actual del sistema MOLDEAS?
\end{enumerate}

\end{block}

}


\frame{
  \frametitle{Consumo de Datos Enlazados Abiertos}

\begin{block}{2-Selección de una regla de asignación de las unidades experimentales a las condiciones de estudio}
\begin{enumerate}
 \item Unidad experimental de este estudio será un repositorio RDF.
\item Base documental $\mathcal{D}$ constituida por $1$ millón de anuncios de licitación.
\item Vocabulario controlado, $\mathcal{V}$, del CPV 2008, formado por 10357 códigos/términos distintos.
\item Cada documento $d \in \mathcal{D}$, etiquetado con al menos un código $v \in \mathcal{V}$.
\item $11$ consultas, $Q_{str}$, proporcionadas por Euroalert.net.
\item Las medidas de evaluación dependen del nº de códigos CPV generados por MOLDEAS.
\end{enumerate}
\end{block}

}

\frame{
  \frametitle{Consumo de Datos Enlazados Abiertos}
\small
\begin{longtable}[c]{|l|p{6cm}|p{3cm}|} 
\hline
\textbf{$Q_{i}$} &  \textbf{Consulta de Usuario-$Q_{str}$} &  \textbf{Nº de Códigos CPV relevantes-$\#Q^{i}_{cpv}$} \\\hline
\endhead
$Q_1$ & ``Comprehensive overview over all environmental technologies renewable energy products'' & $463$ \\ \hline
$Q_2$ & ``Tendering of public works: housing, hospitals, roads, housing developments, station drinking water treatment, reforestation'' & $35$ \\ \hline
$Q_3$ & ``Prefabricated buildings'' & $7$ \\ \hline
$Q_4$ & ``Games for children (parks swings, slides, land of play equipment in the public sphere'' & $26$ \\ \hline
$Q_5$ & ``Vital signs monitor'' &  $277$\\ \hline
\multicolumn{3}{|c|}{\textbf{...}} \\ \hline
 \hline
\end{longtable}

}

\frame{
  \frametitle{Consumo de Datos Enlazados Abiertos}
\small
\begin{longtable}[c]{|l|p{6cm}|p{3cm}|} 
\hline
\textbf{$Q_{i}$} &  \textbf{Consulta de Usuario-$Q_{str}$} &  \textbf{Nº de Códigos CPV relevantes-$\#Q^{i}_{cpv}$} \\\hline
\endhead
$Q_6$ & ``Museum and exhibition and product launch services'' & $1$ \\ \hline
$Q_7$ & ``Voltmeters, instruments measuring electrical quantities, Ammeters, Instruments for checking physical characteristics\ldots'' & $117$ \\ \hline
$Q_8$ & ``Conservation Maintenance of pavements for roads, airfields, bridges, tunnels'' & $13$ \\ \hline
$Q_9$ & ``Wood poles, Wooden sleepers, Lattice towers'' & $10$ \\ \hline
$Q_{10}$ & ``Architectural, construction, engineering and inspection services'' &  $173$\\ \hline
$Q_{11}$ & ``Medical practice and related services'' &  $13$\\ \hline
 \hline
\end{longtable}

}

\frame{
  \frametitle{Consumo de Datos Enlazados Abiertos}
\small
\begin{longtable}[c]{|l|p{5cm}|p{3.5cm}|} 
\hline
\textbf{Método} &  \textbf{Descripción} &  \textbf{Tecnología} \\\hline
\endhead
$M^1$ & Se indexan las descripciones de los códigos CPV y proceso de búsqueda sintáctica de las consultas $Q_{i}$. & Apache Lucene y Solr \\ \hline
$M^2$ & Se extraen una serie de códigos CPV candidatos según jerarquía. & $M^1$ + ponderación \textit{broader/ narrower} \\ \hline
$M^3$ & \ldots según jerarquía con \textit{Spreading Activation}. & $M^1$ + ONTOSPREAD \\ \hline
$M^4$ & \ldots según histórico de las relaciones entre códigos de los anuncios previos. & $M^1$ + Apache Mahout \\ \hline
\hline
\end{longtable}

}

 
 
 \frame{
\frametitle{Consumo de Datos Enlazados Abiertos}
 
 \begin{block}{3-Especificación de las medidas de trabajo en cuanto a la respuesta}<1->
\begin{enumerate}
\item Para cada consulta se recogen los códigos CPV 2008 generados.
\item Se comparan con los indicados en las consultas $Q_{i}$.
\item Se obtienen las medidas Precisión, \textit{Recall}, \textit{Accuracy} y \textit{Specificity} (\textit{PRAS}).
\end{enumerate}

\end{block}

\begin{exampleblock}{5-Ejecución de un experimento piloto}<2->
En primer lugar se realiza una consulta para verificar el proceso de búsqueda en cada método 
y la obtención de medidas.
\end{exampleblock}

}

 \frame{
\frametitle{Consumo de Datos Enlazados Abiertos}
 
 \begin{block}{6-Esquematización de los pasos a seguir}<1->
\begin{enumerate}
 \item A cada consulta $Q_{str}$, identificada como $Q_{i}$, se le aplica un método $M^{i}$, devuelve al $\#Q^{i}_{cpv}$ elementos.
 \item Cada conjunto resultado $Q^{M^i}_{cpv}$ se compara con el conjunto esperado $Q^{i}_{cpv}$ con un \textit{script}.
 \item Se generan los valores \textit{PRAS} para cada método $M^{i}$ y consulta de entrada $Q_{i}$.
\end{enumerate}

\end{block}

 \begin{alertblock}{Otros}<2->
\begin{enumerate}
  \item 4-Especificación de un modelo (N/A).
  \item 7-Determinación del tamaño muestral (ya indicado en el punto 1).
  \item 8-Revisión de las decisiones anteriores.
\end{enumerate}

\end{alertblock}

}

 \frame{
   \frametitle{Consumo de Datos Enlazados Abiertos-\linebreak Resultados Agregados ($\bar{X}$)}
 
\small
\begin{longtable}[c]{|l|c|c|c|c|} 
\hline
\textbf{Método} &  \textbf{Precisión} & \textbf{Recall} & \textbf{Accuracy} & \textbf{Specificity} \\\hline
\endhead
$M^1$ &  $0,28$ & $0,26$ & $0,99$ & $1,00$ \\ \hline
$M^2$ &  $0,11$ & $0,11$ & $0,98$ & $0,99$   \\ \hline
$M^3$ &  $0,23$ & $0,23$ & $0,99$ & $1,00$  \\ \hline
$M^4$ &  $0,03$ & $0,03$ & $0,96$ & $0,98$\\ \hline
\hline
\end{longtable}

}

\frame{
  \frametitle{Consumo de Datos Enlazados Abiertos-Resultados}

\begin{block}{Valoración}
 \begin{enumerate}
\item El tipo y formato de una fuente de datos no es impedimento para la construcción de servicios 
en un dominio determinado.
\item Las relaciones semánticas de los datos se pueden explotar para recuperar información.
\item El enfoque tradicional sintáctico, $M^1$, se comporta más cercano a las expectativas del usuario.
 \end{enumerate}
\end{block}
}
% 

\frame{
  \frametitle{Consumo de Datos Enlazados Abiertos-Conclusiones}

\begin{exampleblock}{Principal Punto Clave}
La \textbf{casuística} de un sistema de soporte a la decisión o de recuperación a la información 
en \textbf{\eproc} es muy \textbf{compleja}, existen \textbf{muchas} \textbf{variables} de \textbf{información} que se pueden optimizar.
\end{exampleblock}
}

